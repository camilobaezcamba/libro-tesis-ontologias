%!TEX root = ../main.tex
\section{Trabajos Futuros}
\label{chap:futuros}

Como trabajo futuro se podrá extender la ontología desarrollada a modo que cumpla con los requrimientos de las nuevas esecificaciones del la version 2 del OCDS, como ser la internacionalizacion de la ontologia y mayor enriquecimiento de la ontologia con otras ontolgias de dominio como por ejemplo una futura ontologia de sistema de registrio de empresas o sistemas de tributacion publica como ser el caso local de la Sectraria de Estado de Tribucacion. 

Ademas de una metodologia para que los implementadores de los demas paises puedar extender, como es en el caso de Paraguay, la ontologia del OCDS a conceptos locales especificos de cada implementacion.

En el area de consultas a base de datos en forma de triplas, se podrá comparar el rendimiento entre distintos sistemas de almacenamiento y consulta de triplas.

Por ultimo, es importante la investigacion de casos de uso de adopcion y comparacion en entornos de produccion entre sistemas de intercambio de informacion enfocados a sintaxis como ser JSON, XML y sistemas enfocados en semantica como ser RDFS y OWL. 

%!TEX root = ../main.tex
\section{Trabajos Futuros}
\label{chap:futuros}

Como trabajo futuro se podrá extender la ontología desarrollada a modo que cumpla con los requerimientos de las nuevas especificaciones de la versión 2 del OCDS, como ser la internacionalización de la ontología y mayor enriquecimiento reutilizando otras ontolgías de dominio como por ejemplo una futura ontología de sistema de registro de empresas o sistemas de tributación pública como ser el caso de la Secretaría de Estado de Tributación en Paraguay. 

Además de una metodologia para que los implementadores de los demás paises puedan extender, como es en el caso de Paraguay, la ontología del OCDS a conceptos locales específicos de cada implementación.

En el área de consultas a base de datos, en forma de triplas, se podrá comparar el rendimiento entre distintos sistemas de almacenamiento y consulta de triplas.

Por último, es importante la investigación de casos de uso de adopción y comparación en entornos de producción entre sistemas de intercambio de información enfocados a sintaxis como ser JSON, XML y sistemas enfocados en semántica como ser RDFS y OWL. 
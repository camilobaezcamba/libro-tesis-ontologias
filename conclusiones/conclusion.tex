%!TEX root = ../main.tex
\section{Conclusiones}
\label{chap:analisis}
A continuación se describen los objetivos específicos y las conclusiones desarrolladas a lo largo del trabajo. \ref{chap:introduccion} 

El objetivo \textbf{O1} es identificar y comparar las bases de representación de conocimientos asociado al dominio de Contrataciones Públicas. En el apartado \ref{section:busqueda de ontologias} se hizo una búsqueda, identificación y comparación de todas las ontologías del dominio de contrataciones públicas, con el fin de su posterior reuso en la construcción de la ontología del trabajo. El resultado del análisas se puede observar en la Tabla \ref{tab:comparacion_ontologias}.

El objetivo \textbf{O2} es describir el proceso de modelado de una ontología utilizando una base de representación de conocimiento de dominio. En el Capítulo \ref{chap:Implementación de la Ontologia} se muestra el proceso de modelado y desarrollo de la ontología utilizando la metodología NeOn.

El objetivo \textbf{O3} es modelar un sistema de extracción y consultas de datos del dominio Contrataciones Públicas. En el Capítulo \ref{chap:Implementación de la Ontologia} se modeló un sistema de extracción y conversión de los datos del formato JSON a una base de información basada en triplas RDF enriquecida con la ontología desarrollada. La arquitectura desarrollada puede verse en la Figura \ref{img:modelo de Implementacion}.

El objetivo \textbf{O4} es mostrar el uso de la ontología con datos de Contrataciones Publicas del Paraguay. En el Capítulo \ref{chap:analisis} se presentaron 5 casos de uso de una fuente de datos almacenados en forma de triplas y enriquecida con una ontología con el fin de mostrar el tipo de consultas y la capacidad de interoperabilidad semántica con otras fuentes externas de datos

En el trabajo se desarrollo de modo experimental el procedimiento necesario para modelar una ontología, integrar con una fuente de datos existente y luego verificar la capacidad de interoperabilidad semántica de los datos. 
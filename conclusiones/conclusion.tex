%!TEX root = ../main.tex
\section{Conclusiones}
\label{chap:analisis}
A continuación se describen los objetivos específicos y las conclusiones desarrolladas a lo largo del trabajo.  

El objetivo \textbf{O1} es identificar y comparar las bases de representación de conocimientos asociado al dominio de Contrataciones Públicas. En el apartado \ref{section:busqueda de ontologias} se hizo una búsqueda, identificación y comparación de todas las ontologías del dominio de contrataciones públicas, con el fin de su posterior reuso en la construcción de la ontología del trabajo. El resultado del análisis se puede observar en la Tabla \ref{tab:comparacion_ontologias}. La DNCP cuenta con su propia ontología centrada más en la sintaxis que en la semántica para enriquecer los datos publicados. Sin embargo, los datos que siguen los delineamientos del OCDS no tienen una ontología asociada ya que el estándar mismo no dispone de una. La ontología OCDSV0 fue utilizada como punto de partida para el desarrollo de la ontología OCDSPY (contribución \textbf{C1}). Cabe destacar también a la ontología PCO la cual fue utilizada para lograr la interoperabilidad semántica entre distintas fuentes de datos.

El objetivo \textbf{O2} es describir el proceso de modelado de una ontología utilizando una base de representación de conocimiento del dominio. En el Capítulo \ref{chap:Implementación de la Ontologia} se muestra el proceso de modelado y desarrollo de la ontología utilizando la metodología NeOn. Cabe mencionar que para el proceso de modelado se utilizaron tanto recursos ontológicos como no-ontológicos, así también patrones de diseño que facilitaron y agilizaron el proceso (contribución \textbf{C6}). Un punto que cabe destacar es que la metodología NeOn no fue utilizada al pie de la letra, en algunos casos específicos se realizaron ajustes ad hoc. Para el proceso de modelado (contribución \textbf{C2}) se utilizó en su mayor parte un recurso no-ontológico basado en JSON-SCHEMA (OCDS).

El objetivo \textbf{O3} es modelar un sistema de extracción y consultas de datos del dominio de Contrataciones Públicas. En el Capítulo \ref{chap:Implementación de la Ontologia} se modeló un sistema de extracción y manipulación (contribución \textbf{C3}) de los datos del formato JSON a una base de información basada en triplas RDF enriquecida con la ontología desarrollada. La arquitectura desarrollada puede verse en la Figura \ref{img:modelo de Implementacion}.

El objetivo \textbf{O4} es mostrar el uso de la ontología con datos de Contrataciones Publicas del Paraguay. En el Capítulo \ref{chap:Contexto experimental} se presentaron 5 casos de uso de una fuente de datos almacenados en forma de triplas y enriquecida con una ontología con el fin de mostrar el tipo de consultas y la capacidad de interoperabilidad semántica con otras fuentes externas de datos. En cada caso se muestran las consultas realizadas al Punto SPARQL (contribución \textbf{C4}) y también se muestran fragmentos de código en los cuáles se evidencia el uso de las ontologías verificando así la adecuación de la misma a los datos disponibilizados por la DNCP (contribución \textbf{C5}).

En el trabajo se desarrolló de modo experimental el procedimiento necesario para modelar una ontología, integrar con una fuente de datos existente y luego verificar la capacidad de interoperabilidad semántica de los datos. 
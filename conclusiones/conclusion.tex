%!TEX root = ../main.tex
\section{Conclusiones}
\label{chap:analisis}
Las conclusiones que se obtienen para cada uno de los objetivos específicos del trabajo descritos en la sección \ref{chap:introduccion} 

El objetivo \ref{obj:1} es identificar y comparar las bases representación de conocimientos asociado al dominio de Contrataciones Públicas.En el apartado \ref{section:busqueda de ontologias} se hizo una busqueda, identificacion y comparacion de todas las ontologias del dominio de contrataciones públicas, con el fin de su posterior reuso en la contruccion de la ontologia del trabajo.

El objetivo \ref{obj:2} es Describir el proceso de modelado de una ontologia utilizando una base de representacion de conocimiento de dominio. Se el capitulo \ref{chap:Implementación de la Ontologia} se muestra el proceso del modelado y desarrollo de la ontologia utilizando la metodología NeOn.

El objetivo \ref{obj:3} es modelar un sistema de extracción y consultas de datos del dominio Contrataciones Públicas. En el Capitulo \ref{chap:Implementación de la Ontologia} se modeló un sistema de extracion, y conversion de los datos del formato JSON a una base de informacion basada en triplas RDF enriquecidad con la ontologia desarrollada. 

El objetivo \ref{obj:4} es mostrar el uso de la ontologia con datos de Contrataciones Publicas del Paraguay. En el capitulo \ref{chap:analisis} se e presentaron 5 casos de uso de una fuente de datos almacenados en forma de triplas y enriquecida con una ontologia con el fin de mostrar el tipo de consultas y la capacidad de interoperabilidad semantica con otras fuentes externas de datos

En el trabajo se desarrollo de modo experimental el procedimiento necesario para modelar una ontologia, integrar con una fuente de datos existente y luego verificar la capacidad de interoperabilidad semantica de los datos. 


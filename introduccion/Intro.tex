%!TEX root = ../main.tex
\chapter{Introducción}
\label{chap:introduccion}




Según el reporte de \textit{Open Contracting Partnership}(OCP)\footnote{https://www.open-contracting.org/?lang=es}, los gobiernos de todo el mundo gastan un estimado de 9.5 billones de dólares anualmente mediante contratos públicos en bienes y servicios. A esta contratación de de bienes y servicios por parte de un Estado se llama Contratación Pública.  Sin embargo, en la mayoría de los países, la información sobre estos contratos no está disponible para el escrutinio público, haciendo al proceso de contratación vulnerable a la corrupción y la mala gestión. Además, los países que sí lo hacen, no proveen un contexto apropiado para entender, reutilizar e integrar dichos datos con otros.

Debido a la creciente importancia de publicar los datos de contrataciones públicas los estados crean portales de datos con el fin de transparentar los procesos y establecer mecanismos para que la ciudadanía, empresas y organizaciones puedan ejercer el rol de contralores y a la vez facilitar a los entes gubernamentales a contratar bienes y servicios. Éste es el caso del portal \textit{Tender Electronic Daily} (TED)\footnote{https://ted.europa.eu/TED/main/HomePage.do}  en la Unión Europea y República Checa\footnote{https://opentender.eu/cz/}. En Paraguay, estos datos pueden ser obtenidos desde el Portal de Datos Abiertos de la Dirección Nacional de Contrataciones Públicas \footnote{https://www.contrataciones.gov.py/datos} (DNCP).

Esta enorme cantidad de datos está disponible en sistemas con representaciones de datos distintas, provocando problemas al momento de interoperar. Por interoperabilidad se entiende que un sistema pueda procesar de manera automática y consistente datos de otros sistemas para enriquecer y complementar la información. Esto se logra a través de la Web Semántica \cite{Semantic20:online}, que es el conjunto de estándares y tecnologías que añade de manera formal el significado y la relación que existe entre los datos de manera a que éstos puedan ser procesables por máquinas.

Según Svátek \cite{svatek2014linked}, los datos de contrataciones públicas son importantes dentro de la Web Semántica \cite{Semantic20:online} porque unifican dos esferas muy diferentes como son: las necesidades públicas por un lado y la oferta de productos y servicios del sector privado por el otro; este escenario es idóneo para la utilización de modelos de datos, metodologías y fuentes de información publicadas de forma independiente. Además, la gran cantidad de datos de contrataciones públicas da lugar para aplicar diversos métodos de análisis de datos que van desde estadísticas agregadas en forma de barras hasta análisis más exhaustivos de minería de datos.



\section{Justificación}

Debido a la gran cantidad de datos disponibles en internet, la integración entre sistemas de información heterogéneos que disponibilizan estos datos se vuelve un problema complejo de resolver. 

Este desafío puede ser abordado a través de tecnologías de la Web Semántica. En ella se integran estándares y herramientas necesarias para la disponibilización de recursos en la web de manera a que sean entendibles por máquinas y a la vez hace posible la integración de recursos de diferentes fuentes. 

Las ontologías proveen un modelo de descripción de significado de los datos que no depende de un contexto en particular y puede ser interpretado tanto por sistemas de información como por personas, logrando así la interoperabilidad semántica. En informática, una ontología se refiere a una definición formal de tipos, propiedades y relaciones entre conceptos de un dominio de conocimiento en particular.

El problema computacional que se quiere resolver es la interoperabilidad sintáctica y semántica de datos de un dominio de conocimiento; en particular con datos de distintas fuentes de publicación y de distintos dominios de conocimiento. El trabajo pretende modelar una ontología para brindar una capa de semántica formal a un estándar de datos abiertos para la publicación de datos estructurados llamado \textit{Open Contracting Data Standard}(OCDS), aplicado a datos de la DNCP utilizando metodologías y herramientas para el desarrollo de ontologías. 


\section{Objetivos}

El objetivo general es verificar experimentalmente la interoperabilidad semántica de datos establecida por una ontología desarrollada en base a un protocolo de intercambio de datos.

A partir del objetivo general se identificaron las siguientes preguntas de investigación.

\begin{description}
    \item[P1.] ¿Cuáles son otros desarrollos ontológicos dentro del dominio de Contrataciones Públicas y con qué intención fueron creados? 
    \item[P2.] ¿Cuáles son las metodologías para el desarrollo de una ontología?
    \item[P3.] ¿Cómo crear una fuente de datos pública en la web de manera que sea procesable por máquinas y humanos?
    \item[P4.] ¿De qué manera ayuda a la interoperabilidad semántica publicar datos en formato de triplas o grafos?
    \item[P5.] ¿Es posible la reutilización de otras ontologías del dominio?
    \item[P6.] ¿Es posible enriquecer los datos  de una fuente de un dominio de conocimiento con otra fuente de datos externa?
    \item[P7.] ¿De qué manera una ontología puede lograr la interpretación y la inferencia automática de datos?
    
\end{description} 

Las preguntas de investigación se enmarcaron en los siguientes objetivos específicos que se irán desarrollando a lo largo de este documento.

\begin{description}
    \item[O1.] Identificar y comparar las bases de representación de conocimiento asociadas al dominio de Contrataciones Públicas.
    \item[O2.] Aplicar un proceso de modelado de una ontología utilizando una base de representación de conocimiento del dominio.
    \item[O3.] Modelar un sistema de extracción y consultas de datos del dominio de Contrataciones Públicas con capacidad de integración semántica.
    \item[O4.] Mostrar, a través de un conjunto de casos, la expresividad e interoperabilidad semántica de datos utilizando la ontología desarrollada.
\end{description} 


\section{Contribuciones}
\label{Contribuciones}
Como resultante del trabajo se realizaron las siguientes contribuciones.

\begin{description}
    \item[C1.] \label{contrib:1}Una Ontología basada en el Open Contracting Data Standard \cite{OCDSReleaseSchema:online}.
    \item[C2.] \label{contrib:2}Descripción del proceso de modelado de una ontología reutilizando un recurso no-ontológico basado en JSON-SCHEMA\cite {JSONSche10:online}.
    \item[C3.] \label{contrib:3}Sistema de extracción y manipulación de datos de Procesos Licitatorios.
    \item[C4.] \label{contrib:4}Sistema de consulta de datos en formato RDF enriquecida con la ontología desarrollada.
    \item[C5.] \label{contrib:6}Verificación parcial de la adecuación de la representación de datos disponibilizada por la DNCP.
    \item[C6.] \label{contrib:5}Descripciones experimentales de la pragmática de incorporación, uso y reuso de ontologías para aumentar la interoperabilidad en los datos publicados.
\end{description} 
 

\section{Estructura del documento}
El trabajo está estructurado de la siguiente manera;en el  Capítulo \ref{chap:Marco Teorico} trataremos el marco teórico del trabajo abordando los temas de la Web Semántica, las ontologías, sus metodologías y herramientas relacionadas. Luego en el Capítulo \ref{chap:MaterialesYMetodos} se expondrá la forma en que se abordarán los objetivos propuestos. En el Capítulo \ref{chap:Desarrollo de la Ontologia} se mostrará el proceso de modelado de la ontología. En el Capítulo \ref{chap:Implementación de la Ontologia} se describirá la implementación del entorno tecnológico de experimentación para luego en el Capítulo \ref{chap:Contexto experimental} mostrar las pruebas realizadas a los datos utilizando la ontología desarrollada. En el Capítulo \ref{chap:conclusiones} se evaluará el cumplimiento de los objetivos propuestos,se mostrarán las conclusiones finales y futuros trabajos que pudieran abordarse. Al final de este documento  se  incluyen las referencias bibliográficas utilizadas para su elaboración y el anexo que provee información complementaria.



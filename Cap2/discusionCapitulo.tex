\section{Discusión del capítulo}
En este capítulo se vieron los conceptos sobre Datos Abiertos, Linked Open Data, Web Semántica y el rol de las ontologías en la misma. Además se dio una introducción a RDF, RDFS y OWL que son formas de representación de ontologías y datos en la Web Semántica. Para poder realizar consultas a las base de datos en RDF se dio una breve introducción del lenguaje SPARQL.

Se revisaron las metodologías más utilizadas para el desarrollo de una ontología para luego hacer una comparación de cada una de ellas presentadas en la Tabla \ref{tab:comparacion}. En cuanto a la madurez y completitud de cada una podemos decir que la metodología NeOn es la que mejor se posiciona, ya que la misma está basada en metodologías de desarrollo de software y metodologías de ingeniería de conocimiento. Además contempla escenarios de reuso de recursos ontológicos y no-ontológicos, los cuales serán necesarios para el desarrollo de este trabajo. Por este motivo se eligió NeOn como metodología base para el desarrollo de la ontología.

Además presentamos las herramientas tecnológicas utilizadas para el desarrollo de ontologías y posterior consulta a través de un servidor de consultas llamado Punto SPARQL.

En el siguiente capítulo se hablará acerca de los recursos necesarios y los pasos seguidos para llegar al objetivo propuesto del trabajo.

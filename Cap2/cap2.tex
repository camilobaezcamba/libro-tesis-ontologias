%!TEX root = ../tesis.tex
\chapter{Web Semántica y Ontologías}
\label{chap:Web Semantica y Ontologias}

\section{Open Data}

\subsection{Linked Open Data (LOD)}
Linked Open Data se enfoca en un método de publicación de datos abiertos estructurados para que puedan ser interconectados. Para lograr esto se utilizan las tecnologías como RDF, RDFa, etc. para estructurar los datos, utilizando URIs para identificar los datos individualmente. Tim Berners Lee define un modelo de 5 estrellas para clasificar e identificar el grado de publicación de datos abiertos.

\begin{table}[]
\centering
\caption{Modelo de publicación de 5 estrellas}
\label{modelo-5-estrellas}
\resizebox{15cm}{!} {
\begin{tabular}{|c|l|}
\hline
\ding{72} & La información está disponible en la web (en cualquier formato) bajo una licencia abierta \\ \hline
\ding{72} \ding{72} & La información está disponible como dato estructurado (Ej. Excel en lugar de una imagen de una tabla) \\ \hline
\ding{72} \ding{72} \ding{72} & Son utilizados formatos no-propietarios (Ej. CSV en lugar de Excel) \\ \hline
\ding{72} \ding{72} \ding{72} \ding{72}  & URIs son utilizadas para que se puedan individualizar los datos \\ \hline
\ding{72} \ding{72} \ding{72} \ding{72} \ding{72}  & Los datos son enlazados con otros datos para proveer un contexto \\ \hline
\end{tabular}
}
\end{table}

\section{Resumen}

\section{Caso 5. Expresividad semántica de la ontología. }
\label{section:caso5}

El termino expresividad se refiere al hecho de poder realizar consultas en términos semánticos o sea a través de conceptos reales. De no ser así, para hacer una consulta a la fuente de datos el experto necesita:

\begin{enumerate}
    \item Conocer toda la estructura de datos.
    \item Analizar e interpretar la estructura.
    \item Crear una consulta que responda a la pregunta que quiera hacer.
\end{enumerate}

Todo esto requiere mayor tiempo y puede crear un sesgo en la interpretación del experto. En este caso la expresividad semántica de la ontología asegura la correcta interpretación de los datos y facilita la consulta de los mismos.


\subsection{Mejora: Definición de un nuevo concepto}
Se puede extender la expresividad semántica de la ontología actualizando el modelo ontológico sin que implique un alto costo debido a que éste se encuentra desacoplado de los datos.

Para probar este caso se utilizó la fuente de datos de procesos licitatorios y la ontología OCDSPY.  Se creó el  concepto  de “Proceso Licitatorio con Contrato”, que se refiere al hecho de que un Proceso Licitatorio ya se encuentra con al menos un contrato firmado.

Se define el concepto a través de una una clase (\textit{PLConContrato}) a partir de una restricción de propiedad. Esta clase se define como una subclase de \textit{Release}, que tiene como restricción todas las instancias de la clase \textit{Release} que tengan al menos una relación en la propiedad \textit{ocds:contracts}.

En el Cuadro \ref{lst:caso5-1} se muestra cómo a través del Punto SPARQL introducimos a la ontología el concepto “Proceso Licitatorio”. Esta extensión de la ontología se dio luego del desarrollo de la misma, demostrando también la forma en que la ontología puede ir expandiéndose a medida que se necesite.

Esta capacidad de extender la ontología nos permite hacer consultas de manera simple sin la necesidad de hacer modificaciones a los datos ni a su estructura y agregando conocimiento de dominio a la ontología.

\hfill \break


\noindent\begin{minipage}[c]{\textwidth}
\begin{lstlisting}[captionpos=b, caption=Extensión de la ontología utilizando restricciones ontológicas, label={lst:caso5-1},  numbers=left,  numberstyle=\tiny\color{mygray},frame=single]
INSERT DATA {
    ocds:PLConContrato rdfs:subClassOf ocds:Release ; 
    owl:equivalentClass    [ 
        rdf:type owl:Class ;
        owl:intersectionOf (   
            ocds:Release [ 
                rdf:type owl:Restriction ;
                owl:onProperty ocds:contracts; 
                owl:minCardinality "1"^^xsd:nonNegativeInteger ;
            ] 
        )
    ] .
}
    
 \end{lstlisting}
\end{minipage}
 De esta manera basta solo con realizar la consulta SPARQL presentada en el Cuadro \ref{lst:caso5-2} para obtener todos los procesos licitatorios con al menos un contrato. La restricción consigue filtrar solo los \textit{Release} cuya propiedad \textit{contracts} tiene al menos una instancia.\hfill \break

\noindent\begin{minipage}[c]{\textwidth}
 \begin{lstlisting}[captionpos=b, caption=Consulta SPARQL utilizando la Clase PLConContrato, label=lst:caso5-2,  numbers=left,  numberstyle=\tiny\color{mygray},frame=single]
SELECT  *
    WHERE { 
    ?a  ?b  ocds:PLConContrato 
    }
 \end{lstlisting}
\end{minipage}
 
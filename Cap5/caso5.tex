\section{Caso 5. Mejora de la expresividad de la ontología. }

Se puede extender la capacidad semántica de la ontología actualizando el modelo ontológico sin que implique un alto costo debido a que éste se encuentra desacoplado de los datos.

Para este caso se utilizó la fuente de datos de procesos licitatorios y la ontología OCDS. Se define el concepto de “Proceso Licitatorio con Contrato” a través de una una clase (\textit{PLConContrato}) a partir de una restricción de propiedad. Esta clase se define como una subclase de \textit{Release}, que tiene como restricción todas las instancias de la clase \textit{Release} que tengan al menos una relación en la propiedad \textit{ocds:contracts}.\hfill \break


\noindent\begin{minipage}[c]{\textwidth}
\begin{lstlisting}[captionpos=b, caption=Extensión de la ontología utilizando restricciones ontológicas, label={lst:caso5-1},  numbers=left,  numberstyle=\tiny\color{mygray},frame=single]
INSERT DATA {
    ocds:PLConContrato rdfs:subClassOf ocds:Release ; 
    owl:equivalentClass    [ 
        rdf:type owl:Class ;
        owl:intersectionOf (   
            ocds:Release [ 
                rdf:type owl:Restriction ;
                owl:onProperty ocds:contracts; 
                owl:minCardinality "1"^^xsd:nonNegativeInteger ;
            ] 
        )
    ] .
}
    
 \end{lstlisting}
\end{minipage}
 De esta manera basta solo con realizar la consulta SPARQL presentada en el Cuadro \ref{lst:caso5-2} para obtener todos los procesos licitatorios con al menos un contrato. La restricción consigue filtrar solo los \textit{Release} cuya propiedad \textit{contracts} tiene al menos una instancia.\hfill \break

\noindent\begin{minipage}[c]{\textwidth}
 \begin{lstlisting}[captionpos=b, caption=Consulta SPARQL utilizando la Clase PLConContrato, label=lst:caso5-2,  numbers=left,  numberstyle=\tiny\color{mygray},frame=single]
SELECT  *
    WHERE { 
    ?a  ?b  ocds:PLConContrato 
    }
 \end{lstlisting}
\end{minipage}
 Esta capacidad de extender la ontología nos permite hacer consultas de manera simple sin la necesidad de hacer modificaciones a los datos y agregando conocimiento de dominio a la ontología
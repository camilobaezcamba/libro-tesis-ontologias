\section{Contexto JSON-LD}

Como se vió anteriormente, existen varias formas de serialización de RDF, en contraste a XML, JSON-LD fue diseñado para ser un formato de intercambio de datos ligero e independiente del lenguaje y es lo suficientemente expresivo como para soportar los conceptos de RDF, además requiere poco esfuerzo para los desarrolladores transformar un documento JSON a JSON-LD. 

Para dar contexto a un objeto JSON es necesario agregar el atributo @context. Éste puede darse de dos formas, definiendo la estructura del contexto como valor de la propiedad, o haciendo referencia (URI) a un documento que contiene la definición del contexto.

Se puede ir agregando @context a los objetos hijos de forma recursiva. Esto es muy importante debido a que se pueden sobrescribir los contextos exclusivamente para un objeto en particular sin que afecte a la definición de los demás. Más adelante en este capítulos se abordará cada uno de los temas.


\begin{lstlisting}[language=json,firstnumber=1]
    "budget": {
        "description": "Adquisicion de Scanner",
        "amount": {
              "amount": 12000000,
              "currency": "PYG"
         }
     }  
    \end{lstlisting}


    \begin{lstlisting}[language=json,firstnumber=1]
        "budget": {
            "@context": "http://girolabs.com.py/ocds/context-budget.json",
            "@type": "Budget",
            "description": "Adquisicion de Scanner",
            "amount": {
                "@context": "http://girolabs.com.py/ocds/context-value.json",
                "@type": "Value",
                "amount": 12000000,
                "currency": "PYG"
           }
       }   
        \end{lstlisting}
    
 
\section{De JSON-LD a RDF}

La conversión de JSON-LD a RDF es un proceso directo. El sujeto (en JSON-LD), que podría ser usado como objeto en otra tripla, es definido por la propiedad @id, todas las demás propiedades son convertidas en predicado del objeto RDF resultante. Finalmente, los valores literales son extraídos directamente de la propiedad o del valor @value de la propiedad (si lo tuviere). Una propiedad puede tener también definido el lenguaje (@language) y/o tipo de información (@type).

Por ejemplo en el Cuadro \ref{lst:EjemploValdezJSONLD} se muestra un objeto JSON-LD, que posee un contexto definido dentro del mismo objeto, este contexto también podría estar definido dentro de un documento diferente y referenciado a través de una URI. Se toma como ejemplo la propiedad foaf:name, la misma se convierte en predicado y el objeto toma el valor del literal “Rodrigo Valdez” definido en el idioma inglés según especifica @language. El cuadro \ref{lst:ValdezRDF} muestra el resultado de la conversión de JSON-LD a RDF/N-QUADS.


\noindent\begin{minipage}{\textwidth}
    \begin{lstlisting}[captionpos=b, caption=Objeto JSON-LD. , label=lst:EjemploValdezJSONLD, language=json,firstnumber=1,  numbers=left,  numberstyle=\tiny\color{mygray},frame=single]
{
    "@context": {
	"@base":"http://pol.una.py/members/"
	"foaf": "http://xmlns.com/foaf/0.1/"
	},
	"@id": "rodrivaldez5",
	"@type": "foaf:Person",
	"foaf:family_name": "Valdez",
	"foaf:givenname": "Rodrigo",
	"foaf:name": {
		"@language": "en",
		"@value": "Rodrigo Valdez"
	} ,
	"foaf:nick": "Rodri",
	"foaf:title": "Software Developer"
}

        \end{lstlisting}
    \end{minipage}


\noindent\begin{minipage}{\textwidth}
\begin{lstlisting}[captionpos=b, caption=Ejemplo en RDF/N-Quads, label=lst:ValdezRDF,  numbers=left, language=SQL, numberstyle=\tiny\color{mygray},frame=single]
<http://pol.una.py/members/rodrivaldez5>
    <http://www.w3.org/1999/02/22-rdf-syntax-ns#type>
        <http://xmlns.com/foaf/0.1/Person> .
        
<http://pol.una.py/members/rodrivaldez5>
    <http://xmlns.com/foaf/0.1/family_name> "Valdez" .
    
<http://pol.una.py/members/rodrivaldez5>
    <http://xmlns.com/foaf/0.1/givenname> "Rodrigo" .
    
<http://pol.una.py/members/rodrivaldez5>
    <http://xmlns.com/foaf/0.1/name> "Rodrigo Valdez"@en .
    
<http://pol.una.py/members/rodrivaldez5>
    <http://xmlns.com/foaf/0.1/nick> "Rodri" .
    
<http://pol.una.py/members/rodrivaldez5>
    <http://xmlns.com/foaf/0.1/title> "Software Developer" .
\end{lstlisting}
\end{minipage}








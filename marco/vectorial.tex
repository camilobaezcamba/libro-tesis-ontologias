\section{Ordenamiento Vectorial}
La Matemática Morfológica requiere establecer una relación de orden entre los colores, representados por vectores, para ser extendida a color. 
En general, una relación binaria $\beta$ en un conjunto $D$ es:
\addsymbol{symbol:beta}\addsymbol{symbol:D}
\begin{enumerate}
\item Reflexivo si $\boldsymbol a \beta \boldsymbol a, \forall \boldsymbol a \in D$,
\item Anti-Simétrico si $\boldsymbol a \beta \boldsymbol b$ y  $\boldsymbol b \beta \boldsymbol a \Rightarrow \boldsymbol a = \boldsymbol b, \forall \boldsymbol a, \boldsymbol b \in  D$,
\item Transitivo $\boldsymbol a \beta \boldsymbol b$ y $\boldsymbol b \beta \boldsymbol c \Rightarrow \boldsymbol a \beta \boldsymbol c, \forall  \boldsymbol a, \boldsymbol b, \boldsymbol c \in D$,
\item Total si $\boldsymbol a \beta \boldsymbol b$ o $\boldsymbol b \beta \boldsymbol a, \forall \boldsymbol a, \boldsymbol b \in D$.
\end{enumerate}

La relación de orden $\leq$ (menor o igual) es requerida para la relación binaria $\beta$.
Cuando la relación de orden $\leq$ cumple con las restricciones 1 y 3 es llamada pre-orden. Una relación de orden es considerada como un ordenamiento si además de cumplir con las restricciones de una relación de pre-orden, también cumple con la restricción 2.
Adicionalmente a todo lo anterior, la relación $\leq$ debe cumplir la restricción 4 para ser considerado como orden total, sino es considerado como orden parcial.

Las técnicas de ordenamiento vectorial pueden ser clasificadas en los siguientes grupos~\cite{barnett1976ordering}:
\begin{itemize}
\item Ordenamiento Marginal (Ordenamiento M): Este ordenamiento cada componente compara el color de manera independiente.
\item Ordenamiento Condicional (Ordenamiento C): Un componente marginal es seleccionado secuencialmente para ordenar los vectores, dependiendo de las diferentes condiciones. Un ejemplo muy utilizado en el Ordenamiento C es el orden lexicográfico, que utiliza todos los componentes disponibles de los vectores a ser ordenados.
\item Ordenamiento Parcial (Ordenamiento P): Este ordenamiento esta basado en la división de los vectores en grupos de equivalencia, tal que entre los grupos existe un orden. Algunos ordenamientos totales son considerados dentro de esta clase.
\item Ordenamiento Reducido (Ordenamiento R): Los vectores son reducidos a valores escalares y posteriormente son clasificados mediante su orden escalar natural. Por ejemplo, un ordenamiento R en $\mathbb{Z}^{n}$ puede considerarse tras definir una transformación $T : \mathbb{Z}^{n} \rightarrow \mathbb{R}$ y posteriormente ordenar mediante orden escalar de su proyección en $\mathbb{Z}^{n}$ por $T$.
\begin{equation}
\forall q,q^{\prime}\in\mathbb{R}^{n} 
q \leq q^{\prime} \Leftrightarrow T\left(q\right) \leq T\left(q^{\prime}\right).
\end{equation}
\end{itemize}
\section{Imágenes Digitales}
El píxel es la unidad básica de una imagen digital. Un píxel contiene información sobre el punto de color. Para poder visualizar, almacenar y procesar dicha información, se debe conocer el espacio de color en el cual está representado.

\addsymbol{symbol:F}
\addsymbol{symbol:e}
\addsymbol{symbol:d}
\addsymbol{symbol:n}
Una imagen digital consiste en un conjunto de píxeles. La definición formal esta dada como $F:e\rightarrow d$, donde $e$ es un subconjunto discreto de una grilla rectangular $\mathbb{N}^{2}$ para imágenes en dos dimensiones y $d$ es el rango de valores permitidos para el píxel.
Los píxeles de imágenes binarias se representan con un único valor en un rango de [0, 1]. Las imágenes en escala de grises utilizan un valor en el rango de [0, 255] como un subconjunto de $\mathbb{Z}$. Un píxel tiene asociado un color que se representa con un vector $(V_{1},V_{2},V_{3},...,V_{n})$ donde $n$ es el número de dimensiones del vector. 
%%%
%%% Conjuto de funciones utilitarias
%%% autor: Maximiliano Báez
%%% fecha: 25/08/2014
%%%

\usepackage{tablefootnote} %para utilizar \footnote{}
\usepackage{amssymb}
\renewcommand{\thefootnote}{\arabic{footnote}}

%%Para la construcción de la lista de simbolos
% Macro for 'List of Symbols', 'List of Notations' etc...
\def\listofsymbols{
    \newpage
\chapter*{Lista de Acrónimos\hfill}
\addcontentsline{toc}{chapter}{Lista de Acrónimos}
\begin{tabbing}
% YOU NEED TO ADD THE FIRST ONE MANUALLY TO ADJUST THE TABBING AND SPACES
%$x$~~~~~~~~~~\=\parbox{5in}{X\dotfill \pageref{symbol:x}}\\
%ADD THE REST OF SYMBOLS WITH THE HELP OF MACRO

%% se añaden nuevos simbolos con el macro \newsymbol y se hace referecnia
% al simbolo utilizando \addsymbol{symbol:LABEL}


\textbf{OCDS} Open Contracting Data Stantard\\
\textbf{W3C} World Wide Web Consortium\\
\textbf{OWL} Ontology Web Language\\
\textbf{CSV} Comma Separated Value\\
\textbf{JSON} Javascript Object Notation\\
\textbf{JSON-LD} Javascript Object Notation - Linked Data\\
\textbf{XML} Extended Markup Language\\
\textbf{URI} Unified Resource Identifier\\
\textbf{DL} Description Logic\\
\textbf{LOV} Linked Open Vocabularies\\
\textbf{LOD} Linked Open Data\\
\textbf{RDF} Resource Description Framework\\
\textbf{RDFS} Resource Description Framework Schema\\
\textbf{OCP} Open Contractic Partnership\\
\textbf{TED} Tender Electronic Daily\\
\textbf{DNCP} Dirección Nacional de Contrataciones Públicas\\
\textbf{ODP} Ontology Design Pattern\\
\textbf{VOWL} Visual Notation for OWL\\
\textbf{FOAF} Friend Of a Friend\\
\textbf{ORSD} Especificación de Requerimientos Ontológicos\\
\textbf{NOR} Not Ontologic Resource\\
\textbf{PCO} Public Contract Ontology\\
\textbf{API} Application Programming Interface\\



\end{tabbing}
    \clearpage{}
}
\def\newsymbol #1: #2#3{$#1$ \> \parbox{5in}{#2 \dotfill \pageref{#3}}\\}
\def\addsymbol#1{\label{#1}}

% Para las imagenes en grilla
% custom commands
\newcommand{\foreign}[1]{{\it #1}}
\DeclareMathOperator*{\argmax}{arg\,max}
%\algsetup{}
% \algsetup{
%     indent=4em,
%     linenosize=\small,
%     linenodelimiter=.
% }

\usepackage{amsmath}

%% se utilizan para referenciar figuras, tablas, secciones y algoritmos
\newcommand{\figref}[1]{Figura \ref{#1}}
\newcommand{\tabref}[1]{Tabla \ref{#1}}
\newcommand{\chapref}[1]{Capítulo \ref{#1}}
\newcommand{\secref}[1]{Sección \ref{#1}}
\newcommand{\algref}[1]{Algoritmo \ref{#1}}
\newcommand{\apenref}[1]{Apéndice \ref{#1}}


%Traducción al español del paquete algorithmic%
% \floatname{algorithm}{Algoritmo}
% \renewcommand{\listalgorithmname}{Lista de algoritmos}
% \renewcommand{\algorithmicrequire}{\textbf{Entrada:}}
% \renewcommand{\algorithmicensure}{\textbf{Salida:}}
% \renewcommand{\algorithmicend}{\textbf{fin}}
% \renewcommand{\algorithmicif}{\textbf{si}}
% \renewcommand{\algorithmicthen}{\textbf{entonces}}
% \renewcommand{\algorithmicelse}{\textbf{si no}}
% \renewcommand{\algorithmicelsif}{\algorithmicelse,\ \algorithmicif}
% \renewcommand{\algorithmicendif}{\algorithmicend\ \algorithmicif}
% \renewcommand{\algorithmicfor}{\textbf{para}}
% \renewcommand{\algorithmicforall}{\textbf{para todo}}
% \renewcommand{\algorithmicdo}{\textbf{hacer}}
% \renewcommand{\algorithmicendfor}{\algorithmicend\ \algorithmicfor}
% \renewcommand{\algorithmicwhile}{\textbf{mientras}}
% \renewcommand{\algorithmicendwhile}{\algorithmicend\ \algorithmicwhile}
% \renewcommand{\algorithmicloop}{\textbf{repetir}}
% \renewcommand{\algorithmicendloop}{\algorithmicend\ \algorithmicloop}
% \renewcommand{\algorithmicrepeat}{\textbf{repetir}}
% \renewcommand{\algorithmicuntil}{\textbf{hasta que}}
% \renewcommand{\algorithmicprint}{\textbf{imprimir}}
% \renewcommand{\algorithmicreturn}{\textbf{retorna}}
% \renewcommand{\algorithmictrue}{\textbf{cierto }}
% \renewcommand{\algorithmicfalse}{\textbf{falso }}
% \renewcommand{\algorithmiccomment}{\textbf{comentario: }}

%Traducción al español del paquete algorithm2c%

\SetKwInput{KwIn}{Entrada}%
\SetKwInput{KwOut}{Salida}%
\SetKwInput{KwData}{Datos}%
\SetKwInput{KwResult}{Resultado}%
\SetKw{KwTo}{a}%
\SetKw{KwRet}{devolver}%
\SetKw{Return}{devolver}%
\SetKwBlock{Begin}{inicio}{fin}%
\SetKwRepeat{Repeat}{repetir}{hasta que}%
%
\SetKwIF{If}{ElseIf}{Else}{si}{entonces}{sin\'o, si}{en otro caso}{fin si}
\SetKwSwitch{Switch}{Case}{Other}{seleccionar}{hacer}{caso}{sin\'o}{fin caso}{fin seleccionar}
\SetKwFor{For}{per}{fai}{fine per}%
\SetKwFor{ForPar}{par}{hacer in paralelo}{fin para}%
\SetKwFor{ForEach}{para cada}{hacer}{fin para cada}
\SetKwFor{ForAll}{para todo}{hacer}{fin para todo}
\SetKwFor{While}{mientras}{hacer}{fin mientras}



\addtocontents{loa}{\protect\thispagestyle{empty}}

% add this to align the list of algorithms
\makeatletter
  \renewcommand*{\listof}[2]{%
    \@ifundefined{ext@#1}{\float@error{#1}}{%
      \@namedef{l@#1}{\@dottedtocline{1}{2em}{1.3em}}% line of the list (change from 1em to the desired value)
      \float@listhead{#2}%
      \begingroup\setlength{\parskip}{\z@}%
        \@starttoc{\@nameuse{ext@#1}}%
      \endgroup}}
  \renewcommand*\l@algocf{\@dottedtocline{1}{1em}{2.3em}}% line of the list (change from 1em to the desired value)
\makeatother

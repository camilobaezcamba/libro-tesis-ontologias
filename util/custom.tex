%%%
%%% Conjuto de funciones utilitarias
%%% autor: Maximiliano Báez
%%% fecha: 25/08/2014
%%%

\usepackage{tablefootnote} %para utilizar \footnote{}
\usepackage{amssymb}
\renewcommand{\thefootnote}{\arabic{footnote}}

%%Para la construcción de la lista de simbolos
% Macro for 'List of Symbols', 'List of Notations' etc...
\def\listofsymbols{
    \newpage
\chapter*{Lista de Acrónimos\hfill}
\addcontentsline{toc}{chapter}{Lista de Símbolos}
\begin{tabbing}
% YOU NEED TO ADD THE FIRST ONE MANUALLY TO ADJUST THE TABBING AND SPACES
$F$~~~~~~~~~~\=\parbox{5in}{Imagen digital\dotfill \pageref{symbol:F}}\\
%$x$~~~~~~~~~~\=\parbox{5in}{X\dotfill \pageref{symbol:x}}\\
%ADD THE REST OF SYMBOLS WITH THE HELP OF MACRO

%% se añaden nuevos simbolos con el macro \newsymbol y se hace referecnia
% al simbolo utilizando \addsymbol{symbol:LABEL}
\newsymbol n: {Dimensión del vector que representa un color}{symbol:xcomp}
\newsymbol e: {Subconjunto discreto de una grilla rectangular $\mathbb{N}^{2}$ para imágenes 2D}{symbol:e}
\newsymbol d: {Rango de valores permitidos para el píxel}{symbol:d}
\newsymbol L^{*}: {Luminiscencia}{symbol:l*}
\newsymbol a^{*}: {Sensación rojo-verde}{symbol:a*}
\newsymbol b^{*}: {Sensación amarillo-azul}{symbol:b*}
\newsymbol H: {Matiz}{symbol:H}
\newsymbol S: {Saturación}{symbol:S}
\newsymbol I: {Intensidad}{symbol:I}
\newsymbol X: {Valor de conversión X RGB a CIELab}{symbol:x}
\newsymbol Y: {Valor de conversión Y RGB a CIELab}{symbol:y}
\newsymbol Z: {Valor de conversión Z RGB a CIELab}{symbol:z}
\newsymbol r: {Nivel de intensidad del píxel RGB en el componente r (red)}{symbol:r}
\newsymbol g: {Nivel de intensidad del píxel RGB en el componente g (green)}{symbol:g}
\newsymbol b: {Nivel de intensidad del píxel RGB en el componente b (blue)}{symbol:b}
\newsymbol f: {Función para convertir de RGB a CIELab}{symbol:f}
\newsymbol q,s,t: {Píxeles de una imagen}{symbol:q}
\newsymbol \Delta E_{Lab}: {Diferencia entre dos colores en CIELab}{symbol:elab}
\newsymbol \Delta L^{*}: {Diferencia de luminiscencia}{symbol:deltal}
\newsymbol \Delta a^{*}: {Diferencia del componente a*}{symbol:deltaa}
\newsymbol \Delta b^{*}: {Diferencia del componente b*}{symbol:deltab}
\newsymbol \Delta E_{rgb}: {Diferencia entre dos colores en RGB}{symbol:ergb}
\newsymbol \Delta r: {Diferencia del componente r}{symbol:deltargbR}
\newsymbol \Delta g: {Diferencia del componente g}{symbol:deltargbG}
\newsymbol \Delta b: {Diferencia del componente b}{symbol:deltargbB}
\newsymbol \varepsilon: {Erosión}{symbol:erosion}
\newsymbol \delta: {Dilatación}{symbol:dilatacion}
\newsymbol \gamma: {Gradiente Morfológica}{symbol:gradiente}
\newsymbol E: {Elemento estructurante}{symbol:estructurante}
\newsymbol D: {Conjunto de elementos}{symbol:D}
\newsymbol win: {Ventanas de la imagen}{symbol:win}
\newsymbol \beta: {Relación binaria}{symbol:beta}
\newsymbol A,B,C: {Etiqueta para los marcadores}{symbol:A}
\newsymbol W: {Etiqueta watershed}{symbol:W}
\newsymbol P: {Etiqueta para denotar un píxel como pendiente}{symbol:P}
\newsymbol VP: {Verdadero Positivo}{symbol:VP}
\newsymbol FP: {Falso Positivo}{symbol:FP}
\newsymbol VN: {Verdadero Negativo}{symbol:VN}
\newsymbol FN: {Falso Negativo}{symbol:FN}
\newsymbol T: {Función para transformar a un valor escalar}{symbol:T}
\newsymbol w: {Vector de pesos}{symbol:w}
\newsymbol V: {Vector de componentes de un píxel}{symbol:V}
\newsymbol i,j,u,v: {Índices de píxeles}{symbol:j}
\newsymbol MxN: {Cantidad de píxeles de una imagen}{symbol:n}
\newsymbol Sp: {Espacio de Color}{symbol:Sp}
\newsymbol \leq_{L}: {Orden lexicográfico}{symbol:lex}
\newsymbol \alpha: {Grado de influencia del primer componente del píxel}{symbol:alpha}
\newsymbol m: {Índice de bits}{symbol:m}
\newsymbol k: {Cantidad de bits de una pixel}{symbol:k}
\newsymbol x,y,z: {Ejemplos de pixeles representados como vectores}{symbol:vec}

\end{tabbing}
    \clearpage{}
}
\def\newsymbol #1: #2#3{$#1$ \> \parbox{5in}{#2 \dotfill \pageref{#3}}\\}
\def\addsymbol#1{\label{#1}}

% Para las imagenes en grilla
% custom commands
\newcommand{\foreign}[1]{{\it #1}}
\DeclareMathOperator*{\argmax}{arg\,max}
%\algsetup{}
% \algsetup{
%     indent=4em,
%     linenosize=\small,
%     linenodelimiter=.
% }

\usepackage{amsmath}

%% se utilizan para referenciar figuras, tablas, secciones y algoritmos
\newcommand{\figref}[1]{Figura \ref{#1}}
\newcommand{\tabref}[1]{Tabla \ref{#1}}
\newcommand{\chapref}[1]{Capítulo \ref{#1}}
\newcommand{\secref}[1]{Sección \ref{#1}}
\newcommand{\algref}[1]{Algoritmo \ref{#1}}
\newcommand{\apenref}[1]{Apéndice \ref{#1}}


%Traducción al español del paquete algorithmic%
% \floatname{algorithm}{Algoritmo}
% \renewcommand{\listalgorithmname}{Lista de algoritmos}
% \renewcommand{\algorithmicrequire}{\textbf{Entrada:}}
% \renewcommand{\algorithmicensure}{\textbf{Salida:}}
% \renewcommand{\algorithmicend}{\textbf{fin}}
% \renewcommand{\algorithmicif}{\textbf{si}}
% \renewcommand{\algorithmicthen}{\textbf{entonces}}
% \renewcommand{\algorithmicelse}{\textbf{si no}}
% \renewcommand{\algorithmicelsif}{\algorithmicelse,\ \algorithmicif}
% \renewcommand{\algorithmicendif}{\algorithmicend\ \algorithmicif}
% \renewcommand{\algorithmicfor}{\textbf{para}}
% \renewcommand{\algorithmicforall}{\textbf{para todo}}
% \renewcommand{\algorithmicdo}{\textbf{hacer}}
% \renewcommand{\algorithmicendfor}{\algorithmicend\ \algorithmicfor}
% \renewcommand{\algorithmicwhile}{\textbf{mientras}}
% \renewcommand{\algorithmicendwhile}{\algorithmicend\ \algorithmicwhile}
% \renewcommand{\algorithmicloop}{\textbf{repetir}}
% \renewcommand{\algorithmicendloop}{\algorithmicend\ \algorithmicloop}
% \renewcommand{\algorithmicrepeat}{\textbf{repetir}}
% \renewcommand{\algorithmicuntil}{\textbf{hasta que}}
% \renewcommand{\algorithmicprint}{\textbf{imprimir}}
% \renewcommand{\algorithmicreturn}{\textbf{retorna}}
% \renewcommand{\algorithmictrue}{\textbf{cierto }}
% \renewcommand{\algorithmicfalse}{\textbf{falso }}
% \renewcommand{\algorithmiccomment}{\textbf{comentario: }}

%Traducción al español del paquete algorithm2c%

\SetKwInput{KwIn}{Entrada}%
\SetKwInput{KwOut}{Salida}%
\SetKwInput{KwData}{Datos}%
\SetKwInput{KwResult}{Resultado}%
\SetKw{KwTo}{a}%
\SetKw{KwRet}{devolver}%
\SetKw{Return}{devolver}%
\SetKwBlock{Begin}{inicio}{fin}%
\SetKwRepeat{Repeat}{repetir}{hasta que}%
%
\SetKwIF{If}{ElseIf}{Else}{si}{entonces}{sin\'o, si}{en otro caso}{fin si}
\SetKwSwitch{Switch}{Case}{Other}{seleccionar}{hacer}{caso}{sin\'o}{fin caso}{fin seleccionar}
\SetKwFor{For}{per}{fai}{fine per}%
\SetKwFor{ForPar}{par}{hacer in paralelo}{fin para}%
\SetKwFor{ForEach}{para cada}{hacer}{fin para cada}
\SetKwFor{ForAll}{para todo}{hacer}{fin para todo}
\SetKwFor{While}{mientras}{hacer}{fin mientras}



\addtocontents{loa}{\protect\thispagestyle{empty}}

% add this to align the list of algorithms
\makeatletter
  \renewcommand*{\listof}[2]{%
    \@ifundefined{ext@#1}{\float@error{#1}}{%
      \@namedef{l@#1}{\@dottedtocline{1}{2em}{1.3em}}% line of the list (change from 1em to the desired value)
      \float@listhead{#2}%
      \begingroup\setlength{\parskip}{\z@}%
        \@starttoc{\@nameuse{ext@#1}}%
      \endgroup}}
  \renewcommand*\l@algocf{\@dottedtocline{1}{1em}{2.3em}}% line of the list (change from 1em to the desired value)
\makeatother

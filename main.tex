\documentclass[draft,fmstyle]{util/ucathesis}
%\documentclass[draft, fmstyle]{util/ucathesis}

% La opción 'final' muestra los graficos, para generar una version sin los graficos utiliza la opcion 'draft'

% paquetes recomendados
%\usepackage[chapter]{theorems}
%\usepackage{symbols}
%\usepackage{url}
\usepackage{amsmath,amsthm}
\usepackage[utf8]{inputenc}
\usepackage[spanish]{babel}
\usepackage{csquotes}
\usepackage[style=numeric,sorting=none,backend=biber]{biblatex}
\usepackage{tabularx}		
\usepackage{pdflscape}
\usepackage{listings}
\usepackage{placeins}
\usepackage[T1]{fontenc}
\usepackage[scaled=0.85]{beramono}
\usepackage{makecell}
\usepackage{listingsutf8}


%\newcolumntype{L}{>{\centering\arraybackslash}m{3cm}}





% para la lista de símbolos
%\usepackage{subfigure} %conflicto con subcaption
\usepackage{array}
\usepackage{perpage}
\usepackage{backnaur}
\usepackage{enumerate}
\usepackage{multirow}
\usepackage{adjustbox}
\usepackage{float}
\usepackage{pifont}

%%agregados
%Hipervinculos
\usepackage{hyperref}
\usepackage{subcaption}
\usepackage{amssymb}
\usepackage[table,xcdraw]{xcolor}
\usepackage{longtable}
\usepackage{tikz}
\usepackage{pgfplots}

%tabla
\usepackage{arydshln}


\usepackage{titlesec}

\usepackage{color}


\colorlet{punct}{red!60!black}
\definecolor{background}{HTML}{EEEEEE}
\definecolor{delim}{RGB}{20,105,176}
\colorlet{numb}{magenta!60!black}

\lstdefinelanguage{json}{
    basicstyle=\linespread{0.8}\footnotesize\ttfamily,
    numbers=left,
    numberstyle=\scriptsize,
    stepnumber=1,
    numbersep=8pt,
    showstringspaces=false,
    breaklines=true,
    frame=lines,
    literate=
     *{0}{{{\color{numb}0}}}{1}
      {1}{{{\color{numb}1}}}{1}
      {2}{{{\color{numb}2}}}{1}
      {3}{{{\color{numb}3}}}{1}
      {4}{{{\color{numb}4}}}{1}
      {5}{{{\color{numb}5}}}{1}
      {6}{{{\color{numb}6}}}{1}
      {7}{{{\color{numb}7}}}{1}
      {8}{{{\color{numb}8}}}{1}
      {9}{{{\color{numb}9}}}{1}
      {:}{{{\color{punct}{:}}}}{1}
      {,}{{{\color{punct}{,}}}}{1}
      {\{}{{{\color{delim}{\{}}}}{1}
      {\}}{{{\color{delim}{\}}}}}{1}
      {[}{{{\color{delim}{[}}}}{1}
      {]}{{{\color{delim}{]}}}}{1},
}

\definecolor{maroon}{rgb}{0.5,0,0}
\definecolor{darkgreen}{rgb}{0,0.5,0}
\lstdefinelanguage{XML}
{
  basicstyle=\linespread{0.8}\footnotesize\ttfamily,
  numberstyle=\scriptsize,
  morestring=[s]{"}{"},
  morecomment=[s]{?}{?},
  morecomment=[s]{!--}{--},
  commentstyle=\color{darkgreen},
  moredelim=[s][\color{black}]{>}{<},
  moredelim=[s][\color{red}]{\ }{=},
  stringstyle=\color{blue},
  breaklines=true,
  identifierstyle=\color{maroon},
  morekeywords={encoding,
    xs:schema,xs:element,xs:complexType,xs:sequence,xs:attribute}
}

\lstdefinelanguage{MySPARQL}{
  language     = SPARQL,
  morekeywords = {SERVICE, MINUS, ORDER, GROUP, INSERT, DATA},
  basicstyle=\linespread{0.8}\footnotesize\ttfamily
}

\lstset{
language=SQL,
morekeywords={PREFIX,java,rdf,rdfs,url}
}

\definecolor{mygreen}{rgb}{0,0.6,0}
\definecolor{mygray}{rgb}{0.5,0.5,0.5}
\definecolor{mymauve}{rgb}{0.58,0,0.82}


%Colores
\definecolor{color1}{HTML}{4F81BD}
\definecolor{color2}{HTML}{9BBB59}

\MakePerPage{footnote}
\addbibresource{referencias.bib}


%se importan las configuraciones customizadas realizadas.
%%%
%%% Conjuto de funciones utilitarias
%%% autor: Maximiliano Báez
%%% fecha: 25/08/2014
%%%

\usepackage{tablefootnote} %para utilizar \footnote{}
\usepackage{amssymb}
\renewcommand{\thefootnote}{\arabic{footnote}}

%%Para la construcción de la lista de simbolos
% Macro for 'List of Symbols', 'List of Notations' etc...
\def\listofsymbols{
    \newpage
\chapter*{Lista de Acrónimos\hfill}
\addcontentsline{toc}{chapter}{Lista de Símbolos}
\begin{tabbing}
% YOU NEED TO ADD THE FIRST ONE MANUALLY TO ADJUST THE TABBING AND SPACES
$F$~~~~~~~~~~\=\parbox{5in}{Imagen digital\dotfill \pageref{symbol:F}}\\
%$x$~~~~~~~~~~\=\parbox{5in}{X\dotfill \pageref{symbol:x}}\\
%ADD THE REST OF SYMBOLS WITH THE HELP OF MACRO

%% se añaden nuevos simbolos con el macro \newsymbol y se hace referecnia
% al simbolo utilizando \addsymbol{symbol:LABEL}
\newsymbol n: {Dimensión del vector que representa un color}{symbol:xcomp}
\newsymbol e: {Subconjunto discreto de una grilla rectangular $\mathbb{N}^{2}$ para imágenes 2D}{symbol:e}
\newsymbol d: {Rango de valores permitidos para el píxel}{symbol:d}
\newsymbol L^{*}: {Luminiscencia}{symbol:l*}
\newsymbol a^{*}: {Sensación rojo-verde}{symbol:a*}
\newsymbol b^{*}: {Sensación amarillo-azul}{symbol:b*}
\newsymbol H: {Matiz}{symbol:H}
\newsymbol S: {Saturación}{symbol:S}
\newsymbol I: {Intensidad}{symbol:I}
\newsymbol X: {Valor de conversión X RGB a CIELab}{symbol:x}
\newsymbol Y: {Valor de conversión Y RGB a CIELab}{symbol:y}
\newsymbol Z: {Valor de conversión Z RGB a CIELab}{symbol:z}
\newsymbol r: {Nivel de intensidad del píxel RGB en el componente r (red)}{symbol:r}
\newsymbol g: {Nivel de intensidad del píxel RGB en el componente g (green)}{symbol:g}
\newsymbol b: {Nivel de intensidad del píxel RGB en el componente b (blue)}{symbol:b}
\newsymbol f: {Función para convertir de RGB a CIELab}{symbol:f}
\newsymbol q,s,t: {Píxeles de una imagen}{symbol:q}
\newsymbol \Delta E_{Lab}: {Diferencia entre dos colores en CIELab}{symbol:elab}
\newsymbol \Delta L^{*}: {Diferencia de luminiscencia}{symbol:deltal}
\newsymbol \Delta a^{*}: {Diferencia del componente a*}{symbol:deltaa}
\newsymbol \Delta b^{*}: {Diferencia del componente b*}{symbol:deltab}
\newsymbol \Delta E_{rgb}: {Diferencia entre dos colores en RGB}{symbol:ergb}
\newsymbol \Delta r: {Diferencia del componente r}{symbol:deltargbR}
\newsymbol \Delta g: {Diferencia del componente g}{symbol:deltargbG}
\newsymbol \Delta b: {Diferencia del componente b}{symbol:deltargbB}
\newsymbol \varepsilon: {Erosión}{symbol:erosion}
\newsymbol \delta: {Dilatación}{symbol:dilatacion}
\newsymbol \gamma: {Gradiente Morfológica}{symbol:gradiente}
\newsymbol E: {Elemento estructurante}{symbol:estructurante}
\newsymbol D: {Conjunto de elementos}{symbol:D}
\newsymbol win: {Ventanas de la imagen}{symbol:win}
\newsymbol \beta: {Relación binaria}{symbol:beta}
\newsymbol A,B,C: {Etiqueta para los marcadores}{symbol:A}
\newsymbol W: {Etiqueta watershed}{symbol:W}
\newsymbol P: {Etiqueta para denotar un píxel como pendiente}{symbol:P}
\newsymbol VP: {Verdadero Positivo}{symbol:VP}
\newsymbol FP: {Falso Positivo}{symbol:FP}
\newsymbol VN: {Verdadero Negativo}{symbol:VN}
\newsymbol FN: {Falso Negativo}{symbol:FN}
\newsymbol T: {Función para transformar a un valor escalar}{symbol:T}
\newsymbol w: {Vector de pesos}{symbol:w}
\newsymbol V: {Vector de componentes de un píxel}{symbol:V}
\newsymbol i,j,u,v: {Índices de píxeles}{symbol:j}
\newsymbol MxN: {Cantidad de píxeles de una imagen}{symbol:n}
\newsymbol Sp: {Espacio de Color}{symbol:Sp}
\newsymbol \leq_{L}: {Orden lexicográfico}{symbol:lex}
\newsymbol \alpha: {Grado de influencia del primer componente del píxel}{symbol:alpha}
\newsymbol m: {Índice de bits}{symbol:m}
\newsymbol k: {Cantidad de bits de una pixel}{symbol:k}
\newsymbol x,y,z: {Ejemplos de pixeles representados como vectores}{symbol:vec}

\end{tabbing}
    \clearpage{}
}
\def\newsymbol #1: #2#3{$#1$ \> \parbox{5in}{#2 \dotfill \pageref{#3}}\\}
\def\addsymbol#1{\label{#1}}

% Para las imagenes en grilla
% custom commands
\newcommand{\foreign}[1]{{\it #1}}
\DeclareMathOperator*{\argmax}{arg\,max}
%\algsetup{}
% \algsetup{
%     indent=4em,
%     linenosize=\small,
%     linenodelimiter=.
% }

\usepackage{amsmath}

%% se utilizan para referenciar figuras, tablas, secciones y algoritmos
\newcommand{\figref}[1]{Figura \ref{#1}}
\newcommand{\tabref}[1]{Tabla \ref{#1}}
\newcommand{\chapref}[1]{Capítulo \ref{#1}}
\newcommand{\secref}[1]{Sección \ref{#1}}
\newcommand{\algref}[1]{Algoritmo \ref{#1}}
\newcommand{\apenref}[1]{Apéndice \ref{#1}}


%Traducción al español del paquete algorithmic%
% \floatname{algorithm}{Algoritmo}
% \renewcommand{\listalgorithmname}{Lista de algoritmos}
% \renewcommand{\algorithmicrequire}{\textbf{Entrada:}}
% \renewcommand{\algorithmicensure}{\textbf{Salida:}}
% \renewcommand{\algorithmicend}{\textbf{fin}}
% \renewcommand{\algorithmicif}{\textbf{si}}
% \renewcommand{\algorithmicthen}{\textbf{entonces}}
% \renewcommand{\algorithmicelse}{\textbf{si no}}
% \renewcommand{\algorithmicelsif}{\algorithmicelse,\ \algorithmicif}
% \renewcommand{\algorithmicendif}{\algorithmicend\ \algorithmicif}
% \renewcommand{\algorithmicfor}{\textbf{para}}
% \renewcommand{\algorithmicforall}{\textbf{para todo}}
% \renewcommand{\algorithmicdo}{\textbf{hacer}}
% \renewcommand{\algorithmicendfor}{\algorithmicend\ \algorithmicfor}
% \renewcommand{\algorithmicwhile}{\textbf{mientras}}
% \renewcommand{\algorithmicendwhile}{\algorithmicend\ \algorithmicwhile}
% \renewcommand{\algorithmicloop}{\textbf{repetir}}
% \renewcommand{\algorithmicendloop}{\algorithmicend\ \algorithmicloop}
% \renewcommand{\algorithmicrepeat}{\textbf{repetir}}
% \renewcommand{\algorithmicuntil}{\textbf{hasta que}}
% \renewcommand{\algorithmicprint}{\textbf{imprimir}}
% \renewcommand{\algorithmicreturn}{\textbf{retorna}}
% \renewcommand{\algorithmictrue}{\textbf{cierto }}
% \renewcommand{\algorithmicfalse}{\textbf{falso }}
% \renewcommand{\algorithmiccomment}{\textbf{comentario: }}

%Traducción al español del paquete algorithm2c%

\SetKwInput{KwIn}{Entrada}%
\SetKwInput{KwOut}{Salida}%
\SetKwInput{KwData}{Datos}%
\SetKwInput{KwResult}{Resultado}%
\SetKw{KwTo}{a}%
\SetKw{KwRet}{devolver}%
\SetKw{Return}{devolver}%
\SetKwBlock{Begin}{inicio}{fin}%
\SetKwRepeat{Repeat}{repetir}{hasta que}%
%
\SetKwIF{If}{ElseIf}{Else}{si}{entonces}{sin\'o, si}{en otro caso}{fin si}
\SetKwSwitch{Switch}{Case}{Other}{seleccionar}{hacer}{caso}{sin\'o}{fin caso}{fin seleccionar}
\SetKwFor{For}{per}{fai}{fine per}%
\SetKwFor{ForPar}{par}{hacer in paralelo}{fin para}%
\SetKwFor{ForEach}{para cada}{hacer}{fin para cada}
\SetKwFor{ForAll}{para todo}{hacer}{fin para todo}
\SetKwFor{While}{mientras}{hacer}{fin mientras}



\addtocontents{loa}{\protect\thispagestyle{empty}}

% add this to align the list of algorithms
\makeatletter
  \renewcommand*{\listof}[2]{%
    \@ifundefined{ext@#1}{\float@error{#1}}{%
      \@namedef{l@#1}{\@dottedtocline{1}{2em}{1.3em}}% line of the list (change from 1em to the desired value)
      \float@listhead{#2}%
      \begingroup\setlength{\parskip}{\z@}%
        \@starttoc{\@nameuse{ext@#1}}%
      \endgroup}}
  \renewcommand*\l@algocf{\@dottedtocline{1}{1em}{2.3em}}% line of the list (change from 1em to the desired value)
\makeatother



 \setcounter{tocdepth}{4}

%Datos de la tesis
\title{Interoperabilidad semántica: Una aplicación de ontologías al Dominio de Contrataciones Públicas de Paraguay}
\author{Jose Rodrigo Valdez y Camilo Báez}
\degree{Informática}

\advisors{Orientadores:}{Phd. Gustavo Gimenez Lugo
\\Phd. Juan Pane}

\logosource{./figuras/politecnica_logo.jpg}
\institution{Universidad Nacional de Asunción}
\faculty{Facultad Politécnica}
\address{San Lorenzo - Paraguay}

%\newtheorem{definicion}{Definicin}
%\numberwithin{algorithm}{chapter}

% \logosource{./graphics/logo.jpg}
% \institution{Universidad Nacional de Asunci\'{o}n}
% \faculty{Facultad Polit\'{e}cnica}
% \address{San Lorenzo - Paraguay}

\usepackage{glossaries}
\usepackage{etoolbox}
\usepackage[xindy]{imakeidx}

\usepackage{hyperref}

\newtoks\customtok


\renewcommand*{\newacronymhook}{%
 \edef\dosetkeys{\noexpand\setkeys{glossentry}{user1={},\the\glskeylisttok}}%
 \dosetkeys
 \ifcsempty{@glo@useri}%
 {%
   \expandafter\customtok\expandafter{\the\glsshorttok}%
 }%
 {%
   \edef\custom{\the\glsshorttok, \csexpandonce{@glo@useri}}%
   \expandafter\customtok\expandafter{\custom}%
 }%
}

\newcommand*{\custompostdesc}[1]{%
  \ifcsempty{glo@#1@useri}{}{ (\glsentryuseri{#1})}%
}

\renewcommand*{\CustomAcronymFields}{%
  user1={},%
  name={\the\glsshorttok},%
  description={\the\glslongtok\noexpand\custompostdesc{\the\glslabeltok}},%
  first={\the\glslongtok\space(\the\customtok)},%
  firstplural={\the\glslongtok\noexpand\acrpluralsuffix
    \space (\the\customtok)}%
  text={\the\glsshorttok},%
  plural={\the\glsshorttok\noexpand\acrpluralsuffix}%
}

\SetCustomStyle

%\makeglossaries
\makeindex

\pgfplotsset{compat=1.12}

\begin{document}

\maketitle     % esto hace las portadas

% Agradecimientos y dedicatoria
\begin{dedicatoria} 

Dedicamos este trabajo a nuestras familias y en especial a quiénes formaron parte de este proceso\dots

\end{dedicatoria}
\begin{thankspage} 
A nuestras familias y amigos que nos dieron su apoyo incondicional en este proceso de formación.

A nuestros compañeros de facultad que vivieron con nosotros la lucha constante de autosuperación para ser profesionales dignos que ayuden a construir este país.

A la Facultad Politécnica que fue nuestro segundo hogar durante nuestros años de estudio.

Al querido Profesor Gustavo Gimenez Lugo y todos los profesores que dedicaron su tiempo en transmitirnos sus conocimientos sin recelos.

\end{thankspage}
\begin{resumen}


    
\textbf{Palabras clave:} Web Semántica,  Ontologías, Contrataciones Públicas, Open Contracting, OWL, Sparql.
\end{resumen}



\begin{resuingles}
Chagas disease is a tropical parasitic disease caused by the protist Trypanosoma cruzi, it is considered by the World Health Organization as a neglected tropical disease. Neglected tropical diseases are those that affect a high amount of the population, mostly on third world countries, and have an expensive, inefficient or inexistent treatment. The first step in determining the effectiveness of new antichagasic drugs that help reduce the costs of treating the disease is the counting of the intracellular parasitic form (amastigote) under a microscope.
The watershed transform is an image segmentation algorithm.
Image segmentation is the process of partitioning a digital image into its constituent regions. Said segmentation is an important process to perform a count of Trypanosoma cruzi amastigotes and thus determine the efficacy of new antichagasic drugs.The watershed algorithm has been used in to solve the segmentation problem, most often in grayscale images. Grayscale images may lose important information for segmentation, this is because it only uses the intensity of the colors. The color images give more information, but implementing the color watershed transform is not trivial. The implementation of the watershed transform by flooding in color images proposed by Meyer, requires an order between the pixels to perform the segmentation. This work extends the watershed algorithm by flooding for color images using the CIELab color space for the segmentation of Trypanosoma Cruzi Amastigotes in microscopic cell images. It uses the Euclidean distance to a reference color for the ordering of the pixels. The proposed method was compared with the implementation of the watershed transform in different color spaces and using different pixel ordering methods. The CIELab color space using the distance to the median of the colors presented the best results with an accuracy of 91 \%. The CIELab color space obtained better results regardless of the reference color used in relation to the methods in the different color spaces and grayscale images. The proposed segmentation method can be used to identify Trypanosoma cruzi amastigotes, and in this way can count and collect statistics necessary to produce new drugs for Chagas disease.

\end{resuingles}




% los siguientes comandos producen 'indices.
\renewcommand{\lstlistingname}{Cuadro}
\renewcommand{\lstlistlistingname}{Índice de \lstlistingname s}

% Tabla de contenidos
\tableofcontents
% Lista de figuras
\listoffigures
% Lista de tablas
\listoftables
% Lista de algoritmos
%\listofalgorithms
\newpage
\renewcommand\lstlistlistingname{Índice de Cuadros}
 \addcontentsline{toc}{chapter}{Índice de Cuadros}
 \lstlistoflistings
 
%\printglossary[type=\acronymtype,title=Lista de Siglas]
%\addcontentsline{toc}{chapter}{Lista de Siglas}
\newacronym{rgb}{RGB}{Red Green Blue}
\newacronym{hsi}{HSI}{Hue Saturation Intensity}
\newacronym{cie}{CIE}{Commission Internationale de l’Eclairage}
\newacronym{oms}{OMS}{Organización Mundial de la Salud}
\newacronym{cielab}{CIELab}{Espacio de Color presentado por la CIE}



% Lista de simbolos
\listofsymbols


\mainmatter  % inician los capitulos de la tesis

% incluye aqui los capitulos (un archivo .tex por capitulo)
% los capitulos
%!TEX root = ../tesis.tex
\chapter{Introducción}
Según el reporte de la Open Contracting Partnership (OCP), los Gobiernos de todo el mundo gastan un estimado de 9.5 billones de dólares anualmente mediante contratos. Sin embargo, en la mayoría de los países, la información sobre estos contratos no está disponible para el escrutinio público, haciendo al proceso de contratación vulnerable a la corrupción y la mala gestión. Además, los países que sí lo hacen, no proveen un contexto apropiado para entender, reutilizar e integrar dichos datos con otros.

Debido a la creciente importancia de publicar los datos de contrataciones públicas  los Estados crean portales de datos con el fin de transparentar los procesos y establecer mecanismos para que la ciudadanía, empresas y organizaciones puedan ejercer un rol de contralores y a la vez facilitar a los entes gubernamentales a contratar bienes y servicios. Este es el caso del portal Tender Electronic Daily (TED) en la Unión Europea, República Checa, y la Municipalidad de Palmares en Costa Rica. En Paraguay, estos datos pueden ser obtenidos desde el Portal de Datos Abiertos de la Dirección Nacional de Contrataciones Públicas (DNCP).

Según Svátek, los datos de contrataciones públicas son importantes dentro de la Web Semántica porque unifican dos esferas muy diferentes como son: las necesidades públicas por un lado y la oferta de productos y servicios del sector privado por el otro; este escenario es idóneo para la utilización de modelos de datos, metodologías y fuentes de información publicadas de forma independiente. Además, la gran cantidad de datos de contrataciones públicas da lugar para aplicar diversos métodos de análisis de datos que van desde estadísticas agregadas en forma de barras hasta análisis más exhaustivos de minería de datos.
Este trabajo nace con la intención de crear una ontología capaz de mejorar la accesibilidad a la información pública, y que dicha información sea comprensible por máquinas con el fin de brindar mayor desarrollo y transparencia al proceso de contrataciones públicas.



\section{Justificación}

Debido a la gran cantidad de datos disponibles en internet, la integración entre sistemas de información heterogéneos que disponibilizan estos datos se vuelve un problema complejo de resolver para poder quitar provecho de los mismos. 

Este desafío puede ser abordado a través de tecnologías de la Semantic Web, en ella se integran estándares y herramientas necesarias para la disponibilización de recursos en la web de manera a que sean entendibles por máquinas y a la vez haciendo posible la integración de recursos de diferentes fuentes. En este aspecto, las ontologías proveen una descripción de los datos que no depende de un contexto en particular y puede ser entendido tanto por sistemas de información como por personas, logrando así la interoperabilidad semántica.

El problema computacional que se quiere resolver es la interoperabilidad sintáctica y semántica de datos de un dominio de conocimiento en particular con datos de distintas fuentes de publicación y de distintos dominios de conocimientos. El trabajo pretende modelar una ontología para brindar una capa de semántica formal a un estándar de datos abiertos para la publicación de datos estructurados llamado Open Contracting Data Standard (OCDS), aplicado a datos de la DNCP utilizando metodologías y herramientas para el desarrollo de ontologías. 


\section{Objetivo General del Proyecto}



\section{Objetivos Específicos}
\label{objetivos}



\section{Contribuciones}
\label{Contribuciones}
\begin{itemize}
\item Serie de mejoras y propuestas de solución de errores a la API de la DNCP.
\item Software de extracción y manipulación de datos de Procesos Licitatorios disponibilizados por la DNCP.
\item Documentación del proceso del desarrollo de una Ontología, reutilizando de un recurso no-ontológico basado en JSON-SCHEMA.
\item Una Ontología basada en el OCDS.
\item Casos de cómo lograr la interoperabilidad entre los datos publicados por la DNCP y otras fuentes de datos.
\end{itemize}
 

\section{Estructura del documento}
El trabajo está estructurado de la siguiente manera; en el \textbf{Capítulo 2} trataremos el marco teórico del trabajo abordando los temas de la web semántica, las ontologías, sus metodologías y herramientas. En el\textbf{ Capítulo 3 }se mostrará el proceso del modelado de la ontología, en el \textbf{Capítulo 4} se implementará el entorno tecnológico de experimentación para luego en el \textbf{Capítulo 5 }mostrar las pruebas realizadas a los datos que utilizan la ontología desarrollada. En el \textbf{Capítulo 6 }se evaluará el cumplimiento de los objetivos propuestos, las conclusiones finales y futuros trabajos que pudieran abordarse. Al final de este documento  se  incluyen las referencias bibliográficas utilizadas para su elaboración y el anexo que provee información complementaria al trabajo.



%!TEX root = ../main.tex
\chapter{Marco Teórico}
\label{chap:Marco Teorico}

\section{Web Semántica y Ontologías}

\subsection{Open Data}

Open Data se refiere al hecho de mantener los datos publicados en internet de tal manera a cumplir con algunos requisitos para que éstos puedan ser utilizados y reutilizados en cualquier momento y desde cualquier sitio \cite{bauer2011linked}.

\subsubsection{Linked Open Data (LOD)}

Linked Open Data se enfoca en un método de publicación de datos abiertos estructurados para que puedan ser interconectados. Para lograr esto se utilizan las tecnologías como RDF, RDFa, etc. para estructurar los datos, utilizando URIs para identificar los datos individualmente. Tim Berners Lee define un modelo de 5 estrellas para clasificar e identificar el grado de publicación de datos abiertos.

\begin{table}[tbp]
\centering
\caption{Modelo de publicación de 5 estrellas}
\label{modelo-5-estrellas}
\resizebox{15cm}{!} {
\begin{tabular}{|c|l|}
\hline
\ding{72} & La información está disponible en la web (en cualquier formato) bajo una licencia abierta \\ \hline
\ding{72} \ding{72} & La información está disponible como dato estructurado (Ej. Excel en lugar de una imagen de una tabla) \\ \hline
\ding{72} \ding{72} \ding{72} & Son utilizados formatos no-propietarios (Ej. CSV en lugar de Excel) \\ \hline
\ding{72} \ding{72} \ding{72} \ding{72}  & URIs son utilizadas para que se puedan individualizar los datos \\ \hline
\ding{72} \ding{72} \ding{72} \ding{72} \ding{72}  & Los datos son enlazados con otros datos para proveer un contexto \\ \hline
\end{tabular}
}
\end{table}

\subsection{La Web Semántica}
La Web Semántica consiste en una serie de estándares y tecnologías propuestas por la W3C que promueven el uso de formatos de datos común y además de un protocolo de datos dentro de la web que nos permite compartir y reusar datos -procesables por máquinas- a través de la web. Se puede pensar la Web Semántica como una manera eficiente de representar datos en la web.

El principal problema de los datos en la web es la dificultad de su utilización a gran escala, ya que no siempre se aplican estándares de publicación de datos de manera a que facilite su procesamiento. Por otra parte, los datos enlazados a través de identificadores comunes en la web nos dan la capacidad de consultar datos de distintas fuentes y contestar preguntas complejas pero interpretar el significado de los mismos no resulta una tarea sencilla. Es por eso que las ontologías juegan un rol fundamental en la web semántica, ya que gracias a ellas podemos representar conocimiento legible y entendible por máquinas y humanos en la web.

\subsection{Ontologías}
Las ontologías son utilizadas para la representación del conocimiento, logrando así que la información esté representada de forma a que pueda ser interpretada por los computadores y humanos \cite{horrocks2011kr}. En ella se definen los conceptos de un determinado dominio, sus propiedades y relaciones entre los mismos, así también reglas para combinar términos y relaciones que permitan extender el vocabulario.
A continuación se pone a conocimiento algunas definiciones hechas acerca de las ontologías en el ámbito de sistemas de información.

\cite{gruber1993translation} definió originalmente el concepto de ontología como “Una especificación explícita de una conceptualización”. \cite{borst1997construction} definió una ontología como “Una especificación formal de una conceptualización compartida”. La definición que se eligió en este trabajo es la de \cite{studer1998knowledge}: “Una ontología se define como una especificación formal y explícita de una conceptualización compartida”.  En esta definición, conceptualización se refiere a un modelo abstracto de algún fenómeno del mundo derivado de la identificación de sus conceptos relevantes; explícita se refiere a que los tipos de conceptos y las restricciones usadas sobre ellos se definen explícitamente; formal se refiere a que la ontología debe ser “legible” por una computadora; y compartida refleja que una ontología capta un conocimiento consensual. \cite{guarino1998formal} también definió como “Un conjunto de axiomas lógicos diseñados para tener en cuenta el significado deseado de un vocabulario”.

Actualmente existen ontologías para diversos dominios de conocimiento. Algunas de las más representativas son: \textit{Simple Knowledge Organization System} (SKOS) \cite{isaac2009skos}, \textit{ Friend of a Friend} (FOAF) \cite{brickley2012foaf}, \textit{Good Relations}, \cite{hepp2008goodrelations}, MGED Ontology \cite{whetzel2006mged} y National Cancer Institute Ontology \cite{golbeck2011national}. Así también existen ontologías que no pertenecen a un dominio específico (de alto nivel) y describen conceptos generales como DOLCE \cite{Masolo02thewonderweb}.
La ontología posee varios usos en Ciencias de la Computación, como ser en el ámbito legal, médico, científico y otros, pero ganó mayor popularidad en el ámbito de la Web Semántica. A continuación veremos las ontologías en el caso de uso de la web semántica.


\subsection{Ontologías en la Web Semántica}
Uno de los principios fundamentales de las ontologías dentro de la web semántica es su reutilización y extensión, esto significa utilizar conceptos de otras ontologías, siempre y cuando sea posible, mejorando el procesamiento y la integración de datos enlazados. En lugar de duplicar esfuerzos definiendo conceptos y propiedades que ya fueron definidos en otra otología, se puede referenciar a estos elementos dentro de la nueva ontología. 

Existe una variedad de tipos de ontologías que dependen del nivel de formalidad, razonamiento y la cantidad de restricciones. En el caso más simple una ontología puede describir una jerarquía de conceptos vinculados por relaciones de subsunción y en casos más sofisticados incluir axiomas para expresar relaciones entre conceptos y restringir la interpretación pretendida.

Las ontologías para la Web Semántica son de tipo ligeras en términos de niveles de formalidad y normalmente pequeñas en tamaño haciéndolas fáciles de manejar y mantener. El poder de expresión de una ontología en la Web Semántica debe ser adecuado, el lenguaje debe ser tan expresivo como para poder describir conceptos con suficiente detalle, pero no tan expresivo como para perder la capacidad de razonamiento. Una ontología más formal o expresiva es computacionalmente más costosa, por lo que se debe encontrar un equilibrio entre los dos puntos. En la Figura \ref{img:expresividad complejidad } se puede ver el espectro de ontologías según su expresividad y complejidad computacional, en donde el lenguaje OWL y la Lógica de Descripciones se encuentran dentro de las más expresivas y costosas computacionalmente.
La capacidad de razonamiento es importante para asegurar la calidad de la ontología construida, por ejemplo, en la etapa de diseño puede ser utilizada para probar si existen conceptos contradictorios y derivar nuevas relaciones.

Las formas de representar ontologías en la Web Semántica son diversas en cuanto a lenguajes y formas de serialización, a continuación se presenta algunas de ellas.

    \begin{figure}[ht!]
    \centering
    \includegraphics[width=150mm]{figuras/Diagramas-ComplejidadOntologica}
    \caption{Expresividad vs Complejidad}
    \label{img:expresividad complejidad }
    \end{figure}



\subsection{RDF y RDFS (RDF)}

Marco de Descripción de Recursos, RDF por sus siglas en inglés Resource Description Framework \cite{rdf}, es un framework para representar información en la Web Semántica. El lenguaje RDF forma parte de la W3C's Semantic Web Activity y es recomendado por la W3C. El modelo de datos RDF está diseñado para que sea procesable por computadoras. 

RDF nos permite representar modelos de datos basados en grafos, los cuales son representados en triplas. Las triplas RDF contienen tres componentes:
El sujeto, que puede ser una referencia tipo URI o un Blank Node
El predicado, que es una referencia tipo URI
El objeto que puede ser una referencia tipo URI, un literal o un Blank Node

En la Figura \ref{img:componentes rdf } se expone un grafo de ejemplo en el cual se visualizan estos tres componentes.

    \begin{figure}[ht!]
    \centering
    \includegraphics[width=150mm]{figuras/Diagramas-RDFGraph}
    \caption{Componentes RDF}
    \label{img:componentes rdf }
    \end{figure}
    
El sujeto es la fuente de la arista y debe ser un recurso. En RDF, un recurso puede ser cualquier cosa que sea identificable de forma única a través de un URI. Comúnmente este identificador es un URL, que es un caso especial de URI. Sin embargo, los URI son más generales que las URL. En particular, no es requerimiento que un URI se pueda utilizar para localizar un documento en Internet. El objeto de una sentencia es el objetivo de la arista. Al igual que el sujeto, puede ser un recurso identificado por un URI, pero alternativamente puede ser un valor literal como una cadena o un número. Los predicados de una sentencia determinan qué tipo de relación se mantiene entre el sujeto y el objeto. También es identificado por un URI.

Un identificador uniforme de recursos (URI por sus siglas inglés Uniform Resource Identifier) es una cadena de caracteres que identifica inequívocamente un recurso en particular. 

Todos los literales tienen una forma léxica que es una cadena Unicode. Un literal en un grafo RDF puede presentarse de dos formas:
\begin{itemize}
    \item Los literales simples tienen una forma léxica y, opcionalmente, una etiqueta de idioma normalizada a minúscula. Ej.: “Hola mundo”@es.
    \item Los literales tipados que tienen una forma léxica y una URI de tipo de datos que es una referencia de URI RDF. Ej.: “Hola mundo”xsd:string.
\end{itemize}

Un Blank Node es un nodo en un documento RDF que representa un recurso para el que no se proporciona un URI o literal y también es llamado recurso anónimo.

RDF es simplemente un modelo de datos y no provee significado semántico de los mismos. RDF Schema (RDFS) en cambio es una extensión de RDF que nos permite definir un vocabulario para el uso en modelos de datos RDF, permite definir tipo de clases de un recurso y propiedades que un recurso puede tener, esto a través de un vocabulario que incluye rdfs:Class, rdf:Property , rdfs:subClassOf, rdfs:subPropertyOf, rdfs:domain, rdfs:range y otras propiedades de documentación como son rdfs:label y rdfs:comment.

La sintaxis de RDF puede escribirse en distintos formatos de serialización: RDF/XML, Turtle, TriG, N-Triples, N-Quads y JSON-LD.

En el cuadro \ref{lst:n-quads} se muestra un ejemplo de triplas en formato RDF/N-Quads :

\definecolor{maroon}{rgb}{0.5,0,0}
\definecolor{darkgreen}{rgb}{0,0.5,0}
\lstdefinelanguage{XML}
{
  basicstyle=\ttfamily,
  numberstyle=\scriptsize,
  morestring=[s]{"}{"},
  morecomment=[s]{?}{?},
  morecomment=[s]{!--}{--},
  commentstyle=\color{darkgreen},
  moredelim=[s][\color{black}]{>}{<},
  moredelim=[s][\color{red}]{\ }{=},
  stringstyle=\color{blue},
  breaklines=true,
  identifierstyle=\color{maroon}
}
\lstset{language=XML, morekeywords={encoding,
    xs:schema,xs:element,xs:complexType,xs:sequence,xs:attribute}}

\begin{lstlisting}[captionpos=b, caption=Ejemplo en RDF/N-Quads, label=lst:n-quads,  numbers=left,  numberstyle=\tiny\color{mygray},
    basicstyle=\tiny,frame=single]
<http://rodrivaldez5.com/> <http://www.w3.org/2000/01/rdf-schema#type> <http://schema.org/person> .  
<http://www.rodrivaldez5.com/> <http://schema.org/name>"Rodrigo Valdez" .
<http://www.rodrivaldez5.com/> <http://schema.org/telephone>"0981530572" .
<http://www.rodrivaldez5.com/> <http://schema.org/jobTitle>"Developer" .
<http://www.rodrivaldez5.com/> <http://schema.org/image> <http://www.rodrivaldez5.com/images/rodri.png> .
\end{lstlisting}


En linea 1, <http://www.rodrivaldez5.com/> es el sujeto, <http://schema.org/image> es el predicado y <http://www.rodrivaldez5.com/images/rodri.png> es el objeto, donde el objeto en este caso es una URI. En la línea 2 se puede ver que el objeto se trata de un dato literal de tipo String.  A continuación se presenta JSON-LD, una forma de representación de datos compatible con RDF.

\subsection{Serialización JSON-LD}

JSON, por sus siglas en inglés JavaScript Object Notation es un formato de texto utilizado para la representación e intercambio de datos en la web. JSON-LD \cite{JSONLDSy39:online}, donde LD proviene de Linked Data, es una extensión de JSON cuya idea es lograr el enlace de datos y agregar una capa de semántica a la web. JSON-LD define un mecanismo para mapear los términos JSON (propiedad y valor) a URIs que poseen información semántica acerca del término. 

Todo objeto JSON-LD es es también un objeto JSON. JSON-LD es compatible con RDF, ya que se puede representar en triplas, en este caso el nombre de la propiedad (JSON) corresponde al predicado (RDF), el sujeto (RDF) corresponde al objeto (JSON) y el valor de la propiedad (JSON) corresponde al objeto (RDF).

A continuación se plantea un ejemplo de transformación de un documento JSON a JSON-LD 

En el Cuadro \ref{lst:json} se muestra un ejemplo de documento JSON y en el Cuadro \ref{lst:json-ld} se muestra su transformación a JSON-LD.  
\newline

\colorlet{punct}{red!60!black}
\definecolor{delim}{RGB}{20,105,176}
\colorlet{numb}{magenta!60!black}
\lstdefinelanguage{json}{
    basicstyle=\normalfont\ttfamily,
    numbers=left,
    numberstyle=\scriptsize,
    stepnumber=1,
    numbersep=8pt,
    showstringspaces=false,
    breaklines=true,
    frame=lines,
    literate=
     *{0}{{{\color{numb}0}}}{1}
      {1}{{{\color{numb}1}}}{1}
      {2}{{{\color{numb}2}}}{1}
      {3}{{{\color{numb}3}}}{1}
      {4}{{{\color{numb}4}}}{1}
      {5}{{{\color{numb}5}}}{1}
      {6}{{{\color{numb}6}}}{1}
      {7}{{{\color{numb}7}}}{1}
      {8}{{{\color{numb}8}}}{1}
      {9}{{{\color{numb}9}}}{1}
      {:}{{{\color{punct}{:}}}}{1}
      {,}{{{\color{punct}{,}}}}{1}
      {\{}{{{\color{delim}{\{}}}}{1}
      {\}}{{{\color{delim}{\}}}}}{1}
      {[}{{{\color{delim}{[}}}}{1}
      {]}{{{\color{delim}{]}}}}{1},
}
\lstset{
    language=json
}  


\noindent\begin{minipage}{\textwidth}
\begin{lstlisting}[captionpos=b, caption=Ejemplo de un documento JSON, label=lst:json,  numbers=left,  numberstyle=\tiny\color{mygray},
    basicstyle=\tiny,frame=single]
{
  "id": "http://www.rodrivaldez5.com/",
  "name": "Rodrigo Valdez",
  "telephone": "0981530572",
  "jobTitle": "Developer",
  "image": "http://www.rodrivaldez5.com/images/rodri.png"
}
\end{lstlisting}
\end{minipage}

\noindent\begin{minipage}{\textwidth}
\begin{lstlisting}[captionpos=b, caption=Ejemplo de un documento JSON-LD, label=lst:json-ld,  numbers=left,  numberstyle=\tiny\color{mygray},
    basicstyle=\tiny,frame=single]
{
"@id": "http://www.rodrivaldez5.com/",
"@type": "http://schema.org/person",
"http://schema.org/name": "Rodrigo Valdez",
"http://schema.org/telephone" : "0981530572",
"http://schema.org/jobTitle": "Developer",
"http://schema.org/image":   {"@id": "http://www.rodrivaldez5.com/images/rodri.png"} 
}
\end{lstlisting}
\end{minipage}
De aquí se puede observar que el principal cambio radica en los nombres de las propiedades, las cuales ahora son URIs válidos, el resultado es equivalente a las triplas RDF del Cuadro \ref{lst:n-quads}

Además JSON-LD permite crear un contexto que contiene el mapeamiento entre el nombre de la propiedad y la URI, dando como resultado el objeto JSON-LD del Cuadro \ref{lst:json-ld-contexto}

\begin{lstlisting}[captionpos=b, caption=Ejemplo de un documento JSON-LD con Contexto, label=lst:json-ld-contexto,  numbers=left,  numberstyle=\tiny\color{mygray},
    basicstyle=\tiny,frame=single]
{
  "@context": {
    "name": "http://schema.org/name",  
    "joTitle": "http://schema.org/jobTitle",  
    "telephone": "http://schema.org/telephone",  
    "image": {
      "@id": "http://schema.org/image",  
      "@type": "@id"  
    },
  
  },
  "@id": "http://www.rodrivaldez5.com/",
  "@type": "http://schema.org/person",
  "image": "http://www.rodrivaldez5.com/images/rodri.png",
  "jobTitle": "Developer",
  "name": "Rodrigo Valdez",
  "telephone": "0981530572"
}
\end{lstlisting}

El contexto también puede definirse en otro documento diferente y solamente haciendo referencia a éste, quedando el JSON final como se muestra en el Cuadro \ref{lst:json-ld-contexto-referencia}

\noindent\begin{minipage}{\textwidth}
\begin{lstlisting}[captionpos=b, caption=Ejemplo de un documento JSON-LD con Contexto referenciado, label=lst:json-ld-contexto-referencia,  numbers=left,  numberstyle=\tiny\color{mygray},
    basicstyle=\tiny,frame=single]
{
"@context": "http://schema.org/",
"@id": "http://www.rodrivaldez5.com/",
"http://schema.org/name": "Rodrigo Valdez",
"http://schema.org/telephone" : "0981530572",
"http://schema.org/jobTitle": "Developer",
"http://schema.org/image":   {"@id": "http://www.rodrivaldez5.com/images/rodri.png"} 
}
\end{lstlisting}
\end{minipage}
El formato JSON es ampliamente utilizado y preferido por los desarrolladores en la web, en comparación con el formato de representación RDF. En la Tabla \ref{prefencia-uso} se muestra la preferencia de uso entre la serialización RDF y JSON-LD en algunas categorías de aplicaciones \cite{rdfjson}.

\FloatBarrier
\begin{table}[ht!]
\footnotesize
\centering
\caption{Preferencias de uso de serialización RDF y JSON-LD}
\label{prefencia-uso}
\resizebox{15cm}{!} {
\begin{tabular}{|l|l|l|}
\hline
 \thead{Categoría de Aplicación} & \thead{RDF o JSON-LD} & \thead{Comentarios}\\\hline
\multicolumn{1}{|m{5cm}|}{Aplicación Web API} & JSON-LD & \multicolumn{1}{m{5cm}|}{La sintaxis está diseñada para integrar fácilmente en sistemas desplegados que ya usan JSON, y proporciona un camino de actualización sin problemas de JSON a JSON-LD}\\\hline
\multicolumn{1}{|m{5cm}|}{Aplicaciones de UI basadas en navegador} & JSON-LD & \multicolumn{1}{m{5cm}|}{La gran cantidad de parsers basados en JSON. Javascript es el lenguaje del navegador}\\\hline
\multicolumn{1}{|m{5cm}|}{Aplicaciones basadas en Inferencia, Razonamiento} & RDF & 
\multicolumn{1}{m{5cm}|}{Gran soporte para razonadores y almacenes de tripletas escalables}\\\hline
\multicolumn{1}{|m{5cm}|}{Herramientas de consultas Expresivas} & RDF & \multicolumn{1}{m{5cm}|}{El estado avanzado de SPARQL 1.1 ayuda a escribir consultas potentes y expresivas}\\ \hline
\end{tabular}
}
\end{table}
\FloatBarrier



\subsection{Web Ontology Language (OWL)}
OWL es un estándar internacional para codificar e intercambiar ontologías y es diseñado para soportar la Web Semántica.

OWL es el lenguaje de representación de conocimientos recomendado por la W3C y es una extensión de RDFS, dándole mayor expresividad a través de operaciones booleanas (intersección, unión, complemento), restricciones de cardinalidad, cuantificación existencial, etc. El mismo está basado en lenguajes de representación de conocimientos llamados Lógica de Descripciones (DL). DL es un lenguaje formal usado para construir ontologías y permite declarar conocimiento de un dominio específico e incluir reglas de razonamiento para poder procesarlo [Kalibatiene and Vasilecas, 2011]

Al estar basado en DL, OWL nos trae consigo las siguientes ventajas
\begin{itemize}
    \item Expresividad: La Lógica de Descripciones nos permite tener expresiones complejas de los conceptos a modelar.
    \item Razonador Automático: la Lógica de Descripciones está basado en lógica formal, eso nos permite desarrollar razonadores capaces de verificar la consistencia de la ontológica e inferir nuevo conocimiento.
\end{itemize}

OWL posee tres sub lenguajes: OWL Lite, OWL DL y OWL Full. Todos permiten describir clases, propiedades e instancias pero difieren uno de otro en el nivel de especificación requerido. OWL Lite está diseñado para usuarios cuyas necesidades de modelado sean simples. OWL DL es lo más cercano a una DL expresiva manteniendo la completitud computacional, esto significa que la ontología es procesable computacionalmente. OWL Full posee mayor expresividad sacrificando la completitud computacional de la ontología. En la Figura \ref{img:subclases owl} se muestra la relación entre los tres sublenguajes.

\begin{figure}[h!]
    \centering
    \includegraphics[width=150mm]{figuras/Diagramas-OwlSubClasses}
    \caption{Subclases OWL}
    \label{img:subclases owl}
    \end{figure}

OWL DL posee todas las características de OWL Lite más sus propias complejidades. OWL Full posee las mismas características de OWL DL agregando otras que resultan en mayor complejidad.

\section{Metodologías para el desarrollo de Ontologías}

En este apartado se dará un resumen de las metodologías más utilizadas para la creación de una ontología; Ontology Development 101, OntoKnowledge , Methontology, y NeOn. Luego se realizará un análisis y elección según los requerimientos de este trabajo.

\subsection{Ontology Development 101}

Esta metodología describe una serie de pasos y recomendaciones a la hora de crear una ontología. 

Se describen a continuación los pasos:
\begin{description}
\item[Paso 1:] Determinar el dominio y alcance de la ontología.
\item[Paso 2:] Considerar la reutilización de ontologías.
\item[Paso 3:] Enumerar términos importantes de la ontología.
\item[Paso 4:] Definir las clases y las jerarquías de las clases.
\item[Paso 5:] Definir las propiedades de las clases.
\item[Paso 6:] Definir las características de cada clase y propiedad.
\item[Paso 7:] Crear instancias.
\end{description}

Además la guía propone algunas buenas prácticas de desarrollo de ontologías relacionadas a la definición de clases como la correcta creación de jerarquías, herencias múltiples,  cuándo crear una clase o una propiedad, cuando es una instancia o una clase, etc. También algunos delineamientos sobre las propiedades como valores por defecto, cardinalidad, etc. Por último, una serie de convenciones de nombres dentro de la ontología. La metodología no tiene en cuenta el proceso de especificación ni el mantenimiento de la ontología.

\subsection{OntoKnowledge} 

Ontoknowledge propone crear ontologías teniendo en cuenta su uso posterior en sistemas de administración de conocimiento, por tanto las ontologías creadas son dependientes de la aplicación. La metodología incluye la identificación de metas a ser logradas a través de herramientas de control basadas en los escenarios de usos.

El proceso que propone la metodología se puede resumir en los siguientes pasos:

\textbf{Estudio de Factibilidad.} Esta debe ser aplicada a toda la aplicación y debe llevarse a cabo antes del desarrollo de la ontología. Aquí se identifica el problema y las áreas de oportunidades, se selecciona la mejor área de enfoque para la solución.

\textbf{Patada Inicial.} El resultado de este proceso es el Documento de Especificación de Requerimientos de la Ontología (ORSD). Se describen las metas y el dominio de la ontología, líneas de diseño (convenciones de nombres), lista de recursos disponibles (libros, revistas, documentación, etc), potenciales usuarios y casos de uso, así como las aplicaciones que utilizarán la ontología. Para esto se propone crear una lista de preguntas de competencia las cuales debe satisfacer la ontología creada. Los conceptos y relaciones más importantes son identificados en un nivel informal. También se debe de buscar ontologías para su potencial reuso, la metodología no provee un delineamiento para identificar dichas ontologías.

\textbf{Refinamiento.} Aquí se construye una ontología sólida orientada a la aplicación acorde al proceso de especificación. Este proceso se divide en dos actividades:

\begin{description}
\item[Actividad 1:] Extracción del conocimiento del los expertos del dominio. Los axiomas de la ontología son definidos y modelados por los expertos del dominio. La metodología propone el uso de una representación intermedia para modelar el conocimiento.
\item[Actividad 2:] Formalización de la ontología. La ontología es implementada en un lenguaje ontológico, el lenguaje es seleccionado según los requerimientos del uso de la ontología.
\end{description}

\textbf{Evaluación.} Este paso sirve para probar la usabilidad de la ontología desarrollada dentro del entorno para la cual fue desarrollada. En este proceso se verifica que se puedan responder las preguntas de competencia y que se satisfagan los requerimientos, además se prueba la ontología en el entorno de software para el cual se desarrolló.

\textbf{Mantenimiento.} En esta etapa es importante recalcar quién es el responsable de mantener la ontología y cómo se debe hacerlo.

La metodología propone un ciclo de vida incremental y cíclico como se muestra en la Figura \ref{img:ontoknowledge}.

\begin{figure}[h!]
    \centering
    \includegraphics[width=150mm]{figuras/Diagramas-OntoKnowledgeProcess}
    \caption{Ciclo de vida de OntoKnowledge}
    \label{img:ontoknowledge}
    \end{figure}

\subsection{Methontology}

Esta metodología incluye la identificación del proceso del desarrollo de una ontología, que consiste en una serie de actividades, el ciclo de vida basado en el refinamiento de un prototipo y técnicas para llevar a cabo cada actividad durante el mantenimiento, el desarrollo y el soporte de las actividades. Todas las actividades están basadas en el proceso de desarrollo de software y las utilizadas en metodologías de ingeniería de conocimiento.

En la Figura \ref{img:methontology} se puede ver los estados de todo el ciclo de vida de la ontología.

\begin{figure}[h!]
    \centering
    \includegraphics[width=150mm]{figuras/Diagramas-MethontologyProcess}
    \caption{Ciclo de vida de Methontology}
    \label{img:methontology}
    \end{figure}
    
    
La metodología propone un ciclo de vida basado en prototipos que permite agregar, cambiar y remover términos en cada prototipo. Para cada prototipo, se propone comenzar con la planificación para identificar las tareas, tiempos y recursos necesarios. Luego, comienza la actividad de especificación, con ella comienzan las actividades de gestión (Control y Calidad) y procesos de soporte (Adquisición de conocimiento, integración, evaluación, documentación y gestión de la configuración), todo esto ejecutado en forma paralela a las actividades de desarrollo (especificación, conceptualización, implementación y mantenimiento) durante todo el ciclo de vida de la tecnología \cite{fernandez1997methontology}.
 
A continuación se describen algunas de las principales actividades:

\begin{enumerate}
\item \textbf{Especificación:} Especificar el propósito de la ontologías, esto incluye usuarios, contexto, nivel de formalidad esperado y granularidad. El resultado de esta fase es un documento de especificación de la ontología en lenguaje natural.
\item \textbf{Adquisición} de conocimiento: Esto ocurre en paralelo a las demás fases, pero específicamente en la parte de especificación. Se consultan libros, manuales, tablas, otras ontologías y expertos para adquirir conocimientos sobre el área a modelar.
\item \textbf{Conceptualización:} Se identifican los términos según sean conceptos, instancias, relaciones o propiedades para crear un modelo conceptual del dominio.
\item \textbf{Integración:} En esta fase se considera el reuso de términos de otras ontologías para no volver a definir y ahorrar trabajo.
\item \textbf{Implementación:} La ontología es representada formalmente en un lenguaje tipo OWL o RDFS.
\item \textbf{Evaluación:} Significa hacer un juzgamiento técnico de la ontología creada, el entorno de desarrollo y la documentación de la misma. La evaluación está compuesta por la verificación y la validación.
\item \textbf{Documentación:} El cual consiste en todos los documentos producidos durante el desarrollo de la ontología que va desde el mismo código de la ontología hasta artículos científicos de la misma.
\end{enumerate}


\subsection{NeOn}
Esta metodología incluye métodos, técnicas y herramientas para llevar a cabo actividades para el proceso de desarrollo de una red ontológica. Está enfocada en la especificación de los requerimientos, la planificación y el reuso de recursos ontológicos y no-ontológicos \cite{suarez2010neon}.

La metodología presenta los siguientes componentes:
\begin{itemize}
    \item Un glosario: Identifica y define los procesos y actividades que pueden estar dentro de el desarrollo de una ontología. En la Figura \ref{img:neon actividades} se muestra la lista de procesos y actividades divididas según la fase de desarrollo.
    \item Nueve escenarios para el desarrollo de ontologías. Se identificaron nueve escenarios flexibles para el desarrollo de ontologías, donde cada escenario está compuesto de diferentes procesos y actividades. En la Figura \ref{img:neon ciclo de vida} se distinguen las relaciones entre las actividades y los escenarios donde las fechas dirigidas con círculos numerados asociados representan los diferentes escenarios. 
    \item Una serie de guías metodológicas para procesos y actividades como el reuso de recursos ontológicos y no-ontológicos, la especificación de los requerimientos, la planificación, etc. Los procesos y actividades están representados en la Figura \ref{img:neon ciclo de vida} con cajas y círculos de color.
\end{itemize}

A continuación se presenta el glosario de términos:
\begin{figure}[h!]
    \centering
    \includegraphics[width=150mm]{figuras/Diagramas-NeonActivities}
    \caption{Actividades de NeOn}
    \label{img:neon actividades}
    \end{figure}


A continuación se presentan los nueve escenarios que propone NeOn.

\begin{description}
    \item[Escenario 1] El escenario va desde la especificación hasta la implementación. La ontología es desarrollada desde el principio, esto significa sin utilizar ninguna base de conocimiento ya construida.
    \item[Escenario 2] Reuso y Reingeniería de recursos no-ontológicos. Este escenario cubre el caso donde los desarrolladores analizan recursos no-ontológicos, y deciden según los requerimientos, qué recurso no-ontológico se puede reutilizar para construir esta ontología. Este escenario también realiza una reingeniería del recurso seleccionado.
    \item[Escenario 3] Reuso de recursos ontológicos. En este escenario los desarrolladores reusan recursos ontológicos, esto se puede hacer de tres maneras: reuso total, reuso de módulos y/o reuso de sentencias.
    \item[Escenario 4] Reuso y reingeniería de recursos ontológicos.Aquí, los desarrolladores hacen tanto un reuso como una reingeniería de los recursos ontológicos.
    \item[Escenario 5] Reuso y unión de recursos ontológicos. Este escenario es propicio cuando se eligen muchas ontologías del mismo dominio de conocimiento para reutilizar y se quiere crear una nueva ontología a partir de ellas.
    \item[Escenario 6] Reuso, unión y reingeniería de recursos ontológicos. Similar al escenario 5, aquí los desarrolladores realizan un proceso de reingeniería antes de unir los recursos reutilizados.
    \item[Escenario 7] Reusos de patrones de diseño de ontologías (ODP). En este escenario se utilizan repositorios de ODP para reutilizarlos.
    \item[Escenario 8] Reestructuración de recursos ontológicos. Se reestructuran recursos ontológicos, esto significa modularizar, recortar, extender y/o especializar el recurso ontológico para integrarla a la ontología construida.
    \item[Escenario 9] Localización de un recurso ontológico. En este escenario se adapta una ontología a otros lenguajes u otras culturas o comunidades, produciendo así una ontología multilenguaje.
\end{description}

\begin{figure}[h!]
    \centering
    \includegraphics[width=150mm]{figuras/Diagramas-NeonProcess}
    \caption{Ciclo de vida de NeOn}
    \label{img:neon ciclo de vida}
    \end{figure}

El proceso de desarrollo de una ontología es definido como un proceso en el cual las necesidades de un usuario son convertidas a una ontología. Esto significa que el proceso se puede ver como un caso específico del proceso de desarrollo de software.	
El ciclo de vida es un modelo que describe cómo construir y mantener un proyecto ontológico, osea cómo ordenar los procesos y actividades en fases. Se pueden tener dos tipos de ciclo de vida que son:

\begin{itemize}
    \item Modelo de ciclo de vida de tipo cascada: En este modelo cada fase debe ser culminada antes de avanzar a la siguiente, no se permite retroceso excepto en el caso de la fase de mantenimiento. Este modelo es ideal cuando el tiempo de desarrollo es acotado y el dominio de conocimiento a modelar es bien conocido.
    \item Modelo de ciclo de vida de tipo iterativo-incremental: cada iteración es similar al modelo de cascada. Es ideal cual existe mucha incertidumbre acerca del alcance de la ontología a modelar.
\end{itemize}

A continuación exponemos los 5 modelos de ciclo de vida:	

\begin{description}
    \item[Modelo cascada de 4 fases.] Representa las etapas de una red de ontologías, comenzando con la Fase de Iniciación y yendo a través de la Fase de Diseño y Fase de Implementación hasta la Fase de Mantenimiento.
    \item[Modelo cascada de 5 fases.] Extiende el modelo de 4 fases con el reuso de recursos  ontológicos tales como son.
    \item[Modelo cascada de 5 fases + mezcla.] Es un caso especial del modelo de 5 fases. Incluye la Fase de Mezcla para obtener un nuevo recurso ontológico a partir de dos o más recursos ontológicos previamente seleccionados en la Fase de Reuso.
    \item[Modelo cascada de 6 fases.] Extiende el modelo de 5 fases con la Fase de Reingeniería. Permite la reingeniería de recursos de conocimiento (ontológicos y no-ontológicos). Puede ocurrir que algunos recursos de conocimiento son transformados en ontologías en la Fase de Reingeniería.
    \item[Modelo cascada de 6 fases + Mezcla.] Extiende el modelo de 6 fases incluyendo la Fase de Mezcla luego de la Fase de Reuso.
\end{description}

Los escenarios están ligados a un modelo dentro del ciclo de vida, como se muestra en la Tabla xx.

\begin{table}[tbp]
\centering
\caption{Escenarios vs Modelo}
\label{escenarios vs modelo}
\resizebox{15cm}{!} {
\begin{tabular}{|l|c|c|c|c|c|}
\hline
 & \multicolumn{1}{|m{2cm}|}{Modelo de 4-fases} & \multicolumn{1}{|m{2cm}|}{Modelo de 5-fases} & \multicolumn{1}{|m{2cm}|}{Modelo de 5-fases + mezcla} & \multicolumn{1}{|m{2cm}|}{Modelo de 6-fases} & \multicolumn{1}{|m{2cm}|}{Modelo de 6-fases + mezcla} \\ \hline
\end{tabular}
}
\end{table}



\section{Herramientas tecnológicas}
\subsection{Protégé}
Protégé es una herramienta de código libre para desarrolladores de ontologías y permite crear sistemas de base de conocimientos. El editor consiste en una interfaz de usuario gráfica donde podemos crear y editar clases, instancias, propiedades y las restricciones. Nos permite guardar la ontología en varios formatos como XML, RDF, TTL y OWL. En la Figura \ref{img:protege} se muestra una captura de pantalla de la herramienta

\begin{figure}[h!]
    \centering
    \includegraphics[width=150mm]{figuras/protege}
    \caption{Herramienta de Desarrollo de Ontologías: Protégé}
    \label{img:protege}
    \end{figure}

\subsection{Visual Notation for OWL}
Además se utilizó una herramienta de visualización gráfica de ontologías llamada Visual Notation for OWL (VOWL) en su versión Web para visualizar de forma gráfica las ontologías estudiadas, en la Figura \ref{img:webvowl} se presenta una captura de pantalla de la herramienta utilizando como ejemplo una ontología llamada FOAF.

\begin{figure}[h!]
    \centering
    \includegraphics[width=150mm]{figuras/webvowl}
    \caption{Herramienta de Visualización de Ontologías: WebVOWL}
    \label{img:webvowl}
    \end{figure}
    

\subsection{Apache Jena}
Jena es un Framework implementado en JAVA para construir aplicaciones de la Web Semántica. Provee una librería para ayudar a desarrolladores a manejar RDF, RDFS, RDFa, OWL y SPARQL y está alineada con las recomendaciones de la W3C. Jena incluye un motor de inferencias basada en reglas para realizar los razonamientos basados en OWL y RDFS, además de una variedad de estrategias para el almacenamiento de triplas RDF en memoria y en disco.

\subsection{Consulta de datos con SPARQL}
Para consultar triplas basadas en RDF han surgido varios lenguajes de consulta, pero SPARQL es el más ampliamente utilizado y fue estandarizado por la W3C. Una consulta SPARQL es formulada usando patrones de grafos que pueden ser combinados con operaciones algebraicas como unión, opcional, filtro, etc. El resultado de la consulta es una lista de relaciones que pueden estar expresadas en tablas o en RDF.

A forma de explicar la semántica y sintaxis completa de este lenguaje en el Cuadro \ref{lst:consulta-sparql} se muestra en un ejemplo la utilización del lenguaje realizando una consulta básica a la base de datos de DBPEDIA \cite{DBpedia:online} donde se listaron los nombres, fechas de nacimiento, fallecimiento y las URIs de personas nacidas en Paraguay antes de 1900, ordenadas por nombre.

\lstset{
    language=SPARQL
}
\begin{lstlisting}[captionpos=b, caption=Ejemplo de consulta SPARQL, label=lst:consulta-sparql,  numbers=left,  numberstyle=\tiny\color{mygray},
    basicstyle=\tiny,frame=single]
PREFIX dbo: <http://dbpedia.org/ontology/>
PREFIX xsd: <http://www.w3.org/2001/XMLSchema#>
PREFIX foaf: <http://xmlns.com/foaf/0.1/>
PREFIX : <http://dbpedia.org/resource/>

SELECT ?name ?birth ?death ?person WHERE { 

?person dbo:birthPlace :Paraguay . 
?person dbo:birthDate ?birth . 
?person foaf:name ?name . 
?person dbo:deathDate ?death . 
FILTER (?birth < "1900-01-01"^^xsd:date) . 

}
ORDER BY ?name
 
\end{lstlisting}

En las primeras líneas se pueden ver los prefijos de las ontologías utilizadas en la consulta, luego se presentan las triplas que componen las restricciones de la consulta. Está fuera del alcance de este documento la explicación detallada de la sintaxis de SPARQL.

Apache Jena Fuseki es un servidor o punto SPARQL que además provee un motor de inferencias de datos. Corre en un sistema operativo como una aplicacion web Java (archivo WAR) asi como tambien en un servidor. Provee el protocolo de consulta y actualización SPARQL 1.1 así como también el protocolo de SPARQL Graph Store.

\section{Discusión del capítulo}
En este capítulo se vieron los conceptos sobre Datos Abiertos, Linked Open Data, Web Semántica y el rol de las ontologías en la misma. Además se dio una introducción a RDF, RDFS y OWL que son formas de representación de ontologías y datos en la Web Semántica. Para poder realizar consultas a las base de datos en RDF se dio una breve introducción del lenguaje SPARQL.

Se revisaron las metodologías más utilizadas para el desarrollo de una ontología para luego hacer una comparación de cada una de ellas presentadas en la Tabla \ref{tab:comparacion}. En cuanto a la madurez y completitud de cada una podemos decir que la metodología NeOn es la que mejor se posiciona, ya que la misma está basada en metodologías de desarrollo de software y metodologías de ingeniería de conocimiento. Además contempla escenarios de reuso de recursos ontológicos y no-ontológicos, los cuales serán necesarios para el desarrollo de este trabajo. Por este motivo se eligió NeOn como metodología base para el desarrollo de la ontología.

Además presentamos las herramientas tecnológicas utilizadas para el desarrollo de ontologías y posterior consulta a través de un servidor de consultas llamado Punto SPARQL.

En el siguiente capítulo se hablará acerca de los recursos necesarios y los pasos seguidos para llegar al objetivo propuesto del trabajo.


%!TEX root = ../main.tex
\chapter{Materiales y Métodos}
\label{chap:MaterialesYMetodos}

En este capítulo se mostrarán los recursos necesarios y los pasos seguidos para llegar a los objetivos propuestos en este trabajo. Se describen a lo largo de este capítulo; La principal fuente de datos utilizada (DNCP \cite{DatosAbiDNCP:online}, ver sección \ref{section:portalDNCP}) y el principal recurso no ontológico utilizado (OCDS, ver sección \ref{section:OCDS}) para el desarrollo de la ontología. Así también una breve reseña del proceso de desarrollo de la ontología (ver sección \ref{section:desarrolloOntologia}), luego explicamos el proceso de despliegue y análisis de los resultados (ver sección \ref{section:despliegue}).

\section{Portal de Datos Abiertos de la DNCP}
\label{section:portalDNCP}

En Paraguay, la DNCP \cite{DatosAbiDNCP:online}, organismo del estado que publica datos de los procesos de contrataciones, posee un portal de datos abiertos donde publica datos en tiempo real de los estados de las licitaciones públicas del país. 

Esta plataforma es fundamental para este y otros trabajos de investigación basados en datos de contrataciones ya que posee los datos de prueba y el escenario de caso de uso. En el proceso se realizaron reuniones e intercambios de correos electrónicos tanto con el equipo técnico encargado de mantener el portal, como el equipo legal de la dirección a fin de conocer el contexto, las funcionalidad, las limitantes y las necesidades de la institución. De no tener este contacto directo, el objetivo de este trabajo pudo haber sido comprometido.

La plataforma posee las siguientes características y funcionalidades principales

\begin{enumerate}
    \item Lista de datos. Con el fin de que los usuarios que necesitan ver, analizar y descargar en formato CSV los datos detallados de todos los registros de contrataciones.
    \item Visualizaciones de datos para usuarios que deseen ver estadísticas e información agregada.
    \item Una API para desarrolladores que permite manipular datos en formato JSON y JSON-LD de manera programática para cualquier otro uso.
    \item Plataforma de Contrataciones Electrónica, destinada a empresas y personas que deseen presentarse o conocer algún proceso de contratación.
\end{enumerate}

Los datos publicados poseen un diccionario y una ontología creada por la DNCP, la cual sirve como contexto para la API desarrollada. Además la DNCP implementó una API siguiendo el estándar OCDS, que posee un esquema de publicación bien definido en JSON Schema.

A continuación hablaremos del estándar utilizado como recurso no ontológico para el desarrollo de la ontología.

\section{Open Contracting Data Standard}
\label{section:OCDS}

El \textit{Open Contracting Data Standard} (OCDS)\footnote{http://standard.open-contracting.org}  es un estándar amigable y flexible para estructurar información de Contrataciones Públicas y es mantenido por la OCP. El estándar describe qué, cuándo y cómo disponibilizar datos y documentos asociados en las diferentes fases del proceso de contratación. El proyecto promueve la divulgación y participación en las contrataciones públicas creando un estándar abierto de datos simple. OCDS posee un esquema de datos detallado de todos los conceptos así como también la estructura de los datos divulgados, dicho esquema esta disponible el sitio web del estándar \cite{OCDSReleaseSchema:online}. Este esquema ayuda a las personas a comprender todos los campos publicados. Además el estándar posee un guía de implementación a modo de facilitar la implementación\footnote{http://standard.open-contracting.org/latest/en/implementation/}.

El estándar ha ganado especial popularidad en los últimos años y fue implementado por mas de 15 países \footnote{https://www.open-contracting.org/resources/es-version-1-1-resumen-sobre-actualizacion-del-estandar/} países hasta el año 2017, donde Paraguay es un de ellos. Es por eso que el mismo ha sido elegido como base de conocimiento principal de dominio de contrataciones públicas.

\section{Planteamiento del problema}
\label{section:planteamientoDelProblema}

A partir del escenario visto en el portal de Datos Abiertos de la DNCP y la implementación del OCDS en este portal, nace la pregunta de qué manera se puede lograr una mejor interoperabilidad de datos de distintas fuentes a fin de obtener, de manera automática, información valiosa para lograr mayor transparencia y mejor toma de decisiones. Así como lo mostrado en el capitulo anterior, esto se puede lograr aplicando conceptos de la web semántica y desarrollando una ontología que permita lograr esta interoperabilidad semántica. 

\section{Desarrollo de la ontología}
\label{section:desarrolloOntologia}

Parte fundamental de este trabajo consiste en el desarrollo de una ontología basada en el estándar de publicación de datos OCDS. Al estándar, como veremos mas adelante en este trabajo, se lo clasificó como un recurso no ontológico, ya que formaliza la sintaxis pero describe en lenguaje natural y no en lenguaje formal, la semántica del dominio de conocimiento. 

Se utilizó como guía metodológica de desarrollo de ontología principal NeOn, ya que la misma, según lo que se muestra en la Tabla  \ref{tab:comparacion}, posee mayor especificación en las actividades a desarrollar  y contempla los casos de reuso de recursos ontológicos y no-ontológicos. 

Así como lo anticipa la metodología, el desarrollo de una ontología no es un proceso lineal, a lo largo de todo este trabajo, la misma es alterada y mejorada constantemente para lograr los objetivos propuestos. En el Capítulo \ref{chap:Desarrollo de la Ontologia} se describirá en detalle todo el proceso para llegar al producto final que representa la mayor contribución de este trabajo.

\section{Despliegue y Análisis de Resultados}
\label{section:despliegue}

Para llevar a cabo este trabajo se desarrolló un software de extracción de datos con los cuales se logró demostrar el funcionamiento de la ontología desarrollada, la cual se describirá en el Capítulo \ref{chap:Implementación de la Ontologia}. El proceso consistió en el desarrollo de un software que se encargó de extraer los datos (a ser utilizados en las pruebas) de la plataforma de datos abiertos de la DNCP por medio de la API para desarrolladores. Este software realiza también un proceso de conversión de datos y posterior enlace con la ontología desarrollada. Finalmente, los convierte en formato RDF para luego ser disponibilizados en un Punto SPARQL implementado en un servidor Apache Jena Fuseki. De esta manera fue posible realizar consultas utilizando los datos mencionados y datos de otras fuentes externas.

Con los datos disponibles para la consulta, describimos 5 casos de uso de los datos enriquecidos con la ontología desarrollada que nos ayudan a mostrar que se pueden lograr los objetivos propuestos de este trabajo. Los 5 casos de uso están descriptos en el Capítulo \ref{chap:Contexto experimental}.

%
\section{Especificación de requerimientos de la ontología}

En esta actividad se define el alcance y los requerimientos de la ontología a desarrollar. Como resultado de esta actividad se obtiene el Documento de Especificación de Requerimientos de la Ontología (ORSD). A continuación se detalla el resultado de dicho proceso.
\begin{enumerate}
\item \textbf{Propósito:} Crear una ontología para el dominio de Contrataciones Públicas para lograr la interoperabilidad semántica con otras fuentes de datos externas utilizando el OCDS como base de conocimiento principal aumentando la formalidad semántica como se muestra en la figura \ref{img:ocds-ocntology-complejidad}

\begin{figure}[h!]
\centering
\includegraphics[width=150mm]{figuras/Diagramas-OenContracting.png}
\caption{Nivel de complejidad semántica de la ontología a desarrollar}
\label{img:ocds-ocntology-complejidad}
\end{figure}
\item \textbf{Alcance:} El alcance de la ontología está delimitada por el vocabulario detallado en la versión 1 del OCDS. Se eligió dicha versión ya en el momento del inicio de esta investigación era la última versión estable del estándar.
\item \textbf{Lenguaje de Implementación.} Se utilizará el lenguaje OWL debido a que es el lenguaje preferido para ontologías en la web semántica recomendado por la W3C \cite{OWLSeman72:online}, y se requiere de una ontología lo suficientemente ligera y explícita que se pueda manejar en la web semántica.
\item \textbf{Grupo Objetivo.}La ontología está orientada a:
\begin{enumerate}
    \item Expertos del dominio de Contrataciones Públicas que quieran realizar consultas ad-hoc sobre datos.
    \item Desarrolladores de software que deseen implementar el OCDS.
    \item Desarrolladores de software que necesiten integrar datos de contrataciones públicas con otras fuentes externas. \end{enumerate}
\item \textbf{Usos de la Ontología. }La ontología se utilizará para crear un esquema de publicación en JSON-LD, esto es debido a que la sintaxis JSON es ampliamente conocida y preferida por los desarrolladores \cite{JSON-37:online}, además el OCDS ya posee un esquema de publicación compatible con esta sintaxis. Los datos utilizados se extraerán del portal de datos abiertos de la DNCP.
\item \textbf{Requerimientos No Funcionales: }
\begin{enumerate}
    \item Se optará, en lo posible, por la reutilización de otras ontologías del dominio de Contrataciones Públicas ampliamente utilizadas.
    \item La ontología desarrollada debe ser procesable dentro de las limitaciones de la web semántica, ósea debe ser una ontología ligera.
    \item Debe soportar múltiples lenguajes: inglés y español inicialmente.
\end{enumerate}
\item \textbf{Requerimientos Funcionales. }
    \begin{enumerate}
        \item Gracias a la ontología se podrán responder las mismas preguntas que se responden a través de los datos publicados en formato de JSON y se podrán responder preguntas de todas las fases del proceso licitatorio.
        \item Debe ser compatible con la versión 1 del OCDS.
        \item Los datos deberán poder ser enriquecidos con otras fuentes de datos provenientes de la DNCP y también fuentes externas como Wikidata o DBpedia.
    \end{enumerate}
\item \textbf{Pre-Glosario de Términos.} El glosario fue extraído del diccionario de datos y de la ontología desarrollada por la DNCP. 
\end{enumerate}



\section{Discusión del Capitulo}

En este capítulo se mostraron los recursos necesarios y los pasos seguidos para la llegar a los objetivos propuestos en este trabajo. Se detalló la importancia de la fuente de datos utilizada, el estándar de datos utilizado como base de conocimiento principal y la herramienta metodológica para el desarrollo de la ontología. Además de la manera en que se pretende mostrar la forma de uso del producto desarrollado.

En el siguiente capítulo comenzaremos con el desarrollo de la ontología utilizando la metodología NeOn.





%!TEX root = ../main.tex
\chapter{Desarrollo de la Ontología}
\label{chap:Desarrollo de la Ontologia}

Luego de haber estudiado las distintas metodologías para el desarrollo de ontologías se decidió utilizar la metodología NeOn ya que la misma, según lo que se muestra en la Tabla  \ref{tab:comparacion_ontologias}, posee mayor especificación en las actividades a desarrollar  y contempla los caso de reuso de recursos ontológicos y no-ontológicos. 

Siguiendo la metodología NeOn se identifica que la ontología a desarrollar sigue las actividades compuestas principalmente por los escenarios 1, 2 y 3 que se exponen en la metodología. El escenario 1, puesto que se quiere modelar una ontología desde el principio y es la base de los demás escenarios, el escenario 2 y 3 ya que se pretende reusar recursos ontológicos y no-ontológicos respectivamente. Los escenario elegidos sirven como guía para el desarrollo de la ontología. A continuación se detalla la primera actividad de la metodología, que consiste en la especificación de los requerimiento de la ontología.


\section{Especificación de requerimientos de la ontología}

En esta actividad se define el alcance y los requerimientos de la ontología a desarrollar. Como resultado de esta actividad se obtiene el Documento de Especificación de Requerimientos de la Ontología (ORSD). A continuación se detalla el resultado de dicho proceso.
\begin{enumerate}
\item \textbf{Propósito:} Crear una ontología para el dominio de Contrataciones Públicas para lograr la interoperabilidad semántica con otras fuentes de datos externas utilizando el OCDS como base de conocimiento principal aumentando la formalidad semántica como se muestra en la figura \ref{img:ocds-ocntology-complejidad}

\begin{figure}[h!]
\centering
\includegraphics[width=150mm]{figuras/Diagramas-OenContracting.png}
\caption{Nivel de complejidad semántica de la ontología a desarrollar}
\label{img:ocds-ocntology-complejidad}
\end{figure}
\item \textbf{Alcance:} El alcance de la ontología está delimitada por el vocabulario detallado en la versión 1 del OCDS. Se eligió dicha versión ya en el momento del inicio de esta investigación era la última versión estable del estándar.
\item \textbf{Lenguaje de Implementación.} Se utilizará el lenguaje OWL debido a que es el lenguaje preferido para ontologías en la web semántica recomendado por la W3C \cite{OWLSeman72:online}, y se requiere de una ontología lo suficientemente ligera y explícita que se pueda manejar en la web semántica.
\item \textbf{Grupo Objetivo.}La ontología está orientada a:
\begin{enumerate}
    \item Expertos del dominio de Contrataciones Públicas que quieran realizar consultas ad-hoc sobre datos.
    \item Desarrolladores de software que deseen implementar el OCDS.
    \item Desarrolladores de software que necesiten integrar datos de contrataciones públicas con otras fuentes externas. \end{enumerate}
\item \textbf{Usos de la Ontología. }La ontología se utilizará para crear un esquema de publicación en JSON-LD, esto es debido a que la sintaxis JSON es ampliamente conocida y preferida por los desarrolladores \cite{JSON-37:online}, además el OCDS ya posee un esquema de publicación compatible con esta sintaxis. Los datos utilizados se extraerán del portal de datos abiertos de la DNCP.
\item \textbf{Requerimientos No Funcionales: }
\begin{enumerate}
    \item Se optará, en lo posible, por la reutilización de otras ontologías del dominio de Contrataciones Públicas ampliamente utilizadas.
    \item La ontología desarrollada debe ser procesable dentro de las limitaciones de la web semántica, ósea debe ser una ontología ligera.
    \item Debe soportar múltiples lenguajes: inglés y español inicialmente.
\end{enumerate}
\item \textbf{Requerimientos Funcionales. }
    \begin{enumerate}
        \item Gracias a la ontología se podrán responder las mismas preguntas que se responden a través de los datos publicados en formato de JSON y se podrán responder preguntas de todas las fases del proceso licitatorio.
        \item Debe ser compatible con la versión 1 del OCDS.
        \item Los datos deberán poder ser enriquecidos con otras fuentes de datos provenientes de la DNCP y también fuentes externas como Wikidata o DBpedia.
    \end{enumerate}
\item \textbf{Pre-Glosario de Términos.} El glosario fue extraído del diccionario de datos y de la ontología desarrollada por la DNCP. 
\end{enumerate}


\section{Planificación de las Actividades}

Aquí se define el modelo de ciclo de vida de la ontología. El ciclo de vida que se adapta a los escenarios elegidos es el Modelo de Cascada de 6 fases, debido a que serán necesarias las fases de reuso tanto de recursos ontológicos como no-ontológicos, reingeniería, y diseño para la implementación. En la Figura \ref{img:secuenciaDeDesarrollo} se muestra el modelo de ciclo de vida a seguir. Se decidió utilizar el Modelo de Ciclo de Vida Cascada ya que el alcance de la ontología es bien delimitado y conocido.

\begin{figure}[ht!]
    \centering
    \includegraphics[width=150mm]{figuras/Diagramas-GraficodeSecuenciasDesarrollo.png}
    \caption{Modelo de Ciclo de Vida de desarrollo de la ontología.}
    \label{img:secuenciaDeDesarrollo}
\end{figure}

A continuación se citan las actividades divididas en las fases del ciclo de vida a desarrollarse según la metodología NeOn:
\begin{enumerate}
\item Fase 1: Iniciación
    \begin{enumerate}
    \item Especificación de Requerimientos Ontológicos (ORSD)
    \item Planificación 
    \end{enumerate}
\item Fase 2: Reuso
    \begin{enumerate}
    \item Reuso de Recursos No-Ontológicos
    \item Reuso de Recursos Ontológicos
    \end{enumerate}
\item Fase 3: Reingeniería
\begin{enumerate}
\item Reingeniería de Recursos No-Ontológicos
\end{enumerate}
\item Fase 4: Diseño
\begin{enumerate}
\item Conceptualización de la Ontología
\end{enumerate}
\item Fase 5: Implementación
\item Fase 6: Mantenimiento

\end{enumerate}
Una vez terminada la planificación se procede al desarrollo de cada una de las demás actividades siguiendo el orden de las fases. La explicación de cada una de las actividades se describen a continuación en este documento.


\section{Reuso de recursos no-ontológicos}
Los recursos no-ontológicos (NOR)\cite{ReusoRecursoNoOntologico} representan recursos de conocimiento cuya semántica no fue formalizada por una ontología. Estos NORs contienen conocimientos de un dominio en particular y representan algún grado de consenso colectivo. Estos recursos están presentes en forma de esquemas de clasificación, tesauros, diccionarios, etc. El principal desafío de los NORs es que la semántica no siempre está formalizada, osea legible por máquina, por lo tanto no son considerados ontologías formales según la definición adoptada.

En esta sección se desarrolla la búsqueda, evaluación y selección de recursos no-ontológicos que servirán como insumo para el desarrollo de la ontología.

\subsection{Búsqueda de recursos no-ontológicos }

En el proceso de búsqueda de documentación y estándares del dominio de Contrataciones Públicas se encontró que la DNCP implementa una API utilizando el estándar de publicación de datos de contrataciones públicas internacional \textit{Open Contracting Data Standard} (OCDS). La intención de la OCP fue crear un estándar que formaliza la sintaxis y no la semántica de cada concepto, a través de JSON Schema\cite{JSONSche10:online}, que es un vocabulario que permite anotar y validar documentos JSON. El estándar posee conceptos explícitamente definidos en lenguaje natural, por lo que la semántica esta definida pero no formalizada como ontología, por ello se lo consideró un recurso no-ontológico. Cabe recordar que cuando se habla de formal, se habla de legible por máquina. 

OCDS es un estándar amigable y flexible para estructurar información de Contrataciones Públicas y es mantenido por la OCP. El estándar describe qué, cuándo y cómo disponibilizar datos y documentos asociados en las diferentes fases del proceso de contratación. El proyecto promueve la divulgación y participación en las contrataciones públicas creando un estándar abierto de datos simple. OCDS posee un esquema de datos detallado de todos los conceptos así como también la estructura de los datos divulgados, dicho esquema esta disponible el sitio web del estándar \cite{OCDSReleaseSchema:online}. Este esquema ayuda a las personas a comprender todos los campos publicados. Además el estándar posee un guía de implementación de modo a facilitar la implementación\footnote{http://standard.open-contracting.org/latest/en/implementation/}.

En Paraguay,  el portal de datos abiertos de la DNCP \cite{DatosAbiDNCP:online}, organismo del estado que publica datos de los procesos de contrataciones , ha implementado el estándar y publica sus datos orientados a diferentes tipos de usuarios:


\begin{enumerate}
    \item Lista de datos. Con el fin de que los usuarios que necesitan ver, analizar y descargar en formato CSV los datos detallados de todos los registros de contrataciones.
    \item Visualizaciones de datos para usuarios que deseen ver estadísticas e información agregada.
    \item Una API para desarrolladores que permite manipular datos en formato JSON y JSON-LD de manera programática para cualquier otro uso.
    \item Plataforma de Contrataciones Electrónica, destinada a empresas y personas que deseen presentarse o conocer algún proceso de contratación.
\end{enumerate}

Los datos publicados poseen un diccionario de datos y una ontología creada por la DNCP, la cual sirve como contexto para la API desarrollada.Esto significa que la DNCP ya realizo un primer trabajo publicando sus datos en formato JSON-LD. Además la DNCP implementó una API siguiendo el estándar OCDS, que posee un esquema de publicación bien definido en JSON Schema.


\subsection{Evaluación de recursos no-ontológicos}
Se eligió el OCDS como recurso no-ontológico a evaluar puesto que el principal objetivo de este trabajo es que la ontología desarrollada sea compatible con dicho estándar de modelo de datos. Para realizar la evaluación se procedió a la extracción de las entradas léxicas, luego al cálculo de precisión y alcance para por último evaluar la factibilidad de reuso del recurso.  A continuación se explica en detalle cada paso.

\subsubsection{Extracción de entradas léxicas}
La meta de esta tarea es extraer las entradas léxicas de los recursos no-ontológicos. Para realizarla, es necesario tomar como entrada los recursos no-ontológicos y extraer sus entradas léxicas usando herramientas de extracción de terminología.	
			
Del esquema JSON de la versión 1 del OCDS se hizo una lista de todas las propiedades del mismo. Para este trabajo consideramos que cada propiedad del esquema consiste en una entrada léxica, donde una entrada léxica corresponde a un concepto o relación dentro del dominio de conocimiento. Asi como nuestra nuestro trabajo \footnote{http://bit.ly/ValdezBaezThesis}, se encontraron 142 propiedades, omitiendo las propiedades repetidas y datos transaccionales del estándar que no representan entradas léxicas, la lista se reduce a 114.

\subsubsection{Cálculo de precisión y alcance}

El objetivo de esta tarea es calcular la precisión de los recursos no-ontológicos candidatos. La precisión es una medida ampliamente utilizada en la recuperación de información y se define como la proporción del material recuperado que realmente es relevante. Esta tarea es llevada a cabo por los desarrolladores de software y los que utilicen la ontología teniendo como entrada las entradas léxicas extraídas de los recursos no-ontológicos y de la terminología reunida del ORSD. Para esto definimos los siguientes términos.

\begin{itemize}
    \item \textit{NOR Lexical Entries}  como el conjunto de entradas léxicas extraídas del recurso no ontológico.	
    \item \textit{ORSD Terminology} como el conjunto de términos identificados incluidos en el ORSD. 
\end{itemize}

Para obtener el número de términos de la especificación de requisitos de ontología (ORSDTerminology) se creó una lista unificada de todas las propiedades encontradas en el diccionario de datos del portal de datos abiertos de la DNCP.  Se encontraron 129 propiedades, de la misma manera que en el OCDS se omitieron propiedades repetidas y datos transaccionales, dejándonos con un total de 109 entradas léxicas. Éstas representan el dominio de conocimiento que se quiere representar con la ontología a construir.

\begin{table}[!htb]
\centering
\caption{Resumen de propiedades y entradas léxicas}
\label{resumen propiedades}
\begin{tabular}{|l|r|r|}
\hline
 & \textbf{Propiedades} & \textbf{Entradas léxicas}\\ \hline
 \textbf{OCDS} & 142 & 114\\ \hline
 \textbf{DNCP} & 129 & 109 \\ \hline
\end{tabular}
\end{table}

Para calcular la cobertura y la precisión del recurso no-ontológico se realizó la unión de términos equivalentes semánticamente. Dicha unión consiste en verificar términos equivalentes entre las entradas léxicas que estaban presentes en los requerimientos y en el recurso no-ontológico analizado. Cabe destacar que la metodología utilizada sólo tomó en cuenta las propiedades del esquema JSON del OCDS y el diccionario de datos de la API desarrollada por la DNCP, esto limita la cobertura, que podría ser más amplia si consideramos todos los términos del dominio de conocimiento. El resultado arrojó que existen 65 entradas léxicas comunes. Con estos datos se puede calcular la cobertura y la precisión a través de las  fórmulas \ref{eq:1} y \ref{eq:2}.

\begin{equation}
    \label{eq:1}
    Precision =  \frac{{NORLexicalEntries}\cap{ORDSTerminology} }{{NORLexicalEntries}}
\end{equation}

\begin{equation}
    \label{eq:3}
    Precision =  \frac{65}{114} = 0.5701754386
\end{equation}

\begin{equation}
    \label{eq:2}
     Coverage = \frac{{NORLexicalEntries}\cap{ORDSTerminology}}{{ORDSTerminology}}
\end{equation}

\begin{equation}
    \label{eq:4}
    Coverage =  \frac{65}{109} = 0.5963302752
\end{equation}

Esto da una cobertura de 0.59 y una precisión de 0.57.


\subsection{Selección del recurso no-ontológico}

Como se puede ver en el esquema del OCDS\footnote{http://standard.open-contracting.org/latest/en/schema/release/}, el mismo posee una especificación formal de la sintaxis de cada una de las propiedades, ya que por medio de un validador construido por la OCP\footnote{http://standard.open-contracting.org/validator/} podemos verificar si el objeto validado cumple con los requerimientos sintácticos de publicación del estándar.

También podemos observar que el estándar contiene explícitamente las descripciones de cada una de las propiedades del esquema JSON, por lo que posee información valiosa y consensuada de cada uno de los conceptos del dominio. Sin embargo, dicha información solamente es entendible por personas, no así procesable por máquinas, por lo que representa un recurso no-ontológico vital para el desarrollo de nuestra ontología.

Como se puede ver en el esquema del OCDS, el mismo posee una especificación formal de la sintaxis de cada una de las propiedades, ya que por medio de un validador construido por la OCP podemos verificar si el objeto validado cumple con los requerimientos sintácticos de publicación del estándar.

Las principales diferencias de cobertura entre los conceptos de la DNCP y la OCDS son que el segundo posee mayor información como es el caso de Periodo, Organización y Detalle de Contacto. También contempla los conceptos de Documentos, Transacciones, Hitos, una fase adicional de Implementación del Contrato y guarda el historial de cambios de todas las propiedades a través de las Adendas. Así también, el dominio de la DNCP agrega conceptos como Modificaciones de Contrato e información más detallada de Proveedores y Convocatorias. En la Figura \ref{img:coberturaontologia} se aprecia la intersección de los principales conceptos entre ambos dominios de conocimientos.

\begin{figure}[ht!]
    \centering
    \includegraphics[width=150mm]{figuras/Diagramas-VennCobertura.png}
    \caption{Cobertura de los terminos de la DNCP y la OCDS}
    \label{img:coberturaontologia}
\end{figure}

    

El OCDS es implementado a nivel internacional, tiene una comunidad de colaboradores expertos en el dominio de distintos países, implementadores que dan al estándar una validez en cuanto a la aceptación de los términos, posee listas de correos y un sistema de gestión de cambios y mejoras activo en la plataforma Github\footnote{https://github.com/open-contracting/standard}. La intención de este trabajo es que la ontología sea utilizada no solamente por la DNCP sino por cualquier ente que realice procesos licitatorios, además, la DNCP está utilizando este estándar para publicación de datos orientados a desarrolladores. Con todo esto se aprecia que la cobertura es suficiente para utilizar el OCDS como recurso no-ontológico para nuestra ontología sacrificando así algunos detalles específicos de implementación local. Estos detalles más específicos se podrían cubrir a través del reuso y extensión de la ontología desarrollada en un futuro trabajo.

\section{Reingeniería de recurso no-ontológicos}
Debido a que se utilizó recursos no-ontológicos, debemos de realizar un proceso de reingeniería para transformar estos recursos de manera a incluir esa información en la ontología. Procedemos de esta manera al proceso de reingeniería del esquema de publicación de OCDS.


\subsection{Ingeniería reversa del recurso no-ontológico}
El primer paso es analizar el recurso no-ontológico para identificar los principales componentes y crear representaciones del recurso. Dentro de esta actividad procederemos a la recolección de los datos, la abstracción conceptual y exploración de la información.

\subsubsection{Recolección de datos}

El propósito de esta actividad es buscar y recolectar todos los datos y la documentación del recurso no-ontológico incluyendo su propósito, componentes, modelo de datos y detalles de implementación.

El OCDS es un estándar abierto de datos para la publicación de información estructurada sobre todas las etapas de un proceso de contratación: desde la planificación hasta la implementación.

La publicación de los datos siguiendo el OCDS permite una mayor transparencia en la contratación pública y puede apoyar al análisis accesible y a profundidad de la eficiencia, efectividad, equidad e integridad de los sistemas de contratación pública. El OCDS fue diseñado con un enfoque en la contratación pública de bienes, obras y servicios, pero puede ampliarse para su uso en otros contextos. 

El OCDS se centra en las siguientes necesidades del usuario:
\begin{itemize}
    \item Fortalecimiento de la transparencia, rendición de cuentas e integridad de los contratos públicos.
    \item Hacer un buen uso del dinero del gobierno.
    \item Permitir al sector privado una competencia justa por contratos públicos.
    \item Control de la eficacia de la prestación de servicios. 
\end{itemize}

El OCDS posee un sitio web donde se encuentra documentado todo el estándar, soportando los idiomas español, inglés y francés. Además posee una guía de implementación del estándar, un validador de la estructura sintáctica del mismo y un mecanismo de gestión de extensiones.

\subsubsection{Modelo conceptual}

En esta actividad se identifica el esquema de recursos así como también los componentes conceptuales y sus relaciones. Además se crea un modelo conceptual dividido por bloques de conceptos relacionados. El conocimiento se expresa mediante representaciones primitivas de conceptos y relaciones entre conceptos. A continuación se representa la conceptualización teniendo como base el OCDS.

Un proceso de contratación se define como toda la información relativa a la planificación, la licitación, las adjudicaciones, los contratos y la ejecución de contratos relacionados con un solo proceso de iniciación.

Un proceso de contratación agrupa información sobre diferentes etapas o fases relacionadas a la vida útil de un contrato, a partir de la planificación, progresando a través de las etapas de iniciación u oferta, luego por la adjudicación y contrato y finalizando con la implementación del producto o servicio como se muestra en la Figura \ref{img:fases del proceso licitatorio }.

\begin{figure}[!htbp]
    \centering
    \includegraphics[width=150mm]{figuras/Diagramas_ProcesoLicitatorio.png}
    \caption{Modelo de fases del Proceso Licitatorio}
    \label{img:fases del proceso licitatorio }
\end{figure}

Las etapas son correlativas, esto significa que necesariamente debe de existir la etapa anterior antes de existir la siguiente. Todas las etapas representan un bloque de información.

Además de estas etapas existe otro concepto importante que es el bloque de Organización, que puede corresponder tanto al comprador como al oferente. Esta Organización puede estar ligada a cualquiera de las etapas del proceso de contratación. A continuación se detallan cada una de las etapas.
\paragraph{Planificación (\textit{Planning})}\hfill \break
Este bloque contiene información necesaria para describir los antecedentes de un proceso de contratación y puede incluir detalles del presupuesto del que se extraen los fondos o proyectos relacionados. Todo proceso de contratación posee una única planificación, dicha planificación tiene un presupuesto (budget) asociado donde finalmente se indica el valor o monto estimado para la adquisición del bien o servicio. Un diagrama de los componentes principales de la clase Planificación puede verse en la Figura \ref{img:Fase de Planificacion}.

\begin{figure}[htbp!]
    \centering
    \includegraphics[width=150mm]{figuras/Diagramas_Planificacion.png}
    \caption{Modelo de fase de Planificación}
    \label{img:Fase de Planificacion}
\end{figure}

\paragraph{Convocatoria (\textit{Tender})}\hfill \break
Este bloque incluye detalles del anuncio indicando que una organización tiene la intención de abastecerse de algunos bienes o servicios y establecer uno o más contratos para estos. Todo proceso de contratación posee una única convocatoria, dicha convocatoria tiene asociada la organización involucrada en el proceso. Además, la convocatoria indica los ítems requeridos, el período establecido, los documentos utilizados, las adendas que pudieran haberse realizado sobre la convocatoria original y la lista de hitos. Se observa el esquema en la Figura \ref{img:Fase de Convocatoria}

\begin{figure}[htbp!]
    \centering
    \includegraphics[width=150mm]{figuras/Diagramas_Convocatoria.png}
    \caption{Modelo de fase de Convocatoria}
    \label{img:Fase de Convocatoria}
\end{figure}

\paragraph{Adjudicación (\textit{Award})}\hfill \break
Este bloque se utiliza para anunciar las adjudicaciones emitidas para una licitación. Todo proceso de contratación puede involucrar una o más adjudicaciones, dichas adjudicaciones indican, para cada proveedor (organización), los ítems adjudicados, el valor o monto adjudicado y los documentos asociados al proceso. Esto se aprecia en la Figura \ref{img:Fase de Adjudiacion}


\begin{figure}[htbp!]
    \centering
    \includegraphics[width=150mm]{figuras/Diagramas_Adjudicacion.png}.
    \caption{Modelo de fase de Adjudicación}
    \label{img:Fase de Adjudiacion}
\end{figure}


\paragraph{Contrato (\textit{Contract})}\hfill \break
Este bloque se utiliza para proporcionar detalles de los contratos que se han celebrado. Todo proceso de contratación puede involucrar uno o más contratos, cada uno de los cuales debe estar asociado a una adjudicación específica. El proveedor (organización) adjudicado está indicado en la adjudicación, no así en el contrato. El contrato también indica los ítems adjudicados, los documentos asociados al proceso y la implementación del contrato. La implementación del contrato contiene información acerca de los hitos, las transacciones realizadas y los documentos utilizados. En la Figura \ref{img:Fase de Contrato} vemos un esquema de la estructura del Contrato.

\begin{figure}[htbp!]
    \centering
    \includegraphics[width=150mm]{figuras/Diagramas_Contrato.png}
    \caption{Modelo de fase de Contrato}
    \label{img:Fase de Contrato}
\end{figure}

\paragraph{Otros componentes del estándar}\hfill \break
El estándar posee además otros componentes o bloques que agrupan y contienen metadatos de publicación de los datos.  A continuación se detallan dichos componentes.

\paragraph{\textit{Releases}}\hfill \break
Para fomentar la mayor apertura de información el estándar está preparado para publicar la información en tiempo real. En cada etapa del proceso de contratación o con cada cambio que ocurra sobre los datos, el estándar prevé la publicación de esa nueva porción de información mediante releases.
Los releases son acumulativos, es decir, durante un proceso de contratación se pueden proporcionar uno o más releases, por ejemplo: describir una licitación, anunciar la adjudicación de contratos y proporcionar actualizaciones sobre los mismos.
Una vez que un release ha sido publicado no puede cambiar. La información actualizada debe ser compartida a través de un nuevo release.
Los releases pueden ser publicados a través de un único sistema o de manera distribuida por diferentes sistemas, pero todos éstos deben estar relacionados a partir de un identificador único denominado Open Contracting ID (OCID).

\paragraph{\textit{Releases}}\hfill \break
Un registro (\textit{Record}) de contratación provee una instantánea del proceso de contratación en un punto dado en el tiempo, que reúne todas las versiones por las cuales pasó ese proceso en un solo lugar.
Un Record contiene tres elementos clave: 
\begin{itemize}
    \item Una lista de todos los releases asociados a un proceso de contratación en particular, 
    \item Un release compilado que contiene la última versión de los datos,
    \item Una versión histórica de releases que contiene la historia con todos los cambios realizados sobre los datos.
\end{itemize}

\paragraph{\textit{Release package}}\hfill \break
Un release package es un esquema de agrupación para la publicación de releases, describe el documento contenedor y metadatos para la publicación de releases.

\paragraph{\textit{Record package}}\hfill \break
Un Record Package es un esquema de agrupación para la publicación de Records, describe el documento contenedor y metadatos para la publicación de Records. En la Figura \ref{img:Record Package} se muestra el modelo conceptual de \textit{Record Package}.



\begin{figure}[ht!]
    \centering
    \includegraphics[width=150mm]{figuras/Diagramas-RecordPackage.png}
    \caption{Record Package}
    \label{img:Record Package}
\end{figure}


\subsubsection{Transformación del recurso no-ontológico}
En esta sección se utilizaron los patrones de diseño para seguir construyendo un modelo conceptual y luego proseguir con la implementación de la ontología. 
\paragraph{Búsqueda de Patrones de Diseño}\hfill \break
El objetivo de este paso es buscar posibles patrones de diseño que conviertan el recurso no-ontológico en un modelo conceptual. 
Una vez analizado el esquema del OCDS se procedió a la búsqueda de patrones de diseño de ontologías, se utilizó el repositorio ontologydesingpatterns.org para realizar esta búsqueda de patrones bien conocidos por la comunidad de desarrolladores de ontologías y recomendado por la metodología NeOn.
Los patrones que son relevantes son aquellos que involucran tiempo, dinero, empresas u organizaciones, lugares y procesos en general. Luego de una búsqueda se encontraron los siguientes patrones de diseño que podrían ser útiles al momento del desarrollo de la ontología.
\begin{enumerate}
    \item Intervalo de tiempo\footnote{http://ontologydesignpatterns.org/wiki/Submissions:TimeInterval}
    \item Precio\footnote{http://ontologydesignpatterns.org/wiki/Submissions:Price} 
    \item Etiquetas\footnote{http://ontologydesignpatterns.org/wiki/Submissions:Tagging}
    \item Lugar\footnote{http://ontologydesignpatterns.org/wiki/Submissions:Place}
    \item Lista\footnote{http://ontologydesignpatterns.org/wiki/Submissions:List}
    \item Secuencia\footnote{http://ontologydesignpatterns.org/wiki/Submissions:Sequence}
    \item Boleta de Pago\footnote{http://ontologydesignpatterns.org/wiki/Submissions:Invoice} 
\end{enumerate}

Además se investigaron otros patrones de reingeniería, pero ninguno de éstos se adecuaba ya que están orientados a patrones jerárquicos de diseño y el dominio modela una secuencia o proceso. 

Se creó un patrón de conversión del estándar de documentacion JSON SCHEMA http://json-schema.org/ a un modelo ontológico. El patrón de conversión tiene los siguientes lineamientos.

\begin{itemize}
    \item Todos los objetos JSON del esquema son potenciales conceptos de la ontología desarrollada. Un Objeto JSON será considerado como clase o individuo en el modelo conceptual.
    \item El hecho de que un objeto esté relacionado a otro no necesariamente indica una relación de jerarquía, puede a la vez tratarse de una relación de contención, uso, correlación o dependencia.
    \item Todos los atributos de un Objeto JSON de tipo \textit{string} o numérico son potenciales propiedades (en la ontología) del concepto generado a partir el Objeto JSON.
    \item Los tipos de datos nos indican el tipo de dato en el formato RDF. Por ejemplo, si dentro de JSON Schema tenemos un campo que solo acepta números, podemos decir que esa propiedad es de tipo \textit{xsd:long}. Así, pueden ser una restricción en los valores de la propiedad.
    \item El nombre del objeto JSON es un candidato para nombrar esta relación. Por ejemplo, un presupuesto posee un objeto de nombre monto, de tipo valor. Por lo que podemos decir que el nombre de la relación entre presupuesto y valor se llamará monto.
    \item Los arreglos son representados a través de la relación uno a muchos dentro del modelo ontológico.
    \item Los atributos \textit{id}, \textit{identifier} o similares son potenciales identificadores de las instancias de cada concepto.
    
\end{itemize}

El esquema del OCDS posee una lista de códigos (\textit{Codelist}) de dos tipos, abiertos y cerrados. Los \textit{Codelist} Abiertos proveen códigos sugeridos, los publicadores pueden extender esta lista con nuevos códigos sin el consenso de otros publicadores. Mientras que los \textit{Codelist} Cerrados proveen códigos mandatorios, los publicadores solo pueden utilizar los valores de la lista oficial, cambios a \textit{Codelist} cerrados deben hacerse a través de la gobernanza y revisión de la lista. Esta lista de códigos fueron transformados utilizando el patrón mencionado a continuación. 

Para los \textit{Codelist} Abiertos se los convirtió en \textit{individuals}, en otros términos, instancias de los conceptos. Para los \textit{Codelist} Cerrados  se los convirtió en \textit{Enumerated Classes}. Las \textit{Enumerated Classes} impiden la declaración de nuevos individuos que pertenecen a esta clase, siendo así más restrictivos.
\paragraph{Utilización de patrones}\hfill \break
Por último, se consideraron los patrones mencionados anteriormente para los siguientes conceptos:
Para el concepto \textit{Value} del OCDS, que representan valores monetarios, se utilizó el patrón \textit{Price}. El patrón describe el precio de alguna entidad, dicho precio contiene un valor y una moneda. En la Figura \ref{img:Modelo de Precio} se muestra el diagrama del patrón.

\begin{figure}[ht!]
    \centering
    \includegraphics[width=150mm]{figuras/Diagramas_Precio.png}
    \caption{Modelo de Precio}
    \label{img:Modelo de Precio}
    
\end{figure}

En cuanto a la implementación, se optó por mantener \textit{Value} y \textit{Currency} como \textit{DataType properties} para mantener la simplicidad de nuestra ontología.

Para el concepto \textit{Period}, que representan periodos de tiempo se utilizó el patrón \textsc{Time Interval}. El patrón describe un intervalo de tiempo que contiene una fecha de inicio, una fecha de fin y una fecha del intervalo. Un posible uso del patrón sería la fecha “Agosto del 2018” tiene un fecha de inicio “1 de Agosto” y una fecha de fin “31 de Agosto”. En la implementación se omitió la propiedad Intervalo de tiempo, ya que no es necesaria para este este dominio. El modelo se puede ver en la Figura \ref{img:Modelo de Intervalo de precio}.

\begin{figure}[ht!]
    \centering
    \includegraphics[width=150mm]{figuras/Diagramas_Tiempo.png}
    \caption{Modelo de Intervalo de tiempo}
    \label{img:Modelo de Intervalo de precio}
    
\end{figure}

Para la relación que existe entre \textit{Planning, Tender, Award, Contract e Implementation}, que tienen una relación de dependencia y de secuencia, se utilizó el patrón \textit{Sequence}. El patrón describe una secuencia de hecho a través de las propiedades Precede, Precede Directamente, Antecede y Antecede Directamente. Por ejemplo, decidir qué camisa utilizaré precede directamente a colocarse la camisa, y precede a colocarse la corbata. En la Figura \ref{img:Modelo de dependencia y secuencia} se presenta el diagrama del patrón. Se utilizó este patrón para describir la secuencia del proceso de contrataciones.

\begin{figure}[ht!]
    \centering
    \includegraphics[width=150mm]{figuras/Diagramas_Follows.png}
    \caption{Modelo de dependencia y secuencia}
    \label{img:Modelo de dependencia y secuencia}
    
\end{figure}

\subsubsection{Construcción de la ontología}

Una vez concluido el modelo conceptual de la ontología procedemos a la construcción de la misma utilizando el lenguaje OWL y la herramienta Protégé.

Dentro de la fase de búsqueda de recursos Ontológicos, se encontró una ontología desarrolladas\footnote{https://github.com/ColinMaudry/open-contracting-ld} por un colaborador al proyecto OCD, la ontología siguió una transformación directa de JSON SCHEMA del estándar, la misma representa un punto de partida para seguir desarrollando la ontología. De esta manera además queremos incentivar la colaboración y hacer partícipe a toda la comunidad del OCDS.

El primer paso fue importar la ontología al software Protégé. Al hacer esto el software reestructuró el código de la ontología e hizo inferencias básicas detalladas a continuación. La estructura del documento del cual se partió estaba agrupada según conceptos semánticos relacionados. Protégé estructuró la ontología según sean \textit{Annotation Property, DataType Property u Object property} y luego ordenó de forma alfabética tomando como referencia el nombre de la propiedad.

Se encontraron algunas inconsistencias sintácticas dentro de la ontología, debido a que la misma fue hecha sin utilizar ninguna herramienta para desarrollo de ontologías, lo cual da lugar a errores sintácticos. Por ejemplo, el rango de la clase \textit{FormerValue} estaba definido como \textit{xsd:Integer}, Protégé infiere de que se trataba de un nuevo tipo de dato,  debido a que no existe este tipo de dato sino que debe ser \textit{xsd:integer}. Así también otro error en el documento en donde se escribió \textit{commnent} en lugar de \textit{commnent}, inmediatamente Protégé infiere como un nuevo tipo de dato. Como aprendizaje, el uso de una herramienta ontológica como Protégé nos ayuda a prevenir errores de sintaxis en la creación de la ontología y además acelera el proceso de desarrollo gracias a una interfaz gráfica que nos ayuda a distinguir entre Clases, \textit{DataType Properties} por citar algunas funcionalidades.

Toda la ontología inicial utilizó el tipo \textit{rdf:Property} para definir las propiedades. Debido a que \textit{owl:ObjectProperty} y \textit{owl:DataTypeProperty} son subclases de \textit{Property}, se decidió especializar las relaciones con estas subclases. Las propiedades que relacionan dos instancias fueron asignadas como owl:ObjectProperty y las propiedades relacionadas entre una instancia y un valor fueron asignadas como owl:DataTypeProperty.

Una propiedad de objeto (\textit{ObjectProperty}) se utiliza para conectar pares de individuos a través de IRIs, mientras que una propiedad de tipo de dato (\textit{DataTypeProperty}) relaciona una instancia de una clase con un literal.

Una propiedad funcional (\textit{FunctionalProperty}) es una propiedad que solamente puede tener un valor Y por cada instancia X, por ejemplo no puede haber dos valores distintos Y1 e Y2 tal que los pares (X,Y1) y (X,Y2) sean instancias de esta propiedad. Tanto los “ObjectProperty” como los \textit{DataTypeProperty} pueden ser declarados como \textit{DataTypeProperty}.


Una consideración importante es que para los \textit{DataType Properties} no es necesario definir como la propiedad functional ya que la mayoría de los motores de razonamiento pueden deducir que dos literales son distintos. Ejemplo: un razonador puede inferir que  el literal “1” es distinto que  "2".

El motor de inferencia agregó un axioma indicando que rdf:descriptions es igual a \textit{owl:AnnotationProperty}, que es una manera de anotar información descriptiva de la propiedad. Además se utilizó la propiedad \textit{rdf:Resource}, clase que no existe en en el estándar RDF, debería indicar la clase RDFS \textit{ rdfs:Resource} para indicar hipervínculos. En este caso se vió que semánticamente se adecua mejor la propiedad \textit{DCTERMS:URI}.

Así, se pudo construir la ontología base partir de un recurso No-Ontológico como ser el Open Contracting Data Estándar. En el siguiente apartado se abocará a la búsqueda y reutilización de recursos Ontológicos para enriquecer la ontología desarrollada.


\section{Reuso de Recursos Ontológicos}
Los pasos para el reuso de una ontología son la búsqueda, análisis y comparación de ontologías desarrolladas y luego la selección e integración de las ontologías a reutilizar. Al mismo tiempo existen distintas maneras de reutilizar una ontología, como veremos más adelante. 

\subsection{Búsqueda de Ontologías}
\label{section:busqueda de ontologias}

Un parámetro importante a la hora de reutilizar ontologías es la cantidad de veces que esta ya fue reutilizada. \textit{Linked Open Vocabularies}(LOV)\footnote{https://lov.linkeddata.es/dataset/lov/} es un repositorio de ontologías bien curado y estable mantenido por la \textit{Open Knowledge Foundation}  (OKF) que nos muestra la cantidad de conexiones entre las ontologías.

\subsubsection{Ontologías Relacionadas}

Con relación a ontologías más generales y relacionadas de alguna manera al dominio que pudieran ayudar a enriquecer las ontologías desarrolladas se encontraron las siguientes:
\begin{itemize}
    \item \textit{Organization}\cite{TheOrgan48:online} : Ontología central para las estructuras de una organización, destinada a apoyar la publicación de datos vinculados de la información de la organización en una serie de dominios. Está diseñada para permitir extensiones específicas de dominio para agregar clasificación de organizaciones y roles, así como extensiones para admitir información relacionada, como actividades de la organización.
    \item \textit{Good Relations}\cite{hepp2008goodrelations} : es un vocabulario estandarizado para datos de productos, precios, tiendas y empresas que pueden integrarse en páginas web estáticas y dinámicas existentes que puede ser procesado por otras computadoras. Esto aumenta la visibilidad de sus productos y servicios en la última generación de motores de búsqueda, sistemas de recomendación y otras aplicaciones.
    \item \textit{DCTERMS}\cite{DCMIDCMI37:online}: Una especificación actualizada de todos los términos de metadatos mantenidos por la Dublin Core Metadata Initiative, que incluye propiedades, esquemas de codificación de vocabulario, esquemas de codificación de sintaxis y clases. 
\end{itemize}

\subsubsection{Ontologías del dominio de Contrataciones Públicas.}

Observando LOV dentro del dominio de estudio, se realizó una búsqueda a través de la etiqueta “Contract”, teniendo como resultado los siguientes recursos ontológicos:

\begin{enumerate}
    \item c4n - Call for Anything vocabulary\footnote{https://lov.linkeddata.es/dataset/lov/vocabs/c4n}
    \item eli - The European Legislation Identifier\footnote{https://lov.linkeddata.es/dataset/lov/vocabs/eli}
    \item ldr - Linked Data Rights\footnote{https://lov.linkeddata.es/dataset/lov/vocabs/ldr}
    \item loted - LOTED ontology\footnote{https://lov.linkeddata.es/dataset/lov/vocabs/loted}
    \item pay - Payments ontology\footnote{https://lov.linkeddata.es/dataset/lov/vocabs/pay}
    \item  pc - Public Contracts Ontology\footnote{https://lov.linkeddata.es/dataset/lov/vocabs/pc}
    \item pproc - PPROC ontology\footnote{https://lov.linkeddata.es/dataset/lov/vocabs/pproc} 
    \item LOTED 2 - LOTED ontology 2 \cite{distinto2014loted2}
    \item DNCP - Ontologia desarrollada por la DNCP\footnote{https://www.contrataciones.gov.py/datos/def/dncp.owl}
\end{enumerate}


Haciendo una búsqueda más exhaustiva dentro del dominio de contrataciones públicas a través de internet se pudo identificar también las ontologías LOTED 2 y la ontología desarrollada por la DNCP.

En este trabajo se pone bajo análisis las ontologías de LOTED 2 , LOTED, Public Contract y PPROC, esta última es la más reciente y en su desarrollo se tomó en cuenta la experiencia de las primeras. Por último también se pone bajo consideración la ontología de la DNCP.

La ontología más referenciada, esto significa que la ontología tiene mayor cantidad de reutilización, en el sector de contrataciones públicas es la Public Contract Ontology (PCO)\cite{klimek2012lod2}  ya que ofrece un medio de expresión para describir los conceptos básicos de este dominio logrando así que otras ontologías puedan extender de ésta fácilmente. Otra ontología que cabe mencionar es la ontología LOTED  ya que fue desarrollada para enriquecer los datos de licitaciones, expuestas por el sistema TED, con enlaces a Geonames y DBpedia. A continuación de LOTED y viendo la necesidad de dar un contexto legal al ámbito de contrataciones fue desarrollada la ontología LOTED2\cite{distinto2014loted2} cuyo principal objetivo es la representación de los conceptos jurídicos relacionados al dominio de las contrataciones públicas, por esto puede resultar un tanto difícil su utilización.

La ontología PPROC \cite{munoz2016pproc} fue desarrollada siguiendo las especificaciones de OWL, utiliza clases de PCO, así como también de otras ontologías como  \textit{Organization Ontology, Schema.org, SKOS} y para la definición de conceptos relacionados a ofertas realizadas utiliza \textit{Good Relations}. Uno de los objetivos específicos de PPROC es la publicación del perfil del contratante de las administraciones que participan en los proceso de licitación.

Tanto LOTED como LOTED2, PCO y PPROC utilizan RDF/XML para la definición de la ontología debido a su fuerte poder de describir los atributos de los recursos. 

Otra ontología que vale mencionar es la desarrollada por la DNCP, la misma está definida en OWL con la serialización XML. Si bien la misma utilizó el lenguaje OWL como sintaxis, no tuvo en cuenta principios importantes en el modelado y desarrollo de una ontología.  Como ejemplo, la ontología desarrollada no tuvo en cuenta el nivel de especificación de los datos, se encuentran definiciones de Clases como \textit{Número, Texto, Fecha, Email,} siendo que estos conceptos podrían modelarse como \textit{DataType Properties} de tipo Texto o Número. De igual manera, la misma representa un diccionario de datos importante y un acercamiento a la formalización sintáctica y semántica de su modelo de datos.

En la  Tabla \ref{tab:comparacion_ontologias} se muestra una comparación entre las cinco ontologías analizadas. Una comparación mas extensa se puede encontrar en el repositorio de archivos\footnote{http://bit.ly/ValdezBaezThesis} y en la Tabla \ref{tab:comparacionOntologiasAnexo} del Anexo \ref{chap:comparaciondeOntologias}.


\newcolumntype{L}[1]{>{\raggedright\let\newline\\\arraybackslash\hspace{0pt}}m{#1}}
\newcolumntype{C}[1]{>{\centering\let\newline\\\arraybackslash\hspace{0pt}}m{#1}}
\newcolumntype{R}[1]{>{\raggedleft\let\newline\\\arraybackslash\hspace{0pt}}m{#1}}



\begin{table}[!htb]
    \caption{Comparación entre las ontologías LOTED, LOTED2, PCO , PPROC, DNCP y OCDSV0.}
    \label{tab:comparacion_ontologias}
    
    \scriptsize 
    \begin{tabular}{|L{1.5cm}|C{1.8cm}|C{2cm}|C{1.8cm}|C{1.8cm}|C{1.8cm}|C{1.8cm}|}
    \hline
     & LOTED & LOTED2 & PCO & PPROC & DNCP  & OCDSV0 \\
    \hline

    
    Cantidad de Axiomas & 172 & 1709 & 450 & 1501 & 802 & 802\\
    \hline

    Cantidad de Clases & 23 & 177 & 22 & 92 & 36 & 36\\
    \hline

    Lenguaje & OWL (RDF/XML) & OWL (RDF/XML) & OWL (RDF/XML) & OWL (RDF/XML) & OWL (RDF/XML)& OWL (RDF/XML)\\
    \hline
    Metodología & Propia & 
    Un enfoque de arriba hacia abajo (extracción de conceptos jurídicos de fuentes legales) y en una de abajo hacia arriba (análisis de las formas estándar).
     & Propia & Ontology Development 101 & Propia & Propia\\
     \hline
    Cantidad de ontologías reutilizadas & 2 (DBpedia, geonames) & 2 & 7 & 6 & 0  & 0\\
    \hline
    Propósito de la ontología & Agregar un nivel de estructura a los datos obtenidos de TED & Dar un contexto legal. Permitir la construcción de aplicaciones legales de Web Semántica para contrataciones públicas. & Modelar contratos públicos en general & Incorporar las técnicas
    semánticas en las herramientas utilizadas por las Administraciones públicas en los procesos de contratación & Base de conocimiento de datos de contrataciones públicas de Paraguay & Base de conocimiento de datos de contrataciones públicas de Paraguay \\
    \hline
    Ultima versión & 03 Jul. 2012 & 24 Jul. 2013 & 10 Oct. 2012 & (1.0.0)29 Oct. 2014 & 2016 & 23 Ago. 2018\\
    \hline
    \end{tabular}
    
    \bigskip
    \small\textit{Nota}. Cabe destacar que todas utilizan OWL(RDF/XML) como lenguaje de desarrollo. LOTED2 tiene mayor cantidad de axiomas y cantidad de clases definidas.
    \end{table}

    \subsection{Reutilización de Ontologías}


    Según NeOn, las ontologías pueden ser reutilizadas de varias maneras según la necesidad:
    
    \begin{enumerate}
       \item se puede utilizar la totalidad de una ontología que cumple con las expectativas para construir la nueva ontología.
        \item  En ciertos casos sólo se necesita utilizar un módulo o una parte de la ontología ya que podrían haber conceptos que interesen y otros que no, por lo tanto no se utilizaría en su totalidad.
        \item  Por último, para tener mayor control de los conceptos reutilizados, se puede utilizar la ontología a nivel de enunciado o tripla, importando solamente el enunciado que interese. 
    \end{enumerate}
       
   


Como en este trabajo se desarrolló una ontología desde el principio utilizando como insumo principal un recurso no-ontológico, se abordará la reutilización por enunciados, de esta forma se enriquece la ontología primeramente desarrollada. La reutilización por enunciado se puede hacer en varios momentos del ciclo de vida del la ontología desarrollada. En este trabajo sólo se utilizarán los conceptos principales de las ontologías antes analizadas, pudiendo seguir reutilizando nuevos enunciados en fases posteriores.

Para los documentos correspondientes a cada fase del proceso licitatorio se hizo una especialización de la clase \textit{Document}, siendo así una subclase del mismo. Por lo tanto tenemos el enunciado del Cuadro \ref{lst:sparql1}.\hfill \break


\noindent\begin{minipage}{\textwidth}

\begin{lstlisting}[captionpos=b, caption={Reutilización de la Clase Document}, label={lst:sparql1},  numbers=left,  numberstyle=\tiny\color{mygray},frame=single]
:Document rdf:type owl:Class ;
    rdfs:subClassOf foaf:Document .
\end{lstlisting}
\end{minipage}

 Al mismo tiempo se reutilizaron de la ontología \textit{Public Contract Ontology}\cite{klimek2012lod2}, los conceptos de \textit{Tender} y \textit{Contract} de un proceso licitatorio (Cuadro \ref{lst:sparql2}).\hfill \break

\noindent\begin{minipage}{\textwidth}
 \begin{lstlisting}[captionpos=b, caption={Conceptos de Tender y Contract}, label={lst:sparql2},  numbers=left,  numberstyle=\tiny\color{mygray},frame=single]
:Tender rdf:type owl:Class ;
    rdfs:subClassOf pc:Tender .
:Contract rdf:type owl:Class ;
    rdfs:subClassOf pc:Contract .
\end{lstlisting}
\end{minipage}

 \textit{Good Relations}\cite{hepp2008goodrelations} es una ontología para modelar compras de bienes y servicios a través de internet, por lo que también se procedió a la reutilización de los conceptos \textit{gr:Offering, gr:BusinessEntity y gr:PriceSpecification} (Cuadro \ref{lst:sparql3}).\hfill \break

\noindent\begin{minipage}{\textwidth}
 \begin{lstlisting}[captionpos=b, caption={Conceptos de Organization y Price}, label={lst:sparql3},  numbers=left,  numberstyle=\tiny\color{mygray},frame=single]
:Item rdf:type owl:Class ;
    rdfs:subClassOf gr:Offering .
:Organization rdf:type owl:Class ;
    owl:equivalentClass gr:BusinessEntity .
:Value rdf:type owl:Class ;
    owl:equivalentClass gr:PriceSpecification .


 \end{lstlisting}
\end{minipage}

 Se reutilizó la ontología de Schema.org\footnote{https://schema.org} para conceptos básicos como los presentados en el Cuadro \ref{lst:sparql4}.\hfill \break

\noindent\begin{minipage}{\textwidth}
 \begin{lstlisting}[captionpos=b, caption={Reutilización de Schema.org}, label={lst:sparql4},  numbers=left,  numberstyle=\tiny\color{mygray},frame=single]
:Address rdf:type owl:Class ;
    rdfs:subClassOf sch:PostalAddress .
:ContactPoint rdf:type owl:Class ;
    rdfs:subClassOf sch:ContactPoint .
 \end{lstlisting}
\end{minipage}

 Además se utilizó la ontología \textit{Sequence}\footnote{http://www.ontologydesignpatterns.org/cp/owl/sequence.owl} para modelar la relación de secuencia que existe dentro del proceso licitatorio. En el Cuadro \ref{lst:sparql5} muestra como ejemplo la propiedad que indica que la Planificación precede directamente al Llamado.\hfill \break


\noindent\begin{minipage}{\textwidth}
 \begin{lstlisting}[captionpos=b, caption={Reutilización del patrón de Secuencia}, label={lst:sparql5},  numbers=left,  numberstyle=\tiny\color{mygray},frame=single]
:planningPrecedes rdf:type owl:ObjectProperty ;
    rdfs:subPropertyOf seq:directlyPrecedes ;
    rdfs:domain :Planning ;
    rdfs:range :Tender .




 \end{lstlisting}
\end{minipage}

Estos enunciados utilizados fueron elegidos teniendo en cuenta todas las ontologías del dominio y también la forma de reuso que existían entre las mismas. 

Con esto se concluyó con la reutilización de ontologías, paso necesario para poder enriquecer la semántica de la ontología desarrollada. La Figura \ref{img:grafo ontologia desarrolla} es una representación visual en forma de grafo de la ontología desarrolla. En la Tabla \ref{tab:comparacion_ontologias_ocdspy} se muestra una pequeña comparación entre las ontologías OCDSV0 y OCDSPY. Allí se observa que la cantidad final de axiomas y clases aumentó luego de haber finalizado el desarrollo de la ontología OCDSPY. En el siguiente capítulo se procederá al uso de la ontología desarrollada y posteriormente se realizarán pruebas con la misma.


\begin{table}[!htb]
    \centering
    \caption{Comparación entre la ontología OCDSV0 y la ontología desarrollada OCDSPY}
    \label{tab:comparacion_ontologias_ocdspy}
    \scriptsize 
    \begin{tabular}{|L{6cm}|C{2.5cm}|C{2.5cm}|}
    \hline
     &  OCDSV0 & OCDSPY \\
    \hline
    Cantidad de Axiomas & 587 & 1210 \\
    \hline
    Cantidad de Clases & 22 & 36 \\
    \hline
    Lenguaje & OWL (RDF/TTL) & OWL (RDF/TTL) \\
    \hline
    Metodología & Propia (no especificada) & NeOn \\
     \hline
    Cantidad de ontologías reutilizadas & 0 & 6\\
    \hline
    Propósito de la ontología & Agregar un nivel semántico al OCDS  & Agregar un nivel semántico al OCDS \\
    \hline
    Ultima versión & 23 Agosto 2018 & 24 Abril 2019 \\
    \hline
    \end{tabular}
    \bigskip
\end{table}

\begin{figure}[ht!]
    \includegraphics[width=180mm]{figuras/grafoOCDS.png}
    \caption{Grafo de la ontología desarrollada}
    \label{img:grafo ontologia desarrolla}
    \end{figure}

    \begin{figure}[ht!]
        \includegraphics[width=150mm]{figuras/zoomTender.png}
        \caption{Acercamiento del Grafo de la ontología desarrollada en Convocatoria.}
        \label{img:zoomTender}
        \end{figure}


        \begin{figure}[ht!]
            \includegraphics[width=150mm]{figuras/zoomContract.png}
            \caption{Acercamiento del Grafo de la ontología desarrollada en Contracto.}
            \label{img:zoomContract}
            \end{figure}



\section{Discusión del Capitulo}

En este capítulo se desarrolló una ontología teniendo como base de conocimiento un recurso no-ontológico (OCDS). Se utilizó la metodología NeOn y el software Protégé para el desarrollo de ontología. Las actividades principales fueron la planificación de actividades, especificación de los requerimientos, el reuso de recursos ontológicos y no-ontológicos. El resultado de todo este proceso fue una ontología desarrollada en lenguaje OWL que responde a la base de conocimiento de la OCDS y al esquema de datos de la DNCP teniendo en cuenta las buenas prácticas de desarrollo de ontologías propuestas por la metodología.





%!TEX root = ../main.tex
\chapter{Implementación}
\label{chap:Implementación de la Ontologia}

La intención del trabajo es enriquecer los datos publicados por la DNCP en formato JSON a través de inserción de las propiedades necesarias para convertir un objeto JSON a JSON-LD para luego transformarlos a un formato de triplas RDF. Una vez construida la base de datos en formato RDF se podrá montar a un Punto SPARQL para realizar consultas.


\begin{figure}[h!]
   \centering
   \includegraphics[width=150mm]{figuras/Diagramas-Implementacion.png}

   \caption{Modelo de Implementación}
   \label{img:modelo de Implementacion}
\end{figure}

A continuación se explicarán algunos conceptos relacionados y también las tareas realizadas para lograr montar correctamente el Punto SPARQL.


\section{Contexto JSON-LD}

Como se vió anteriormente, existen varias formas de serialización de RDF, en contraste a XML, JSON-LD fue diseñado para ser un formato de intercambio de datos ligero e independiente del lenguaje y es lo suficientemente expresivo como para soportar los conceptos de RDF, además requiere poco esfuerzo para los desarrolladores transformar un documento JSON a JSON-LD\cite{JSONLDJS41:online}. 

Para dar contexto a un objeto JSON es necesario agregar el atributo @context. Éste puede darse de dos formas, definiendo la estructura del contexto como valor de la propiedad, o haciendo referencia (URI) a un documento que contiene la definición del contexto como ya se vió en la sección \ref{section:serializacion_jsonld}.

Se puede ir agregando @context a los objetos hijos de forma recursiva. Esto es muy importante debido a que se pueden sobrescribir los contextos exclusivamente para un objeto en particular sin que afecte a la definición de los demás. Más adelante en este capítulo se abordarán cada uno de los temas.



 
\section{De JSON a JSON-LD}
\label{section:jsonajsonld}

Se consultó la API de la DNCP que sigue el formato de OCDS, a modo de ejemplo se utlizó el proceso de licitación número 193399 a través de la siguiente URL https://www.contrataciones.gov.py/datos/api/v2/doc/ocds/record-package/193399.

Se extrajo solamente el objeto \textit{Compiled Release} (ver sección \ref{record}), que posee toda la información del proceso licitatorio. Una versión resumida del objeto se puede visualizar en el Cuadro \ref{lst:json1}. Se puede ver que el objeto tiene las siguientes propiedades: language, ocid, date, tag, initiationType, planning, tender, buyer, awards, contracts.\hfill \break

\noindent\begin{minipage}{\textwidth}
\begin{lstlisting}[captionpos=b, caption=Objeto JSON del OCDS de un Compiled Release, label=lst:json1,  numbers=left, language=json, numberstyle=\tiny\color{mygray},frame=single]
"compiledRelease": {
    "language": "es",
    "ocid": "ocds-03ad3f-193399",
    "id": "193399-adquisicion-scanner",
    "date": "2018-12-18T23:39:23Z",
    "tag": [
        "compiled"
    ],
    "initiationType": "tender",
    "planning": { ...
    },
    "tender": { ...
    },
    "buyer": { ...
    },
    "awards": [ ...
    ],
    "contracts": [ ...
    ]
}
\end{lstlisting}
\end{minipage}


Para enriquecer semánticamente un objeto debemos agregar un contexto y los respectivos @id para poder convertirlo a JSON-LD. 

Se encontró que no sería suficiente un solo contexto para todo el objeto. Esto se debe a las colisiones que existen entre propiedades que poseen el mismo nombre y distintos significados.

Dado que la definición de los contextos se da en forma de cascada, es decir se toma la definición más próxima, se definió un contexto para cada objeto hijo.

A continuación se muestra la colisión entre conceptos, donde \textit{amount} (del objeto \textit{budget}) se refiere al valor monetario (número y moneda) del presupuesto y la siguiente propiedad \textit{amount} (del objeto \textit{amount}) se refiere al valor numérico específicamente.\hfill \break

\noindent\begin{minipage}{\textwidth}
\begin{lstlisting}[captionpos=b, caption=Objeto JSON con colisión semántica entre conceptos, label=lst:oJson,language=json,firstnumber=1,  numbers=left,  numberstyle=\tiny\color{mygray},frame=single]
    "budget": {
        "description": "Adquisicion de Scanner",
        "amount": {
              "amount": 12000000,
              "currency": "PYG"
         }
     }  
    \end{lstlisting}
\end{minipage}
\noindent
\begin{minipage}{\textwidth}
    \begin{lstlisting}[captionpos=b, caption=Objeto JSON-LD sin colisión semántica entre conceptos , label=lst:oJsonLd, language=json,firstnumber=1,  numbers=left,  numberstyle=\tiny\color{mygray},frame=single]
    "budget": {
        "@context": "http://purl.org/onto-ocds/ocds/context-budget.json",
        "@type": "Budget",
        "description": "Adquisicion de Scanner",
        "amount": {
            "@context": "http://purl.org/onto-ocds/ocds/context-value.json",
            "@type": "Value",
            "amount": 12000000,
            "currency": "PYG"
       }
   }   
        \end{lstlisting}
    \end{minipage}

En el Cuadro \ref{lst:oJsonLd} se observa la solución implementada de manera a que el objeto \textit{budget} tenga la definición de su atributo \textit{amount} y el objeto \textit{amount} tenga la definición de su atributo \textit{amount} por separado evitando así inconsistencias.

Si bien las propiedades tienen el mismo nombre (o identificador) tienen definiciones distintas en cuanto a la sintaxis y a la semántica. En el caso de la propiedad de la línea 5 se refiere al valor monetario del presupuesto que está compuesto por la moneda y el valor numérico y está representado mediante un objeto JSON. En el caso de la propiedad de la línea 8 se refiere únicamente al valor numérico y está representado a través de un atributo de tipo numérico. 


A modo de ejemplo, se procedió a crear un contexto para el objeto planning que se muestra en el Cuadro \ref{lst:json2} y todos los objetos hijos del mismo. En el Cuadro \ref{lst:json3} se puede observar que se agregaron las propiedades @context en las líneas 2,6,10, la propiedad @type en las líneas 3,7,11 y la propiedad @id en la línea 4.

Se utilizó la herramienta JSON-LD PLAYGROUND \cite{JSONLDPl78:online} y RDF Translator\cite{RDFTrans0:online} para verificar la sintaxis de los documentos creados.\hfill \break

\noindent\begin{minipage}{\textwidth}
\begin{lstlisting}[captionpos=b, caption=Objeto JSON del OCDS de un Planning, label=lst:json2,  numbers=left, language=json, firstnumber=1, numberstyle=\tiny\color{mygray},frame=single]
"planning": {
    "budget": { 
        "description": "Adquisicion de Scanner",
        "amount": {
            "amount": 12000000,
            "currency": "PYG"
        }
    },
    "url": "https://www.contrataciones.gov.py/datos/id/planificaciones/193399-adquisicion-scanner"
}
\end{lstlisting}
\end{minipage}

\hfill \break

\noindent\begin{minipage}{\textwidth}
\begin{lstlisting}[captionpos=b, caption=Objeto JSON-LD del OCDS de un Planning, label=lst:json3,  numbers=left, language=json, firstnumber=1, numberstyle=\tiny\color{mygray},frame=single]
"planning": {
    "@context": "http://example.com/ocds/context-planning.json",
    "@type": "Planning",
    "@id": "https://www.contrataciones.gov.py/datos/id/planificaciones/193399-adquisicion-scanner"
    "budget": { 
        "@context": "http://example.com/ocds/context-budget.json",
    "@type": "Budget",
        "description": "Adquisicion de Scanner",
        "amount": {
            "@context": "http://example.com/ocds/context-value.json",
            "@type": "Value",
            "amount": 12000000,
            "currency": "PYG"
        }
    },
    "url": "https://www.contrataciones.gov.py/datos/id/planificaciones/193399-adquisicion-scanner"
}
\end{lstlisting}
\end{minipage}

Los documentos de los contextos sirven para referenciar a cada propiedad con la ontología creada. En el Cuadro \ref{lst:json4} se muestra el contenido del documento context-planning.json. Nótese de que sólo están definidas las propiedades hijas del objeto \textit{planning}, los demás ancestros se encuentran definidas en el documento del padre.

Además sólo se agregó la propiedad @id al objeto \textit{planning}, el objeto \textit{amount} no posee un @id debido a que la implementación de la DNCP tampoco provee un identificador único para ese objeto, esto significa que a conversión a RDF de dicho objeto derivará en un Blank Node, por lo que no será posible referenciar dicha instancia más allá de este documento. Eso tampoco es una necesidad para el nivel de agregación que se requiere ya que no es necesario identificar individualmente el monto. El monto siempre estará referenciado a través del objeto \textit{planning}

\hfill \break

\noindent\begin{minipage}{\textwidth}
\begin{lstlisting}[captionpos=b, caption=Contexto del Objeto Planning, label=lst:json4,  numbers=left, language=json, firstnumber=1, numberstyle=\tiny\color{mygray},frame=single]
"@context": {
    "Planning": {
        "@id": "http://w3id.org/ocds/ns#Planning"
    },
    "budget": {
        "@id": "http://w3id.org/ocds/ns#budget"
        "@type": "@id"
    },
    "url": {
        "@id": "http://w3id.org/ocds/ns#planningUrl"
        "@type": "http://www.w3/org/2001/XMLSchema#anyURI"
    }
}

\end{lstlisting}
\end{minipage}

Un inconveniente al momento de generar los valores para los atributos del @id, es que los objetos no posean un valor válido, en este caso una URI que desreferencie dicho objetos. El procedimiento consiste en verificar que el objeto posea un atributo “uri” cuyo valor sea una URI, de no ser el caso se procede a verificar si posee los atributos “url” e “id” sucesivamente. En el caso de que posea estos atributos pero con valores inválidos, osea no sean una URI propiamente dicha, se crea una URI ficticia a fin de que tenga un @id con formato válido. Si un objeto no tiene un @id válido, entonces no podrá ser convertido luego a formato RDF.

Otra excepción fue el caso de los Codelist abiertos y cerrados. En cada caso se procedió a transformar el valor del atributo correspondiente, que en este caso era un String que no era una URI, a una URI que referencia a una instancia del respectivo CodeList. A modo de ejemplo se transformaron valores de los atributos de \textit{TenderStatus}, \textit{AwardStatus}, \textit{ContractStatus}, \textit{Classification Scheme} que referencian a las instancias que se encuentran en la ontología desarrollada. Los demás Codelist ya poseen las instancias respectivas dentro de la ontología desarrollada, pero la conversión de los mismos en los datos de la DNCP quedó como futuro trabajo.

Se realizó el mismo procedimiento con los demás objetos pertenecientes al JSON. Como futuro trabajo se podrá optimizar el tiempo de carga del documento JSON-LD agregando contextos sólo cuando exista una colisión semántica, de esta manera se ahorrarán líneas de datos innecesarias.

Como siguiente paso queda convertir de manera programática los objetos publicados en JSON a JSON-LD para su posterior serialización a RDF.


\section{De JSON-LD a RDF}

La conversión de JSON-LD a RDF es un proceso directo. El sujeto (en JSON-LD), que podría ser usado como objeto en otra tripla, es definido por la propiedad @id, todas las demás propiedades son convertidas en predicado del objeto RDF resultante. Finalmente, los valores literales son extraídos directamente de la propiedad o del valor @value de la propiedad (si lo tuviere). Una propiedad puede tener también definido el lenguaje (@language) y/o tipo de informacion (@type).

Por ejemplo en la figura \ref{img:EjemploBaranJSONLD} se muestra un objeto JSON-LD, que posee un contexto definido dentro del mismo objeto, este contexto también podría estar definido dentro de un documento diferente y referenciado a través de una URI. Tomemos como ejemplo la propiedad foaf:name, la misma se convierte en predicado y el objeto toma el valor del literal “Benjamin Baran” definido en el idioma inglés según especifica @language.



\begin{figure}[h!]
    \centering
    \includegraphics[width=150mm]{figuras/BaranJSONLD.png}
    \caption{Objeto JSON-LD}
    \label{img:EjemploBaranJSONLD}
    \end{figure}




\begin{figure}[h!]
    \centering
    \includegraphics[width=150mm]{figuras/BaranRDF.png}
    \caption{Grafo RDF}
    \label{img:EjemploBaranRDF}
    \end{figure}




\section{Scrapper y convertidor de JSON-LD}

En la sección anterior se vió el proceso para convertir un objeto JSON a JSON-LD, en esta sección nos enfocaremos en automatizar el proceso de manera programática y tener una base de datos en formato RDF.

Como primer insumo se utilizó la API V2 de la DNCP, se utilizó como base un software desarrollado en Java por Diego Torres también para un trabajo de grado \footnote{https://github.com/diegotorrespy/scraper}, el mismo disponibilizó el trabajo bajo una licencia de libre uso. El software fue optimizado y adaptado según las necesidades de este trabajo, mejorando la velocidad de consulta de los registros y puesto de manera publica en un repositorio Git\footnote{https://github.com/camilobaezcamba/scraper}.

Como primer paso se utilizó el servicio de buscador de contratos. Se utilizó el buscador para extraer los identificadores de los procesos licitatorios en un rango de fechas. Para este trabajo se decidió trabajar con contrataciones del 1 de Enero del 2016 al 31 de Diciembre del 2016, ya que los mismos contemplan un año completo de ejercicio fiscal y las variaciones de los contratos son mínimas con relación al año 2018.



Como siguiente paso, con los Identificadores recogidos, se consultó la información del proceso licitatorio de los servicios del Estándar Open Contracting. Al realizar las experimentaciones previas se detectó de que el servicio tiende a fallar cuando se solicitan varios procesos licitatorios por segundo, por lo que el software hace varios intentos hasta conseguir descargar el registro. La cantidad y el intervalo entre cada intento es parametrizable dentro del software desarrollado.

El Récord package contiene mucha información, incluyendo todos los Releases del proceso licitatorio. Para este trabajo se consideró solamente el Compiled Release que se encuentra dentro del arreglo de Records, el mismo contiene un recopilado de todos los Releases y la versión final de todos los atributos. Con esta decisión, se pierde el historial de cambios que contempla el OCDS, aunque la DNCP tampoco implementó la publicación del historial de cambios.

El Compiled Release es guardado en un sistema de base de datos no relacional, que nos permite guardar la información en formato JSON, se utilizó este sistema por su versatilidad para el manejo de documentos con este formato.

El siguiente paso, luego de almacenar todos los procesos licitatorios en el sistema de base de datos  (MongoDB), fue la conversión del objeto JSON a JSON-LD esto se hace sistemáticamente identificando las propiedades que son de Tipo JsonObject y agregando las propiedades @id, @context y @type según el nombre de la propiedad, siguiendo el procedimiento y las salvedades expuestas anteriormente. Luego de la conversión se procede a guardar ese nuevo Objeto JSON en una nueva colección en la base de datos mencionada.

El proceso siguiente fue la conversión de los objetos JSON-LD a RDF utilizando la librería Apache Jena. Una vez incluida la librería en el proyecto fue posible realizar esta conversión pasando los objetos JSON-LD e indicando la ubicación del TDB en la que se almacenarán los datos en RDF. TDB es una capa de almacenamiento de grafos persistente para Jena.

De la misma manera se obtuvo los datos de todos los Proveedores del Estado del servicio en JSON-LD que brinda la DNCP, se almacenó en la base de datos MongoDB para su posterior conversión a RDF. 

El software desarrollado \footnote{http://bit.ly/ValdezBaezThesis} va desde la obtención de los datos hasta la transformación y almacenamiento a RDF. Para este trabajo se optó por utilizar Apache Jena, por el gran soporte para JAVA, que es el lenguaje utilizado en el Scrapper de datos. Además de contar con un servidor SPARQL llamado Apache Jena Fuseki.



\section{Punto SPARQL}

Para este trabajo se implementó el Jena Fuseki en un servidor para realizar las consultas. Cabe mencionar que todo lo desarrollado se hizo en un entorno experimental, como futuro trabajo queda hacer las optimizaciones correspondientes de tiempo de ejecución y uso de recursos para un entorno de producción o industrial.

Se consulta la colección de la base de datos MongDB generada y se utiliza la librería JENA para transformar y convertir la serialización de JSON-LD a RDF. Luego de la serialización se procede a guardar los datos en un TDB, un sistema de almacenamiento y consulta de RDF, para que luego este sirva como insumo para el punto SPARQL implementado utilizando Apache Jena FUSEKI.

Previo al despliegue del servidor fue necesario realizar una serie de configuraciones.

\begin{enumerate}
    \item Se indicaron los conjuntos de datos (TDB).
    \item Se configuraron 2 servicios independientes (Procesos Licitatorios y Proveedores)
    \item Se activaron los servicios de Razonamiento de FUSEKI indicando que utilizaremos las Reglas Básicas de Razonamiento de OWL. 
    \item Se realizaron ajustes a la cantidad máxima de memoria (RAM) utilizada ya que la configurada por defecto ocasiona interrupciones o paradas inesperadas del servidor. 

\end{enumerate}

Una vez desplegado el servidor se procedió a cargar las ontologías correspondientes. Desde este momento el punto SPARQL quedó totalmente configurado para realizar las consultas necesarias.

La Figura \ref{img:modelo de Implementacion} muestra el proceso de conversión, almacenamiento y disponibilización de datos.

\section{Conjuntos de datos para las pruebas de interoperabilidad}

Para probar el funcionamiento de la ontología creada se procedió al uso de una parte de la ontología implementada por la DNCP correspondiente a los proveedores del estado. Con el conjunto de datos de Proveedores podemos obtener datos adicionales sobre algún proveedor que se está presentando en algún proceso licitatorio.

Se utilizó para esto la API V2 de proveedores \footnote{\url{https://www.contrataciones.gov.py/datos/api/v2/doc/proveedores/}}, dicha API ya se encuentra publicada con el formato JSON-LD. Se realizaron consultas para obtener todos los registros de proveedores del estado y volcarlos a la base de datos MongoDB. Luego, utilizando la librería Jena, se procedió a convertir los objetos JSON-LD a RDF para posteriormente almacenarlos en la base de datos TDB.

Al momento de realizar las pruebas de interoperabilidad solamente se trabajó con un subconjunto (100) de los procesos licitatorios del año 2016. De esta manera se logró realizar consultas rápidas ya que se partió de una base de datos ligera y la intención fue probar la interoperabilidad semántica entre distintas fuentes de datos. Esto es meramente experimental y no fue una prioridad mejorar los tiempos de respuesta.

Como se había explicado anteriormente hemos utilizado los conjuntos de datos de Procesos Licitatorios y Proveedores para realizar consultas sobre el Punto SPARQL. A continuación expondremos algunos inconvenientes y las soluciones aplicadas referente a la manipulación de datos.

Para obtener los datos de la API V2 de la DNCP se procedió a registrarnos de manera a poder realizar peticiones sin limitaciones.

Al realizar la consulta a la API “Buscador de licitaciones” pasando como rango de fechas 01/01/2016 - 31/12/2016 se encontró que existen alrededor de 12.000 procesos licitatorios.
Dicho servicio retorna datos mínimos de cada proceso licitatorio de manera paginada, en este caso 1000. Luego, a partir del id de llamado (identificador único) se procedió a realizar las consultas al servicio de “Licitaciones” el cual está alineado al OCDS (estándar en el cual se basa la ontología desarrollada) para obtener los datos de manera más detallada.

En este proceso se identificaron varios inconvenientes en la comunicación con los servicios. El tiempo de respuesta se ve afectado por la cantidad de datos solicitados y por el intervalo de tiempo entre estas solicitudes de datos. Entonces se procedió a ajustar los parámetros de las peticiones, logrando conseguir un tiempo óptimo con las siguientes salvedades.

\begin{enumerate}

    \item Obtener los id de llamado (Buscador) de manera paginada (1000 registros).
    \item Cada 1000 registros obtenidos, lanzar 100 solicitudes paralelas de procesos licitatorios (Servicio OCDS) y esperar la respuesta de todas las peticiones para volver a realizar otras 100 solicitudes hasta completar el lote.
    \item A medida que se obtienen los procesos licitatorios se almacenan en la Base de datos MongoDB.
  
\end{enumerate}

La intención no fue optimizar el desempeño del software desarrollado sino lo necesario como para obtener los datos sin mayores inconvenientes y poder realizar todo el proceso de manera automatizada.










 \section{Discusión del Capítulo }

 Se puso a conocimiento los procesos realizados con relación a la manipulación de los datos así como también las cuestiones de implementación del software desarrollado y la utilización y configuración del Punto SPARQL (Jena Fuseki). Con todo esto puesto a punto, en la siguiente sección explicaremos el proceso de pruebas y resultados obtenidos.




%!TEX root = ../main.tex
\chapter{Contexto experimental}
\label{chap:Contexto experimental}


En esta sección se expondrán los experimentos realizados basados en distintas fuentes de datos y distintas ontologías. La intención de este capítulo es demostrar que gracias a la utilización de ontologías es posible la interoperabilidad semántica y sintáctica entre distintas fuentes de datos. Para esto mostramos como realizar consultas al Punto SPARQL y pretendemos dar una pequeña demostración de casos prácticos utilizando fuentes de datos y ontologías reales ya existentes. 

Luego del despliegue del Punto SPARQL se procedió a organizar los experimentos. Para ello se definió el siguiente procedimiento:

Clasificar las consultas en casos específicos según complejidad sintáctica, semántica o computacional que se requiere teniendo también en cuenta el conjunto de fuentes de datos y ontologías utilizados para realizar las consultas correspondientes.
En cada caso se definieron consultas lo suficientemente claras y concisas que demuestren el funcionamiento correcto de la ontología utilizada para esa fuente de datos.

En el \textbf{caso 1} mostramos las bondades de trabajar con triplas RDF y ontologías. En el \textbf{caso 2} consiste en mostrar la integración e interoperabilidad de dos fuentes y la facilidad de realizar consultas a modo de enriquecer la información obtenida. En el \textbf{caso 3} se pretende mostrar la integración e interoperabilidad con una ontología de dominio y una ontología general, para esto se utilizaron los datos de Procesos Licitatorios de la DNCP y también los datos disponibles en Wikidata. En el \textbf{Caso 4} se pretende mostrar la capacidad de reutilización e inferencia para la integración de datos utilizando la fuente de Procesos Licitatorios de la DNCP y la fuente de Procesos Licitatorios de Zaragoza, donde ambos reutilizan la ontología de Public Contract dentro de sus respectivas ontologías. Por último, en el \textbf{caso 5} se mejora la expresividad semántica de la ontología a fin de asegurar la correcta interpretación de los datos y facilita la consulta de los mismos.

En SPARQL es necesario definir primeramente todos los prefijos de las ontologías a utilizar en una determinada consulta. Los prefijos se definen de manera a facilitar la lectura de las consultas. De manera a facilitar la lectura de este documento las agrupamos en el Cuadro \ref{lst:prefijos}, los prefijos deben ir declarados antes de las consultas a realizar y solamente las ontologías utilizadas en esa consulta en específico.



\noindent\begin{minipage}[t]{\textwidth}
\begin{lstlisting}[captionpos=b, caption={Prefijos de las consultas SPARQL}, label={lst:prefijos},  numbers=left,  numberstyle=\tiny\color{mygray},frame=single]
    PREFIX foaf: <http://xmlns.com/foaf/0.1/>
    PREFIX co: <http://rhizomik.net/ontologies/copyrightonto.owl#>
    PREFIX oc: <http://opencoinage.org/rdf/>
    PREFIX ocd: <http://dati.camera.it/ocd/>
    PREFIX rdf: <http://www.w3.org/1999/02/22-rdf-syntax-ns#>
    PREFIX owl: <http://www.w3.org/2002/07/owl#>
    PREFIX lic: <http://example.org/licitaciones>
    PREFIX prov: <http://example.org/proveedores>
    PREFIX ocds: <http://purl.org/onto-ocds/ocds#>
    PREFIX dncp: <http://purl.org/onto-ocds/ocds/dncp.owl#>
    PREFIX pc: <http://purl.org/procurement/public-contracts#>
    PREFIX xsd:   <http://www.w3.org/2001/XMLSchema#>
    PREFIX fd: <http://foodable.co/ns/>
    PREFIX wd: <http://www.wikidata.org/entity/>
    PREFIX wdt: <http://www.wikidata.org/prop/direct/>
    PREFIX wikibase: <http://wikiba.se/ontology#>
    PREFIX p: <http://www.wikidata.org/prop/>
    PREFIX ps: <http://www.wikidata.org/prop/statement/>
    PREFIX pq: <http://www.wikidata.org/prop/qualifier/>
    PREFIX rdfs: <http://www.w3.org/2000/01/rdf-schema#>
    PREFIX bd: <http://www.bigdata.com/rdf#>
    PREFIX dbpedia-owl: <http://dbpedia.org/ontology/>
    PREFIX pproc: <http://contsem.unizar.es/def/sector-publico/pproc#> 
    PREFIX dcterms: <http://purl.org/dc/terms/> 
    PREFIX pc: <http://purl.org/procurement/public-contracts#> 
    PREFIX s: <http://schema.org/> 
    
 \end{lstlisting}
\end{minipage}

 A continuación se explica en cada caso los datos utilizados y cual es la intención. También se muestra cuál fue la consulta realizada al Punto SPARQL.


\section{Caso 1. Consultas de una fuente de datos a una ontología}


Aquí se pretende dar una demostración de cómo realizar una consulta simple a una fuente de datos y una ontología. Para esto se utilizó la Ontología OCDS y los datos de la DNCP. La consulta consiste en obtener de la base de datos toda la información referente a un proceso licitatorio, escificamente el proceso cuyo identificacor es <http://www.contrataciones.gov.py/datos/api/v2/doc/release/302438-adquisicion-semillas-gdc-1>, ya que es un Proceso Licitatorio que contemple las fases de Planificación, Convocatoria, Adjudicacion y Contratacion.

En el cuadro xx se muestra la consulta realizada al Punto SPARQL




\begin{lstlisting}[captionpos=b, caption=Información referente al proceso licitatorio cuyo identificacor es, label=lst:caso1,  numbers=left,  numberstyle=\tiny\color{mygray},
    basicstyle=\ttfamily,frame=single]

    SELECT *  
    where {    	
    <http://www.contrataciones.gov.py/datos/api/v2/doc/release/302438-adquisicion-semillas-gdc-1> rdf:type ocds:Release .
    <http://www.contrataciones.gov.py/datos/api/v2/doc/release/302438-adquisicion-semillas-gdc-1> ?propiedadNivel1 ?recursoNivel1 .   
    OPTIONAL { ?recursoNivel1 ?propiedadNivel2 ?recursoNivel2}
    }
    
    
 \end{lstlisting}


 Si se considera el resultado de esta consulta un árbol, donde la raíz es el Proceso Licitatorio, entonces la consulta traerá la información del proceso licitatorio hasta el segundo nivel del árbol.

En la imagen xx se muestran los resultados obtenidos. Los nodos de un nivel representan un recurso o un literal. Se puede observar que el resultado contiene tanto los datos propios del Proceso de Contratación (Release) como es “ocds:ReleaseTag” cuyo valor es “compiled” y también contiene datos del Comprador (ocds:buyer) del bien o servicio como puede ser el título (ocds:organizationName) cuyo valor es “Gobierno Departamental de Central”. El valor “compiled” fue posible obtener a partir de la sentencia de la línea 4 y el valor “Gobierno Departamental de Central” a partir de la sentencia de la línea 5.

En la línea 3 encontramos la restricción cuya intención es asegurar que el recurso se trate de un Release, esto es posible debido a la utilización de la ontología desarrollada. En la línea 4 se consulta por el conjunto de triplas que cumplan la condición de tener como sujeto la IRI del proceso licitatorio. Con dicha sentencia, se obtiene una serie de triplas correspondientes a todas las relaciones de dicho proceso. Con la sentencia de la línea 5 se obtienen las triplas que tengan como sujeto el objeto proveniente del resultado de la primera sentencia. En otras otras palabras, se obtienen datos provenientes del segundo nivel del árbol. Haciendo esta segunda sentencia opcional, se puede también desplegar aquellas triplas que no tienen un segundo nivel en el árbol, por ende se muestran los datos directos del Proceso Licitatorio.

Como podemos ver en el cuadro xx, no es necesario conocer las propiedades, tipos de datos ni relaciones existentes entre los conceptos para realizar la consulta, es decir nos permite realizar consultas donde el modelo de datos no es necesariamente conocido. El único dato necesario fue la URI de un Proceso de Contratación. En cambio, si comparamos con una consulta en una base de datos relacional, para obtener todos los datos de un Proceso de Contratación, posiblemente se requeriría realizar un JOIN entre varias tablas relacionadas. Para poder realizar esto necesitaríamos conocer cuáles son las tablas involucradas y cuáles son las columnas para aplicar el JOIN correctamente.


\begin{figure}[h!]
    \centering
    \includegraphics[width=150mm]{figuras/caso1Resultado.png}
    \caption{Despliegue de resultado de la consulta de l caso 1}
    \label{img:caso1Resultado}
 \end{figure}

\section{Caso 2. Dos fuentes de datos y dos ontologías.}
\label{section:caso2}

Este caso consiste en la integración e interoperabilidad de dos fuentes con semántica enriquecida por el uso de dos ontologías. Se pretende demostrar la facilidad de realizar consultas a dos fuentes de datos a modo de enriquecer la información obtenida. El caso presenta una situación en el que un mismo proveedor de datos posee dos fuentes de datos distintas, que tienen que interoperar entre si a modo de enriquecer la información obtenida. Este escenario es muy común ya que normalmente aunque nos referimos a un mismo organismo publicador de datos éste posee sistemas de información distintos que pueden o no estar interconectados.


En el capítulo \ref{chap:Implementación de la Ontologia} se desarrolló un sistema que recogió estos datos y los almacenó en dos almacenes de datos distintos y los disponibilizó en dos Puntos SPARQL distintos. En este caso el publicador de los datos es el mismo, pero las ontologías son distintas. Se ejecutó una consulta al Punto SPARQL que utiliza por un lado la ontología de OCDS y los datos de los procesos licitatorios de la DNCP y por otro lado la ontología desarrollada por la DNCP y los datos de Proveedores. En la Figura \ref{img:Diagramacaso2Endpoint} se puede describir gráficamente la situación. Si bien es posible que el mismo proveedor conozca y pueda interconectar internamente entre si ambas fuentes de datos a través de comandos SQL, esto no es verdad para un agente externo de la DNCP, que es el caso dentro de la web semántica, quien no conoce los datos y posee únicamente estos dos puntos de accesos independientes uno de otro.

 \begin{figure}[ht!]
    \centering
    \includegraphics[width=150mm]{figuras/Diagramas-Caso2-Ilustracion.png}
    \caption{Diagrama de fuentes de datos}
    \label{img:Diagramacaso2Endpoint}
 \end{figure}


La consulta presentada en el Cuadro \ref{lst:caso2} consiste en obtener datos de una organización cuyo RUC es 80008004-1. Se utiliza el comando SERVICE para realizar consultas a una fuente de datos distinta indicando su URL.

\noindent\begin{minipage}[c]{\textwidth}
\begin{lstlisting}[captionpos=b, caption={Consulta a dos fuentes de datos}, label={lst:caso2},  numbers=left,  numberstyle=\tiny\color{mygray},frame=single]
SELECT  *
    WHERE {   
            { 
                SELECT DISTINCT   ?c ?d
                WHERE { 
                    ?organization rdf:type          ocds:Organization ;
                                  ocds:identifier   ?identifier .
                    ?identifier   ocds:identifierId "80008004-1" .
                    ?organization ?a                ?b
                    OPTIONAL { 
                        ?b  ?c  ?d 
                    }
                  }
            }
            UNION {
                SERVICE <http://localhost:3030/proveedores_completas_no_inf/sparql>
                  {  ?empresa rdf:type dncp:Proveedor ;
                              dncp:ruc "80008004-1" ;
                              ?x       ?y
                  }
            }
    }
    
 \end{lstlisting}
\end{minipage}

 En la Figura \ref{img:caso2Resultado} se observan en la columna izquierda los datos obtenidos a partir de los Procesos Licitatorios utilizando la ontología OCDSPY y en la columna derecha se observan los datos obtenidos a partir de la fuente de datos de Proveedores que utiliza la ontología de la DNCP.


 \begin{figure}[ht!]
    \centering
    \includegraphics[width=150mm]{figuras/caso2Resultado.png}
    \caption{Despliegue de resultado de la consulta del caso 2}
    \label{img:caso2Resultado}
 \end{figure}



A continuación se presentan las mejoras obtenidas con relación a los datos publicados en el portal de datos de la DNCP.
 
 \subsection{Mejora 1: Igualdad semántica de propiedades}
 Se puede observar que ambas fuentes de datos se relacionan entre sí a partir de la propiedad \textit{ocds:identifierId} y \textit{dncp:ruc}. Sin bien es posible realizar esta relación debido a que un experto de dominio puede deducir que ambas propiedades denotan los mismos valores, esta igualdad no está explícitamente definida en la ontología. 
 
 En el caso que se desee expresar de manera explícita esta información bastaría agregar la propiedad \textit{owl:equivalentProperty}  a la definición de la propiedad \textit{identifierId}. Un ejemplo de cómo quedaría la definición de la propiedad \textit{identifierId} puede verse en el Cuadro \ref{lst:caso2-2}. Con esto es posible utilizar indistintamente cualquiera de las dos propiedades para referirse a la misma relación.\hfill \break

\noindent\begin{minipage}{\textwidth}
 \begin{lstlisting}[captionpos=b, caption={Declaración de equivalencia semántica entre dncp:ruc y ocds:identifierID}, label=lst:caso2-2,  numbers=left,  numberstyle=\tiny\color{mygray},frame=single]
###  http://purl.org/onto-ocds/ocds#identifierId
:identifierId rdf:type owl:DatatypeProperty ,
                        owl:FunctionalProperty ;
                owl:equivalentProperty dncp:ruc ;
                rdfs:domain :Identifier ;
                rdfs:range xsd:integer ,
                            xsd:string ;
                rdfs:comment "The identifier of the organization in the selected scheme."@en ;
                rdfs:label "Identifier ID"@en .
 \end{lstlisting}
\end{minipage}

 Las propiedades equivalentes (\textit{owl:equivalentProperty}) tienen los mismos valores, pero pueden ser semánticamente diferentes, en este caso \textit{dncp:ruc} y \textit{ocds:identifierId} tienen el mismo valor, y \textit{dncp:ruc} se refiere al “Código Único del Contribuyente (RUC) de la Secretaría de Estado y Tributación (SET)” y \textit{ocds:identifierID} se refiere al “Identificador Único de la Organización”. Equivalencia no quiere decir que sean iguales. Para indicar que dos propiedades son iguales se debe utilizar la propiedad \textit{owl:sameAs}
 En la Tabla \ref{table:semanticaID} se expresa la comparación entre las propiedades mencionadas.

 \begin{table}[!htb]
    \centering
    \caption{Definición Semántica de Indentificador y RUC}
    \label{table:semanticaID}
    \resizebox{15cm}{!} {
    \begin{tabular}{|c|l|l|}
    \hline
    
Propiedad & Valor &  Significado \\ \hline

ocds:identifierID  & 80008004-1 &  Identificador de la Organización \\ \hline
dncp:ruc & 80008004-1 &  Código Único del Contribuyente de la Secretaría de Estado y Tributación \\ \hline

    \end{tabular}
    }
    \end{table}
    
 Sin Embargo, Que dos individuos tengan una propiedad con el mismo valor no necesariamente implica que se trate del mismo individuo. En ontologías, a menos que haya una sentencia que indique que dos individuos son iguales o diferentes, no se puede asumir ninguna de las dos situaciones. Esto situación da pie a la mejora 2.

 \subsection{Mejora 2: Igualdad semántica de instancias}
 \label{section:caso2mejora2}
 Como en la fuente de datos no está definida la equivalencia entre ambas instancias, entonces los datos obtenidos de las distintas fuentes deben ser asumidos como iguales por el conocimiento del experto de dominio. La forma de expresar explícitamente la igualdad semántica entre dos instancias seria agregando la sentencia de Cuadro \ref{lst:caso2-1}. En la Figura \ref{img:DiagramaCaso3} se observa como quedaría esta unión semántica entre ambas instancias. Esto aseguraría de que una maquina pueda interpretar la equivalencia entre ambas instancias. 
 
 
\noindent\begin{minipage}[c]{\textwidth}
    \begin{lstlisting}[captionpos=b, caption=Declaración de igualdad semántica de dos instancias, label=lst:caso2-1,  numbers=left,  numberstyle=\tiny\color{mygray},frame=single]
<http://www.contrataciones.gov.py/datos/api/v2/doc/proveedores/ruc/80008004-1> 
owl:sameAs <http://www.contrataciones.gov.py/datos/id/proveedores/fax-comtel-srl>  .

     \end{lstlisting}
\end{minipage}
     \begin{figure}[ht!]
        \centering
        \includegraphics[width=150mm]{figuras/Diagramas-Caso2.png}
        \caption{Grafo de la consulta del caso 2}
        \label{img:DiagramaCaso2}
     \end{figure}

Cabe mencionar, que no se debe hacer una cambio de la sintaxis o de la estructura de los mismos, que podría hacerse llamando a ambas instancias o propiedades con el mismo identificador o mismo nombre de propiedad, sino que solamente bastará por hacer esa conexión semántica.

En este caso se observa que ambos conjuntos de datos pertenecen al mismo Proveedor/Publicador y los datos de las organizaciones se complementan en cuanto a información se refiere. Por una parte podemos obtener datos como el nombre del punto de contacto (\textit{ocds:contactPointName}), email y teléfono. Por otra parte se pueden obtener datos como el código de tipo de proveedor, el cual en este ejemplo es “SRL”. 

    
\section{Caso 3. Dos fuentes de datos donde una es externa y dos ontologías}

Se pretende demostrar la integración e interoperabilidad con una ontología específica y una genérica. Para esto, se utilizaron los datos de Procesos Licitatorios de la DNCP junto con la ontología OCDS, y también los datos disponibles en Wikidata junto con su ontología propia. Con esto se enriqueció la fuente de datos primaria a partir de una fuente de datos externa, la cual permitió obtener más información relevante y contextual.

La consulta realizada en el cuadro  \ref{lst:caso3-1}consiste en obtener la cantidad de todos los Procesos Licitatorios cuyos proveedores residen en países que no son limítrofes con Paraguay, agrupando estas cantidades de acuerdo al país.

\noindent\begin{minipage}[c]{\textwidth}
\begin{lstlisting}[captionpos=b, caption=Integracion con una fuente de datos externa utilizando dos ontologias, label=lst:caso3-1,  numbers=left,  numberstyle=\tiny\color{mygray},
    basicstyle=\footnotesize\ttfamily,frame=single]
select *
    where {
      {
      select ?label (COUNT(?label) AS ?cantidad)
    where {
    ?plicitatorio rdf:type ocds:Release;
                  ocds:contracts ?contract.
      ?contract ocds:contractSuppliers ?org.
      ?org ocds:address ?b .
    ?b ocds:countryName ?label
          MINUS  { SERVICE <https://query.wikidata.org/sparql>
                        { 
        SELECT DISTINCT  (str(?labelWhitLang) as ?label)
            WHERE
                { wd:Q733   wdt:P47    ?paislimitrofe .
                
                    ?paislimitrofe wdt:P297    ?codigoPais.
                            ?paislimitrofe     rdfs:label ?labelWhitLang. 
                    FILTER (langMatches( lang(?labelWhitLang), "ES" ) )
                }        
        }
        }
        } group by ?label
      }
    
    }
 \end{lstlisting}
\end{minipage}

Primeramente se seleccionan todos los Procesos Licitatorios junto con el país donde residen (\textit{ocds:countryName}) excluyendo (MINUS) todos los países que sean limítrofes con Paraguay. En Wikidata, wd:Q733 es el identificador de país de Paraguay mientras que wdt:P47 es el identificador de la propiedad “comparte frontera con” y rdf:label es el nombre de país. El comando SERVICE nos permite realizar consultas a otro punto Sparql, en este caso Wikidata. La única información necesaria para lograr este enlace es conocer la URI que representa el Punto Sparql para así poder realizar consultas sobre los datos externos.


Se utilizó el nombre del país en ambos conjuntos de datos como elemento de relación entre países para conseguir suprimir del resultado los países que son limítrofes con Paraguay. Se tomó la decisión de utilizar este atributo debido a que no hay una relación directa entre un país en el conjunto de datos de la DNCP y el mismo país en el conjunto de datos de Wikidata. En la Figura \ref{img:caso3Resultado1} se muestran los resultados obtenidos.


\begin{figure}[ht!]
    \centering
    \includegraphics[width=150mm]{figuras/caso3Resultado1.png}
    \caption{Despliegue de resultado de la consulta del caso 3}
    \label{img:caso3Resultado1}
 \end{figure}


En la Figura \ref{img:caso3Resultado2} se ve el resultado si se suprime el filtro por países limítrofes correspondientes a las líneas desde la 11 a la 24. En la Figura \ref{img:DiagramaCaso3} se presenta un diagrama de relaciones entre las dos fuentes de datos, las ontologías utilizadas y la consulta.


 \begin{figure}[ht!]
    \centering
    \includegraphics[width=150mm]{figuras/caso3Resultado2.png}
    \caption{Despliegue de resultado de la consulta del caso 3}
    \label{img:caso3Resultado2}
 \end{figure}


 \begin{figure}[ht!]
    \centering
    \includegraphics[width=150mm]{figuras/Diagramas-Caso3.png}
    \caption{Despliegue de resultado de la consulta del caso 3}
    \label{img:DiagramaCaso3}
 \end{figure}

 Esta consulta podría tener inconvenientes si la forma de escribir del nombre de los país cambiara, es el caso de ISRAEL en la línea 5 de la Figura \ref{img:caso3Resultado1} que se encuentra en mayúsculas. Para evitar esto, lo óptimo sería que los atributos no sean de tipo texto sino más bien sean objetos dentro del conjunto de datos de la DNCP y éstos estén relacionados con otros semejantes dentro de la Web Semántica.

En el cuadro \ref{lst:3-2} se muestra la definición actual de la propiedad “countryName” que pertenece a la clase “Address". Se observa que el tipo de propiedad es “DatatypeProperty” el cual representa un literal.\hfill \break 

\noindent\begin{minipage}[c]{\textwidth}
\begin{lstlisting}[captionpos=b, caption=Definicion de la propiedad countryName, label={lst:3-2},  numbers=left,  numberstyle=\tiny\color{mygray},
    basicstyle=\footnotesize\ttfamily,frame=single]
###  http://purl.org/onto-ocds/ocds#countryName
:countryName rdf:type owl:DatatypeProperty ,
    owl:FunctionalProperty ;
    rdfs:domain :Address ;
    rdfs:range xsd:string ;
    rdfs:comment "The country name. For example, United States."@en ;
    rdfs:label "Country name"@en .
 \end{lstlisting}
\end{minipage}

 En el cuadro {lst:caso3-3} se muestra una posible solución al inconveniente mencionado. Primeramente se debe renombrar la propiedad \textit{countryName} a \textit{country} para que de esta forma denote la relación con un país y no con el nombre del país solamente. Luego se debe cambiar el rango de valores de \textit{rdfs: range xsd:string}  a \textit{rdfs:range :Country }donde \textit{Country} se refiere a una nueva clase a ser definida en la ontología. Esta última será la que represente al concepto “País”. Luego deberá definirse una instancia del país (por cada país que exista en la fuente de datos) de tipo \textit{Country}. Finalmente faltaría relacionar este recurso con otro semejante dentro de la Web Semántica, en este caso se puede realizar con los datos disponibles en Wikidata a través de la sentencia \textit{owl:sameAs wd:Q733} donde \textit{wd:Q733} es el identificador único (URI) de Paraguay en Wikidata.\hfill \break

\noindent\begin{minipage}[c]{\textwidth}
 \begin{lstlisting}[captionpos=b, caption=Declaracion de la Clase Country, label={lst:caso3-3},  numbers=left,  numberstyle=\tiny\color{mygray},
    basicstyle=\footnotesize\ttfamily,frame=single]
PREFIX wd: <http://www.wikidata.org/entity/>
###  http://purl.org/onto-ocds/ocds#country
:country rdf:type owl:ObjectProperty ,
                      owl:FunctionalProperty ;
             rdfs:domain :Address ;
             rdfs:range :Country ;
             rdfs:comment "El pais donde reside"@es ;
             rdfs:label "Pais de residencia"@en .

###  http://purl.org/onto-ocds/ocds#Country
:Country rdf:type owl:Class ;
         rdfs:comment "Representa un pais."@es ;
         rdfs:label "Pais"@es .

###  http://purl.org/onto-ocds/ocds#countryParaguay
:countryParaguay rdf:type owl:NamedIndividual ,
                                    :Country ;
                           dcterms:title "Republica del Paraguay"@es ;
                           rdfs:label "Paraguay"@es .
    owl:sameAs wd:Q733
 \end{lstlisting}
\end{minipage}

\section{Caso 4. Inferencia y Reutilización de ontologías. }

En este caso se pretende demostrar la capacidad de reutilización e inferencia para la integración de datos. Utilizamos la fuente de Procesos Licitatorios de la DNCP y la fuente de Procesos Licitatorios de Zaragoza, donde ambos reutilizan la ontología de Public Contract dentro de sus respectivas ontologías.

Para obtener las convocatorias de la DNCP es necesario que el Punto SPARQL utiliza el motor de inferencia ya que los datos no están anotados directamente con la propiedad \textit{pc:Tender} pero sí están anotados con \textit{ocds:Tender} la cual, en la ontología de OCDS, está definida como una subclase de \textit{pc:Tender} (como se muestra en el Cuadro \ref{lst:caso4-1}). \hfill \break

\noindent\begin{minipage}[c]{\textwidth}
\begin{lstlisting}[captionpos=b, caption=Extensión de la ontología reutilizando PC, label={lst:caso4-1},  numbers=left,  numberstyle=\tiny\color{mygray},frame=single]
###  http://purl.org/onto-ocds/ocds#Tender
:Tender rdf:type owl:Class ;
        rdfs:subClassOf pc:Tender ;
        rdfs:comment "Data regarding tender process - publicly inviting prospective contractors to submit bids for evaluation and selecting a winner or winners"@en ;
        rdfs:label "Tender"@en .
 \end{lstlisting}
\end{minipage}
 

 La consulta realizada en el Cuadro \ref{lst:caso4-2} consiste en listar los primeros 6 contratos encontrados (3 de Paraguay y 3 de Zaragoza)”.\hfill \break
 
\noindent\begin{minipage}[c]{\textwidth}
 \begin{lstlisting}[captionpos=b, caption=Consulta a dos fuentes de datos utilizando el mismo concepto, label={lst:caso4-2},  numbers=left,  numberstyle=\tiny\color{mygray},frame=single]
SELECT ?tender
WHERE {
    #Paraguay
    { 
        SELECT * 
        WHERE {
            ?tender ?b pc:Tender 
        } 
        LIMIT 3 
    }
    UNION 
    #Zaragoza
    { 
        SERVICE <http://datos.zaragoza.es/sparql> {
            SELECT * 
            WHERE {
                ?tender ?b pc:Tender 
            } 
            LIMIT 3
        }
    }
}
 \end{lstlisting}
\end{minipage}
 Los resultados obtenidos se muestran en la Figura \ref{img:caso4Resultado}. Allí se muestra el identificador de cada convocatoria. Con esta consulta, podemos estar seguros de que todos los recursos de la lista corresponden a convocatorias, aunque las convocatorias no son del mismo publicador de datos.



\begin{figure}[ht!]
    \centering
    \includegraphics[width=150mm]{figuras/caso4Resultado.png}
    \caption{Despliegue de resultado de la consulta del caso 4}
    \label{img:caso4Resultado}
 \end{figure}


 En la Figura \ref{img:Diagramas-Caso 4} se observa que ambas fuentes de datos reutilizan la ontología de Public Contract, en este caso la propiedad “Tender”. Gracias a esto es posible relacionar los datos en una consulta al Punto Sparql logrando así la interoperabilidad semántica, dando seguridad de que ambas fuentes de datos se refieren al mismo concepto.

 \begin{figure}[ht!]
    \centering
    \includegraphics[width=150mm]{figuras/Diagramas-Caso4.png}
    \caption{Diagrama de la consulta del caso 4}
    \label{img:Diagramas-Caso 4}
 \end{figure}

\section{Caso 5. Mejora de la expresividad de la ontología. }

Se puede extender la capacidad semántica de la ontología actualizando el modelo ontológico sin que implique un alto costo debido a que éste se encuentra desacoplado de los datos.

Para este caso se utilizó la fuente de datos de procesos licitatorios y la ontología OCDS. Se define el concepto de “Proceso Licitatorio con Contrato” a través de una una clase (\textit{PLConContrato}) a partir de una restricción de propiedad. Esta clase se define como una subclase de \textit{Release}, que tiene como restricción todas las instancias de la clase \textit{Release} que tengan al menos una relación en la propiedad \textit{ocds:contracts}.\hfill \break


\noindent\begin{minipage}[c]{\textwidth}
\begin{lstlisting}[captionpos=b, caption=Extension de la ontologia utilizando restricciones ontologicas, label={lst:caso5-1},  numbers=left,  numberstyle=\tiny\color{mygray},frame=single]
INSERT DATA {
    ocds:PLConContrato rdfs:subClassOf ocds:Release ; 
    owl:equivalentClass    [ 
        rdf:type owl:Class ;
        owl:intersectionOf (   
            ocds:Release [ rdf:type owl:Restriction ;
                                        owl:onProperty ocds:contracts; 
                    owl:minCardinality "1"^^xsd:nonNegativeInteger ;
                            ] )
                ] .
}
    
 \end{lstlisting}
\end{minipage}
 De esta manera basta solo con realizar la consulta SPARQL presentada en el Cuadro \ref{lst:caso5-2} para obtener todos los procesos licitatorios con al menos un contrato. La restricción consigue filtrar solo los \textit{Release} cuya propiedad \textit{contracts} tiene al menos una instancia.\hfill \break

\noindent\begin{minipage}[c]{\textwidth}
 \begin{lstlisting}[captionpos=b, caption=Consulta SPARQL utilizando la Clase PLConContrato, label=lst:caso5-2,  numbers=left,  numberstyle=\tiny\color{mygray},frame=single]
SELECT  *
    WHERE
      { ?a  ?b  ocds:PLConContrato }
    
 \end{lstlisting}
\end{minipage}
 Esta capacidad de extender la ontología nos permite hacer consultas de manera simple sin la necesidad de hacer modificaciones a los datos y agregando conocimiento de dominio a la ontología

\section{Análisis}
Cabe mencionar que para lograr la interoperabilidad tanto sintáctica como semántica fue necesario un previo procesamiento de los datos. Primeramente se procedió a la obtención de las muestras y luego a su preparación (ver \ref{chap:Implementación de la Ontologia}) para finalmente disponibilizarlos en un Punto SPARQL. Un escenario ideal sería en el que las fuentes de datos (utilizando ontologías) estén disponibles en formato RDF en un Punto SPARQL.

Teniendo en cuenta las fuentes de datos utilizadas en los experimentos, si quisiéramos lograr la interoperabilidad sintáctica y semántica sin el uso de la Web Semántica sería más costoso en términos de preparación de los datos o inclusive quizás no sería posible. Por ejemplo: los datos de Procesos Licitatorios disponibles en la API de la DNCP que siguen los estándares de la OCDS podrían ser enlazados con datos de otros países que también sigan las especificaciones de la OCDS pero esta relación no se encuentra explícita en los datos. Por ende, según la definición de Interoperabilidad Semántica (ver \ref{section:interoperabilidadsemantica}), solo se lograría la interoperabilidad sintáctica.

\section{Discusión del Capítulo}

En este capítulo se vieron diversos casos en como pueden ser realizadas las consultas teniendo en cuenta la diversidad de fuentes de datos y ontologías que pueden ser utilizadas para lograr la interoperabilidad tanto sintáctica como semántica. Con ayuda de cada uno de ellos podemos responder a las preguntas y los objetivos propuestos para este trabajo.
 
En el siguiente capítulo analizaremos y expondremos los objetivos alcanzados.



%%!TEX root = ../main.tex
\chapter{Conclusiones}
\label{chap:Conclusiones}


% %!TEX root = ../main.tex
\chapter{Resultados}
\label{chap:resultados}

En este capítulo se definen las métricas utilizadas para la evaluación de la extensión de la transformada de watershed implementada y se presentan los resultados obtenidos en los experimentos.

\section{Métricas de Evaluación}
El objetivo de la segmentación es separar los objetos deseados del fondo para el estudio futuro~\cite{carrasco}. A partir de los objetos segmentados se pueden clasificar los píxeles en cuatro resultados posibles:
\addsymbol{symbol:VP} \addsymbol{symbol:VN} 
\addsymbol{symbol:FP} \addsymbol{symbol:FN} 
\begin{itemize}
\item Verdadero Positivo ($VP$), píxel que pertenece al objeto segmentado y es segmentado como tal.
\item Falso Positivo ($FP$), píxel que no pertenece al objeto segmentado y es segmentado como parte del mismo erróneamente.
\item Verdadero Negativo ($VN$), píxel que no pertenece al objeto segmentado y no es segmentado como parte del mismo.
\item Falso Negativo ($FN$), píxel que pertenece al objeto segmentado y no es segmentado como tal.
\end{itemize}
La clasificación de los resultados obtenidos para las distintas variables se puede observar en la Figura \ref{img:vpfp}. Las zonas en donde el objeto a segmentar y la segmentación realizada se interceptan corresponde a los píxeles considerados como $VP$, en donde el objeto no intercepta a la segmentación realizada corresponde a los $FN$. Posteriormente los píxeles que son parte de la segmentación realizada pero no el objeto son los $FP$ y los píxeles del fondo de la imagen que si corresponden al fondo son los $VN$.
\begin{figure}[H]
\centering
\includegraphics[width=160mm]{./imagenes/metricas-imagen.png}
\caption{Resultados posibles de la segmentación}
\label{img:vpfp}
\end{figure}
Para el problema estudiado la imagen ideal representa la segmentación hecha por un profesional, y la imagen segmentada representa la segmentación de los amastigotes obtenidos automáticamente. Se puede considerar que los $VP$ son los píxeles que pertenecen al amastigote y fueron segmentados correctamente, los $FP$ son los píxeles que fueron segmentados como parte de el amastigote pero no lo son, los $FN$ son los píxeles que forman parte del amastigote pero no fueron segmentados como tal y los $VN$ son los píxeles que forman parte del fondo y fueron segmentados correctamente.

Para la comparación de los resultados obtenidos, se utilizan las siguientes métricas:
\begin{itemize}
\item Sensibilidad ($SEN$): La probabilidad de segmentar correctamente un amastigote.
\begin{equation}\label{eq:sensibilidad}
  SEN = \cfrac{VP}{VP + FN}.
\end{equation}
\item Especificidad ($ESP$): La probabilidad de segmentar correctamente el fondo.
\begin{equation}\label{eq:especificidad}
  ESP = \cfrac{VN}{VN + FP}.
\end{equation}
\item Exactitud ($EX$): La probabilidad de que un amastigote o el fondo obtengan una segmentación correcta.
\begin{equation}\label{eq:exactitud}
  EX = \cfrac{VP + VN}{VP + VN + FP + FN}.
\end{equation}
\end{itemize}
%!TEX root = ../main.tex
\section{Experimentos}
Para los experimentos se utilizó una base de datos con 100 imágenes microscópicas de células infectadas con amastigotes de Trypanosoma Cruzi~\cite{noguera2013mathematical}.

Los diferentes órdenes  utilizados se muestran en la Tabla \ref{metodos-ordenamiento}. Los métodos propuestos son los que utilizan la distancia a un color de referencia en el espacio de color CIELab, en nuestro caso se propuso la utilización de la distancia euclidiana al color origen, media y mediana de la imagen. Para las imágenes en escala de grises se utilizó el orden natural entre las intensidades de los píxeles, en el espacio de color HSI se utilizó el orden lexicográfico y en el espacio de color RGB se utilizaron los órdenes lexicográfico, alpha-lexicográfico, el ordenamiento propuesto por Meyer y los ordenamientos propuesto por Vázquez.

\begin{table}[!htb]
\centering
\caption{Métodos de ordenamiento}
\label{metodos-ordenamiento}
\resizebox{15cm}{!} {
\begin{tabular}{|l|l|l|}
\hline
\multicolumn{1}{|c|}{\textbf{Método}} & \multicolumn{1}{c|}{\textbf{Espacio de color}} & \multicolumn{1}{c|}{\textbf{Observación}}  \\ \hline
\textbf{MEDIANA-CIELAB} & CIELab & Espacio propuesto \\ \hline
\textbf{MEDIA-CIELAB} & CIELab & Espacio propuesto \\ \hline
\textbf{DISTANCIA-EUCLIDIANA-CIELAB} & CIELab & Espacio propuesto \\ \hline
\textbf{LEXICOGRAFICO-RGB} & RGB & Sección \ref{chap:marco-lex} \\ \hline
\textbf{ALGORITMO-MEYER} & RGB & Sección \ref{chap:marco-meyer} \\ \hline
\textbf{ALPHA-MOD-LEXICOGRAFICO-RGB} & RGB & Sección \ref{chap:marco-alphalex} \\ \hline
\textbf{DISTANCIA-EUCLIDIANA-RGB} & RGB & Sección \ref{chap:marco-distanciaeuclidianta} \\ \hline
\textbf{ESCALA-DE-GRISES} & Escala de Gris & Sección \ref{chap:marco-lex}  \\ \hline
\textbf{VAZQUEZ-ET-AL} & RGB & Sección \ref{chap:marco-lex} \\ \hline
\textbf{ENTRELAZADO-RGB} & RGB & Sección \ref{chap:marco-entrelazado} \\ \hline
\textbf{LEXICOGRAFICO-HSI} & HSI & Sección \ref{chap:marco-lex} \\ \hline
\textbf{ENTROPIA-RGB} & RGB & Sección \ref{chap:marco-vazquez}  \\ \hline
\textbf{MAXIMO-RGB} & RGB & Sección \ref{chap:marco-vazquez}  \\ \hline
\textbf{MINIMO-RGB} & RGB & Sección \ref{chap:marco-vazquez} \\ \hline
\textbf{MODA-RGB} & RGB & Sección \ref{chap:marco-vazquez}  \\ \hline
\textbf{MODA-MAXIMO-RGB} & RGB & Sección \ref{chap:marco-vazquez} \\ \hline
\textbf{MODA-MINIMO-RGB} & RGB & Sección \ref{chap:marco-vazquez}  \\ \hline
\textbf{SUAVIDAD-RGB} & RGB & Sección \ref{chap:marco-vazquez} \\ \hline
\textbf{VARIANZA-RGB} & RGB & Sección \ref{chap:marco-vazquez}   \\ \hline
\end{tabular}
}
\end{table}


\section{Resultados Obtenidos}
\label{chap:resultados}

En esta sección se brinda un resumen de los resultados obtenidos al realizar los experimentos y se presentan ejemplos visuales de los experimentos.

\subsection{Resultados numéricos}

A continuación se muestran los resultados promedios obtenidos para los $VP$, $FP$, $VN$ y $FN$ en la Tabla \ref{resultados} (Ver resultados de manera extensiva en el Anexo \ref{chap:ApendiceA}). Los valores presentados muestran que el ordenamiento \textbf{MEDIANA-CIELAB} obtuvo los mayores resultados para las cuatro variables a considerar. En todos los resultados para los distintos métodos existe una gran cantidad de $FP$, esto indica que al realizar la segmentación los objetos obtenidos son de un tamaño mayor a los objetos en la imagen ideal. En los resultados se observa también que los métodos \textbf{MEDIANA-CIELAB} y \textbf{MEDIA-CIELAB} fueron los únicos que obtuvieron una cantidad mayor al 40\% en $VP$ por lo cual la distancia a colores de referencia que utilizan información de la imagen, como la mediana y la media, son los más adecuados para la segmentación de los amastigotes de Trypanosoma Cruzi.

\begin{table*}[]
\centering
\caption{Resultados de la Comparación}
\label{resultados}
\resizebox{15cm}{!} {
\begin{tabular}{|l|l|l|l|l|}
\hline
\multicolumn{1}{|c|}{\textbf{Método}} & \multicolumn{1}{c|}{\textbf{FP}} & \multicolumn{1}{c|}{\textbf{VP}} & \multicolumn{1}{c|}{\textbf{FN}}	&	\multicolumn{1}{c|}{\textbf{VN}} \\ \hline
\textbf{MEDIANA-CIELAB} & 53.9871 & 46.0129 & 3.241 & 96.759\\ \hline
\textbf{MEDIA-CIELAB} & 58.0464 & 41.9536 & 3.8889 & 96.1111\\ \hline
\textbf{DISTANCIA-EUCLIDIANA-CIELAB} & 62.382 & 37.618 & 3.388 & 96.612\\ \hline
\textbf{LEXICOGRAFICO-RGB} & 62.5968 & 37.4032 & 4.5841 & 95.4159\\ \hline
\textbf{ALGORITMO-MEYER} & 61.9621 & 38.0379 & 4.2933 & 95.7067 \\ \hline
\textbf{ALPHA-MOD-LEXICOGRAFICO-RGB} & 64.4714 & 32.6382 & 4.7328 & 95.2672\\ \hline
\textbf{DISTANCIA-EUCLIDIANA-RGB} & 64.4714 & 35.5286 & 4.6877 & 95.3123 \\ \hline
\textbf{ESCALA-DE-GRISES} & 62.4645 & 37.5355 & 4.9666 & 95.0334 \\ \hline
\textbf{VAZQUEZ-ET-AL} & 66.7119 & 33.2881 & 4.8933 & 95.1067\\ \hline
\textbf{ENTRELAZADO-RGB} & 67.171 & 32.829 & 5.5399 & 94.4601 \\ \hline
\textbf{LEXICOGRAFICO-HSI} & 65.1106 & 34.8894 & 5.6917 & 94.3083 \\ \hline
\textbf{ENTROPIA-RGB} & 66.6292 & 33.3708 & 5.5686 & 94.4314 \\ \hline
\textbf{MAXIMO-RGB} & 67.0666 & 32.9334 & 5.0553 & 94.9447 \\ \hline
\textbf{MINIMO-RGB} & 65.8297 & 34.1703 & 5.6771 & 94.3229 \\ \hline
\textbf{MODA-RGB} & 64.8527 & 35.1473 & 4.6959 & 95.3041 \\ \hline
\textbf{MODA-MAXIMO-RGB} & 64.8527 & 35.1473 & 4.6959 & 95.3041 \\ \hline
\textbf{MODA-MINIMO-RGB} & 64.8527 & 35.1473 & 4.6959 & 95.3041 \\ \hline
\textbf{SUAVIDAD-RGB} & 67.6638 & 32.3362 & 4.4133 & 95.5867 \\ \hline
\textbf{VARIANZA-RGB} & 66.7964 & 33.2036 & 4.8919 & 95.1081 \\ \hline
\end{tabular}
}
\end{table*}

La comparación de las métricas de sensibilidad, exactitud y especificidad de la implementación utilizando distintos métodos de orden entre píxeles, se puede observar en la Tabla \ref{resultadosCielab}. Los valores indican el promedio obtenido por cada método de ordenamiento para todas las imágenes.El método \textbf{MEDIANA-CIELAB} presenta una mejora de 3.14\% en Especificidad y 1.34\% en Exactitud en comparación con el siguiente mejor método. 


\begin{table}[]
\centering
\caption{Resultados en espacio de color CIELab}
\label{resultadosCielab}
\resizebox{15cm}{!} {
\begin{tabular}{|l|l|l|l|}
\hline
\multicolumn{1}{|c|}{\textbf{Método}} & \multicolumn{1}{c|}{\textbf{$SEN$}}& \multicolumn{1}{c|}{\textbf{$ESP$}}&  \multicolumn{1}{c|}{\textbf{$EX$}}  \\ \hline
\textbf{MEDIANA-CIELAB} & 26.18\%  & 93.19\%  & 91.14\%  \\ \hline
\textbf{MEDIA-CIELAB} & 27.14\%  & 92.95\% & 90.94\%  \\ \hline
\textbf{DISTANCIA-EUCLIDIANA-CIELAB} & 27.24\%  & 91.76\% & 89.80\% \\ \hline
\textbf{LEXICOGRAFICO-RGB} & 27.41\%  & 90.05\% & 88.17\% \\ \hline
\textbf{ALGORITMO-MEYER} & 27.02\%  & 90.40\%  & 88.48\% \\ \hline
\textbf{ALPHA-MOD-LEXICOGRAFICO-RGB} & 27.31\% & 89.14\% & 87.27\% \\ \hline
\textbf{DISTANCIA-EUCLIDIANA-RGB} & 27.40\% & 88.73\% & 86.8\%  \\ \hline
\textbf{ESCALA-DE-GRISES} & 25.5\%  & 88.69\%  & 86.81\%  \\ \hline
\textbf{VAZQUEZ-ET-AL} & 27.37\%  & 88.08\%  & 86.25\% \\ \hline
\textbf{ENTRELAZADO-RGB} & 27.27\% & 88.02\% & 86.17\% \\ \hline
\textbf{LEXICOGRAFICO-HSI} & 27.39\%  & 87.96\% & 86.15\%  \\ \hline
\textbf{ENTROPIA-RGB} & 27.24\%  & 87.53\%  & 85.73\% \\ \hline
\textbf{MAXIMO-RGB} & 27.36\%  & 87.40\%  & 85.62\%  \\ \hline
\textbf{MINIMO-RGB} & 27.53\%  & 87.95\%  & 86.15\%  \\ \hline
\textbf{MODA-RGB} & 27.48\%  & 89.03\%  & 87.15\% \\ \hline
\textbf{MODA-MAXIMO-RGB} & 27.48\%  & 89.03\% & 87.15\%   \\ \hline
\textbf{MODA-MINIMO-RGB} & 27.48\%  & 89.03\% &  87.15\%   \\ \hline
\textbf{SUAVIDAD-RGB} & 27.55\% & 88.17\%  & 86.34\% \\ \hline
\textbf{VARIANZA-RGB} & 27.47\%  & 88.17\%  & 86.34\% \\ \hline
\end{tabular}
}
\end{table}

Para obtener un detalle de los datos se contabilizó además la cantidad total de imágenes en las que cada espacio (3 métodos en CIELab, 14 métodos en RGB y 1 método en HSI) de color obtuvo el mejor exactitud (ver Figura \ref{exp:exactitud}). En este conteo se puede ver que el espacio de color CIELab obtuvo la mejor segmentación en la mayor cantidad de las imágenes.

\begin{figure}[h!]
\centering
\includegraphics[width=160mm]{./imagenes/exactitud.png}
\caption{Cantidad de imágenes que dan el mejor resultado en exactitud por método}
\label{exp:exactitud}
\end{figure}

\subsection{Resultados visuales}

A continuación se muestran los ejemplos visuales de los experimentos realizados en el trabajo. La Figura \ref{exp:lab} muestra los resultados obtenidos en el espacio de color CIELab. En la Figura \ref{exp:lab}(a) se muestra una imagen del caso de estudio. Además se muestra la gradiente en la Figura \ref{exp:lab}(b) utilizada para la segmentación. La segmentación obtenida en la Figura \ref{exp:lab}(c) y la segmentación ideal se muestran en la Figura \ref{exp:lab} (d). Este ejemplo será utilizado posteriormente para la comparación con los distintos experimentos realizados.

\begin{figure}[h!]
\centering
\includegraphics[width=150mm]{./imagenes/ejemplo-medialab.png}
\caption{Resultados obtenidos en el espacio de Color CIELab}
\label{exp:lab}
\end{figure}

\subsubsection{Comparación con imágenes en escala de grises}

Los resultados de la implementación utilizando el algoritmo de Vincent y Soille en escala de grises en la Figura \ref{exp:gris} muestran como los amastigotes segmentados en la esquina inferior izquierda en \ref{exp:gris}(c) poseen un tamaño mayor a lo que se ve en la segmentación ideal y en comparación con la segmentación realizada en el espacio CIELab.
\begin{figure}[h]
\centering
\includegraphics[width=150mm]{./imagenes/ejemplo-gris.png}
\caption{Resultados obtenidos utilizando el algoritmo de Vincent y Soille}
\label{exp:gris}
\end{figure}

En base a un análisis visual de los resultados se nota que los amastigotes segmentados tienden a ser de mayor tamaño que los esperados en la imagen ideal, esta tendencia se mantiene tanto en escala de gris como en CIELab pero en este último espacio se ve un tamaño más cercano a la imagen ideal, esto refleja la tendencia obtenida en las tablas anteriores.

\subsubsection{Comparación con imágenes a color}

Los resultados de la implementación utilizando el algoritmo de Meyer en el espacio de color RGB se pueden observar en la Figura \ref{exp:rgb}. 
\begin{figure}[h]
\centering
\includegraphics[width=150mm]{./imagenes/ejemplo-rgb.png}
\caption{Resultados obtenidos utilizando el algoritmo de Meyer}
\label{exp:rgb}
\end{figure}

Realizando un análisis visual de los resultados se puede ver que presenta la misma tendencia que al realizar la segmentación en escala de grises o en el espacio CIELab. Los amastigotes segmentados poseen un tamaño mayor al esperado en la imagen ideal así como lo indica la alta presencia de $FP$ en los resultados numéricos.

Comparando con los resultados obtenidos en escala de grises se puede ver una mejora al utilizar el color y comparando con la segmentación en el espacio de color CIELab se puede ver un resultado similar, tomando los amastigotes segmentados en la esquina inferior izquierda existe un mayor error de incremento de tamaño en el espacio de color RGB en comparación con el espacio de color CIELab.


En la Figura \ref{img:r1}(a) se muestra la imagen original y en \ref{img:r1}(b) la segmentación ideal.

\begin{figure}[H]
\centering
\includegraphics[width=100mm]{./imagenes/r1.png}
\caption{(a) Imagen original (b) Segmentación ideal}
\label{img:r1}
\end{figure}

En la Figura \ref{img:r2}(a) se muestra la segmentación obtenida en el espacio CIELab con el ordenamiento de la distancia a la mediana y en la Figura \ref{img:r2}(b) el espacio RGB utilizando el ordenamiento lexicógrafo, el cual obtuvo los mejores resultados entre los distintos espacios de color y ordenamientos que no utilizan el espacio CIELab. Se puede ver cómo la segmentación en el espacio de color CIELab logra delimitar los objetos con mayor precisión. Ambos métodos generan segmentaciones de los cuerpos de mayor tamaño que en la segmentación ideal, pero este error es menor en las segmentaciones con el espacio de color CIELab.

\begin{figure}[H]
\centering
\includegraphics[width=100mm]{./imagenes/r2.png}
\caption{(a) Segmentación en CIELab (b) Segmentación en RGB}
\label{img:r2}
\end{figure}



 %!TEX root = ../main.tex
\chapter{Conclusiones y Trabajos Futuros}
\label{chap:conclusiones}

%!TEX root = ../main.tex
\section{Conclusiones}
\label{chap:analisis}
Las conclusiones que se obtienen para cada uno de los objetivos específicos del trabajo descritos en la sección \ref{chap:introduccion} 

El objetivo \ref{obj:1} es identificar y comparar las bases representación de conocimientos asociado al dominio de Contrataciones Públicas. Se ...

El objetivo \ref{obj:2} es Describir el proceso de modelado de una ontologia utilizando una base de representacion de conocimiento de dominio. Se ... 

El objetivo \ref{obj:3} es modelar un sistema de extracción y consultas de datos del dominio Contrataciones Públicas. Se ... 

El objetivo \ref{obj:4} es mostrar el uso de la ontologia con datos de Contrataciones Publicas del Paraguay. Se ...

Parrafo de cierre ....
%!TEX root = ../tesis.tex
\section{Trabajos Futuros}
\label{chap:futuros}

Como trabajo futuro se podrían probar nuevos métodos de ordenamiento de los píxeles a color y otros espacios de color. Además, se podría automatizar el proceso de selección de los marcadores de la imagen y mejorar el pre-procesamiento de las imágenes utilizadas, a modo de obtener mejores gradientes morfológicas. 

Finalmente, se puede probar con más bases de datos de imágenes para observar como se comporta el algoritmo en otros escenarios.


\appendix   % inician los apendices de tu tesis
	
% los capitulos que incluyas a partir de aqui aparecen 
% como apendices
% \chapter{Tablas de resultados por método de orden}
\label{chap:ApendiceA}
A continuación se muestra detalladamente por método y número de imagen los resultados obtenidos en los experimentos.
\fontsize{9}{11}\selectfont
\section{MEDIANA-CIELAB}
\begin{longtable}[c]{|l|l|l|l|l|l|l|l|}
\hline
\textbf{Imagen} & \textbf{ESP} & \textbf{SEN} & \textbf{EX} & \textbf{FP\%} & \textbf{VP\%} & \textbf{FN\%} & \textbf{VN\%} \\ \hline
\endfirsthead
%
\endhead
%
1               & 99,76        & 87,94        & 99,59       & 16,25         & 83,75         & 0,17          & 99,83         \\ \hline
2               & 99,91        & 72,14        & 99,78       & 19,84         & 80,16         & 0,14          & 99,86         \\ \hline
3               & 99,06        & 26,75        & 94,8        & 35,89         & 64,11         & 4,42          & 95,58         \\ \hline
4               & 99,79        & 35,86        & 98,8        & 27,44         & 72,56         & 1             & 99            \\ \hline
5               & 99,92        & 18,77        & 98,68       & 22,37         & 77,63         & 1,24          & 98,76         \\ \hline
6               & 97,63        & 27,05        & 96,19       & 80,87         & 19,13         & 1,53          & 98,47         \\ \hline
7               & 99,94        & 12,28        & 98,46       & 21,9          & 78,1          & 1,49          & 98,51         \\ \hline
8               & 99,66        & 16,21        & 93,51       & 20,68         & 79,32         & 6,28          & 93,72         \\ \hline
9               & 16,06        & 30,14        & 16,34       & 99,27         & 0,73          & 8,19          & 91,81         \\ \hline
10              & 99,94        & 8,96         & 96,45       & 14,81         & 85,19         & 3,5           & 96,5          \\ \hline
11              & 99,58        & 25,55        & 96,24       & 25,82         & 74,18         & 3,41          & 96,59         \\ \hline
12              & 98,73        & 30,97        & 93,84       & 34,63         & 65,37         & 5,15          & 94,85         \\ \hline
13              & 98,71        & 34,15        & 95,69       & 43,38         & 56,62         & 3,17          & 96,83         \\ \hline
14              & 99,68        & 28,34        & 97,32       & 24,71         & 75,29         & 2,4           & 97,6          \\ \hline
15              & 96,07        & 34,68        & 94,15       & 77,87         & 22,13         & 2,14          & 97,86         \\ \hline
16              & 98,94        & 40,76        & 96,84       & 41,13         & 58,87         & 2,19          & 97,81         \\ \hline
17              & 99,76        & 29,1         & 98,48       & 31,26         & 68,74         & 1,29          & 98,71         \\ \hline
18              & 100          & 21,94        & 98,96       & 0             & 100           & 1,04          & 98,96         \\ \hline
19              & 99,24        & 21,24        & 94,75       & 37,04         & 62,96         & 4,62          & 95,38         \\ \hline
20              & 99,72        & 13,49        & 94,15       & 22,81         & 77,19         & 5,66          & 94,34         \\ \hline
21              & 99,53        & 29,29        & 98,19       & 45,15         & 54,85         & 1,37          & 98,63         \\ \hline
22              & 99,68        & 34,23        & 98,39       & 31,5          & 68,5          & 1,32          & 98,68         \\ \hline
23              & 99,92        & 21,03        & 99,26       & 32            & 68            & 0,66          & 99,34         \\ \hline
24              & 99,45        & 26,38        & 96,84       & 35,94         & 64,06         & 2,67          & 97,33         \\ \hline
25              & 19,98        & 9,2          & 19,57       & 99,55         & 0,45          & 15,25         & 84,75         \\ \hline
26              & 98,85        & 9,57         & 93,52       & 65,42         & 34,58         & 5,49          & 94,51         \\ \hline
27              & 99,76        & 8,63         & 99,08       & 78,38         & 21,62         & 0,69          & 99,31         \\ \hline
28              & 7,41         & 25,91        & 8,3         & 98,61         & 1,39          & 33,45         & 66,55         \\ \hline
29              & 99,74        & 22,7         & 96,78       & 22,55         & 77,45         & 3             & 97            \\ \hline
30              & 86,6         & 36,54        & 84,28       & 88,26         & 11,74         & 3,45          & 96,55         \\ \hline
31              & 19,2         & 27,69        & 19,59       & 98,38         & 1,62          & 15,31         & 84,69         \\ \hline
32              & 99,81        & 21,15        & 97,45       & 23            & 77            & 2,38          & 97,62         \\ \hline
33              & 99,61        & 24,02        & 96,71       & 28,71         & 71,29         & 2,96          & 97,04         \\ \hline
34              & 99,72        & 31,26        & 98,15       & 27,67         & 72,33         & 1,59          & 98,41         \\ \hline
35              & 98,16        & 24,12        & 94,78       & 61,41         & 38,59         & 3,57          & 96,43         \\ \hline
36              & 99,28        & 29,79        & 96,88       & 40,24         & 59,76         & 2,47          & 97,53         \\ \hline
37              & 99,63        & 32,94        & 97,16       & 22,72         & 77,28         & 2,52          & 97,48         \\ \hline
38              & 99,5         & 31,13        & 96,19       & 24,1          & 75,9          & 3,4           & 96,6          \\ \hline
39              & 99,83        & 38,62        & 98,61       & 18,09         & 81,91         & 1,23          & 98,77         \\ \hline
40              & 99,5         & 28,22        & 97,64       & 39,69         & 60,31         & 1,9           & 98,1          \\ \hline
41              & 12,67        & 28,62        & 13,59       & 98,03         & 1,97          & 25,63         & 74,37         \\ \hline
42              & 99,81        & 27,26        & 98,62       & 29,27         & 70,73         & 1,2           & 98,8          \\ \hline
43              & 99,3         & 34,09        & 97,96       & 49,48         & 50,52         & 1,37          & 98,63         \\ \hline
44              & 98,58        & 29,34        & 95,9        & 54,68         & 45,32         & 2,8           & 97,2          \\ \hline
45              & 99,87        & 8,86         & 96,79       & 29,54         & 70,46         & 3,1           & 96,9          \\ \hline
46              & 99,5         & 41,01        & 97,88       & 29,92         & 70,08         & 1,66          & 98,34         \\ \hline
47              & 83,01        & 14,2         & 79,75       & 96            & 4             & 4,9           & 95,1          \\ \hline
48              & 99,32        & 26,7         & 98,14       & 60,57         & 39,43         & 1,21          & 98,79         \\ \hline
49              & 98,91        & 10,32        & 96,2        & 76,95         & 23,05         & 2,78          & 97,22         \\ \hline
50              & 99,29        & 21,65        & 95,76       & 40,95         & 59,05         & 3,62          & 96,38         \\ \hline
51              & 99,75        & 27,96        & 98,32       & 30,49         & 69,51         & 1,45          & 98,55         \\ \hline
52              & 99,54        & 28,43        & 97,48       & 35,38         & 64,62         & 2,1           & 97,9          \\ \hline
53              & 89,82        & 17,34        & 88,52       & 96,99         & 3,01          & 1,65          & 98,35         \\ \hline
54              & 99,52        & 23,87        & 96,63       & 33,7          & 66,3          & 2,95          & 97,05         \\ \hline
55              & 91,6         & 33,62        & 90,54       & 93,07         & 6,93          & 1,33          & 98,67         \\ \hline
56              & 97,58        & 30,97        & 95,03       & 66,32         & 33,68         & 2,73          & 97,27         \\ \hline
57              & 96,99        & 14,23        & 91,93       & 76,48         & 23,52         & 5,44          & 94,56         \\ \hline
58              & 98,99        & 30,99        & 94,59       & 32,02         & 67,98         & 4,6           & 95,4          \\ \hline
59              & 94,17        & 26,14        & 92,95       & 92,42         & 7,58          & 1,41          & 98,59         \\ \hline
60              & 79,73        & 33,2         & 77,12       & 91,11         & 8,89          & 4,76          & 95,24         \\ \hline
61              & 98,01        & 27,07        & 96,79       & 80,73         & 19,27         & 1,29          & 98,71         \\ \hline
62              & 88,36        & 27,54        & 86,32       & 92,4          & 7,6           & 2,77          & 97,23         \\ \hline
63              & 99,75        & 39,3         & 98,95       & 32,44         & 67,56         & 0,81          & 99,19         \\ \hline
64              & 99,74        & 30,47        & 98,73       & 36,01         & 63,99         & 1,03          & 98,97         \\ \hline
65              & 98,95        & 20,6         & 95,1        & 49,69         & 50,31         & 3,98          & 96,02         \\ \hline
66              & 99,89        & 1,39         & 97,73       & 77,12         & 22,88         & 2,17          & 97,83         \\ \hline
67              & 99,64        & 24,86        & 98,08       & 40,43         & 59,57         & 1,58          & 98,42         \\ \hline
68              & 96,98        & 20,3         & 94,06       & 79,04         & 20,96         & 3,14          & 96,86         \\ \hline
69              & 99,67        & 29,59        & 98,96       & 51,64         & 48,36         & 0,73          & 99,27         \\ \hline
70              & 99,27        & 22,16        & 96,89       & 50,82         & 49,18         & 2,44          & 97,56         \\ \hline
71              & 94,13        & 35,48        & 92,89       & 88,52         & 11,48         & 1,45          & 98,55         \\ \hline
72              & 99,91        & 28,26        & 99,29       & 26,18         & 73,82         & 0,62          & 99,38         \\ \hline
73              & 82,39        & 24,35        & 79,03       & 92,16         & 7,84          & 5,34          & 94,66         \\ \hline
74              & 99,58        & 21,94        & 97,55       & 41,31         & 58,69         & 2,06          & 97,94         \\ \hline
75              & 99,76        & 21,68        & 98,15       & 34,53         & 65,47         & 1,63          & 98,37         \\ \hline
76              & 99,84        & 22,43        & 98,56       & 29,09         & 70,91         & 1,3           & 98,7          \\ \hline
77              & 96,95        & 24,77        & 94,78       & 79,84         & 20,16         & 2,35          & 97,65         \\ \hline
78              & 99,58        & 19,55        & 98,24       & 55,7          & 44,3          & 1,36          & 98,64         \\ \hline
79              & 98,7         & 25,89        & 97,16       & 69,86         & 30,14         & 1,6           & 98,4          \\ \hline
80              & 98,36        & 24,99        & 95,33       & 60,4          & 39,6          & 3,18          & 96,82         \\ \hline
81              & 99,75        & 14,18        & 97,38       & 37,77         & 62,23         & 2,4           & 97,6          \\ \hline
82              & 99,36        & 29,22        & 97,78       & 48,55         & 51,45         & 1,62          & 98,38         \\ \hline
83              & 98,33        & 7,64         & 96,2        & 90,09         & 9,91          & 2,21          & 97,79         \\ \hline
84              & 99,82        & 23,22        & 98,49       & 30,86         & 69,14         & 1,34          & 98,66         \\ \hline
85              & 92,6         & 37,11        & 90,31       & 82,19         & 17,81         & 2,85          & 97,15         \\ \hline
86              & 99,57        & 23,68        & 97,53       & 39,96         & 60,04         & 2,07          & 97,93         \\ \hline
87              & 96,22        & 26,36        & 93,58       & 78,49         & 21,51         & 2,92          & 97,08         \\ \hline
88              & 99,18        & 21,32        & 96,77       & 54,5          & 45,5          & 2,47          & 97,53         \\ \hline
89              & 99,26        & 34,32        & 98,38       & 61,04         & 38,96         & 0,9           & 99,1          \\ \hline
90              & 66,33        & 18,56        & 65,13       & 98,6          & 1,4           & 3,06          & 96,94         \\ \hline
91              & 99,14        & 13,01        & 95,7        & 61,38         & 38,62         & 3,52          & 96,48         \\ \hline
92              & 98,29        & 23,44        & 96,04       & 70,16         & 29,84         & 2,36          & 97,64         \\ \hline
93              & 98,42        & 24,24        & 96,53       & 71,47         & 28,53         & 1,97          & 98,03         \\ \hline
94              & 99,27        & 27,19        & 98,48       & 71,04         & 28,96         & 0,8           & 99,2          \\ \hline
95              & 95,14        & 27,93        & 93,68       & 88,68         & 11,32         & 1,65          & 98,35         \\ \hline
96              & 99,48        & 23,69        & 97,75       & 48,12         & 51,88         & 1,77          & 98,23         \\ \hline
97              & 99,31        & 35,16        & 97,02       & 34,62         & 65,38         & 2,36          & 97,64         \\ \hline
98              & 83,71        & 25,84        & 81,91       & 95,15         & 4,85          & 2,77          & 97,23         \\ \hline
99              & 85,11        & 28,26        & 84,5        & 97,97         & 2,03          & 0,91          & 99,09         \\ \hline
100             & 96,91        & 23,59        & 94,21       & 77,45         & 22,55         & 2,92          & 97,08         \\ \hline
\label{anx:medianalab}
\end{longtable}
\section{MEDIA-CIELAB}
\begin{longtable}[c]{|l|l|l|l|l|l|l|l|}
\hline
\textbf{Imagen} & \textbf{ESP} & \textbf{SEN} & \textbf{EX} & \textbf{FP\%} & \textbf{VP\%} & \textbf{FN\%} & \textbf{VN\%} \\ \hline
\endfirsthead
%
\endhead
%
1               & 99,73        & 88,31        & 99,57       & 17,36         & 82,64         & 0,17          & 99,83         \\ \hline
2               & 99,91        & 72,14        & 99,78       & 19,84         & 80,16         & 0,14          & 99,86         \\ \hline
3               & 98,99        & 26,75        & 94,73       & 37,67         & 62,33         & 4,43          & 95,57         \\ \hline
4               & 99,7         & 36,61        & 98,72       & 34,62         & 65,38         & 0,99          & 99,01         \\ \hline
5               & 99,89        & 20,78        & 98,69       & 25,76         & 74,24         & 1,21          & 98,79         \\ \hline
6               & 97,7         & 29,28        & 96,31       & 79,07         & 20,93         & 1,48          & 98,52         \\ \hline
7               & 99,88        & 12,06        & 98,4        & 36,36         & 63,64         & 1,49          & 98,51         \\ \hline
8               & 98,28        & 32,15        & 93,4        & 40,21         & 59,79         & 5,21          & 94,79         \\ \hline
9               & 98,81        & 26           & 97,35       & 69,05         & 30,95         & 1,51          & 98,49         \\ \hline
10              & 99,93        & 8,91         & 96,44       & 16,67         & 83,33         & 3,51          & 96,49         \\ \hline
11              & 98,76        & 44,3         & 96,3        & 37,18         & 62,82         & 2,6           & 97,4          \\ \hline
12              & 98,69        & 30,92        & 93,8        & 35,34         & 64,66         & 5,16          & 94,84         \\ \hline
13              & 98,81        & 33,89        & 95,77       & 41,69         & 58,31         & 3,18          & 96,82         \\ \hline
14              & 99,63        & 28,85        & 97,29       & 27,33         & 72,67         & 2,38          & 97,62         \\ \hline
15              & 95,48        & 36,04        & 93,62       & 79,56         & 20,44         & 2,11          & 97,89         \\ \hline
16              & 98,92        & 40,76        & 96,82       & 41,6          & 58,4          & 2,19          & 97,81         \\ \hline
17              & 97,89        & 32,75        & 96,72       & 77,82         & 22,18         & 1,24          & 98,76         \\ \hline
18              & 100          & 21,94        & 98,96       & 0             & 100           & 1,04          & 98,96         \\ \hline
19              & 99,28        & 20,89        & 94,77       & 36,25         & 63,75         & 4,64          & 95,36         \\ \hline
20              & 99,64        & 16,75        & 94,28       & 23,53         & 76,47         & 5,46          & 94,54         \\ \hline
21              & 99,53        & 29,08        & 98,18       & 45,57         & 54,43         & 1,37          & 98,63         \\ \hline
22              & 99,7         & 34,55        & 98,41       & 29,87         & 70,13         & 1,31          & 98,69         \\ \hline
23              & 99,93        & 21,82        & 99,28       & 26,79         & 73,21         & 0,65          & 99,35         \\ \hline
24              & 98,64        & 29,64        & 96,18       & 55,27         & 44,73         & 2,57          & 97,43         \\ \hline
25              & 98,88        & 8,99         & 95,46       & 75,84         & 24,16         & 3,52          & 96,48         \\ \hline
26              & 98,81        & 9,52         & 93,48       & 66,29         & 33,71         & 5,5           & 94,5          \\ \hline
27              & 99,77        & 8,62         & 99,08       & 78,24         & 21,76         & 0,69          & 99,31         \\ \hline
28              & 98,32        & 22,66        & 94,7        & 59,55         & 40,45         & 3,81          & 96,19         \\ \hline
29              & 99,74        & 22,84        & 96,79       & 22,04         & 77,96         & 3             & 97            \\ \hline
30              & 86,67        & 36,57        & 84,34       & 88,2          & 11,8          & 3,45          & 96,55         \\ \hline
31              & 24,18        & 27,69        & 24,34       & 98,28         & 1,72          & 12,55         & 87,45         \\ \hline
32              & 99,81        & 21,21        & 97,45       & 22,77         & 77,23         & 2,38          & 97,62         \\ \hline
33              & 99,59        & 24,34        & 96,7        & 29,82         & 70,18         & 2,94          & 97,06         \\ \hline
34              & 99,72        & 31,21        & 98,15       & 27,46         & 72,54         & 1,59          & 98,41         \\ \hline
35              & 97,65        & 24,02        & 94,29       & 67,13         & 32,87         & 3,59          & 96,41         \\ \hline
36              & 99,28        & 31,38        & 96,93       & 39,15         & 60,85         & 2,42          & 97,58         \\ \hline
37              & 99,61        & 33,24        & 97,15       & 23,25         & 76,75         & 2,51          & 97,49         \\ \hline
38              & 68,02        & 38,42        & 66,59       & 94,25         & 5,75          & 4,4           & 95,6          \\ \hline
39              & 99,83        & 40           & 98,64       & 17,53         & 82,47         & 1,21          & 98,79         \\ \hline
40              & 99,83        & 29,01        & 97,98       & 17,64         & 82,36         & 1,87          & 98,13         \\ \hline
41              & 12,69        & 28,62        & 13,61       & 98,03         & 1,97          & 25,59         & 74,41         \\ \hline
42              & 99,8         & 27,5         & 98,62       & 29,89         & 70,11         & 1,2           & 98,8          \\ \hline
43              & 95,53        & 34,44        & 94,27       & 86,1          & 13,9          & 1,42          & 98,58         \\ \hline
44              & 99,35        & 27,97        & 96,6        & 36,51         & 63,49         & 2,83          & 97,17         \\ \hline
45              & 99,68        & 24,64        & 97,14       & 27            & 73            & 2,58          & 97,42         \\ \hline
46              & 99,51        & 41,01        & 97,88       & 29,64         & 70,36         & 1,66          & 98,34         \\ \hline
47              & 84,09        & 14,19        & 80,77       & 95,74         & 4,26          & 4,84          & 95,16         \\ \hline
48              & 98,39        & 27,31        & 97,23       & 78,03         & 21,97         & 1,21          & 98,79         \\ \hline
49              & 98,92        & 10,53        & 96,21       & 76,52         & 23,48         & 2,78          & 97,22         \\ \hline
50              & 99,28        & 21,58        & 95,75       & 41,17         & 58,83         & 3,62          & 96,38         \\ \hline
51              & 99,74        & 27,96        & 98,31       & 31,33         & 68,67         & 1,45          & 98,55         \\ \hline
52              & 99,58        & 28,39        & 97,52       & 33,28         & 66,72         & 2,1           & 97,9          \\ \hline
53              & 89,63        & 16,81        & 88,32       & 97,13         & 2,87          & 1,66          & 98,34         \\ \hline
54              & 99,29        & 26,2         & 96,5        & 40,49         & 59,51         & 2,87          & 97,13         \\ \hline
55              & 91,81        & 33,69        & 90,75       & 92,9          & 7,1           & 1,32          & 98,68         \\ \hline
56              & 97,73        & 31,01        & 95,18       & 64,82         & 35,18         & 2,73          & 97,27         \\ \hline
57              & 97,07        & 14,32        & 92,01       & 75,89         & 24,11         & 5,43          & 94,57         \\ \hline
58              & 98,26        & 31,52        & 93,94       & 44,33         & 55,67         & 4,6           & 95,4          \\ \hline
59              & 94,11        & 26,14        & 92,89       & 92,5          & 7,5           & 1,41          & 98,59         \\ \hline
60              & 28,99        & 36,25        & 29,4        & 97,05         & 2,95          & 11,59         & 88,41         \\ \hline
61              & 96,53        & 28,16        & 95,35       & 87,57         & 12,43         & 1,29          & 98,71         \\ \hline
62              & 89,1         & 25,65        & 86,97       & 92,44         & 7,56          & 2,82          & 97,18         \\ \hline
63              & 99,74        & 39,3         & 98,94       & 32,67         & 67,33         & 0,81          & 99,19         \\ \hline
64              & 99,71        & 30,47        & 98,7        & 38,96         & 61,04         & 1,03          & 98,97         \\ \hline
65              & 98,87        & 20,9         & 95,04       & 51,13         & 48,87         & 3,97          & 96,03         \\ \hline
66              & 99,9         & 1,35         & 97,73       & 77,03         & 22,97         & 2,17          & 97,83         \\ \hline
67              & 99,65        & 24,67        & 98,09       & 40,17         & 59,83         & 1,58          & 98,42         \\ \hline
68              & 97           & 20,25        & 94,08       & 78,98         & 21,02         & 3,14          & 96,86         \\ \hline
69              & 99,67        & 29,85        & 98,96       & 51,42         & 48,58         & 0,72          & 99,28         \\ \hline
70              & 12,27        & 25,3         & 12,68       & 99,09         & 0,91          & 16,26         & 83,74         \\ \hline
71              & 94,09        & 35,22        & 92,85       & 88,67         & 11,33         & 1,46          & 98,54         \\ \hline
72              & 99,9         & 26,93        & 99,27       & 29,65         & 70,35         & 0,63          & 99,37         \\ \hline
73              & 82,95        & 24,4         & 79,56       & 91,91         & 8,09          & 5,3           & 94,7          \\ \hline
74              & 99,58        & 21,81        & 97,54       & 41,67         & 58,33         & 2,07          & 97,93         \\ \hline
75              & 99,52        & 22,48        & 97,94       & 50,15         & 49,85         & 1,61          & 98,39         \\ \hline
76              & 99,63        & 23,69        & 98,37       & 47,73         & 52,27         & 1,28          & 98,72         \\ \hline
77              & 97,01        & 24,72        & 94,83       & 79,56         & 20,44         & 2,35          & 97,65         \\ \hline
78              & 99,71        & 22,35        & 98,41       & 43,4          & 56,6          & 1,31          & 98,69         \\ \hline
79              & 89,88        & 31,55        & 88,65       & 93,68         & 6,32          & 1,62          & 98,38         \\ \hline
80              & 97,56        & 25,63        & 94,59       & 68,86         & 31,14         & 3,18          & 96,82         \\ \hline
81              & 99,73        & 14,58        & 97,36       & 39,53         & 60,47         & 2,39          & 97,61         \\ \hline
82              & 98,91        & 29,38        & 97,34       & 61,61         & 38,39         & 1,63          & 98,37         \\ \hline
83              & 97,68        & 7,68         & 95,57       & 92,62         & 7,38          & 2,22          & 97,78         \\ \hline
84              & 98,29        & 24,05        & 97,01       & 80,11         & 19,89         & 1,34          & 98,66         \\ \hline
85              & 88,9         & 37,3         & 86,76       & 87,33         & 12,67         & 2,96          & 97,04         \\ \hline
86              & 79,15        & 26,91        & 77,75       & 96,57         & 3,43          & 2,48          & 97,52         \\ \hline
87              & 94,82        & 26,19        & 92,22       & 83,42         & 16,58         & 2,97          & 97,03         \\ \hline
88              & 99,18        & 21,38        & 96,77       & 54,57         & 45,43         & 2,47          & 97,53         \\ \hline
89              & 99,15        & 34,81        & 98,28       & 64            & 36            & 0,9           & 99,1          \\ \hline
90              & 0,01         & 18,91        & 0,48        & 99,52         & 0,48          & 99,64         & 0,36          \\ \hline
91              & 99,13        & 12,92        & 95,68       & 61,9          & 38,1          & 3,53          & 96,47         \\ \hline
92              & 98,27        & 23,18        & 96,01       & 70,63         & 29,37         & 2,37          & 97,63         \\ \hline
93              & 98,47        & 24,07        & 96,57       & 70,95         & 29,05         & 1,97          & 98,03         \\ \hline
94              & 99,26        & 27,24        & 98,48       & 71,09         & 28,91         & 0,8           & 99,2          \\ \hline
95              & 93,96        & 28,03        & 92,53       & 90,66         & 9,34          & 1,67          & 98,33         \\ \hline
96              & 99,33        & 24,13        & 97,61       & 54,11         & 45,89         & 1,76          & 98,24         \\ \hline
97              & 99,09        & 35,61        & 96,82       & 40,88         & 59,12         & 2,35          & 97,65         \\ \hline
98              & 87,33        & 26,02        & 85,42       & 93,81         & 6,19          & 2,65          & 97,35         \\ \hline
99              & 85,09        & 28,48        & 84,48       & 97,96         & 2,04          & 0,91          & 99,09         \\ \hline
100             & 96,71        & 23,68        & 94,03       & 78,43         & 21,57         & 2,92          & 97,08         \\ \hline
\label{anx:medialab}
\end{longtable}
\section{DISTANCIA-EUCLIDIANA-CIELAB}
\begin{longtable}[c]{|l|l|l|l|l|l|l|l|}
\hline
\textbf{Imagen} & \textbf{ESP} & \textbf{SEN} & \textbf{EX} & \textbf{FP\%} & \textbf{VP\%} & \textbf{FN\%} & \textbf{VN\%} \\ \hline
\endfirsthead
%
\endhead
%
1               & 99,74        & 85,16        & 99,53       & 17,74         & 82,26         & 0,21          & 99,79         \\ \hline
2               & 99,91        & 74,29        & 99,79       & 19,38         & 80,62         & 0,13          & 99,87         \\ \hline
3               & 99,2         & 26,04        & 94,89       & 32,93         & 67,07         & 4,46          & 95,54         \\ \hline
4               & 99,85        & 24,29        & 98,68       & 28,26         & 71,74         & 1,18          & 98,82         \\ \hline
5               & 99,88        & 21,95        & 98,7        & 26,07         & 73,93         & 1,19          & 98,81         \\ \hline
6               & 97,43        & 30,03        & 96,05       & 80,5          & 19,5          & 1,47          & 98,53         \\ \hline
7               & 99,94        & 11,94        & 98,45       & 23,53         & 76,47         & 1,49          & 98,51         \\ \hline
8               & 55,84        & 39,16        & 54,6        & 93,4          & 6,6           & 7,99          & 92,01         \\ \hline
9               & 5,47         & 30,14        & 5,97        & 99,35         & 0,65          & 20,76         & 79,24         \\ \hline
10              & 99,93        & 9,02         & 96,44       & 16,49         & 83,51         & 3,5           & 96,5          \\ \hline
11              & 98,14        & 49,88        & 95,96       & 44,15         & 55,85         & 2,36          & 97,64         \\ \hline
12              & 99,22        & 27,06        & 94,02       & 27,01         & 72,99         & 5,4           & 94,6          \\ \hline
13              & 99,06        & 33,8         & 96,01       & 36,05         & 63,95         & 3,18          & 96,82         \\ \hline
14              & 99,67        & 27,73        & 97,29       & 25,81         & 74,19         & 2,42          & 97,58         \\ \hline
15              & 95,42        & 33,53        & 93,49       & 80,91         & 19,09         & 2,2           & 97,8          \\ \hline
16              & 98,97        & 40,59        & 96,87       & 40,49         & 59,51         & 2,19          & 97,81         \\ \hline
17              & 97,93        & 33,02        & 96,76       & 77,38         & 22,62         & 1,24          & 98,76         \\ \hline
18              & 100          & 20,86        & 98,94       & 0             & 100           & 1,06          & 98,94         \\ \hline
19              & 98,1         & 22,98        & 93,78       & 57,48         & 42,52         & 4,57          & 95,43         \\ \hline
20              & 54,47        & 27,75        & 52,74       & 95,96         & 4,04          & 8,4           & 91,6          \\ \hline
21              & 99,6         & 28,95        & 98,25       & 41,57         & 58,43         & 1,37          & 98,63         \\ \hline
22              & 99,59        & 35,32        & 98,32       & 36,26         & 63,74         & 1,29          & 98,71         \\ \hline
23              & 99,93        & 21,37        & 99,28       & 26,92         & 73,08         & 0,65          & 99,35         \\ \hline
24              & 98,37        & 34,69        & 96,1        & 55,87         & 44,13         & 2,4           & 97,6          \\ \hline
25              & 98,91        & 8,82         & 95,48       & 75,73         & 24,27         & 3,52          & 96,48         \\ \hline
26              & 96,73        & 9,53         & 91,52       & 84,39         & 15,61         & 5,61          & 94,39         \\ \hline
27              & 99,62        & 8,63         & 98,94       & 85,29         & 14,71         & 0,69          & 99,31         \\ \hline
28              & 87,12        & 23,78        & 84,09       & 91,5          & 8,5           & 4,21          & 95,79         \\ \hline
29              & 99,8         & 21,87        & 96,81       & 18,45         & 81,55         & 3,03          & 96,97         \\ \hline
30              & 98,62        & 33,91        & 95,61       & 45,5          & 54,5          & 3,17          & 96,83         \\ \hline
31              & 93,35        & 24,71        & 90,21       & 84,87         & 15,13         & 3,73          & 96,27         \\ \hline
32              & 99,79        & 20,25        & 97,4        & 25,59         & 74,41         & 2,41          & 97,59         \\ \hline
33              & 99,19        & 24,02        & 96,31       & 45,71         & 54,29         & 2,97          & 97,03         \\ \hline
34              & 99,71        & 30,53        & 98,13       & 28,47         & 71,53         & 1,61          & 98,39         \\ \hline
35              & 94,35        & 26,99        & 91,27       & 81,38         & 18,62         & 3,57          & 96,43         \\ \hline
36              & 99,7         & 28,69        & 97,24       & 22,47         & 77,53         & 2,5           & 97,5          \\ \hline
37              & 99,49        & 33,79        & 97,05       & 28,31         & 71,69         & 2,5           & 97,5          \\ \hline
38              & 99,07        & 33,56        & 95,91       & 35,26         & 64,74         & 3,29          & 96,71         \\ \hline
39              & 99,86        & 37,64        & 98,63       & 15,08         & 84,92         & 1,25          & 98,75         \\ \hline
40              & 99,74        & 29,57        & 97,91       & 24,4          & 75,6          & 1,86          & 98,14         \\ \hline
41              & 96,32        & 25,31        & 92,23       & 70,36         & 29,64         & 4,53          & 95,47         \\ \hline
42              & 99,72        & 26,55        & 98,53       & 38,4          & 61,6          & 1,21          & 98,79         \\ \hline
43              & 98,97        & 33,28        & 97,63       & 59,53         & 40,47         & 1,39          & 98,61         \\ \hline
44              & 99,28        & 27,89        & 96,52       & 39,21         & 60,79         & 2,83          & 97,17         \\ \hline
45              & 99,03        & 30,71        & 96,72       & 47,47         & 52,53         & 2,39          & 97,61         \\ \hline
46              & 99,59        & 40,11        & 97,94       & 26,38         & 73,62         & 1,69          & 98,31         \\ \hline
47              & 80,31        & 14,42        & 77,18       & 96,48         & 3,52          & 5,05          & 94,95         \\ \hline
48              & 89,34        & 28,02        & 88,34       & 95,82         & 4,18          & 1,32          & 98,68         \\ \hline
49              & 67,63        & 11,46        & 65,91       & 98,89         & 1,11          & 3,97          & 96,03         \\ \hline
50              & 64,41        & 23,49        & 62,55       & 96,96         & 3,04          & 5,35          & 94,65         \\ \hline
51              & 99,92        & 27,8         & 98,48       & 12,82         & 87,18         & 1,45          & 98,55         \\ \hline
52              & 99,55        & 27,98        & 97,48       & 34,89         & 65,11         & 2,11          & 97,89         \\ \hline
53              & 89,79        & 16,68        & 88,48       & 97,11         & 2,89          & 1,66          & 98,34         \\ \hline
54              & 98,85        & 28,23        & 96,16       & 50,56         & 49,44         & 2,8           & 97,2          \\ \hline
55              & 96,84        & 31,29        & 95,65       & 84,44         & 15,56         & 1,3           & 98,7          \\ \hline
56              & 97,81        & 30,68        & 95,24       & 64,29         & 35,71         & 2,74          & 97,26         \\ \hline
57              & 95,73        & 15,84        & 90,85       & 80,55         & 19,45         & 5,41          & 94,59         \\ \hline
58              & 98,05        & 38,44        & 94,19       & 42,25         & 57,75         & 4,16          & 95,84         \\ \hline
59              & 91,05        & 27,4         & 89,91       & 94,7          & 5,3           & 1,44          & 98,56         \\ \hline
60              & 77,71        & 32,01        & 75,14       & 92,12         & 7,88          & 4,95          & 95,05         \\ \hline
61              & 96,66        & 31,18        & 95,53       & 85,95         & 14,05         & 1,23          & 98,77         \\ \hline
62              & 76,75        & 28,06        & 75,11       & 95,97         & 4,03          & 3,15          & 96,85         \\ \hline
63              & 99,41        & 38,78        & 98,61       & 53,13         & 46,87         & 0,82          & 99,18         \\ \hline
64              & 99,89        & 29,4         & 98,85       & 20,66         & 79,34         & 1,04          & 98,96         \\ \hline
65              & 1,92         & 23,35        & 2,97        & 98,78         & 1,22          & 67,38         & 32,62         \\ \hline
66              & 96,95        & 3,26         & 94,89       & 97,66         & 2,34          & 2,19          & 97,81         \\ \hline
67              & 99,64        & 23,75        & 98,06       & 41,37         & 58,63         & 1,6           & 98,4          \\ \hline
68              & 97           & 20,2         & 94,08       & 79,03         & 20,97         & 3,15          & 96,85         \\ \hline
69              & 99,73        & 29,42        & 99,01       & 47,35         & 52,65         & 0,73          & 99,27         \\ \hline
70              & 74,41        & 23,89        & 72,85       & 97,11         & 2,89          & 3,16          & 96,84         \\ \hline
71              & 99,37        & 31,26        & 97,94       & 48,35         & 51,65         & 1,46          & 98,54         \\ \hline
72              & 85,77        & 21,86        & 85,22       & 98,68         & 1,32          & 0,79          & 99,21         \\ \hline
73              & 81,17        & 25,02        & 77,92       & 92,45         & 7,55          & 5,37          & 94,63         \\ \hline
74              & 99,56        & 21,83        & 97,52       & 42,96         & 57,04         & 2,07          & 97,93         \\ \hline
75              & 99,44        & 21,52        & 97,84       & 55,12         & 44,88         & 1,63          & 98,37         \\ \hline
76              & 99,63        & 23,46        & 98,36       & 48,5          & 51,5          & 1,28          & 98,72         \\ \hline
77              & 98,83        & 26,12        & 96,63       & 59,15         & 40,85         & 2,27          & 97,73         \\ \hline
78              & 97,99        & 25,05        & 96,76       & 82,46         & 17,54         & 1,29          & 98,71         \\ \hline
79              & 81,04        & 31,34        & 79,99       & 96,55         & 3,45          & 1,8           & 98,2          \\ \hline
80              & 88,03        & 25,93        & 85,47       & 91,46         & 8,54          & 3,5           & 96,5          \\ \hline
81              & 31,28        & 17,36        & 30,89       & 99,28         & 0,72          & 7,01          & 92,99         \\ \hline
82              & 98,82        & 28,72        & 97,24       & 63,97         & 36,03         & 1,64          & 98,36         \\ \hline
83              & 95,69        & 7,71         & 93,63       & 95,87         & 4,13          & 2,27          & 97,73         \\ \hline
84              & 97,69        & 24,68        & 96,43       & 84,12         & 15,88         & 1,34          & 98,66         \\ \hline
85              & 89,26        & 37,07        & 87,1        & 87,03         & 12,97         & 2,95          & 97,05         \\ \hline
86              & 78,97        & 27,12        & 77,58       & 96,57         & 3,43          & 2,48          & 97,52         \\ \hline
87              & 97,79        & 26,48        & 95,09       & 67,98         & 32,02         & 2,87          & 97,13         \\ \hline
88              & 98,95        & 21,07        & 96,54       & 60,99         & 39,01         & 2,48          & 97,52         \\ \hline
89              & 99,23        & 33,59        & 98,34       & 62,66         & 37,34         & 0,91          & 99,09         \\ \hline
90              & 82,02        & 17,84        & 80,41       & 97,51         & 2,49          & 2,51          & 97,49         \\ \hline
91              & 94,97        & 13,8         & 91,73       & 89,76         & 10,24         & 3,64          & 96,36         \\ \hline
92              & 97,81        & 23,09        & 95,56       & 75,37         & 24,63         & 2,38          & 97,62         \\ \hline
93              & 98,43        & 23,81        & 96,53       & 71,62         & 28,38         & 1,98          & 98,02         \\ \hline
94              & 99,41        & 27,38        & 98,62       & 66,31         & 33,69         & 0,8           & 99,2          \\ \hline
95              & 93,91        & 29,07        & 92,5        & 90,42         & 9,58          & 1,65          & 98,35         \\ \hline
96              & 98,88        & 23,89        & 97,16       & 66,73         & 33,27         & 1,77          & 98,23         \\ \hline
97              & 99,04        & 35,89        & 96,79       & 41,94         & 58,06         & 2,34          & 97,66         \\ \hline
98              & 82,9         & 25,5         & 81,11       & 95,43         & 4,57          & 2,81          & 97,19         \\ \hline
99              & 85,15        & 28,41        & 84,53       & 97,96         & 2,04          & 0,91          & 99,09         \\ \hline
100             & 82,92        & 24,31        & 80,77       & 94,85         & 5,15          & 3,37          & 96,63         \\ \hline
\label{anx:distancialab}
\end{longtable}
\section{LEXICOGRAFICO-RGB}
\begin{longtable}[c]{|l|l|l|l|l|l|l|l|}
\hline
\textbf{Imagen} & \textbf{ESP} & \textbf{SEN} & \textbf{EX} & \textbf{FP\%} & \textbf{VP\%} & \textbf{FN\%} & \textbf{VN\%} \\ \hline
\endfirsthead
%
\endhead
%
1               & 99,77        & 82,93        & 99,53       & 16,14         & 83,86         & 0,24          & 99,76         \\ \hline
2               & 99,91        & 69,29        & 99,76       & 21,77         & 78,23         & 0,15          & 99,85         \\ \hline
3               & 99,18        & 25,98        & 94,86       & 33,59         & 66,41         & 4,46          & 95,54         \\ \hline
4               & 99,65        & 39,1         & 98,71       & 36,4          & 63,6          & 0,95          & 99,05         \\ \hline
5               & 99,87        & 22,8         & 98,7        & 27,36         & 72,64         & 1,18          & 98,82         \\ \hline
6               & 97,69        & 30,57        & 96,33       & 78,44         & 21,56         & 1,45          & 98,55         \\ \hline
7               & 99,93        & 11,6         & 98,44       & 24,63         & 75,37         & 1,5           & 98,5          \\ \hline
8               & 32,13        & 40,33        & 32,73       & 95,48         & 4,52          & 12,89         & 87,11         \\ \hline
9               & 82,09        & 26,95        & 80,98       & 97,01         & 2,99          & 1,79          & 98,21         \\ \hline
10              & 99,94        & 7,57         & 96,39       & 17,58         & 82,42         & 3,56          & 96,44         \\ \hline
11              & 98,59        & 49,09        & 96,35       & 37,84         & 62,16         & 2,38          & 97,62         \\ \hline
12              & 98,84        & 26,91        & 93,65       & 35,74         & 64,26         & 5,43          & 94,57         \\ \hline
13              & 98,82        & 34,15        & 95,79       & 41,33         & 58,67         & 3,17          & 96,83         \\ \hline
14              & 99,71        & 27,39        & 97,32       & 23,74         & 76,26         & 2,43          & 97,57         \\ \hline
15              & 96,21        & 32,5         & 94,22       & 78,35         & 21,65         & 2,21          & 97,79         \\ \hline
16              & 98,94        & 39,85        & 96,82       & 41,55         & 58,45         & 2,22          & 97,78         \\ \hline
17              & 97,83        & 35,1         & 96,7        & 77,13         & 22,87         & 1,2           & 98,8          \\ \hline
18              & 100          & 22,66        & 98,97       & 0             & 100           & 1,03          & 98,97         \\ \hline
19              & 98,16        & 23,02        & 93,84       & 56,7          & 43,3          & 4,57          & 95,43         \\ \hline
20              & 55,67        & 27,79        & 53,87       & 95,85         & 4,15          & 8,23          & 91,77         \\ \hline
21              & 99,55        & 31,63        & 98,25       & 42,01         & 57,99         & 1,32          & 98,68         \\ \hline
22              & 99,76        & 34,17        & 98,46       & 25,83         & 74,17         & 1,32          & 98,68         \\ \hline
23              & 89,23        & 24,52        & 88,69       & 98,13         & 1,87          & 0,7           & 99,3          \\ \hline
24              & 99,46        & 32,7         & 97,08       & 30,7          & 69,3          & 2,44          & 97,56         \\ \hline
25              & 98,71        & 8,8          & 95,29       & 78,72         & 21,28         & 3,53          & 96,47         \\ \hline
26              & 98,83        & 9,54         & 93,5        & 65,86         & 34,14         & 5,5           & 94,5          \\ \hline
27              & 99,53        & 8,69         & 98,85       & 87,77         & 12,23         & 0,69          & 99,31         \\ \hline
28              & 91,7         & 23,38        & 88,43       & 87,59         & 12,41         & 4,03          & 95,97         \\ \hline
29              & 99,8         & 22,31        & 96,82       & 18,68         & 81,32         & 3,01          & 96,99         \\ \hline
30              & 98,36        & 36,5         & 95,48       & 47,93         & 52,07         & 3,05          & 96,95         \\ \hline
31              & 92,77        & 25,51        & 89,69       & 85,53         & 14,47         & 3,71          & 96,29         \\ \hline
32              & 99,8         & 20,89        & 97,43       & 24,12         & 75,88         & 2,39          & 97,61         \\ \hline
33              & 98,48        & 24,55        & 95,64       & 60,83         & 39,17         & 2,97          & 97,03         \\ \hline
34              & 99,74        & 30,92        & 98,17       & 26,05         & 73,95         & 1,6           & 98,4          \\ \hline
35              & 90,37        & 25,97        & 87,43       & 88,56         & 11,44         & 3,77          & 96,23         \\ \hline
36              & 99,75        & 28,83        & 97,29       & 19,72         & 80,28         & 2,49          & 97,51         \\ \hline
37              & 99,58        & 33,41        & 97,13       & 24,64         & 75,36         & 2,51          & 97,49         \\ \hline
38              & 99,32        & 33,63        & 96,14       & 28,61         & 71,39         & 3,28          & 96,72         \\ \hline
39              & 99,86        & 38,07        & 98,63       & 15,33         & 84,67         & 1,24          & 98,76         \\ \hline
40              & 99,76        & 29,97        & 97,94       & 23,1          & 76,9          & 1,85          & 98,15         \\ \hline
41              & 99,18        & 25,14        & 94,91       & 34,82         & 65,18         & 4,41          & 95,59         \\ \hline
42              & 99,81        & 28,07        & 98,64       & 28,51         & 71,49         & 1,19          & 98,81         \\ \hline
43              & 98,37        & 33,8         & 97,05       & 69,7          & 30,3          & 1,39          & 98,61         \\ \hline
44              & 99,25        & 29,96        & 96,57       & 38,44         & 61,56         & 2,76          & 97,24         \\ \hline
45              & 99,15        & 30,21        & 96,81       & 44,69         & 55,31         & 2,4           & 97,6          \\ \hline
46              & 99,66        & 39,76        & 98          & 22,93         & 77,07         & 1,7           & 98,3          \\ \hline
47              & 75,23        & 14,31        & 72,34       & 97,2          & 2,8           & 5,37          & 94,63         \\ \hline
48              & 88,37        & 28,16        & 87,39       & 96,14         & 3,86          & 1,33          & 98,67         \\ \hline
49              & 75,44        & 11,66        & 73,49       & 98,52         & 1,48          & 3,57          & 96,43         \\ \hline
50              & 94,57        & 23,13        & 91,33       & 83,15         & 16,85         & 3,72          & 96,28         \\ \hline
51              & 99,89        & 28,13        & 98,46       & 16,3          & 83,7          & 1,44          & 98,56         \\ \hline
52              & 99,53        & 28,74        & 97,48       & 35,67         & 64,33         & 2,09          & 97,91         \\ \hline
53              & 89,72        & 17,72        & 88,43       & 96,95         & 3,05          & 1,65          & 98,35         \\ \hline
54              & 98,81        & 28,58        & 96,13       & 51,14         & 48,86         & 2,79          & 97,21         \\ \hline
55              & 93,35        & 32,25        & 92,23       & 91,74         & 8,26          & 1,33          & 98,67         \\ \hline
56              & 97,42        & 31,75        & 94,91       & 67,16         & 32,84         & 2,71          & 97,29         \\ \hline
57              & 2,12         & 17,56        & 3,06        & 98,85         & 1,15          & 71,73         & 28,27         \\ \hline
58              & 98,2         & 31,26        & 93,87       & 45,39         & 54,61         & 4,62          & 95,38         \\ \hline
59              & 91,15        & 27,73        & 90,01       & 94,58         & 5,42          & 1,43          & 98,57         \\ \hline
60              & 79,41        & 29,71        & 76,61       & 92,08         & 7,92          & 5,01          & 94,99         \\ \hline
61              & 96,17        & 32,04        & 95,07       & 87,22         & 12,78         & 1,22          & 98,78         \\ \hline
62              & 77,15        & 28,43        & 75,52       & 95,85         & 4,15          & 3,12          & 96,88         \\ \hline
63              & 99,79        & 37,48        & 98,96       & 29,94         & 70,06         & 0,83          & 99,17         \\ \hline
64              & 99,85        & 30,17        & 98,83       & 24,81         & 75,19         & 1,03          & 98,97         \\ \hline
65              & 1,85         & 23,35        & 2,91        & 98,79         & 1,21          & 68,1          & 31,9          \\ \hline
66              & 92,95        & 2,7          & 90,96       & 99,15         & 0,85          & 2,3           & 97,7          \\ \hline
67              & 99,61        & 24,71        & 98,05       & 42,39         & 57,61         & 1,58          & 98,42         \\ \hline
68              & 96,45        & 20,35        & 93,56       & 81,55         & 18,45         & 3,16          & 96,84         \\ \hline
69              & 99,65        & 29,5         & 98,93       & 53,5          & 46,5          & 0,73          & 99,27         \\ \hline
70              & 3,54         & 25,3         & 4,21        & 99,17         & 0,83          & 40,24         & 59,76         \\ \hline
71              & 99,41        & 30,86        & 97,97       & 47,04         & 52,96         & 1,47          & 98,53         \\ \hline
72              & 85,84        & 22,58        & 85,29       & 98,63         & 1,37          & 0,78          & 99,22         \\ \hline
73              & 73,72        & 25,1         & 70,9        & 94,45         & 5,55          & 5,88          & 94,12         \\ \hline
74              & 99,58        & 21,4         & 97,53       & 42,17         & 57,83         & 2,08          & 97,92         \\ \hline
75              & 99,41        & 21,6         & 97,8        & 56,56         & 43,44         & 1,63          & 98,37         \\ \hline
76              & 99,6         & 23,84        & 98,34       & 50,12         & 49,88         & 1,28          & 98,72         \\ \hline
77              & 98,62        & 26,31        & 96,44       & 62,81         & 37,19         & 2,27          & 97,73         \\ \hline
78              & 99,64        & 23,65        & 98,36       & 47,41         & 52,59         & 1,29          & 98,71         \\ \hline
79              & 81,19        & 31,39        & 80,13       & 96,52         & 3,48          & 1,8           & 98,2          \\ \hline
80              & 97           & 25,93        & 94,07       & 72,86         & 27,14         & 3,19          & 96,81         \\ \hline
81              & 44,83        & 17,16        & 44,06       & 99,12         & 0,88          & 5,01          & 94,99         \\ \hline
82              & 2,87         & 34,04        & 3,58        & 99,2          & 0,8           & 34,69         & 65,31         \\ \hline
83              & 98,82        & 7,64         & 96,68       & 86,47         & 13,53         & 2,2           & 97,8          \\ \hline
84              & 97,75        & 24,78        & 96,49       & 83,74         & 16,26         & 1,34          & 98,66         \\ \hline
85              & 89,12        & 37,34        & 86,98       & 87,09         & 12,91         & 2,95          & 97,05         \\ \hline
86              & 82,26        & 26,98        & 80,78       & 95,98         & 4,02          & 2,39          & 97,61         \\ \hline
87              & 98,09        & 25,82        & 95,36       & 65,28         & 34,72         & 2,89          & 97,11         \\ \hline
88              & 98,08        & 21,55        & 95,71       & 73,6          & 26,4          & 2,49          & 97,51         \\ \hline
89              & 99,3         & 30,21        & 98,36       & 62,87         & 37,13         & 0,96          & 99,04         \\ \hline
90              & 79,22        & 17,75        & 77,68       & 97,85         & 2,15          & 2,6           & 97,4          \\ \hline
91              & 87,76        & 13,88        & 84,81       & 95,5          & 4,5           & 3,92          & 96,08         \\ \hline
92              & 86,4         & 23,41        & 84,5        & 94,93         & 5,07          & 2,68          & 97,32         \\ \hline
93              & 98,61        & 22,96        & 96,69       & 69,84         & 30,16         & 2             & 98            \\ \hline
94              & 99,15        & 27,87        & 98,38       & 73,37         & 26,63         & 0,8           & 99,2          \\ \hline
95              & 94           & 28,13        & 92,57       & 90,57         & 9,43          & 1,67          & 98,33         \\ \hline
96              & 98,92        & 24,48        & 97,21       & 65,32         & 34,68         & 1,76          & 98,24         \\ \hline
97              & 99,14        & 34,99        & 96,85       & 39,78         & 60,22         & 2,37          & 97,63         \\ \hline
98              & 84,27        & 25,62        & 82,44       & 95,03         & 4,97          & 2,76          & 97,24         \\ \hline
99              & 85,35        & 28,62        & 84,73       & 97,91         & 2,09          & 0,9           & 99,1          \\ \hline
100             & 94,82        & 23,97        & 92,22       & 84,99         & 15,01         & 2,97          & 97,03         \\ \hline
\label{anx:lexrgb}
\end{longtable}
\section{ALGORITMO-MEYER}
\begin{longtable}[c]{|l|l|l|l|l|l|l|l|}
\hline
\textbf{Imagen} & \textbf{ESP} & \textbf{SEN} & \textbf{EX} & \textbf{FP\%} & \textbf{VP\%} & \textbf{FN\%} & \textbf{VN\%} \\ \hline
\endhead
%
1               & 99,66        & 85,34        & 99,46       & 21,77         & 78,23         & 0,21          & 99,79         \\ \hline
2               & 99,91        & 73,57        & 99,79       & 19,53         & 80,47         & 0,13          & 99,87         \\ \hline
3               & 99,15        & 25,81        & 94,83       & 34,36         & 65,64         & 4,48          & 95,52         \\ \hline
4               & 99,86        & 22,55        & 98,66       & 28,87         & 71,13         & 1,2           & 98,8          \\ \hline
5               & 99,87        & 21,85        & 98,68       & 27,72         & 72,28         & 1,19          & 98,81         \\ \hline
6               & 97,65        & 30,93        & 96,3        & 78,51         & 21,49         & 1,45          & 98,55         \\ \hline
7               & 99,92        & 10,33        & 98,41       & 30,23         & 69,77         & 1,52          & 98,48         \\ \hline
8               & 75,6         & 38,81        & 72,88       & 88,76         & 11,24         & 6,06          & 93,94         \\ \hline
9               & 5,61         & 30,14        & 6,11        & 99,35         & 0,65          & 20,34         & 79,66         \\ \hline
10              & 99,93        & 9,02         & 96,44       & 16,92         & 83,08         & 3,5           & 96,5          \\ \hline
11              & 96,95        & 49,29        & 94,8        & 56,72         & 43,28         & 2,41          & 97,59         \\ \hline
12              & 98,88        & 25,39        & 93,59       & 36,14         & 63,86         & 5,54          & 94,46         \\ \hline
13              & 98,66        & 34,24        & 95,64       & 44,37         & 55,63         & 3,17          & 96,83         \\ \hline
14              & 99,64        & 27,99        & 97,27       & 27,46         & 72,54         & 2,41          & 97,59         \\ \hline
15              & 95,9         & 32,34        & 93,91       & 79,73         & 20,27         & 2,22          & 97,78         \\ \hline
16              & 98,94        & 39,99        & 96,82       & 41,44         & 58,56         & 2,21          & 97,79         \\ \hline
17              & 98,06        & 30,36        & 96,84       & 77,71         & 22,29         & 1,29          & 98,71         \\ \hline
18              & 100          & 21,58        & 98,95       & 0             & 100           & 1,05          & 98,95         \\ \hline
19              & 97,96        & 23,37        & 93,67       & 58,86         & 41,14         & 4,56          & 95,44         \\ \hline
20              & 56,55        & 27,79        & 54,69       & 95,77         & 4,23          & 8,11          & 91,89         \\ \hline
21              & 99,58        & 28,39        & 98,22       & 42,92         & 57,08         & 1,38          & 98,62         \\ \hline
22              & 99,69        & 34,48        & 98,4        & 30,99         & 69,01         & 1,31          & 98,69         \\ \hline
23              & 99,93        & 19,8         & 99,26       & 30,43         & 69,57         & 0,67          & 99,33         \\ \hline
24              & 98,57        & 36,02        & 96,33       & 51,81         & 48,19         & 2,35          & 97,65         \\ \hline
25              & 99,28        & 8,81         & 95,84       & 67,24         & 32,76         & 3,51          & 96,49         \\ \hline
26              & 98,77        & 9,59         & 93,44       & 66,88         & 33,12         & 5,5           & 94,5          \\ \hline
27              & 99,65        & 8,57         & 98,96       & 84,56         & 15,44         & 0,69          & 99,31         \\ \hline
28              & 97,25        & 22,82        & 93,68       & 70,59         & 29,41         & 3,84          & 96,16         \\ \hline
29              & 99,8         & 21,9         & 96,81       & 18,25         & 81,75         & 3,03          & 96,97         \\ \hline
30              & 98,5         & 34,77        & 95,53       & 46,97         & 53,03         & 3,13          & 96,87         \\ \hline
31              & 92,66        & 25,4         & 89,58       & 85,76         & 14,24         & 3,72          & 96,28         \\ \hline
32              & 99,79        & 20,7         & 97,42       & 24,82         & 75,18         & 2,39          & 97,61         \\ \hline
33              & 98,4         & 24,69        & 95,57       & 61,93         & 38,07         & 2,97          & 97,03         \\ \hline
34              & 99,69        & 30,14        & 98,1        & 30,18         & 69,82         & 1,62          & 98,38         \\ \hline
35              & 97,8         & 24,9         & 94,47       & 64,89         & 35,11         & 3,55          & 96,45         \\ \hline
36              & 99,74        & 28,83        & 97,29       & 19,92         & 80,08         & 2,49          & 97,51         \\ \hline
37              & 99,58        & 33,43        & 97,13       & 24,67         & 75,33         & 2,51          & 97,49         \\ \hline
38              & 98,52        & 33,39        & 95,37       & 46,62         & 53,38         & 3,32          & 96,68         \\ \hline
39              & 99,86        & 38,39        & 98,64       & 15,01         & 84,99         & 1,24          & 98,76         \\ \hline
40              & 99,81        & 27,51        & 97,92       & 20,56         & 79,44         & 1,91          & 98,09         \\ \hline
41              & 96,91        & 24,25        & 92,73       & 67,54         & 32,46         & 4,56          & 95,44         \\ \hline
42              & 99,84        & 28,1         & 98,66       & 25,84         & 74,16         & 1,19          & 98,81         \\ \hline
43              & 98,57        & 33,39        & 97,24       & 67,09         & 32,91         & 1,4           & 98,6          \\ \hline
44              & 99,28        & 26,29        & 96,47       & 40,37         & 59,63         & 2,9           & 97,1          \\ \hline
45              & 98,69        & 32,68        & 96,46       & 53,42         & 46,58         & 2,33          & 97,67         \\ \hline
46              & 99,65        & 40,11        & 98          & 23,35         & 76,65         & 1,69          & 98,31         \\ \hline
47              & 77,59        & 14,36        & 74,59       & 96,9          & 3,1           & 5,22          & 94,78         \\ \hline
48              & 89,2         & 28,2         & 88,2        & 95,85         & 4,15          & 1,32          & 98,68         \\ \hline
49              & 67,63        & 11,33        & 65,91       & 98,91         & 1,09          & 3,97          & 96,03         \\ \hline
50              & 96,12        & 22,44        & 92,77       & 78,43         & 21,57         & 3,7           & 96,3          \\ \hline
51              & 99,91        & 27,72        & 98,47       & 13,3          & 86,7          & 1,45          & 98,55         \\ \hline
52              & 99,51        & 29,05        & 97,48       & 35,96         & 64,04         & 2,08          & 97,92         \\ \hline
53              & 89,21        & 17,47        & 87,92       & 97,13         & 2,87          & 1,66          & 98,34         \\ \hline
54              & 99,21        & 27,04        & 96,45       & 42,41         & 57,59         & 2,84          & 97,16         \\ \hline
55              & 91,19        & 33,39        & 90,13       & 93,42         & 6,58          & 1,34          & 98,66         \\ \hline
56              & 97,27        & 30,54        & 94,72       & 69,21         & 30,79         & 2,76          & 97,24         \\ \hline
57              & 2,33         & 17,56        & 3,26        & 98,84         & 1,16          & 69,72         & 30,28         \\ \hline
58              & 97,88        & 31,07        & 93,56       & 49,63         & 50,37         & 4,65          & 95,35         \\ \hline
59              & 91,76        & 26,33        & 90,59       & 94,48         & 5,52          & 1,45          & 98,55         \\ \hline
60              & 79,42        & 29,96        & 76,64       & 92,02         & 7,98          & 4,99          & 95,01         \\ \hline
61              & 96           & 30,67        & 94,88       & 88,15         & 11,85         & 1,25          & 98,75         \\ \hline
62              & 74,15        & 27,96        & 72,6        & 96,38         & 3,62          & 3,27          & 96,73         \\ \hline
63              & 99,67        & 37,16        & 98,84       & 39,87         & 60,13         & 0,84          & 99,16         \\ \hline
64              & 99,72        & 26,65        & 98,65       & 41,02         & 58,98         & 1,08          & 98,92         \\ \hline
65              & 2,33         & 23,35        & 3,36        & 98,78         & 1,22          & 62,95         & 37,05         \\ \hline
66              & 92,08        & 8,38         & 90,24       & 97,68         & 2,32          & 2,19          & 97,81         \\ \hline
67              & 99,62        & 24,29        & 98,05       & 42,4          & 57,6          & 1,59          & 98,41         \\ \hline
68              & 97,1         & 19,48        & 94,15       & 79,04         & 20,96         & 3,17          & 96,83         \\ \hline
69              & 99,72        & 29,59        & 99,01       & 47,37         & 52,63         & 0,73          & 99,27         \\ \hline
70              & 4,37         & 25,3         & 5,02        & 99,16         & 0,84          & 35,26         & 64,74         \\ \hline
71              & 99,52        & 29,73        & 98,06       & 42,77         & 57,23         & 1,49          & 98,51         \\ \hline
72              & 85,83        & 20,29        & 85,27       & 98,77         & 1,23          & 0,8           & 99,2          \\ \hline
73              & 73,15        & 25,12        & 70,37       & 94,56         & 5,44          & 5,92          & 94,08         \\ \hline
74              & 99,56        & 21,94        & 97,52       & 42,9          & 57,1          & 2,06          & 97,94         \\ \hline
75              & 99,44        & 21,76        & 97,84       & 54,87         & 45,13         & 1,63          & 98,37         \\ \hline
76              & 96,95        & 23,08        & 95,73       & 88,65         & 11,35         & 1,32          & 98,68         \\ \hline
77              & 98,79        & 26,4         & 96,6        & 59,69         & 40,31         & 2,26          & 97,74         \\ \hline
78              & 99,59        & 24,58        & 98,33       & 49,23         & 50,77         & 1,28          & 98,72         \\ \hline
79              & 81,17        & 31,61        & 80,12       & 96,5          & 3,5           & 1,79          & 98,21         \\ \hline
80              & 96,31        & 25,96        & 93,41       & 76,73         & 23,27         & 3,21          & 96,79         \\ \hline
81              & 77,05        & 16,81        & 75,37       & 97,95         & 2,05          & 2,99          & 97,01         \\ \hline
82              & 19,23        & 32,48        & 19,53       & 99,08         & 0,92          & 7,51          & 92,49         \\ \hline
83              & 90,12        & 7,71         & 88,18       & 98,16         & 1,84          & 2,41          & 97,59         \\ \hline
84              & 99,75        & 24,73        & 98,45       & 36,75         & 63,25         & 1,31          & 98,69         \\ \hline
85              & 88,91        & 37,11        & 86,77       & 87,37         & 12,63         & 2,96          & 97,04         \\ \hline
86              & 82,25        & 25,6         & 80,73       & 96,18         & 3,82          & 2,43          & 97,57         \\ \hline
87              & 97,99        & 26,28        & 95,27       & 66,09         & 33,91         & 2,87          & 97,13         \\ \hline
88              & 99,38        & 20,68        & 96,94       & 48,52         & 51,48         & 2,49          & 97,51         \\ \hline
89              & 99,21        & 33,82        & 98,32       & 62,94         & 37,06         & 0,91          & 99,09         \\ \hline
90              & 77,6         & 17,93        & 76,1        & 97,98         & 2,02          & 2,65          & 97,35         \\ \hline
91              & 93,36        & 13,23        & 90,16       & 92,34         & 7,66          & 3,72          & 96,28         \\ \hline
92              & 98,19        & 22,38        & 95,91       & 72,26         & 27,74         & 2,39          & 97,61         \\ \hline
93              & 98,42        & 23,11        & 96,5        & 72,39         & 27,61         & 2             & 98            \\ \hline
94              & 99,05        & 27,51        & 98,27       & 75,79         & 24,21         & 0,8           & 99,2          \\ \hline
95              & 93,81        & 29,02        & 92,4        & 90,57         & 9,43          & 1,65          & 98,35         \\ \hline
96              & 99,36        & 23,58        & 97,62       & 53,64         & 46,36         & 1,77          & 98,23         \\ \hline
97              & 99,14        & 35,57        & 96,87       & 39,47         & 60,53         & 2,35          & 97,65         \\ \hline
98              & 84,65        & 24,85        & 82,79       & 95,05         & 4,95          & 2,77          & 97,23         \\ \hline
99              & 84,77        & 28,99        & 84,17       & 97,97         & 2,03          & 0,91          & 99,09         \\ \hline
100             & 96,93        & 22,89        & 94,2        & 77,87         & 22,13         & 2,95          & 97,05         \\ \hline
\label{anx:meyer}
\end{longtable}
\section{ALPHA-MOD-LEXICOGRAFICO-RGB}
\begin{longtable}[c]{|l|l|l|l|l|l|l|l|}
\hline
\textbf{Imagen} & \textbf{ESP} & \textbf{SEN} & \textbf{EX} & \textbf{FP\%} & \textbf{VP\%} & \textbf{FN\%} & \textbf{VN\%} \\ \hline
\endfirsthead
%
\endhead
%
1               & 99,75        & 82,19        & 99,5        & 17,5          & 82,5          & 0,26          & 99,74         \\ \hline
2               & 99,91        & 72,14        & 99,77       & 20,47         & 79,53         & 0,14          & 99,86         \\ \hline
3               & 99,17        & 25,08        & 94,81       & 34,51         & 65,49         & 4,52          & 95,48         \\ \hline
4               & 99,52        & 39,02        & 98,58       & 44,17         & 55,83         & 0,95          & 99,05         \\ \hline
5               & 99,87        & 21,63        & 98,68       & 28,67         & 71,33         & 1,2           & 98,8          \\ \hline
6               & 97,15        & 30,6         & 95,8        & 81,77         & 18,23         & 1,46          & 98,54         \\ \hline
7               & 99,92        & 10,1         & 98,41       & 30,16         & 69,84         & 1,52          & 98,48         \\ \hline
8               & 96,83        & 37,31        & 92,44       & 51,6          & 48,4          & 4,9           & 95,1          \\ \hline
9               & 7            & 30,14        & 7,46        & 99,34         & 0,66          & 17            & 83            \\ \hline
10              & 96,9         & 18,6         & 93,9        & 80,68         & 19,32         & 3,24          & 96,76         \\ \hline
11              & 23,71        & 52,4         & 25          & 96,86         & 3,14          & 8,67          & 91,33         \\ \hline
12              & 98,81        & 25,52        & 93,52       & 37,6          & 62,4          & 5,53          & 94,47         \\ \hline
13              & 98,69        & 34,24        & 95,67       & 43,72         & 56,28         & 3,17          & 96,83         \\ \hline
14              & 97,58        & 29,2         & 95,32       & 70,78         & 29,22         & 2,42          & 97,58         \\ \hline
15              & 94,48        & 36,49        & 92,67       & 82,43         & 17,57         & 2,12          & 97,88         \\ \hline
16              & 98,9         & 39,79        & 96,77       & 42,61         & 57,39         & 2,22          & 97,78         \\ \hline
17              & 97,37        & 31,2         & 96,18       & 82,1          & 17,9          & 1,28          & 98,72         \\ \hline
18              & 100          & 22,66        & 98,97       & 0             & 100           & 1,03          & 98,97         \\ \hline
19              & 97,98        & 22,82        & 93,65       & 59,23         & 40,77         & 4,59          & 95,41         \\ \hline
20              & 55,61        & 27,67        & 53,8        & 95,87         & 4,13          & 8,25          & 91,75         \\ \hline
21              & 99,59        & 28,24        & 98,23       & 42,68         & 57,32         & 1,38          & 98,62         \\ \hline
22              & 88,69        & 35,76        & 87,64       & 94            & 6             & 1,44          & 98,56         \\ \hline
23              & 99,92        & 21,48        & 99,27       & 29,78         & 70,22         & 0,65          & 99,35         \\ \hline
24              & 98,55        & 33,72        & 96,24       & 53,71         & 46,29         & 2,43          & 97,57         \\ \hline
25              & 99,16        & 8,81         & 95,72       & 70,61         & 29,39         & 3,51          & 96,49         \\ \hline
26              & 97,49        & 9,36         & 92,23       & 80,81         & 19,19         & 5,58          & 94,42         \\ \hline
27              & 99,73        & 8,6          & 99,04       & 80,78         & 19,22         & 0,69          & 99,31         \\ \hline
28              & 88,33        & 23,48        & 85,23       & 90,81         & 9,19          & 4,17          & 95,83         \\ \hline
29              & 99,69        & 21,48        & 96,69       & 26,55         & 73,45         & 3,05          & 96,95         \\ \hline
30              & 98,52        & 34,32        & 95,54       & 46,84         & 53,16         & 3,15          & 96,85         \\ \hline
31              & 77,37        & 26,18        & 75,03       & 94,74         & 5,26          & 4,38          & 95,62         \\ \hline
32              & 99,77        & 20,31        & 97,39       & 27,25         & 72,75         & 2,41          & 97,59         \\ \hline
33              & 96,98        & 25,08        & 94,22       & 75,1          & 24,9          & 2,99          & 97,01         \\ \hline
34              & 99,72        & 28,34        & 98,08       & 29,93         & 70,07         & 1,66          & 98,34         \\ \hline
35              & 97,37        & 25,37        & 94,08       & 68,38         & 31,62         & 3,54          & 96,46         \\ \hline
36              & 99,45        & 29,37        & 97,02       & 34,32         & 65,68         & 2,48          & 97,52         \\ \hline
37              & 99,21        & 35,36        & 96,84       & 36,86         & 63,14         & 2,45          & 97,55         \\ \hline
38              & 98,83        & 33,44        & 95,67       & 40,76         & 59,24         & 3,31          & 96,69         \\ \hline
39              & 99,8         & 37,84        & 98,57       & 20,42         & 79,58         & 1,25          & 98,75         \\ \hline
40              & 99,84        & 27,62        & 97,96       & 17,39         & 82,61         & 1,91          & 98,09         \\ \hline
41              & 91,56        & 24,62        & 87,7        & 84,86         & 15,14         & 4,79          & 95,21         \\ \hline
42              & 99,75        & 28,48        & 98,58       & 34,81         & 65,19         & 1,18          & 98,82         \\ \hline
43              & 98,69        & 33,04        & 97,34       & 65,41         & 34,59         & 1,4           & 98,6          \\ \hline
44              & 99           & 22,96        & 96,06       & 52,11         & 47,89         & 3,03          & 96,97         \\ \hline
45              & 97,6         & 33,62        & 95,44       & 67,09         & 32,91         & 2,33          & 97,67         \\ \hline
46              & 99,58        & 40,31        & 97,93       & 26,87         & 73,13         & 1,68          & 98,32         \\ \hline
47              & 81,04        & 14,35        & 77,87       & 96,36         & 3,64          & 5,01          & 94,99         \\ \hline
48              & 88,26        & 28,3         & 87,28       & 96,16         & 3,84          & 1,33          & 98,67         \\ \hline
49              & 82,92        & 11,39        & 80,73       & 97,94         & 2,06          & 3,26          & 96,74         \\ \hline
50              & 83,87        & 23,12        & 81,11       & 93,62         & 6,38          & 4,18          & 95,82         \\ \hline
51              & 99,87        & 27,47        & 98,43       & 18,45         & 81,55         & 1,46          & 98,54         \\ \hline
52              & 99,63        & 28,87        & 97,58       & 30,03         & 69,97         & 2,08          & 97,92         \\ \hline
53              & 90,56        & 18,02        & 89,26       & 96,63         & 3,37          & 1,62          & 98,38         \\ \hline
54              & 98,63        & 27,79        & 95,92       & 55,44         & 44,56         & 2,83          & 97,17         \\ \hline
55              & 89,52        & 34,32        & 88,52       & 94,26         & 5,74          & 1,35          & 98,65         \\ \hline
56              & 97,44        & 31,07        & 94,91       & 67,45         & 32,55         & 2,73          & 97,27         \\ \hline
57              & 1,49         & 17,56        & 2,48        & 98,85         & 1,15          & 78,23         & 21,77         \\ \hline
58              & 97,65        & 31,87        & 93,39       & 51,62         & 48,38         & 4,61          & 95,39         \\ \hline
59              & 91,28        & 26,28        & 90,11       & 94,78         & 5,22          & 1,46          & 98,54         \\ \hline
60              & 77,4         & 33,06        & 74,91       & 91,98         & 8,02          & 4,9           & 95,1          \\ \hline
61              & 95,4         & 32,1         & 94,31       & 89,11         & 10,89         & 1,23          & 98,77         \\ \hline
62              & 88,64        & 25,6         & 86,52       & 92,74         & 7,26          & 2,83          & 97,17         \\ \hline
63              & 99,61        & 33,14        & 98,73       & 46,49         & 53,51         & 0,89          & 99,11         \\ \hline
64              & 99,9         & 28,94        & 98,86       & 19,23         & 80,77         & 1,05          & 98,95         \\ \hline
65              & 1,9          & 23,35        & 2,95        & 98,79         & 1,21          & 67,62         & 32,38         \\ \hline
66              & 94,45        & 2,3          & 92,43       & 99,07         & 0,93          & 2,27          & 97,73         \\ \hline
67              & 99,58        & 24,71        & 98,02       & 44,64         & 55,36         & 1,58          & 98,42         \\ \hline
68              & 96,97        & 19,69        & 94,04       & 79,58         & 20,42         & 3,17          & 96,83         \\ \hline
69              & 99,77        & 28,29        & 99,03       & 44,35         & 55,65         & 0,74          & 99,26         \\ \hline
70              & 4,19         & 25,3         & 4,85        & 99,17         & 0,84          & 36,23         & 63,77         \\ \hline
71              & 99,29        & 33,95        & 97,92       & 49,33         & 50,67         & 1,41          & 98,59         \\ \hline
72              & 85,51        & 19,81        & 84,94       & 98,82         & 1,18          & 0,81          & 99,19         \\ \hline
73              & 73,06        & 25,84        & 70,32       & 94,43         & 5,57          & 5,87          & 94,13         \\ \hline
74              & 98,83        & 21,58        & 96,81       & 66,81         & 33,19         & 2,09          & 97,91         \\ \hline
75              & 98,95        & 22,32        & 97,37       & 69,12         & 30,88         & 1,63          & 98,37         \\ \hline
76              & 98,2         & 21,86        & 96,93       & 82,97         & 17,03         & 1,33          & 98,67         \\ \hline
77              & 94,27        & 27,06        & 92,25       & 87,2          & 12,8          & 2,35          & 97,65         \\ \hline
78              & 99,23        & 24,02        & 97,97       & 65,09         & 34,91         & 1,29          & 98,71         \\ \hline
79              & 81,68        & 31,72        & 80,62       & 96,39         & 3,61          & 1,78          & 98,22         \\ \hline
80              & 92,38        & 26,45        & 89,66       & 86,99         & 13,01         & 3,32          & 96,68         \\ \hline
81              & 76,52        & 16,93        & 74,87       & 97,98         & 2,02          & 3,01          & 96,99         \\ \hline
82              & 2,88         & 34,04        & 3,58        & 99,2          & 0,8           & 34,65         & 65,35         \\ \hline
83              & 90,52        & 7,57         & 88,57       & 98,12         & 1,88          & 2,4           & 97,6          \\ \hline
84              & 97,8         & 24,63        & 96,54       & 83,5          & 16,5          & 1,34          & 98,66         \\ \hline
85              & 88,51        & 34,82        & 86,29       & 88,43         & 11,57         & 3,08          & 96,92         \\ \hline
86              & 82,01        & 26           & 80,51       & 96,17         & 3,83          & 2,42          & 97,58         \\ \hline
87              & 94,64        & 26,6         & 92,07       & 83,67         & 16,33         & 2,96          & 97,04         \\ \hline
88              & 99,27        & 20,86        & 96,85       & 52,14         & 47,86         & 2,48          & 97,52         \\ \hline
89              & 99           & 34,58        & 98,13       & 67,7          & 32,3          & 0,9           & 99,1          \\ \hline
90              & 77,68        & 18,01        & 76,19       & 97,97         & 2,03          & 2,64          & 97,36         \\ \hline
91              & 89,99        & 13,72        & 86,94       & 94,61         & 5,39          & 3,84          & 96,16         \\ \hline
92              & 98,38        & 22,31        & 96,09       & 70,11         & 29,89         & 2,39          & 97,61         \\ \hline
93              & 98,1         & 21,86        & 96,16       & 76,9          & 23,1          & 2,04          & 97,96         \\ \hline
94              & 98,88        & 26,88        & 98,09       & 79,15         & 20,85         & 0,81          & 99,19         \\ \hline
95              & 93,84        & 29,22        & 92,44       & 90,47         & 9,53          & 1,65          & 98,35         \\ \hline
96              & 99,25        & 23,96        & 97,52       & 57,2          & 42,8          & 1,77          & 98,23         \\ \hline
97              & 99,12        & 35,85        & 96,86       & 39,82         & 60,18         & 2,34          & 97,66         \\ \hline
98              & 83,27        & 24,61        & 81,44       & 95,49         & 4,51          & 2,83          & 97,17         \\ \hline
99              & 84,47        & 28,62        & 83,87       & 98,03         & 1,97          & 0,91          & 99,09         \\ \hline
100             & 95,76        & 23,25        & 93,09       & 82,7          & 17,3          & 2,97          & 97,03         \\ \hline
\label{anx:alpha}
\end{longtable}
\section{DISTANCIA-EUCLIDIANA-RGB}
\begin{longtable}[c]{|l|l|l|l|l|l|l|l|}
\hline
\textbf{Imagen} & \textbf{ESP} & \textbf{SEN} & \textbf{EX} & \textbf{FP\%} & \textbf{VP\%} & \textbf{FN\%} & \textbf{VN\%} \\ \hline
\endfirsthead
%
\endhead
%
1               & 99,66        & 86,46        & 99,48       & 21,42         & 78,58         & 0,19          & 99,81         \\ \hline
2               & 99,91        & 75           & 99,79       & 19,23         & 80,77         & 0,12          & 99,88         \\ \hline
3               & 99,09        & 25,81        & 94,77       & 36,11         & 63,89         & 4,48          & 95,52         \\ \hline
4               & 99,77        & 36,69        & 98,8        & 28,53         & 71,47         & 0,99          & 99,01         \\ \hline
5               & 99,87        & 21,42        & 98,68       & 27,86         & 72,14         & 1,2           & 98,8          \\ \hline
6               & 97,34        & 30,84        & 95,99       & 80,57         & 19,43         & 1,45          & 98,55         \\ \hline
7               & 99,91        & 11,37        & 98,42       & 30,77         & 69,23         & 1,5           & 98,5          \\ \hline
8               & 71,16        & 39,08        & 68,79       & 90,26         & 9,74          & 6,38          & 93,62         \\ \hline
9               & 4,51         & 30,14        & 5,02        & 99,36         & 0,64          & 24,13         & 75,87         \\ \hline
10              & 99,62        & 12,08        & 96,27       & 43,93         & 56,07         & 3,4           & 96,6          \\ \hline
11              & 98,51        & 49,72        & 96,31       & 38,81         & 61,19         & 2,36          & 97,64         \\ \hline
12              & 98,64        & 28,33        & 93,57       & 38,13         & 61,87         & 5,34          & 94,66         \\ \hline
13              & 98,71        & 34,06        & 95,68       & 43,6          & 56,4          & 3,18          & 96,82         \\ \hline
14              & 99,61        & 28,6         & 97,26       & 28,76         & 71,24         & 2,39          & 97,61         \\ \hline
15              & 95,95        & 35,95        & 94,07       & 77,76         & 22,24         & 2,11          & 97,89         \\ \hline
16              & 99,1         & 39,73        & 96,97       & 37,67         & 62,33         & 2,22          & 97,78         \\ \hline
17              & 97,91        & 32,47        & 96,74       & 77,78         & 22,22         & 1,25          & 98,75         \\ \hline
18              & 100          & 23,38        & 98,98       & 0             & 100           & 1,02          & 98,98         \\ \hline
19              & 97,98        & 23,02        & 93,67       & 59,02         & 40,98         & 4,58          & 95,42         \\ \hline
20              & 55,3         & 28,35        & 53,56       & 95,8          & 4,2           & 8,22          & 91,78         \\ \hline
21              & 99,55        & 28,8         & 98,19       & 44,72         & 55,28         & 1,37          & 98,63         \\ \hline
22              & 89,02        & 35,06        & 87,95       & 93,94         & 6,06          & 1,45          & 98,55         \\ \hline
23              & 99,93        & 20,58        & 99,27       & 29,07         & 70,93         & 0,66          & 99,34         \\ \hline
24              & 98,51        & 33,72        & 96,2        & 54,35         & 45,65         & 2,43          & 97,57         \\ \hline
25              & 99,16        & 8,78         & 95,72       & 70,63         & 29,37         & 3,51          & 96,49         \\ \hline
26              & 98,45        & 9,49         & 93,14       & 71,98         & 28,02         & 5,52          & 94,48         \\ \hline
27              & 99,69        & 8,57         & 99,01       & 82,58         & 17,42         & 0,69          & 99,31         \\ \hline
28              & 97,05        & 23,15        & 93,51       & 71,72         & 28,28         & 3,83          & 96,17         \\ \hline
29              & 99,72        & 21,63        & 96,72       & 24,74         & 75,26         & 3,04          & 96,96         \\ \hline
30              & 98,51        & 34,57        & 95,54       & 46,87         & 53,13         & 3,14          & 96,86         \\ \hline
31              & 92,64        & 25,73        & 89,57       & 85,64         & 14,36         & 3,7           & 96,3          \\ \hline
32              & 99,79        & 20,31        & 97,41       & 25,53         & 74,47         & 2,41          & 97,59         \\ \hline
33              & 98,49        & 25,01        & 95,67       & 60,17         & 39,83         & 2,95          & 97,05         \\ \hline
34              & 99,7         & 30,29        & 98,11       & 29,68         & 70,32         & 1,62          & 98,38         \\ \hline
35              & 97,13        & 25,92        & 93,88       & 69,79         & 30,21         & 3,52          & 96,48         \\ \hline
36              & 99,64        & 28,87        & 97,2        & 25,56         & 74,44         & 2,5           & 97,5          \\ \hline
37              & 99,43        & 34,01        & 97          & 30,44         & 69,56         & 2,49          & 97,51         \\ \hline
38              & 98,77        & 32,79        & 95,59       & 42,41         & 57,59         & 3,34          & 96,66         \\ \hline
39              & 99,82        & 39,37        & 98,62       & 18,43         & 81,57         & 1,22          & 98,78         \\ \hline
40              & 99,78        & 29,37        & 97,94       & 21,53         & 78,47         & 1,86          & 98,14         \\ \hline
41              & 92,11        & 25,21        & 88,25       & 83,66         & 16,34         & 4,73          & 95,27         \\ \hline
42              & 99,82        & 27,95        & 98,65       & 27,49         & 72,51         & 1,19          & 98,81         \\ \hline
43              & 99,02        & 33,16        & 97,67       & 58,56         & 41,44         & 1,4           & 98,6          \\ \hline
44              & 98,76        & 25,8         & 95,94       & 54,53         & 45,47         & 2,93          & 97,07         \\ \hline
45              & 54,25        & 36,11        & 53,64       & 97,31         & 2,69          & 3,96          & 96,04         \\ \hline
46              & 99,56        & 40,5         & 97,92       & 27,44         & 72,56         & 1,68          & 98,32         \\ \hline
47              & 74,61        & 14,32        & 71,74       & 97,26         & 2,74          & 5,42          & 94,58         \\ \hline
48              & 89           & 28,3         & 88,01       & 95,91         & 4,09          & 1,32          & 98,68         \\ \hline
49              & 75,59        & 11,3         & 73,62       & 98,56         & 1,44          & 3,57          & 96,43         \\ \hline
50              & 95,72        & 23,09        & 92,43       & 79,56         & 20,44         & 3,68          & 96,32         \\ \hline
51              & 99,9         & 28,21        & 98,47       & 15,23         & 84,77         & 1,44          & 98,56         \\ \hline
52              & 99,63        & 29,09        & 97,59       & 29,94         & 70,06         & 2,08          & 97,92         \\ \hline
53              & 90,27        & 17,67        & 88,97       & 96,8          & 3,2           & 1,64          & 98,36         \\ \hline
54              & 99,2         & 27,49        & 96,46       & 42,31         & 57,69         & 2,82          & 97,18         \\ \hline
55              & 91,41        & 33,65        & 90,36       & 93,21         & 6,79          & 1,33          & 98,67         \\ \hline
56              & 97,27        & 30,74        & 94,73       & 69,09         & 30,91         & 2,75          & 97,25         \\ \hline
57              & 2,33         & 17,56        & 3,26        & 98,84         & 1,16          & 69,77         & 30,23         \\ \hline
58              & 97,83        & 31,45        & 93,53       & 49,94         & 50,06         & 4,63          & 95,37         \\ \hline
59              & 91,54        & 27,08        & 90,38       & 94,47         & 5,53          & 1,44          & 98,56         \\ \hline
60              & 79,05        & 30,67        & 76,33       & 91,98         & 8,02          & 4,97          & 95,03         \\ \hline
61              & 96,53        & 27,36        & 95,34       & 87,86         & 12,14         & 1,3           & 98,7          \\ \hline
62              & 70,06        & 28,32        & 68,66       & 96,82         & 3,18          & 3,43          & 96,57         \\ \hline
63              & 99,82        & 30,67        & 98,9        & 30,95         & 69,05         & 0,92          & 99,08         \\ \hline
64              & 99,74        & 26,95        & 98,67       & 39,52         & 60,48         & 1,08          & 98,92         \\ \hline
65              & 2,28         & 23,35        & 3,31        & 98,78         & 1,22          & 63,47         & 36,53         \\ \hline
66              & 94,77        & 2,62         & 92,74       & 98,89         & 1,11          & 2,26          & 97,74         \\ \hline
67              & 99,61        & 24,67        & 98,05       & 42,59         & 57,41         & 1,58          & 98,42         \\ \hline
68              & 97,22        & 19,86        & 94,28       & 78,01         & 21,99         & 3,15          & 96,85         \\ \hline
69              & 99,69        & 29,59        & 98,98       & 50            & 50            & 0,73          & 99,27         \\ \hline
70              & 4,22         & 25,3         & 4,87        & 99,16         & 0,84          & 36,07         & 63,93         \\ \hline
71              & 99,36        & 30,93        & 97,92       & 49,04         & 50,96         & 1,47          & 98,53         \\ \hline
72              & 85,76        & 22,95        & 85,21       & 98,61         & 1,39          & 0,78          & 99,22         \\ \hline
73              & 73,95        & 24,86        & 71,11       & 94,46         & 5,54          & 5,88          & 94,12         \\ \hline
74              & 99,53        & 21,63        & 97,49       & 44,73         & 55,27         & 2,07          & 97,93         \\ \hline
75              & 99,43        & 21,72        & 97,82       & 55,67         & 44,33         & 1,63          & 98,37         \\ \hline
76              & 98,05        & 23,57        & 96,81       & 83,03         & 16,97         & 1,3           & 98,7          \\ \hline
77              & 98,89        & 26,5         & 96,71       & 57,34         & 42,66         & 2,26          & 97,74         \\ \hline
78              & 99,23        & 24,95        & 97,98       & 64,5          & 35,5          & 1,28          & 98,72         \\ \hline
79              & 72,15        & 32,04        & 71,3        & 97,57         & 2,43          & 2             & 98            \\ \hline
80              & 58,61        & 26,67        & 57,29       & 97,3          & 2,7           & 5,12          & 94,88         \\ \hline
81              & 34,91        & 17,2         & 34,42       & 99,25         & 0,75          & 6,34          & 93,66         \\ \hline
82              & 2,85         & 34,04        & 3,55        & 99,2          & 0,8           & 34,89         & 65,11         \\ \hline
83              & 90,33        & 7,79         & 88,39       & 98,1          & 1,9           & 2,4           & 97,6          \\ \hline
84              & 98,48        & 24,98        & 97,21       & 77,56         & 22,44         & 1,32          & 98,68         \\ \hline
85              & 89,01        & 36,93        & 86,86       & 87,32         & 12,68         & 2,97          & 97,03         \\ \hline
86              & 82,25        & 26,25        & 80,75       & 96,09         & 3,91          & 2,41          & 97,59         \\ \hline
87              & 97,57        & 26,15        & 94,86       & 70,29         & 29,71         & 2,89          & 97,11         \\ \hline
88              & 99,31        & 20,9         & 96,88       & 50,97         & 49,03         & 2,48          & 97,52         \\ \hline
89              & 99,22        & 33,65        & 98,33       & 62,8          & 37,2          & 0,91          & 99,09         \\ \hline
90              & 79,22        & 17,62        & 77,67       & 97,87         & 2,13          & 2,6           & 97,4          \\ \hline
91              & 92,59        & 13,58        & 89,44       & 92,91         & 7,09          & 3,74          & 96,26         \\ \hline
92              & 98,38        & 22,68        & 96,1        & 69,71         & 30,29         & 2,38          & 97,62         \\ \hline
93              & 98,27        & 22,15        & 96,33       & 74,97         & 25,03         & 2,03          & 97,97         \\ \hline
94              & 98,93        & 27,56        & 98,15       & 77,92         & 22,08         & 0,8           & 99,2          \\ \hline
95              & 93,85        & 28,97        & 92,44       & 90,54         & 9,46          & 1,65          & 98,35         \\ \hline
96              & 99,22        & 24           & 97,5        & 57,94         & 42,06         & 1,77          & 98,23         \\ \hline
97              & 99,14        & 35,2         & 96,86       & 39,76         & 60,24         & 2,36          & 97,64         \\ \hline
98              & 84,28        & 26,05        & 82,47       & 94,94         & 5,06          & 2,74          & 97,26         \\ \hline
99              & 84,57        & 28,48        & 83,97       & 98,03         & 1,97          & 0,91          & 99,09         \\ \hline
100             & 98           & 23,66        & 95,27       & 68,9          & 31,1          & 2,89          & 97,11         \\ \hline
\label{anx:eucli}
\end{longtable}
\section{ESCALA-DE-GRISES}
\begin{longtable}[c]{|l|l|l|l|l|l|l|l|}
\hline
\textbf{Imagen} & \textbf{ESP} & \textbf{SEN} & \textbf{EX} & \textbf{FP\%} & \textbf{VP\%} & \textbf{FN\%} & \textbf{VN\%} \\ \hline
\endfirsthead
%
\endhead
%
1               & 99,59        & 85,53        & 99,39       & 25,04         & 74,96         & 0,21          & 99,79         \\ \hline
2               & 99,9         & 75,71        & 99,78       & 21,48         & 78,52         & 0,12          & 99,88         \\ \hline
3               & 99,4         & 21,99        & 94,84       & 30,45         & 69,55         & 4,68          & 95,32         \\ \hline
4               & 99,67        & 24,38        & 98,51       & 46,04         & 53,96         & 1,18          & 98,82         \\ \hline
5               & 99,96        & 9,33         & 98,58       & 22,12         & 77,88         & 1,38          & 98,62         \\ \hline
6               & 97,16        & 22,5         & 95,65       & 85,85         & 14,15         & 1,63          & 98,37         \\ \hline
7               & 99,97        & 5,74         & 98,38       & 23,08         & 76,92         & 1,6           & 98,4          \\ \hline
8               & 18,45        & 41,32        & 20,13       & 96,12         & 3,88          & 20,21         & 79,79         \\ \hline
9               & 5,42         & 30,14        & 5,91        & 99,35         & 0,65          & 20,93         & 79,07         \\ \hline
10              & 99,93        & 6,68         & 96,36       & 20,53         & 79,47         & 3,59          & 96,41         \\ \hline
11              & 99,4         & 46,61        & 97,02       & 21,27         & 78,73         & 2,48          & 97,52         \\ \hline
12              & 99,36        & 23,7         & 93,91       & 25,79         & 74,21         & 5,63          & 94,37         \\ \hline
13              & 99,53        & 28,62        & 96,21       & 24,88         & 75,12         & 3,41          & 96,59         \\ \hline
14              & 99,51        & 27,22        & 97,12       & 34,58         & 65,42         & 2,44          & 97,56         \\ \hline
15              & 95,15        & 26,53        & 93,01       & 85            & 15            & 2,43          & 97,57         \\ \hline
16              & 98,89        & 40,47        & 96,79       & 42,39         & 57,61         & 2,2           & 97,8          \\ \hline
17              & 98,33        & 32           & 97,14       & 73,96         & 26,04         & 1,25          & 98,75         \\ \hline
18              & 100          & 25,91        & 99,03       & 0             & 100           & 0,98          & 99,02         \\ \hline
19              & 94,83        & 23,91        & 90,75       & 77,98         & 22,02         & 4,67          & 95,33         \\ \hline
20              & 52,73        & 28,03        & 51,14       & 96,06         & 3,94          & 8,62          & 91,38         \\ \hline
21              & 99,63        & 28,87        & 98,28       & 39,61         & 60,39         & 1,37          & 98,63         \\ \hline
22              & 99,6         & 35,89        & 98,34       & 35,59         & 64,41         & 1,28          & 98,72         \\ \hline
23              & 90,41        & 23,28        & 89,86       & 98,01         & 1,99          & 0,71          & 99,29         \\ \hline
24              & 98,36        & 30,36        & 95,93       & 59,39         & 40,61         & 2,55          & 97,45         \\ \hline
25              & 98,91        & 8,44         & 95,47       & 76,53         & 23,47         & 3,54          & 96,46         \\ \hline
26              & 95,9         & 10,53        & 90,87       & 86,12         & 13,88         & 5,52          & 94,48         \\ \hline
27              & 99,73        & 7,72         & 99,04       & 82,39         & 17,61         & 0,69          & 99,31         \\ \hline
28              & 96,17        & 21,1         & 92,58       & 78,29         & 21,71         & 3,96          & 96,04         \\ \hline
29              & 99,71        & 21,42        & 96,71       & 25,16         & 74,84         & 3,05          & 96,95         \\ \hline
30              & 98,46        & 32,98        & 95,41       & 48,96         & 51,04         & 3,21          & 96,79         \\ \hline
31              & 88,16        & 23,32        & 85,19       & 91,37         & 8,63          & 4,01          & 95,99         \\ \hline
32              & 99,71        & 22,37        & 97,39       & 29,9          & 70,1          & 2,35          & 97,65         \\ \hline
33              & 98,11        & 23,21        & 95,23       & 67,15         & 32,85         & 3,03          & 96,97         \\ \hline
34              & 99,69        & 27,37        & 98,03       & 32,41         & 67,59         & 1,68          & 98,32         \\ \hline
35              & 99,02        & 15,46        & 95,21       & 56,84         & 43,16         & 3,93          & 96,07         \\ \hline
36              & 99,56        & 27,32        & 97,06       & 30,95         & 69,05         & 2,55          & 97,45         \\ \hline
37              & 99,5         & 33,08        & 97,03       & 28,41         & 71,59         & 2,52          & 97,48         \\ \hline
38              & 99,28        & 29,71        & 95,92       & 32,17         & 67,83         & 3,47          & 96,53         \\ \hline
39              & 99,76        & 40,32        & 98,57       & 22,88         & 77,12         & 1,2           & 98,8          \\ \hline
40              & 99,87        & 16,96        & 97,26       & 19,05         & 80,95         & 2,63          & 97,37         \\ \hline
41              & 98,07        & 23,7         & 93,78       & 57,16         & 42,84         & 4,54          & 95,46         \\ \hline
42              & 99,88        & 24,32        & 98,64       & 22,56         & 77,44         & 1,25          & 98,75         \\ \hline
43              & 99,28        & 30,01        & 97,86       & 53,44         & 46,56         & 1,46          & 98,54         \\ \hline
44              & 98,44        & 19,99        & 95,41       & 66,03         & 33,97         & 3,16          & 96,84         \\ \hline
45              & 46,69        & 35,28        & 46,3        & 97,74         & 2,26          & 4,63          & 95,37         \\ \hline
46              & 99,58        & 39,91        & 97,92       & 27,16         & 72,84         & 1,69          & 98,31         \\ \hline
47              & 85,11        & 14,01        & 81,73       & 95,52         & 4,48          & 4,8           & 95,2          \\ \hline
48              & 97,43        & 27,38        & 96,28       & 85            & 15            & 1,22          & 98,78         \\ \hline
49              & 87,78        & 10,81        & 85,43       & 97,28         & 2,72          & 3,11          & 96,89         \\ \hline
50              & 70,82        & 22,29        & 68,62       & 96,49         & 3,51          & 4,96          & 95,04         \\ \hline
51              & 99,87        & 27,56        & 98,43       & 18,4          & 81,6          & 1,46          & 98,54         \\ \hline
52              & 99,65        & 28,47        & 97,59       & 29,1          & 70,9          & 2,09          & 97,91         \\ \hline
53              & 99,11        & 17,94        & 97,66       & 73,25         & 26,75         & 1,48          & 98,52         \\ \hline
54              & 99,28        & 26,3         & 96,49       & 40,74         & 59,26         & 2,86          & 97,14         \\ \hline
55              & 92,05        & 30           & 90,92       & 93,44         & 6,56          & 1,39          & 98,61         \\ \hline
56              & 93,31        & 28,5         & 90,83       & 85,53         & 14,47         & 2,95          & 97,05         \\ \hline
57              & 2,21         & 18,92        & 3,22        & 98,77         & 1,23          & 70,19         & 29,81         \\ \hline
58              & 98,24        & 23           & 93,37       & 52,49         & 47,51         & 5,15          & 94,85         \\ \hline
59              & 92,7         & 25,15        & 91,49       & 94,07         & 5,93          & 1,45          & 98,55         \\ \hline
60              & 68,85        & 30,79        & 66,71       & 94,44         & 5,56          & 5,65          & 94,35         \\ \hline
61              & 88,96        & 28,56        & 87,92       & 95,66         & 4,34          & 1,39          & 98,61         \\ \hline
62              & 92,11        & 25,97        & 89,89       & 89,73         & 10,27         & 2,72          & 97,28         \\ \hline
63              & 99,77        & 28,08        & 98,82       & 37,88         & 62,12         & 0,96          & 99,04         \\ \hline
64              & 99,73        & 25,57        & 98,64       & 41,61         & 58,39         & 1,1           & 98,9          \\ \hline
65              & 4,85         & 23,35        & 5,76        & 98,75         & 1,25          & 44,96         & 55,04         \\ \hline
66              & 85,96        & 7,07         & 84,23       & 98,88         & 1,12          & 2,37          & 97,63         \\ \hline
67              & 99,68        & 24,21        & 98,1        & 38,59         & 61,41         & 1,59          & 98,41         \\ \hline
68              & 96,93        & 18,63        & 93,96       & 80,65         & 19,35         & 3,21          & 96,79         \\ \hline
69              & 99,89        & 24,89        & 99,13       & 29,03         & 70,97         & 0,77          & 99,23         \\ \hline
70              & 6,03         & 25,3         & 6,62        & 99,15         & 0,85          & 28,33         & 71,67         \\ \hline
71              & 99,56        & 26,78        & 98,03       & 43,12         & 56,88         & 1,55          & 98,45         \\ \hline
72              & 99,98        & 17,39        & 99,26       & 13,25         & 86,75         & 0,72          & 99,28         \\ \hline
73              & 72,44        & 24,8         & 69,69       & 94,76         & 5,24          & 6             & 94            \\ \hline
74              & 99,51        & 21,24        & 97,47       & 45,94         & 54,06         & 2,08          & 97,92         \\ \hline
75              & 99,38        & 20,74        & 97,76       & 58,79         & 41,21         & 1,65          & 98,35         \\ \hline
76              & 99,85        & 19,43        & 98,51       & 31,68         & 68,32         & 1,35          & 98,65         \\ \hline
77              & 25,61        & 29,54        & 25,73       & 98,78         & 1,22          & 7,87          & 92,13         \\ \hline
78              & 99,37        & 22,81        & 98,08       & 61,9          & 38,1          & 1,31          & 98,69         \\ \hline
79              & 80,87        & 31,88        & 79,83       & 96,52         & 3,48          & 1,79          & 98,21         \\ \hline
80              & 96,56        & 23,39        & 93,54       & 77,33         & 22,67         & 3,31          & 96,69         \\ \hline
81              & 83,2         & 14,08        & 81,29       & 97,67         & 2,33          & 2,85          & 97,15         \\ \hline
82              & 0,62         & 34,04        & 1,38        & 99,21         & 0,79          & 70,97         & 29,03         \\ \hline
83              & 99,59        & 7,14         & 97,46       & 70,73         & 29,27         & 2,16          & 97,84         \\ \hline
84              & 99,65        & 23,41        & 98,33       & 46,12         & 53,88         & 1,34          & 98,66         \\ \hline
85              & 88,57        & 36,32        & 86,4        & 87,94         & 12,06         & 3,01          & 96,99         \\ \hline
86              & 81,39        & 24,68        & 79,87       & 96,48         & 3,52          & 2,48          & 97,52         \\ \hline
87              & 97,75        & 22,62        & 94,9        & 71,7          & 28,3          & 3,02          & 96,98         \\ \hline
88              & 99,35        & 20,18        & 96,9        & 50,19         & 49,81         & 2,5           & 97,5          \\ \hline
89              & 99,19        & 29,55        & 98,25       & 66,57         & 33,43         & 0,97          & 99,03         \\ \hline
90              & 81,18        & 16,89        & 79,58       & 97,76         & 2,24          & 2,55          & 97,45         \\ \hline
91              & 93,6         & 13,65        & 90,44       & 91,95         & 8,05          & 3,65          & 96,35         \\ \hline
92              & 98,24        & 20,89        & 95,91       & 73,06         & 26,94         & 2,44          & 97,56         \\ \hline
93              & 98,84        & 22,45        & 96,89       & 66,51         & 33,49         & 2,01          & 97,99         \\ \hline
94              & 99,29        & 25,43        & 98,48       & 71,83         & 28,17         & 0,82          & 99,18         \\ \hline
95              & 93,93        & 27,84        & 92,5        & 90,76         & 9,24          & 1,68          & 98,32         \\ \hline
96              & 99,12        & 23,96        & 97,4        & 60,96         & 39,04         & 1,77          & 98,23         \\ \hline
97              & 99,17        & 35,77        & 96,91       & 38,53         & 61,47         & 2,34          & 97,66         \\ \hline
98              & 82,1         & 25,9         & 80,35       & 95,56         & 4,44          & 2,82          & 97,18         \\ \hline
99              & 84,58        & 27,68        & 83,97       & 98,08         & 1,92          & 0,92          & 99,08         \\ \hline
100             & 97,94        & 21,93        & 95,15       & 71,08         & 28,92         & 2,95          & 97,05         \\ \hline
\label{anx:gris}
\end{longtable}
\section{VAZQUEZ-ET-AL}
\begin{longtable}[c]{|l|l|l|l|l|l|l|l|}
\hline
\textbf{Imagen} & \textbf{ESP} & \textbf{SEN} & \textbf{EX} & \textbf{FP\%} & \textbf{VP\%} & \textbf{FN\%} & \textbf{VN\%} \\ \hline
\endfirsthead
%
\endhead
%
1               & 99,68        & 85,34        & 99,47       & 20,96         & 79,04         & 0,21          & 99,79         \\ \hline
2               & 99,91        & 75           & 99,79       & 19,23         & 80,77         & 0,12          & 99,88         \\ \hline
3               & 99,12        & 25,61        & 94,79       & 35,47         & 64,53         & 4,49          & 95,51         \\ \hline
4               & 99,65        & 39,18        & 98,71       & 36,44         & 63,56         & 0,95          & 99,05         \\ \hline
5               & 99,88        & 21,31        & 98,68       & 27,17         & 72,83         & 1,2           & 98,8          \\ \hline
6               & 97,44        & 30,96        & 96,09       & 79,91         & 20,09         & 1,45          & 98,55         \\ \hline
7               & 99,62        & 11,02        & 98,13       & 66,43         & 33,57         & 1,51          & 98,49         \\ \hline
8               & 92,05        & 38,55        & 88,11       & 72,13         & 27,87         & 5,05          & 94,95         \\ \hline
9               & 3,66         & 30,14        & 4,19        & 99,36         & 0,64          & 28,14         & 71,86         \\ \hline
10              & 97,48        & 12,31        & 94,22       & 83,7          & 16,3          & 3,46          & 96,54         \\ \hline
11              & 98,51        & 49,75        & 96,31       & 38,75         & 61,25         & 2,35          & 97,65         \\ \hline
12              & 98,81        & 27,77        & 93,69       & 35,51         & 64,49         & 5,37          & 94,63         \\ \hline
13              & 98,72        & 33,98        & 95,69       & 43,34         & 56,66         & 3,18          & 96,82         \\ \hline
14              & 98,8         & 29,03        & 96,49       & 54,83         & 45,17         & 2,4           & 97,6          \\ \hline
15              & 95,03        & 36,34        & 93,2        & 80,93         & 19,07         & 2,11          & 97,89         \\ \hline
16              & 98,94        & 39,88        & 96,81       & 41,66         & 58,34         & 2,22          & 97,78         \\ \hline
17              & 97,93        & 32,62        & 96,76       & 77,53         & 22,47         & 1,25          & 98,75         \\ \hline
18              & 100          & 23,38        & 98,98       & 0             & 100           & 1,02          & 98,98         \\ \hline
19              & 97,98        & 22,86        & 93,66       & 59,13         & 40,87         & 4,58          & 95,42         \\ \hline
20              & 55,41        & 28,31        & 53,66       & 95,8          & 4,2           & 8,21          & 91,79         \\ \hline
21              & 99,52        & 29,19        & 98,17       & 45,91         & 54,09         & 1,37          & 98,63         \\ \hline
22              & 89,04        & 35,64        & 87,98       & 93,84         & 6,16          & 1,44          & 98,56         \\ \hline
23              & 99,93        & 20,81        & 99,27       & 29,39         & 70,61         & 0,66          & 99,34         \\ \hline
24              & 98,51        & 33,83        & 96,2        & 54,28         & 45,72         & 2,43          & 97,57         \\ \hline
25              & 99,25        & 8,79         & 95,8        & 68,39         & 31,61         & 3,51          & 96,49         \\ \hline
26              & 97,68        & 9,46         & 92,41       & 79,42         & 20,58         & 5,56          & 94,44         \\ \hline
27              & 99,7         & 8,64         & 99,01       & 82,36         & 17,64         & 0,69          & 99,31         \\ \hline
28              & 34,24        & 25,04        & 33,8        & 98,12         & 1,88          & 9,92          & 90,08         \\ \hline
29              & 99,71        & 21,45        & 96,7        & 25,36         & 74,64         & 3,05          & 96,95         \\ \hline
30              & 98,51        & 32,8         & 95,46       & 48,17         & 51,83         & 3,22          & 96,78         \\ \hline
31              & 80,1         & 25,45        & 77,59       & 94,22         & 5,78          & 4,28          & 95,72         \\ \hline
32              & 99,77        & 20,5         & 97,4        & 26,56         & 73,44         & 2,4           & 97,6          \\ \hline
33              & 98,35        & 24,62        & 95,52       & 62,61         & 37,39         & 2,97          & 97,03         \\ \hline
34              & 99,71        & 29,9         & 98,11       & 29,47         & 70,53         & 1,62          & 98,38         \\ \hline
35              & 96,56        & 25,87        & 93,33       & 73,54         & 26,46         & 3,55          & 96,45         \\ \hline
36              & 99,62        & 29,42        & 97,19       & 26,73         & 73,27         & 2,48          & 97,52         \\ \hline
37              & 99,32        & 34,98        & 96,94       & 33,42         & 66,58         & 2,46          & 97,54         \\ \hline
38              & 98,77        & 32,75        & 95,58       & 42,6          & 57,4          & 3,34          & 96,66         \\ \hline
39              & 99,82        & 38,16        & 98,59       & 19,21         & 80,79         & 1,24          & 98,76         \\ \hline
40              & 99,77        & 29,46        & 97,93       & 22,55         & 77,45         & 1,86          & 98,14         \\ \hline
41              & 92,12        & 25,15        & 88,26       & 83,66         & 16,34         & 4,73          & 95,27         \\ \hline
42              & 99,77        & 27,26        & 98,58       & 33,48         & 66,52         & 1,2           & 98,8          \\ \hline
43              & 99,02        & 32,75        & 97,66       & 58,89         & 41,11         & 1,4           & 98,6          \\ \hline
44              & 98,71        & 26,48        & 95,92       & 54,87         & 45,13         & 2,9           & 97,1          \\ \hline
45              & 57,97        & 35,02        & 57,19       & 97,17         & 2,83          & 3,78          & 96,22         \\ \hline
46              & 99,54        & 40,38        & 97,9        & 28,6          & 71,4          & 1,68          & 98,32         \\ \hline
47              & 81,73        & 14,35        & 78,53       & 96,23         & 3,77          & 4,97          & 95,03         \\ \hline
48              & 88,01        & 28,6         & 87,04       & 96,19         & 3,81          & 1,33          & 98,67         \\ \hline
49              & 38,65        & 11,32        & 37,81       & 99,42         & 0,58          & 6,75          & 93,25         \\ \hline
50              & 94,88        & 23,05        & 91,62       & 82,36         & 17,64         & 3,71          & 96,29         \\ \hline
51              & 99,89        & 28,13        & 98,46       & 15,69         & 84,31         & 1,44          & 98,56         \\ \hline
52              & 99,61        & 29,13        & 97,57       & 31,14         & 68,86         & 2,08          & 97,92         \\ \hline
53              & 90,49        & 18,15        & 89,19       & 96,64         & 3,36          & 1,62          & 98,38         \\ \hline
54              & 99,22        & 27,44        & 96,47       & 41,87         & 58,13         & 2,82          & 97,18         \\ \hline
55              & 91,27        & 33,99        & 90,22       & 93,26         & 6,74          & 1,33          & 98,67         \\ \hline
56              & 97,32        & 30,9         & 94,78       & 68,58         & 31,42         & 2,74          & 97,26         \\ \hline
57              & 1,49         & 17,56        & 2,48        & 98,85         & 1,15          & 78,22         & 21,78         \\ \hline
58              & 97,78        & 31,52        & 93,49       & 50,39         & 49,61         & 4,62          & 95,38         \\ \hline
59              & 91,64        & 26,98        & 90,47       & 94,43         & 5,57          & 1,44          & 98,56         \\ \hline
60              & 78,96        & 31,34        & 76,29       & 91,85         & 8,15          & 4,93          & 95,07         \\ \hline
61              & 96,53        & 27,01        & 95,34       & 87,99         & 12,01         & 1,31          & 98,69         \\ \hline
62              & 70,22        & 28,22        & 68,81       & 96,81         & 3,19          & 3,43          & 96,57         \\ \hline
63              & 99,55        & 30,87        & 98,64       & 52,3          & 47,7          & 0,92          & 99,08         \\ \hline
64              & 99,74        & 26,49        & 98,67       & 39,72         & 60,28         & 1,08          & 98,92         \\ \hline
65              & 1,88         & 23,35        & 2,93        & 98,79         & 1,21          & 67,85         & 32,15         \\ \hline
66              & 94,51        & 2,42         & 92,49       & 99,02         & 0,98          & 2,27          & 97,73         \\ \hline
67              & 99,6         & 24,9         & 98,05       & 42,77         & 57,23         & 1,58          & 98,42         \\ \hline
68              & 97,77        & 19,63        & 94,8        & 74,21         & 25,79         & 3,14          & 96,86         \\ \hline
69              & 99,77        & 28,37        & 99,04       & 43,7          & 56,3          & 0,74          & 99,26         \\ \hline
70              & 2,01         & 25,3         & 2,73        & 99,18         & 0,82          & 54,24         & 45,76         \\ \hline
71              & 99,35        & 31           & 97,91       & 49,57         & 50,43         & 1,47          & 98,53         \\ \hline
72              & 85,73        & 23,91        & 85,2        & 98,56         & 1,44          & 0,77          & 99,23         \\ \hline
73              & 74,2         & 24,84        & 71,34       & 94,41         & 5,59          & 5,86          & 94,14         \\ \hline
74              & 99,53        & 21,67        & 97,49       & 44,81         & 55,19         & 2,07          & 97,93         \\ \hline
75              & 98,38        & 22,3         & 96,81       & 77,51         & 22,49         & 1,64          & 98,36         \\ \hline
76              & 98           & 23,54        & 96,76       & 83,39         & 16,61         & 1,3           & 98,7          \\ \hline
77              & 94,68        & 26,74        & 92,63       & 86,49         & 13,51         & 2,35          & 97,65         \\ \hline
78              & 99,08        & 24,86        & 97,84       & 68,33         & 31,67         & 1,28          & 98,72         \\ \hline
79              & 72,19        & 31,99        & 71,34       & 97,57         & 2,43          & 2             & 98            \\ \hline
80              & 93,11        & 26,33        & 90,35       & 85,86         & 14,14         & 3,3           & 96,7          \\ \hline
81              & 36,79        & 17,29        & 36,25       & 99,23         & 0,77          & 6,03          & 93,97         \\ \hline
82              & 9,59         & 32,64        & 10,11       & 99,17         & 0,83          & 13,97         & 86,03         \\ \hline
83              & 90,32        & 7,79         & 88,38       & 98,1          & 1,9           & 2,4           & 97,6          \\ \hline
84              & 97,73        & 25,07        & 96,47       & 83,72         & 16,28         & 1,33          & 98,67         \\ \hline
85              & 89,08        & 35,07        & 86,84       & 87,82         & 12,18         & 3,05          & 96,95         \\ \hline
86              & 81,49        & 26,51        & 80,02       & 96,21         & 3,79          & 2,42          & 97,58         \\ \hline
87              & 96,98        & 26,13        & 94,3        & 74,59         & 25,41         & 2,91          & 97,09         \\ \hline
88              & 99,33        & 20,81        & 96,9        & 50,1          & 49,9          & 2,48          & 97,52         \\ \hline
89              & 99,24        & 33,85        & 98,36       & 62            & 38            & 0,91          & 99,09         \\ \hline
90              & 76,85        & 17,65        & 75,37       & 98,08         & 1,92          & 2,68          & 97,32         \\ \hline
91              & 86,74        & 13,78        & 83,83       & 95,86         & 4,14          & 3,97          & 96,03         \\ \hline
92              & 98,33        & 22,66        & 96,05       & 70,34         & 29,66         & 2,38          & 97,62         \\ \hline
93              & 98,57        & 22,01        & 96,63       & 71,31         & 28,69         & 2,02          & 97,98         \\ \hline
94              & 98,87        & 27,42        & 98,09       & 78,88         & 21,12         & 0,8           & 99,2          \\ \hline
95              & 93,84        & 29,12        & 92,43       & 90,5          & 9,5           & 1,65          & 98,35         \\ \hline
96              & 99,34        & 23,86        & 97,61       & 54,25         & 45,75         & 1,77          & 98,23         \\ \hline
97              & 99,18        & 35,12        & 96,89       & 38,7          & 61,3          & 2,36          & 97,64         \\ \hline
98              & 83,67        & 25,1         & 81,85       & 95,29         & 4,71          & 2,8           & 97,2          \\ \hline
99              & 84,61        & 28,41        & 84          & 98,03         & 1,97          & 0,91          & 99,09         \\ \hline
100             & 97,88        & 23,81        & 95,16       & 70,02         & 29,98         & 2,88          & 97,12         \\ \hline
\label{anx:vazquez}
\end{longtable}
\section{ENTRELAZADO-RGB}
\begin{longtable}[c]{|l|l|l|l|l|l|l|l|}
\hline
\textbf{Imagen} & \textbf{ESP} & \textbf{SEN} & \textbf{EX} & \textbf{FP\%} & \textbf{VP\%} & \textbf{FN\%} & \textbf{VN\%} \\ \hline
\endfirsthead
%
\endhead
%
1               & 99,68        & 70,87        & 99,27       & 23,9          & 76,1          & 0,42          & 99,58         \\ \hline
2               & 99,87        & 77,14        & 99,76       & 25            & 75            & 0,11          & 99,89         \\ \hline
3               & 99,45        & 24,77        & 95,05       & 26,23         & 73,77         & 4,52          & 95,48         \\ \hline
4               & 99,37        & 39,93        & 98,45       & 50,21         & 49,79         & 0,94          & 99,06         \\ \hline
5               & 99,65        & 23,75        & 98,5        & 48,51         & 51,49         & 1,17          & 98,83         \\ \hline
6               & 97,42        & 30,24        & 96,06       & 80,41         & 19,59         & 1,47          & 98,53         \\ \hline
7               & 99,61        & 8,15         & 98,06       & 73,61         & 26,39         & 1,56          & 98,44         \\ \hline
8               & 92,77        & 37,85        & 88,72       & 70,57         & 29,43         & 5,07          & 94,93         \\ \hline
9               & 4,11         & 30,14        & 4,64        & 99,36         & 0,64          & 25,83         & 74,17         \\ \hline
10              & 99,92        & 8,74         & 96,42       & 18,65         & 81,35         & 3,51          & 96,49         \\ \hline
11              & 98,54        & 49,09        & 96,31       & 38,64         & 61,36         & 2,38          & 97,62         \\ \hline
12              & 99,25        & 27,34        & 94,07       & 26,04         & 73,96         & 5,38          & 94,62         \\ \hline
13              & 98,7         & 34,42        & 95,68       & 43,52         & 56,48         & 3,16          & 96,84         \\ \hline
14              & 98,67        & 27,99        & 96,34       & 58,12         & 41,88         & 2,43          & 97,57         \\ \hline
15              & 95,75        & 31,68        & 93,75       & 80,63         & 19,37         & 2,25          & 97,75         \\ \hline
16              & 98,94        & 39,79        & 96,82       & 41,59         & 58,41         & 2,22          & 97,78         \\ \hline
17              & 98,13        & 34,29        & 96,98       & 74,78         & 25,22         & 1,21          & 98,79         \\ \hline
18              & 99,99        & 23,38        & 98,96       & 4,41          & 95,59         & 1,02          & 98,98         \\ \hline
19              & 97,68        & 24,02        & 93,44       & 61,28         & 38,72         & 4,53          & 95,47         \\ \hline
20              & 45,91        & 28,67        & 44,8        & 96,46         & 3,54          & 9,7           & 90,3          \\ \hline
21              & 99,62        & 31,2         & 98,31       & 38,76         & 61,24         & 1,33          & 98,67         \\ \hline
22              & 89,32        & 33,85        & 88,22       & 93,99         & 6,01          & 1,47          & 98,53         \\ \hline
23              & 95,5         & 23,4         & 94,9        & 95,83         & 4,17          & 0,67          & 99,33         \\ \hline
24              & 98,6         & 32,4         & 96,24       & 53,85         & 46,15         & 2,47          & 97,53         \\ \hline
25              & 99,06        & 8,71         & 95,62       & 73,18         & 26,82         & 3,52          & 96,48         \\ \hline
26              & 98,72        & 9,4          & 93,39       & 68,11         & 31,89         & 5,51          & 94,49         \\ \hline
27              & 99,36        & 8,48         & 98,68       & 90,93         & 9,07          & 0,69          & 99,31         \\ \hline
28              & 87,9         & 23,74        & 84,83       & 91,02         & 8,98          & 4,18          & 95,82         \\ \hline
29              & 99,62        & 21,3         & 96,62       & 30,67         & 69,33         & 3,06          & 96,94         \\ \hline
30              & 86,43        & 36,26        & 84,09       & 88,47         & 11,53         & 3,47          & 96,53         \\ \hline
31              & 98,19        & 24,84        & 94,83       & 60,35         & 39,65         & 3,54          & 96,46         \\ \hline
32              & 99,66        & 20,63        & 97,29       & 35,09         & 64,91         & 2,4           & 97,6          \\ \hline
33              & 97,17        & 26,06        & 94,44       & 73,08         & 26,92         & 2,95          & 97,05         \\ \hline
34              & 99,64        & 30,48        & 98,05       & 33,58         & 66,42         & 1,61          & 98,39         \\ \hline
35              & 75,13        & 26,57        & 72,91       & 95,13         & 4,87          & 4,47          & 95,53         \\ \hline
36              & 96,92        & 32,02        & 94,67       & 72,85         & 27,15         & 2,45          & 97,55         \\ \hline
37              & 99,36        & 33,74        & 96,93       & 33,12         & 66,88         & 2,5           & 97,5          \\ \hline
38              & 98,75        & 30,69        & 95,46       & 44,47         & 55,53         & 3,44          & 96,56         \\ \hline
39              & 99,77        & 39,21        & 98,57       & 22,2          & 77,8          & 1,22          & 98,78         \\ \hline
40              & 99,7         & 29,01        & 97,85       & 28,05         & 71,95         & 1,87          & 98,13         \\ \hline
41              & 98,89        & 24,59        & 94,61       & 42,44         & 57,56         & 4,46          & 95,54         \\ \hline
42              & 99,85        & 27,65        & 98,67       & 23,98         & 76,02         & 1,19          & 98,81         \\ \hline
43              & 92,56        & 33,97        & 91,35       & 91,27         & 8,73          & 1,47          & 98,53         \\ \hline
44              & 99,02        & 28,67        & 96,31       & 45,89         & 54,11         & 2,81          & 97,19         \\ \hline
45              & 93,81        & 31,75        & 91,71       & 84,77         & 15,23         & 2,48          & 97,52         \\ \hline
46              & 99,51        & 39,91        & 97,86       & 30,06         & 69,94         & 1,69          & 98,31         \\ \hline
47              & 81,81        & 14,45        & 78,61       & 96,19         & 3,81          & 4,96          & 95,04         \\ \hline
48              & 88,98        & 27,78        & 87,98       & 95,99         & 4,01          & 1,33          & 98,67         \\ \hline
49              & 81,6         & 11,67        & 79,46       & 98,04         & 1,96          & 3,3           & 96,7          \\ \hline
50              & 91,01        & 22,42        & 87,9        & 89,39         & 10,61         & 3,9           & 96,1          \\ \hline
51              & 99,83        & 25,84        & 98,35       & 24,94         & 75,06         & 1,49          & 98,51         \\ \hline
52              & 99,57        & 29,24        & 97,54       & 32,86         & 67,14         & 2,07          & 97,93         \\ \hline
53              & 90,51        & 17,54        & 89,2        & 96,74         & 3,26          & 1,63          & 98,37         \\ \hline
54              & 98,42        & 27,64        & 95,72       & 58,97         & 41,03         & 2,84          & 97,16         \\ \hline
55              & 91,68        & 32,25        & 90,6        & 93,28         & 6,72          & 1,35          & 98,65         \\ \hline
56              & 97,69        & 31,01        & 95,15       & 65,2          & 34,8          & 2,73          & 97,27         \\ \hline
57              & 1,5          & 17,56        & 2,48        & 98,85         & 1,15          & 78,13         & 21,87         \\ \hline
58              & 98,05        & 30,34        & 93,67       & 48,1          & 51,9          & 4,69          & 95,31         \\ \hline
59              & 91,35        & 27,64        & 90,21       & 94,48         & 5,52          & 1,43          & 98,57         \\ \hline
60              & 79,23        & 31,11        & 76,53       & 91,81         & 8,19          & 4,93          & 95,07         \\ \hline
61              & 96,53        & 32,04        & 95,42       & 86,08         & 13,92         & 1,22          & 98,78         \\ \hline
62              & 90,41        & 25,24        & 88,22       & 91,62         & 8,38          & 2,79          & 97,21         \\ \hline
63              & 99,69        & 36,58        & 98,86       & 38,5          & 61,5          & 0,85          & 99,15         \\ \hline
64              & 99,55        & 27,26        & 98,49       & 52,53         & 47,47         & 1,08          & 98,92         \\ \hline
65              & 3,75         & 23,35        & 4,71        & 98,76         & 1,24          & 51,38         & 48,62         \\ \hline
66              & 54,9         & 16,55        & 54,06       & 99,18         & 0,82          & 3,3           & 96,7          \\ \hline
67              & 89,87        & 24,98        & 88,52       & 95,02         & 4,98          & 1,74          & 98,26         \\ \hline
68              & 94,82        & 19,74        & 91,97       & 86,91         & 13,09         & 3,23          & 96,77         \\ \hline
69              & 99,88        & 24,28        & 99,11       & 32,12         & 67,88         & 0,78          & 99,22         \\ \hline
70              & 1,4          & 25,3         & 2,14        & 99,19         & 0,81          & 62,95         & 37,05         \\ \hline
71              & 99,42        & 30,37        & 97,97       & 46,98         & 53,02         & 1,48          & 98,52         \\ \hline
72              & 99,96        & 18,48        & 99,25       & 21,13         & 78,87         & 0,71          & 99,29         \\ \hline
73              & 72,59        & 25,6         & 69,87       & 94,57         & 5,43          & 5,93          & 94,07         \\ \hline
74              & 98,94        & 21,74        & 96,92       & 64,48         & 35,52         & 2,08          & 97,92         \\ \hline
75              & 97,07        & 23,42        & 95,55       & 85,6          & 14,4          & 1,63          & 98,37         \\ \hline
76              & 97,3         & 23,23        & 96,07       & 87,31         & 12,69         & 1,32          & 98,68         \\ \hline
77              & 16,61        & 30,08        & 17,02       & 98,89         & 1,11          & 11,56         & 88,44         \\ \hline
78              & 99,23        & 24,39        & 97,98       & 64,74         & 35,26         & 1,28          & 98,72         \\ \hline
79              & 71,86        & 31,83        & 71,01       & 97,61         & 2,39          & 2,01          & 97,99         \\ \hline
80              & 96,32        & 26,27        & 93,43       & 76,48         & 23,52         & 3,19          & 96,81         \\ \hline
81              & 14,33        & 17,35        & 14,42       & 99,43         & 0,57          & 14,14         & 85,86         \\ \hline
82              & 0,4          & 34,04        & 1,16        & 99,22         & 0,78          & 79,15         & 20,85         \\ \hline
83              & 98,9         & 7,42         & 96,75       & 86            & 14            & 2,2           & 97,8          \\ \hline
84              & 99,69        & 24,29        & 98,38       & 42,07         & 57,93         & 1,32          & 98,68         \\ \hline
85              & 88,88        & 33,62        & 86,6        & 88,45         & 11,55         & 3,12          & 96,88         \\ \hline
86              & 81,3         & 26,37        & 79,83       & 96,26         & 3,74          & 2,43          & 97,57         \\ \hline
87              & 96,45        & 26,9         & 93,82       & 77,04         & 22,96         & 2,89          & 97,11         \\ \hline
88              & 99,22        & 20,61        & 96,79       & 54,12         & 45,88         & 2,49          & 97,51         \\ \hline
89              & 99,26        & 32,53        & 98,36       & 62,28         & 37,72         & 0,93          & 99,07         \\ \hline
90              & 80,79        & 17,62        & 79,21       & 97,7          & 2,3           & 2,56          & 97,44         \\ \hline
91              & 91,81        & 13,89        & 88,7        & 93,41         & 6,59          & 3,75          & 96,25         \\ \hline
92              & 87,55        & 23,08        & 85,61       & 94,56         & 5,44          & 2,65          & 97,35         \\ \hline
93              & 98,58        & 21,72        & 96,63       & 71,41         & 28,59         & 2,03          & 97,97         \\ \hline
94              & 99,72        & 25,75        & 98,91       & 49,91         & 50,09         & 0,81          & 99,19         \\ \hline
95              & 99,19        & 29,32        & 97,68       & 55,35         & 44,65         & 1,56          & 98,44         \\ \hline
96              & 99,46        & 23,48        & 97,72       & 49,63         & 50,37         & 1,77          & 98,23         \\ \hline
97              & 99,2         & 34,34        & 96,88       & 38,7          & 61,3          & 2,39          & 97,61         \\ \hline
98              & 83,14        & 26,26        & 81,37       & 95,23         & 4,77          & 2,77          & 97,23         \\ \hline
99              & 97,56        & 27,75        & 96,81       & 88,96         & 11,04         & 0,8           & 99,2          \\ \hline
100             & 90,02        & 23,45        & 87,57       & 91,77         & 8,23          & 3,14          & 96,86         \\ \hline
\label{anx:entrelazado}
\end{longtable}
\section{LEXICOGRAFICO-HSI}
\begin{longtable}[c]{|l|l|l|l|l|l|l|l|}
\hline
\textbf{Imagen} & \textbf{ESP} & \textbf{SEN} & \textbf{EX} & \textbf{FP\%} & \textbf{VP\%} & \textbf{FN\%} & \textbf{VN\%} \\ \hline
\endfirsthead
%
\endhead
%
1               & 99,67        & 84,6         & 99,46       & 21,38         & 78,62         & 0,22          & 99,78         \\ \hline
2               & 99,9         & 75           & 99,77       & 22,22         & 77,78         & 0,12          & 99,88         \\ \hline
3               & 99,11        & 25,02        & 94,74       & 36,26         & 63,74         & 4,52          & 95,48         \\ \hline
4               & 99,64        & 39,93        & 98,72       & 36,59         & 63,41         & 0,94          & 99,06         \\ \hline
5               & 99,86        & 22,27        & 98,68       & 29,05         & 70,95         & 1,19          & 98,81         \\ \hline
6               & 97,46        & 30,72        & 96,1        & 79,93         & 20,07         & 1,45          & 98,55         \\ \hline
7               & 99,62        & 11,94        & 98,13       & 65,1          & 34,9          & 1,5           & 98,5          \\ \hline
8               & 32,52        & 40,42        & 33,1        & 95,45         & 4,55          & 12,74         & 87,26         \\ \hline
9               & 5,26         & 30,14        & 5,76        & 99,35         & 0,65          & 21,41         & 78,59         \\ \hline
10              & 97,48        & 12,42        & 94,22       & 83,57         & 16,43         & 3,46          & 96,54         \\ \hline
11              & 98,52        & 49,72        & 96,32       & 38,64         & 61,36         & 2,36          & 97,64         \\ \hline
12              & 98,86        & 26,12        & 93,62       & 35,93         & 64,07         & 5,49          & 94,51         \\ \hline
13              & 98,67        & 34,15        & 95,65       & 44,19         & 55,81         & 3,18          & 96,82         \\ \hline
14              & 98,84        & 28,94        & 96,53       & 54,04         & 45,96         & 2,4           & 97,6          \\ \hline
15              & 95           & 36,19        & 93,17       & 81,07         & 18,93         & 2,12          & 97,88         \\ \hline
16              & 99,43        & 39,34        & 97,27       & 27,85         & 72,15         & 2,23          & 97,77         \\ \hline
17              & 97,91        & 33,24        & 96,75       & 77,39         & 22,61         & 1,24          & 98,76         \\ \hline
18              & 100          & 23,38        & 98,98       & 0             & 100           & 1,02          & 98,98         \\ \hline
19              & 98,03        & 22,71        & 93,7        & 58,72         & 41,28         & 4,59          & 95,41         \\ \hline
20              & 54,12        & 28,39        & 52,46       & 95,9          & 4,1           & 8,38          & 91,62         \\ \hline
21              & 99,52        & 29,23        & 98,18       & 45,77         & 54,23         & 1,37          & 98,63         \\ \hline
22              & 89,2         & 35,44        & 88,14       & 93,78         & 6,22          & 1,44          & 98,56         \\ \hline
23              & 99,92        & 20,92        & 99,26       & 32,61         & 67,39         & 0,66          & 99,34         \\ \hline
24              & 98,61        & 33,11        & 96,27       & 53,11         & 46,89         & 2,45          & 97,55         \\ \hline
25              & 99,22        & 8,77         & 95,78       & 69,11         & 30,89         & 3,51          & 96,49         \\ \hline
26              & 98,57        & 9,42         & 93,25       & 70,44         & 29,56         & 5,52          & 94,48         \\ \hline
27              & 99,74        & 8,69         & 99,06       & 79,8          & 20,2          & 0,69          & 99,31         \\ \hline
28              & 36,19        & 24,91        & 35,65       & 98,07         & 1,93          & 9,45          & 90,55         \\ \hline
29              & 99,7         & 21,54        & 96,7        & 25,97         & 74,03         & 3,05          & 96,95         \\ \hline
30              & 98,45        & 34,64        & 95,48       & 47,81         & 52,19         & 3,14          & 96,86         \\ \hline
31              & 91,64        & 25,61        & 88,62       & 87,18         & 12,82         & 3,75          & 96,25         \\ \hline
32              & 99,75        & 20,7         & 97,38       & 28,51         & 71,49         & 2,39          & 97,61         \\ \hline
33              & 98,49        & 24,9         & 95,66       & 60,34         & 39,66         & 2,95          & 97,05         \\ \hline
34              & 99,71        & 30,19        & 98,12       & 28,78         & 71,22         & 1,62          & 98,38         \\ \hline
35              & 97,15        & 24,38        & 93,83       & 70,92         & 29,08         & 3,59          & 96,41         \\ \hline
36              & 99,63        & 29,1         & 97,19       & 25,93         & 74,07         & 2,49          & 97,51         \\ \hline
37              & 99,38        & 34,18        & 96,97       & 32,02         & 67,98         & 2,48          & 97,52         \\ \hline
38              & 98,77        & 32,75        & 95,58       & 42,56         & 57,44         & 3,34          & 96,66         \\ \hline
39              & 99,79        & 39,66        & 98,6        & 20,4          & 79,6          & 1,21          & 98,79         \\ \hline
40              & 99,8         & 28,84        & 97,94       & 20,86         & 79,14         & 1,88          & 98,12         \\ \hline
41              & 98,98        & 24,66        & 94,7        & 40,35         & 59,65         & 4,45          & 95,55         \\ \hline
42              & 99,89        & 26,99        & 98,7        & 19,73         & 80,27         & 1,2           & 98,8          \\ \hline
43              & 98,47        & 33,39        & 97,14       & 68,57         & 31,43         & 1,4           & 98,6          \\ \hline
44              & 98,75        & 25,78        & 95,93       & 54,77         & 45,23         & 2,93          & 97,07         \\ \hline
45              & 58,59        & 34,84        & 57,79       & 97,14         & 2,86          & 3,75          & 96,25         \\ \hline
46              & 99,58        & 39,76        & 97,92       & 26,87         & 73,13         & 1,7           & 98,3          \\ \hline
47              & 81,45        & 14,39        & 78,26       & 96,28         & 3,72          & 4,98          & 95,02         \\ \hline
48              & 88,33        & 27,97        & 87,35       & 96,18         & 3,82          & 1,33          & 98,67         \\ \hline
49              & 72,62        & 11,39        & 70,74       & 98,7          & 1,3           & 3,71          & 96,29         \\ \hline
50              & 90,54        & 23,38        & 87,49       & 89,48         & 10,52         & 3,87          & 96,13         \\ \hline
51              & 99,85        & 27,31        & 98,41       & 20,85         & 79,15         & 1,46          & 98,54         \\ \hline
52              & 99,61        & 29,24        & 97,58       & 30,79         & 69,21         & 2,07          & 97,93         \\ \hline
53              & 90,66        & 18,2         & 89,36       & 96,57         & 3,43          & 1,62          & 98,38         \\ \hline
54              & 98,7         & 27,89        & 95,99       & 54            & 46            & 2,82          & 97,18         \\ \hline
55              & 91,43        & 33,84        & 90,38       & 93,16         & 6,84          & 1,33          & 98,67         \\ \hline
56              & 97,45        & 31,32        & 94,92       & 67,22         & 32,78         & 2,72          & 97,28         \\ \hline
57              & 1,49         & 17,56        & 2,47        & 98,85         & 1,15          & 78,29         & 21,71         \\ \hline
58              & 97,83        & 31,26        & 93,52       & 50,12         & 49,88         & 4,64          & 95,36         \\ \hline
59              & 91,8         & 27,26        & 90,64       & 94,27         & 5,73          & 1,43          & 98,57         \\ \hline
60              & 79,29        & 32,05        & 76,63       & 91,56         & 8,44          & 4,86          & 95,14         \\ \hline
61              & 97,4         & 27,47        & 96,2        & 84,37         & 15,63         & 1,29          & 98,71         \\ \hline
62              & 75,27        & 25,97        & 73,62       & 96,48         & 3,52          & 3,31          & 96,69         \\ \hline
63              & 99,72        & 29,96        & 98,8        & 41,22         & 58,78         & 0,93          & 99,07         \\ \hline
64              & 99,74        & 26,8         & 98,67       & 39,45         & 60,55         & 1,08          & 98,92         \\ \hline
65              & 1,81         & 23,35        & 2,87        & 98,79         & 1,21          & 68,62         & 31,38         \\ \hline
66              & 91,68        & 7,86         & 89,83       & 97,92         & 2,08          & 2,21          & 97,79         \\ \hline
67              & 99,59        & 25,02        & 98,03       & 43,65         & 56,35         & 1,58          & 98,42         \\ \hline
68              & 96,13        & 19,79        & 93,23       & 83,2          & 16,8          & 3,19          & 96,81         \\ \hline
69              & 99,77        & 26,11        & 99,02       & 45,55         & 54,45         & 0,76          & 99,24         \\ \hline
70              & 1,5          & 25,3         & 2,23        & 99,19         & 0,81          & 61,44         & 38,56         \\ \hline
71              & 99,39        & 30,86        & 97,95       & 47,98         & 52,02         & 1,47          & 98,53         \\ \hline
72              & 99,96        & 22,46        & 99,29       & 17,33         & 82,67         & 0,67          & 99,33         \\ \hline
73              & 73,65        & 24,84        & 70,83       & 94,52         & 5,48          & 5,9           & 94,1          \\ \hline
74              & 99,51        & 21,63        & 97,47       & 45,65         & 54,35         & 2,07          & 97,93         \\ \hline
75              & 98,71        & 21,16        & 97,11       & 74,36         & 25,64         & 1,65          & 98,35         \\ \hline
76              & 98,03        & 23,73        & 96,8        & 83,07         & 16,93         & 1,3           & 98,7          \\ \hline
77              & 93,6         & 26,85        & 91,58       & 88,47         & 11,53         & 2,37          & 97,63         \\ \hline
78              & 98,84        & 24,39        & 97,59       & 73,51         & 26,49         & 1,29          & 98,71         \\ \hline
79              & 72,23        & 31,93        & 71,38       & 97,57         & 2,43          & 2             & 98            \\ \hline
80              & 91,03        & 26,78        & 88,38       & 88,6          & 11,4          & 3,35          & 96,65         \\ \hline
81              & 22,48        & 17,49        & 22,34       & 99,36         & 0,64          & 9,49          & 90,51         \\ \hline
82              & 0,26         & 34,04        & 1,02        & 99,22         & 0,78          & 85,62         & 14,38         \\ \hline
83              & 90,46        & 7,75         & 88,51       & 98,08         & 1,92          & 2,4           & 97,6          \\ \hline
84              & 99,76        & 24,93        & 98,46       & 35,69         & 64,31         & 1,31          & 98,69         \\ \hline
85              & 89,07        & 36,85        & 86,91       & 87,29         & 12,71         & 2,97          & 97,03         \\ \hline
86              & 82,34        & 26,23        & 80,83       & 96,07         & 3,93          & 2,41          & 97,59         \\ \hline
87              & 96,74        & 26,52        & 94,08       & 75,76         & 24,24         & 2,9           & 97,1          \\ \hline
88              & 99,36        & 20,87        & 96,93       & 49,18         & 50,82         & 2,48          & 97,52         \\ \hline
89              & 99,3         & 32,76        & 98,4        & 60,9          & 39,1          & 0,92          & 99,08         \\ \hline
90              & 78,31        & 17,62        & 76,79       & 97,95         & 2,05          & 2,63          & 97,37         \\ \hline
91              & 89,99        & 13,75        & 86,95       & 94,59         & 5,41          & 3,83          & 96,17         \\ \hline
92              & 97,22        & 22,6         & 94,97       & 79,87         & 20,13         & 2,41          & 97,59         \\ \hline
93              & 98,59        & 22,03        & 96,64       & 71,05         & 28,95         & 2,02          & 97,98         \\ \hline
94              & 98,97        & 27,42        & 98,19       & 77,34         & 22,66         & 0,8           & 99,2          \\ \hline
95              & 93,82        & 29,07        & 92,41       & 90,55         & 9,45          & 1,65          & 98,35         \\ \hline
96              & 99,2         & 23,96        & 97,48       & 58,67         & 41,33         & 1,77          & 98,23         \\ \hline
97              & 99,27        & 34,42        & 96,96       & 36,43         & 63,57         & 2,39          & 97,61         \\ \hline
98              & 81,95        & 26,69        & 80,23       & 95,46         & 4,54          & 2,79          & 97,21         \\ \hline
99              & 84,75        & 27,97        & 84,14       & 98,04         & 1,96          & 0,92          & 99,08         \\ \hline
100             & 98,2         & 23,7         & 95,46       & 66,62         & 33,38         & 2,88          & 97,12         \\ \hline
\label{anx:lexhsi}
\end{longtable}
\section{ENTROPIA-RGB}
\begin{longtable}[c]{|l|l|l|l|l|l|l|l|}
\hline
\textbf{Imagen} & \textbf{ESP} & \textbf{SEN} & \textbf{EX} & \textbf{FP\%} & \textbf{VP\%} & \textbf{FN\%} & \textbf{VN\%} \\ \hline
\endfirsthead
%
\endhead
%
1               & 99,69        & 69,57        & 99,26       & 23,94         & 76,06         & 0,44          & 99,56         \\ \hline
2               & 99,87        & 71,43        & 99,74       & 26,47         & 73,53         & 0,14          & 99,86         \\ \hline
3               & 99,34        & 25,55        & 95          & 29,07         & 70,93         & 4,48          & 95,52         \\ \hline
4               & 99,46        & 40,27        & 98,54       & 46,1          & 53,9          & 0,93          & 99,07         \\ \hline
5               & 99,67        & 24,07        & 98,52       & 47,33         & 52,67         & 1,16          & 98,84         \\ \hline
6               & 98,02        & 29,91        & 96,64       & 76,08         & 23,92         & 1,46          & 98,54         \\ \hline
7               & 99,62        & 8,15         & 98,07       & 73,21         & 26,79         & 1,56          & 98,44         \\ \hline
8               & 33,14        & 39,65        & 33,62       & 95,49         & 4,51          & 12,67         & 87,33         \\ \hline
9               & 5,45         & 30,14        & 5,94        & 99,35         & 0,65          & 20,83         & 79,17         \\ \hline
10              & 99,89        & 9,08         & 96,41       & 22,75         & 77,25         & 3,5           & 96,5          \\ \hline
11              & 98,49        & 49,29        & 96,27       & 39,29         & 60,71         & 2,38          & 97,62         \\ \hline
12              & 98,86        & 26,85        & 93,67       & 35,26         & 64,74         & 5,44          & 94,56         \\ \hline
13              & 98,77        & 34,33        & 95,75       & 42,25         & 57,75         & 3,17          & 96,83         \\ \hline
14              & 98,78        & 27,73        & 96,43       & 56,37         & 43,63         & 2,44          & 97,56         \\ \hline
15              & 95,96        & 31,89        & 93,96       & 79,71         & 20,29         & 2,24          & 97,76         \\ \hline
16              & 98,97        & 39,58        & 96,83       & 41,1          & 58,9          & 2,23          & 97,77         \\ \hline
17              & 96,89        & 34,82        & 95,78       & 82,94         & 17,06         & 1,22          & 98,78         \\ \hline
18              & 99,99        & 23,38        & 98,96       & 4,41          & 95,59         & 1,02          & 98,98         \\ \hline
19              & 97,92        & 23,52        & 93,64       & 59,22         & 40,78         & 4,55          & 95,45         \\ \hline
20              & 52,5         & 28,67        & 50,96       & 95,99         & 4,01          & 8,58          & 91,42         \\ \hline
21              & 99,49        & 31,67        & 98,19       & 45,38         & 54,62         & 1,32          & 98,68         \\ \hline
22              & 89,42        & 31,22        & 88,27       & 94,37         & 5,63          & 1,53          & 98,47         \\ \hline
23              & 91,9         & 24,41        & 91,34       & 97,54         & 2,46          & 0,68          & 99,32         \\ \hline
24              & 98,68        & 32,35        & 96,32       & 52,37         & 47,63         & 2,47          & 97,53         \\ \hline
25              & 99,2         & 8,69         & 95,76       & 69,82         & 30,18         & 3,52          & 96,48         \\ \hline
26              & 98,79        & 9,52         & 93,45       & 66,71         & 33,29         & 5,5           & 94,5          \\ \hline
27              & 99,68        & 8,77         & 99          & 82,63         & 17,37         & 0,69          & 99,31         \\ \hline
28              & 86,87        & 23,97        & 83,86       & 91,59         & 8,41          & 4,22          & 95,78         \\ \hline
29              & 99,67        & 21,3         & 96,66       & 27,88         & 72,12         & 3,06          & 96,94         \\ \hline
30              & 98,43        & 34,57        & 95,46       & 48,27         & 51,73         & 3,14          & 96,86         \\ \hline
31              & 98,4         & 24,91        & 95,03       & 57,3          & 42,7          & 3,53          & 96,47         \\ \hline
32              & 99,72        & 21,02        & 97,36       & 30,49         & 69,51         & 2,39          & 97,61         \\ \hline
33              & 97,9         & 25,61        & 95,12       & 67,25         & 32,75         & 2,94          & 97,06         \\ \hline
34              & 99,67        & 30,34        & 98,08       & 31,88         & 68,12         & 1,61          & 98,39         \\ \hline
35              & 91,46        & 25,35        & 88,44       & 87,56         & 12,44         & 3,76          & 96,24         \\ \hline
36              & 96,98        & 31,02        & 94,69       & 73,11         & 26,89         & 2,49          & 97,51         \\ \hline
37              & 99,58        & 32,88        & 97,11       & 24,94         & 75,06         & 2,53          & 97,47         \\ \hline
38              & 97,96        & 34           & 94,87       & 54,2          & 45,8          & 3,31          & 96,69         \\ \hline
39              & 99,79        & 38,96        & 98,58       & 21,06         & 78,94         & 1,23          & 98,77         \\ \hline
40              & 99,72        & 29,22        & 97,87       & 26,56         & 73,44         & 1,87          & 98,13         \\ \hline
41              & 68,09        & 25,82        & 65,66       & 95,28         & 4,72          & 6,25          & 93,75         \\ \hline
42              & 99,88        & 27,65        & 98,7        & 20,05         & 79,95         & 1,19          & 98,81         \\ \hline
43              & 98,53        & 33,51        & 97,19       & 67,71         & 32,29         & 1,39          & 98,61         \\ \hline
44              & 98           & 28,97        & 95,34       & 63,19         & 36,81         & 2,83          & 97,17         \\ \hline
45              & 98,51        & 32,1         & 96,26       & 57,04         & 42,96         & 2,36          & 97,64         \\ \hline
46              & 99,61        & 39,8         & 97,95       & 25,68         & 74,32         & 1,7           & 98,3          \\ \hline
47              & 73,4         & 14,41        & 70,6        & 97,37         & 2,63          & 5,5           & 94,5          \\ \hline
48              & 88,92        & 27,99        & 87,92       & 95,98         & 4,02          & 1,33          & 98,67         \\ \hline
49              & 83,58        & 11,4         & 81,37       & 97,85         & 2,15          & 3,24          & 96,76         \\ \hline
50              & 88,39        & 23,24        & 85,43       & 91,3          & 8,7           & 3,97          & 96,03         \\ \hline
51              & 99,84        & 26,08        & 98,36       & 23,68         & 76,32         & 1,48          & 98,52         \\ \hline
52              & 99,5         & 29,4         & 97,47       & 36,29         & 63,71         & 2,07          & 97,93         \\ \hline
53              & 90,47        & 17,87        & 89,17       & 96,69         & 3,31          & 1,63          & 98,37         \\ \hline
54              & 98,48        & 27,84        & 95,78       & 57,84         & 42,16         & 2,83          & 97,17         \\ \hline
55              & 91,9         & 32,8         & 90,83       & 93            & 7             & 1,34          & 98,66         \\ \hline
56              & 97,58        & 30,99        & 95,04       & 66,29         & 33,71         & 2,73          & 97,27         \\ \hline
57              & 1,5          & 17,56        & 2,49        & 98,85         & 1,15          & 78,11         & 21,89         \\ \hline
58              & 98,02        & 30,68        & 93,66       & 48,19         & 51,81         & 4,67          & 95,33         \\ \hline
59              & 91,37        & 26,94        & 90,22       & 94,6          & 5,4           & 1,44          & 98,56         \\ \hline
60              & 79,42        & 30,75        & 76,68       & 91,83         & 8,17          & 4,94          & 95,06         \\ \hline
61              & 95,51        & 32,15        & 94,42       & 88,86         & 11,14         & 1,23          & 98,77         \\ \hline
62              & 81,15        & 25,24        & 79,27       & 95,55         & 4,45          & 3,1           & 96,9          \\ \hline
63              & 99,81        & 32,43        & 98,92       & 30,36         & 69,64         & 0,9           & 99,1          \\ \hline
64              & 98,2         & 28,94        & 97,19       & 80,67         & 19,33         & 1,06          & 98,94         \\ \hline
65              & 3,23         & 23,35        & 4,22        & 98,77         & 1,23          & 55,08         & 44,92         \\ \hline
66              & 53,3         & 16,59        & 52,49       & 99,21         & 0,79          & 3,4           & 96,6          \\ \hline
67              & 95,69        & 24,79        & 94,21       & 89,11         & 10,89         & 1,64          & 98,36         \\ \hline
68              & 96,46        & 19,62        & 93,55       & 82,02         & 17,98         & 3,19          & 96,81         \\ \hline
69              & 99,82        & 26,28        & 99,06       & 40,43         & 59,57         & 0,76          & 99,24         \\ \hline
70              & 1,4          & 25,3         & 2,13        & 99,19         & 0,81          & 63,07         & 36,93         \\ \hline
71              & 99,37        & 30,9         & 97,93       & 48,82         & 51,18         & 1,47          & 98,53         \\ \hline
72              & 99,96        & 20,05        & 99,27       & 17            & 83            & 0,69          & 99,31         \\ \hline
73              & 73,13        & 25,85        & 70,4        & 94,42         & 5,58          & 5,87          & 94,13         \\ \hline
74              & 99,13        & 21,58        & 97,1        & 60,11         & 39,89         & 2,08          & 97,92         \\ \hline
75              & 99,43        & 21,45        & 97,82       & 55,69         & 44,31         & 1,64          & 98,36         \\ \hline
76              & 99,48        & 23,12        & 98,21       & 56,97         & 43,03         & 1,29          & 98,71         \\ \hline
77              & 23,24        & 28,74        & 23,4        & 98,85         & 1,15          & 8,7           & 91,3          \\ \hline
78              & 99,21        & 23,74        & 97,94       & 66,05         & 33,95         & 1,3           & 98,7          \\ \hline
79              & 72,16        & 31,61        & 71,3        & 97,6          & 2,4           & 2,01          & 97,99         \\ \hline
80              & 93,24        & 26,31        & 90,48       & 85,63         & 14,37         & 3,29          & 96,71         \\ \hline
81              & 14,65        & 17,33        & 14,72       & 99,42         & 0,58          & 13,88         & 86,12         \\ \hline
82              & 0,41         & 34,04        & 1,17        & 99,22         & 0,78          & 78,86         & 21,14         \\ \hline
83              & 99,52        & 7,53         & 97,36       & 72,53         & 27,47         & 2,19          & 97,81         \\ \hline
84              & 97,81        & 24,93        & 96,54       & 83,31         & 16,69         & 1,33          & 98,67         \\ \hline
85              & 89,07        & 36,72        & 86,91       & 87,33         & 12,67         & 2,98          & 97,02         \\ \hline
86              & 82,15        & 26,42        & 80,66       & 96,09         & 3,91          & 2,41          & 97,59         \\ \hline
87              & 96,82        & 26,36        & 94,15       & 75,44         & 24,56         & 2,9           & 97,1          \\ \hline
88              & 99,23        & 20,54        & 96,79       & 53,98         & 46,02         & 2,49          & 97,51         \\ \hline
89              & 99,33        & 32,3         & 98,43       & 60,03         & 39,97         & 0,93          & 99,07         \\ \hline
90              & 78,68        & 17,49        & 77,15       & 97,93         & 2,07          & 2,63          & 97,37         \\ \hline
91              & 94,52        & 13,7         & 91,29       & 90,58         & 9,42          & 3,66          & 96,34         \\ \hline
92              & 97,6         & 22,78        & 95,35       & 77,21         & 22,79         & 2,4           & 97,6          \\ \hline
93              & 98,74        & 21,21        & 96,77       & 69,52         & 30,48         & 2,04          & 97,96         \\ \hline
94              & 99,69        & 26,97        & 98,9        & 50,87         & 49,13         & 0,8           & 99,2          \\ \hline
95              & 99,17        & 29,12        & 97,65       & 56,15         & 43,85         & 1,56          & 98,44         \\ \hline
96              & 99,26        & 23,96        & 97,53       & 56,89         & 43,11         & 1,77          & 98,23         \\ \hline
97              & 99,27        & 33,77        & 96,93       & 37,01         & 62,99         & 2,41          & 97,59         \\ \hline
98              & 83,01        & 25,99        & 81,24       & 95,31         & 4,69          & 2,79          & 97,21         \\ \hline
99              & 85,21        & 28,62        & 84,6        & 97,93         & 2,07          & 0,91          & 99,09         \\ \hline
100             & 97,35        & 23,23        & 94,63       & 74,91         & 25,09         & 2,92          & 97,08         \\ \hline
\label{anx:entropia}
\end{longtable}
\section{MAXIMO-RGB}
% Please add the following required packages to your document preamble:
% \usepackage{longtable}
% Note: It may be necessary to compile the document several times to get a multi-page table to line up properly
\begin{longtable}[c]{|l|l|l|l|l|l|l|l|}
\hline
\textbf{Imagen} & \textbf{ESP} & \textbf{SEN} & \textbf{EX} & \textbf{FP\%} & \textbf{VP\%} & \textbf{FN\%} & \textbf{VN\%} \\ \hline
\endfirsthead
%
\endhead
%
1               & 99,68        & 84,79        & 99,47       & 20,93         & 79,07         & 0,22          & 99,78         \\ \hline
2               & 99,91        & 75,71        & 99,79       & 20,3          & 79,7          & 0,12          & 99,88         \\ \hline
3               & 99,12        & 25,63        & 94,79       & 35,47         & 64,53         & 4,49          & 95,51         \\ \hline
4               & 99,65        & 38,94        & 98,71       & 36,5          & 63,5          & 0,95          & 99,05         \\ \hline
5               & 99,88        & 21,31        & 98,68       & 27,17         & 72,83         & 1,2           & 98,8          \\ \hline
6               & 97,18        & 30,96        & 95,83       & 81,44         & 18,56         & 1,45          & 98,55         \\ \hline
7               & 99,62        & 11,02        & 98,13       & 66,43         & 33,57         & 1,51          & 98,49         \\ \hline
8               & 31,2         & 40,52        & 31,88       & 95,52         & 4,48          & 13,18         & 86,82         \\ \hline
9               & 3,66         & 30,14        & 4,19        & 99,36         & 0,64          & 28,15         & 71,85         \\ \hline
10              & 97,49        & 12,36        & 94,22       & 83,62         & 16,38         & 3,46          & 96,54         \\ \hline
11              & 98,51        & 49,79        & 96,31       & 38,76         & 61,24         & 2,35          & 97,65         \\ \hline
12              & 98,76        & 28,16        & 93,67       & 36,11         & 63,89         & 5,35          & 94,65         \\ \hline
13              & 98,71        & 34,06        & 95,68       & 43,6          & 56,4          & 3,18          & 96,82         \\ \hline
14              & 98,93        & 28,94        & 96,61       & 52,07         & 47,93         & 2,4           & 97,6          \\ \hline
15              & 95,04        & 36,22        & 93,21       & 80,94         & 19,06         & 2,12          & 97,88         \\ \hline
16              & 99,04        & 39,85        & 96,92       & 39,12         & 60,88         & 2,22          & 97,78         \\ \hline
17              & 97,93        & 32,71        & 96,75       & 77,53         & 22,47         & 1,24          & 98,76         \\ \hline
18              & 100          & 23,38        & 98,98       & 0             & 100           & 1,02          & 98,98         \\ \hline
19              & 97,95        & 22,86        & 93,63       & 59,52         & 40,48         & 4,59          & 95,41         \\ \hline
20              & 54,14        & 28,39        & 52,47       & 95,9          & 4,1           & 8,38          & 91,62         \\ \hline
21              & 99,51        & 29,01        & 98,16       & 46,44         & 53,56         & 1,37          & 98,63         \\ \hline
22              & 88,89        & 35,76        & 87,84       & 93,9          & 6,1           & 1,44          & 98,56         \\ \hline
23              & 99,93        & 20,58        & 99,27       & 29,07         & 70,93         & 0,66          & 99,34         \\ \hline
24              & 98,54        & 33,72        & 96,23       & 53,87         & 46,13         & 2,43          & 97,57         \\ \hline
25              & 99,27        & 8,73         & 95,82       & 67,86         & 32,14         & 3,51          & 96,49         \\ \hline
26              & 97,6         & 9,48         & 92,34       & 79,92         & 20,08         & 5,56          & 94,44         \\ \hline
27              & 99,72        & 8,64         & 99,03       & 81,3          & 18,7          & 0,69          & 99,31         \\ \hline
28              & 36,05        & 24,92        & 35,52       & 98,08         & 1,92          & 9,48          & 90,52         \\ \hline
29              & 99,7         & 21,39        & 96,7        & 25,72         & 74,28         & 3,05          & 96,95         \\ \hline
30              & 98,53        & 32,8         & 95,47       & 47,94         & 52,06         & 3,22          & 96,78         \\ \hline
31              & 78,09        & 25,73        & 75,69       & 94,67         & 5,33          & 4,37          & 95,63         \\ \hline
32              & 99,76        & 20,57        & 97,39       & 27,5          & 72,5          & 2,4           & 97,6          \\ \hline
33              & 98,36        & 24,73        & 95,53       & 62,47         & 37,53         & 2,97          & 97,03         \\ \hline
34              & 99,7         & 30           & 98,1        & 29,73         & 70,27         & 1,62          & 98,38         \\ \hline
35              & 97,11        & 25,76        & 93,85       & 70,05         & 29,95         & 3,53          & 96,47         \\ \hline
36              & 99,63        & 29,33        & 97,2        & 25,87         & 74,13         & 2,48          & 97,52         \\ \hline
37              & 99,31        & 34,73        & 96,92       & 34,1          & 65,9          & 2,47          & 97,53         \\ \hline
38              & 98,78        & 32,7         & 95,59       & 42,39         & 57,61         & 3,34          & 96,66         \\ \hline
39              & 99,82        & 38,19        & 98,59       & 19,12         & 80,88         & 1,24          & 98,76         \\ \hline
40              & 99,76        & 29,45        & 97,93       & 23,05         & 76,95         & 1,86          & 98,14         \\ \hline
41              & 91,67        & 25,05        & 87,83       & 84,46         & 15,54         & 4,76          & 95,24         \\ \hline
42              & 99,83        & 26,4         & 98,62       & 28,29         & 71,71         & 1,21          & 98,79         \\ \hline
43              & 98,44        & 33,22        & 97,1        & 69,12         & 30,88         & 1,4           & 98,6          \\ \hline
44              & 98,76        & 25,66        & 95,93       & 54,68         & 45,32         & 2,93          & 97,07         \\ \hline
45              & 93,19        & 34,22        & 91,2        & 85,03         & 14,97         & 2,41          & 97,59         \\ \hline
46              & 99,59        & 39,64        & 97,93       & 26,4          & 73,6          & 1,7           & 98,3          \\ \hline
47              & 82,08        & 14,36        & 78,86       & 96,16         & 3,84          & 4,95          & 95,05         \\ \hline
48              & 88,13        & 28,02        & 87,15       & 96,23         & 3,77          & 1,34          & 98,66         \\ \hline
49              & 38,68        & 11,32        & 37,85       & 99,42         & 0,58          & 6,75          & 93,25         \\ \hline
50              & 94,3         & 23,18        & 91,07       & 83,79         & 16,21         & 3,73          & 96,27         \\ \hline
51              & 99,9         & 27,64        & 98,46       & 15,5          & 84,5          & 1,45          & 98,55         \\ \hline
52              & 99,6         & 29,13        & 97,56       & 31,54         & 68,46         & 2,08          & 97,92         \\ \hline
53              & 90,49        & 18,07        & 89,2        & 96,65         & 3,35          & 1,62          & 98,38         \\ \hline
54              & 99,22        & 27,44        & 96,47       & 41,87         & 58,13         & 2,82          & 97,18         \\ \hline
55              & 91,23        & 34,06        & 90,19       & 93,26         & 6,74          & 1,33          & 98,67         \\ \hline
56              & 97,27        & 30,78        & 94,73       & 69,08         & 30,92         & 2,75          & 97,25         \\ \hline
57              & 1,49         & 17,56        & 2,48        & 98,85         & 1,15          & 78,22         & 21,78         \\ \hline
58              & 97,82        & 31,41        & 93,52       & 50,06         & 49,94         & 4,63          & 95,37         \\ \hline
59              & 91,64        & 27,03        & 90,48       & 94,42         & 5,58          & 1,44          & 98,56         \\ \hline
60              & 79,17        & 31,34        & 76,48       & 91,77         & 8,23          & 4,91          & 95,09         \\ \hline
61              & 96,53        & 27,07        & 95,34       & 87,97         & 12,03         & 1,31          & 98,69         \\ \hline
62              & 70,21        & 28,22        & 68,8        & 96,81         & 3,19          & 3,43          & 96,57         \\ \hline
63              & 99,69        & 30,74        & 98,78       & 42,68         & 57,32         & 0,92          & 99,08         \\ \hline
64              & 99,74        & 26,65        & 98,66       & 40            & 60            & 1,08          & 98,92         \\ \hline
65              & 1,84         & 23,35        & 2,9         & 98,79         & 1,21          & 68,24         & 31,76         \\ \hline
66              & 94,48        & 2,54         & 92,45       & 98,98         & 1,02          & 2,27          & 97,73         \\ \hline
67              & 99,6         & 24,98        & 98,05       & 42,89         & 57,11         & 1,58          & 98,42         \\ \hline
68              & 97,32        & 19,75        & 94,38       & 77,44         & 22,56         & 3,15          & 96,85         \\ \hline
69              & 99,79        & 25,94        & 99,03       & 44,3          & 55,7          & 0,76          & 99,24         \\ \hline
70              & 1,5          & 25,3         & 2,23        & 99,19         & 0,81          & 61,38         & 38,62         \\ \hline
71              & 99,36        & 30,96        & 97,92       & 49,1          & 50,9          & 1,47          & 98,53         \\ \hline
72              & 85,81        & 24,03        & 85,28       & 98,54         & 1,46          & 0,77          & 99,23         \\ \hline
73              & 74,13        & 24,82        & 71,28       & 94,43         & 5,57          & 5,87          & 94,13         \\ \hline
74              & 99,53        & 21,65        & 97,49       & 44,77         & 55,23         & 2,07          & 97,93         \\ \hline
75              & 98,41        & 22,05        & 96,84       & 77,34         & 22,66         & 1,64          & 98,36         \\ \hline
76              & 98           & 23,61        & 96,76       & 83,37         & 16,63         & 1,3           & 98,7          \\ \hline
77              & 94,68        & 26,93        & 92,64       & 86,41         & 13,59         & 2,34          & 97,66         \\ \hline
78              & 99,08        & 24,86        & 97,83       & 68,48         & 31,52         & 1,28          & 98,72         \\ \hline
79              & 72,15        & 31,93        & 71,3        & 97,58         & 2,42          & 2             & 98            \\ \hline
80              & 92,61        & 26,52        & 89,88       & 86,62         & 13,38         & 3,31          & 96,69         \\ \hline
81              & 36,8         & 17,27        & 36,25       & 99,23         & 0,77          & 6,03          & 93,97         \\ \hline
82              & 9,6          & 32,62        & 10,12       & 99,17         & 0,83          & 13,97         & 86,03         \\ \hline
83              & 90,49        & 7,75         & 88,55       & 98,08         & 1,92          & 2,39          & 97,61         \\ \hline
84              & 97,72        & 25,12        & 96,46       & 83,73         & 16,27         & 1,33          & 98,67         \\ \hline
85              & 89,08        & 36,67        & 86,91       & 87,34         & 12,66         & 2,98          & 97,02         \\ \hline
86              & 81,59        & 26,37        & 80,11       & 96,21         & 3,79          & 2,42          & 97,58         \\ \hline
87              & 96,79        & 26,22        & 94,12       & 75,67         & 24,33         & 2,91          & 97,09         \\ \hline
88              & 99,36        & 20,84        & 96,93       & 48,88         & 51,12         & 2,48          & 97,52         \\ \hline
89              & 99,3         & 33,26        & 98,4        & 60,53         & 39,47         & 0,91          & 99,09         \\ \hline
90              & 78,49        & 17,44        & 76,96       & 97,96         & 2,04          & 2,63          & 97,37         \\ \hline
91              & 92,21        & 13,76        & 89,08       & 93,15         & 6,85          & 3,74          & 96,26         \\ \hline
92              & 98,39        & 22,75        & 96,11       & 69,54         & 30,46         & 2,38          & 97,62         \\ \hline
93              & 98,56        & 22,06        & 96,61       & 71,49         & 28,51         & 2,02          & 97,98         \\ \hline
94              & 98,87        & 27,38        & 98,09       & 78,96         & 21,04         & 0,8           & 99,2          \\ \hline
95              & 93,82        & 29,07        & 92,41       & 90,55         & 9,45          & 1,65          & 98,35         \\ \hline
96              & 99,15        & 23,96        & 97,42       & 60,3          & 39,7          & 1,77          & 98,23         \\ \hline
97              & 99,17        & 35,28        & 96,89       & 38,9          & 61,1          & 2,36          & 97,64         \\ \hline
98              & 83,6         & 25,1         & 81,78       & 95,31         & 4,69          & 2,8           & 97,2          \\ \hline
99              & 84,45        & 28,19        & 83,84       & 98,06         & 1,94          & 0,92          & 99,08         \\ \hline
100             & 51,95        & 25,85        & 50,99       & 97,99         & 2,01          & 5,17          & 94,83         \\ \hline
\label{anx:max}
\end{longtable}
\section{MINIMO-RGB}
\begin{longtable}[c]{|l|l|l|l|l|l|l|l|}
\hline
\textbf{Imagen} & \textbf{ESP} & \textbf{SEN} & \textbf{EX} & \textbf{FP\%} & \textbf{VP\%} & \textbf{FN\%} & \textbf{VN\%} \\ \hline
\endfirsthead
%
\endhead
%
1               & 99,64        & 84,23        & 99,42       & 22,92         & 77,08         & 0,23          & 99,77         \\ \hline
2               & 99,9         & 76,43        & 99,78       & 21,9          & 78,1          & 0,11          & 99,89         \\ \hline
3               & 99,26        & 24,55        & 94,86       & 32,56         & 67,44         & 4,54          & 95,46         \\ \hline
4               & 99,64        & 39,6         & 98,72       & 36,36         & 63,64         & 0,94          & 99,06         \\ \hline
5               & 99,87        & 22,16        & 98,68       & 28,18         & 71,82         & 1,19          & 98,81         \\ \hline
6               & 97,38        & 30,48        & 96,02       & 80,57         & 19,43         & 1,46          & 98,54         \\ \hline
7               & 99,61        & 12,06        & 98,13       & 65            & 35            & 1,5           & 98,5          \\ \hline
8               & 34,8         & 40,15        & 35,2        & 95,32         & 4,68          & 12,05         & 87,95         \\ \hline
9               & 5,28         & 30,14        & 5,78        & 99,35         & 0,65          & 21,34         & 78,66         \\ \hline
10              & 98,15        & 12,47        & 94,87       & 78,79         & 21,21         & 3,43          & 96,57         \\ \hline
11              & 97,46        & 49,75        & 95,3        & 51,95         & 48,05         & 2,38          & 97,62         \\ \hline
12              & 98,85        & 26,18        & 93,61       & 36,07         & 63,93         & 5,48          & 94,52         \\ \hline
13              & 98,66        & 34,15        & 95,64       & 44,35         & 55,65         & 3,18          & 96,82         \\ \hline
14              & 98,96        & 28,94        & 96,65       & 51,16         & 48,84         & 2,39          & 97,61         \\ \hline
15              & 95,03        & 36,16        & 93,19       & 81            & 19            & 2,12          & 97,88         \\ \hline
16              & 99,44        & 39,31        & 97,28       & 27,55         & 72,45         & 2,23          & 97,77         \\ \hline
17              & 97,89        & 34,08        & 96,74       & 77,17         & 22,83         & 1,22          & 98,78         \\ \hline
18              & 100          & 23,38        & 98,98       & 0             & 100           & 1,02          & 98,98         \\ \hline
19              & 98           & 23,06        & 93,69       & 58,67         & 41,33         & 4,57          & 95,43         \\ \hline
20              & 54,19        & 28,39        & 52,52       & 95,89         & 4,11          & 8,37          & 91,63         \\ \hline
21              & 99,51        & 29,57        & 98,17       & 46,01         & 53,99         & 1,36          & 98,64         \\ \hline
22              & 89,22        & 35,06        & 88,15       & 93,83         & 6,17          & 1,45          & 98,55         \\ \hline
23              & 99,91        & 20,92        & 99,26       & 32,85         & 67,15         & 0,66          & 99,34         \\ \hline
24              & 98,26        & 35,66        & 96,03       & 56,85         & 43,15         & 2,37          & 97,63         \\ \hline
25              & 99,22        & 8,77         & 95,78       & 69,2          & 30,8          & 3,51          & 96,49         \\ \hline
26              & 98,56        & 9,51         & 93,24       & 70,45         & 29,55         & 5,51          & 94,49         \\ \hline
27              & 99,71        & 8,68         & 99,03       & 81,61         & 18,39         & 0,69          & 99,31         \\ \hline
28              & 35,85        & 25,08        & 35,33       & 98,07         & 1,93          & 9,51          & 90,49         \\ \hline
29              & 99,7         & 21,63        & 96,71       & 25,59         & 74,41         & 3,04          & 96,96         \\ \hline
30              & 98,46        & 34,53        & 95,48       & 47,81         & 52,19         & 3,14          & 96,86         \\ \hline
31              & 87,26        & 25,68        & 84,44       & 91,18         & 8,82          & 3,93          & 96,07         \\ \hline
32              & 99,74        & 20,7         & 97,38       & 28,82         & 71,18         & 2,39          & 97,61         \\ \hline
33              & 98,49        & 24,9         & 95,66       & 60,31         & 39,69         & 2,95          & 97,05         \\ \hline
34              & 99,71        & 30,19        & 98,12       & 28,95         & 71,05         & 1,62          & 98,38         \\ \hline
35              & 97,16        & 24,49        & 93,84       & 70,8          & 29,2          & 3,59          & 96,41         \\ \hline
36              & 99,63        & 29,1         & 97,19       & 25,93         & 74,07         & 2,49          & 97,51         \\ \hline
37              & 99,4         & 34,48        & 97          & 31,12         & 68,88         & 2,47          & 97,53         \\ \hline
38              & 98,77        & 32,84        & 95,59       & 42,37         & 57,63         & 3,34          & 96,66         \\ \hline
39              & 99,79        & 39,69        & 98,6        & 20,28         & 79,72         & 1,21          & 98,79         \\ \hline
40              & 99,79        & 29,49        & 97,95       & 20,93         & 79,07         & 1,86          & 98,14         \\ \hline
41              & 98,98        & 24,69        & 94,7        & 40,26         & 59,74         & 4,45          & 95,55         \\ \hline
42              & 99,89        & 27,05        & 98,7        & 19,27         & 80,73         & 1,2           & 98,8          \\ \hline
43              & 98,48        & 33,45        & 97,14       & 68,46         & 31,54         & 1,4           & 98,6          \\ \hline
44              & 98,69        & 25,84        & 95,88       & 55,73         & 44,27         & 2,93          & 97,07         \\ \hline
45              & 58,75        & 35,09        & 57,95       & 97,11         & 2,89          & 3,72          & 96,28         \\ \hline
46              & 99,58        & 39,8         & 97,93       & 26,8          & 73,2          & 1,7           & 98,3          \\ \hline
47              & 81,51        & 14,39        & 78,32       & 96,26         & 3,74          & 4,98          & 95,02         \\ \hline
48              & 88,24        & 28,06        & 87,26       & 96,19         & 3,81          & 1,33          & 98,67         \\ \hline
49              & 72,73        & 11,49        & 70,86       & 98,69         & 1,31          & 3,7           & 96,3          \\ \hline
50              & 90,49        & 23,46        & 87,45       & 89,5          & 10,5          & 3,87          & 96,13         \\ \hline
51              & 99,85        & 27,64        & 98,41       & 21,4          & 78,6          & 1,45          & 98,55         \\ \hline
52              & 99,62        & 29,26        & 97,58       & 30,54         & 69,46         & 2,07          & 97,93         \\ \hline
53              & 90,78        & 18,25        & 89,48       & 96,52         & 3,48          & 1,62          & 98,38         \\ \hline
54              & 98,68        & 28,03        & 95,98       & 54,17         & 45,83         & 2,82          & 97,18         \\ \hline
55              & 91,44        & 34,06        & 90,39       & 93,12         & 6,88          & 1,32          & 98,68         \\ \hline
56              & 97,53        & 31,42        & 95          & 66,45         & 33,55         & 2,72          & 97,28         \\ \hline
57              & 1,5          & 17,56        & 2,48        & 98,85         & 1,15          & 78,21         & 21,79         \\ \hline
58              & 97,82        & 31,26        & 93,51       & 50,15         & 49,85         & 4,64          & 95,36         \\ \hline
59              & 91,79        & 27,26        & 90,63       & 94,28         & 5,72          & 1,43          & 98,57         \\ \hline
60              & 79,3         & 31,97        & 76,64       & 91,57         & 8,43          & 4,86          & 95,14         \\ \hline
61              & 96,74        & 27,53        & 95,55       & 87,13         & 12,87         & 1,3           & 98,7          \\ \hline
62              & 80,44        & 27,38        & 78,66       & 95,36         & 4,64          & 3,04          & 96,96         \\ \hline
63              & 99,7         & 33,27        & 98,82       & 40,07         & 59,93         & 0,89          & 99,11         \\ \hline
64              & 99,74        & 26,8         & 98,67       & 39,24         & 60,76         & 1,08          & 98,92         \\ \hline
65              & 1,82         & 23,35        & 2,87        & 98,79         & 1,21          & 68,55         & 31,45         \\ \hline
66              & 91,53        & 7,94         & 89,69       & 97,94         & 2,06          & 2,21          & 97,79         \\ \hline
67              & 99,59        & 24,98        & 98,03       & 43,69         & 56,31         & 1,58          & 98,42         \\ \hline
68              & 96,19        & 19,87        & 93,29       & 82,94         & 17,06         & 3,18          & 96,82         \\ \hline
69              & 99,77        & 26,11        & 99,02       & 45,55         & 54,45         & 0,76          & 99,24         \\ \hline
70              & 1,49         & 25,3         & 2,23        & 99,19         & 0,81          & 61,45         & 38,55         \\ \hline
71              & 99,39        & 30,76        & 97,94       & 48,18         & 51,82         & 1,47          & 98,53         \\ \hline
72              & 99,95        & 22,34        & 99,28       & 18,86         & 81,14         & 0,67          & 99,33         \\ \hline
73              & 73,67        & 24,86        & 70,85       & 94,51         & 5,49          & 5,9           & 94,1          \\ \hline
74              & 99,3         & 22,08        & 97,28       & 54,03         & 45,97         & 2,07          & 97,93         \\ \hline
75              & 98,64        & 21,92        & 97,06       & 74,66         & 25,34         & 1,64          & 98,36         \\ \hline
76              & 98           & 23,76        & 96,76       & 83,29         & 16,71         & 1,3           & 98,7          \\ \hline
77              & 93,53        & 26,92        & 91,52       & 88,56         & 11,44         & 2,37          & 97,63         \\ \hline
78              & 98,85        & 24,58        & 97,6        & 73,23         & 26,77         & 1,29          & 98,71         \\ \hline
79              & 72,29        & 31,99        & 71,43       & 97,56         & 2,44          & 1,99          & 98,01         \\ \hline
80              & 91,14        & 26,79        & 88,48       & 88,47         & 11,53         & 3,35          & 96,65         \\ \hline
81              & 22,88        & 17,47        & 22,73       & 99,36         & 0,64          & 9,33          & 90,67         \\ \hline
82              & 0,26         & 34,04        & 1,02        & 99,22         & 0,78          & 85,58         & 14,42         \\ \hline
83              & 90,54        & 7,79         & 88,6        & 98,06         & 1,94          & 2,39          & 97,61         \\ \hline
84              & 97,09        & 25,22        & 95,85       & 86,74         & 13,26         & 1,34          & 98,66         \\ \hline
85              & 89,08        & 36,96        & 86,92       & 87,25         & 12,75         & 2,97          & 97,03         \\ \hline
86              & 82,28        & 26,35        & 80,78       & 96,07         & 3,93          & 2,4           & 97,6          \\ \hline
87              & 96,67        & 26,49        & 94,02       & 76,16         & 23,84         & 2,9           & 97,1          \\ \hline
88              & 99,34        & 21,06        & 96,91       & 49,62         & 50,38         & 2,48          & 97,52         \\ \hline
89              & 99,3         & 32,89        & 98,4        & 60,68         & 39,32         & 0,92          & 99,08         \\ \hline
90              & 78,35        & 17,69        & 76,83       & 97,94         & 2,06          & 2,63          & 97,37         \\ \hline
91              & 90,07        & 13,76        & 87,02       & 94,55         & 5,45          & 3,83          & 96,17         \\ \hline
92              & 97,29        & 22,77        & 95,05       & 79,3          & 20,7          & 2,4           & 97,6          \\ \hline
93              & 98,6         & 22,06        & 96,65       & 70,9          & 29,1          & 2,02          & 97,98         \\ \hline
94              & 98,92        & 27,47        & 98,14       & 78,14         & 21,86         & 0,8           & 99,2          \\ \hline
95              & 93,84        & 28,97        & 92,43       & 90,55         & 9,45          & 1,65          & 98,35         \\ \hline
96              & 99,2         & 23,96        & 97,48       & 58,65         & 41,35         & 1,77          & 98,23         \\ \hline
97              & 99,26        & 34,79        & 96,96       & 36,47         & 63,53         & 2,37          & 97,63         \\ \hline
98              & 82,07        & 26,94        & 80,36       & 95,39         & 4,61          & 2,78          & 97,22         \\ \hline
99              & 84,66        & 28,04        & 84,05       & 98,05         & 1,95          & 0,92          & 99,08         \\ \hline
100             & 98,27        & 23,75        & 95,53       & 65,58         & 34,42         & 2,88          & 97,12         \\ \hline
\label{anx:min}
\end{longtable}
\section{MODA-RGB}
\begin{longtable}[c]{|l|l|l|l|l|l|l|l|}
\hline
\textbf{Imagen} & \textbf{ESP} & \textbf{SEN} & \textbf{EX} & \textbf{FP\%} & \textbf{VP\%} & \textbf{FN\%} & \textbf{VN\%} \\ \hline
\endfirsthead
%
\endhead
%
1               & 99,64        & 84,23        & 99,42       & 22,92         & 77,08         & 0,23          & 99,77         \\ \hline
2               & 99,89        & 74,29        & 99,77       & 22,96         & 77,04         & 0,13          & 99,87         \\ \hline
3               & 99,43        & 24,47        & 95,02       & 27,02         & 72,98         & 4,54          & 95,46         \\ \hline
4               & 99,55        & 39,6         & 98,62       & 42,16         & 57,84         & 0,94          & 99,06         \\ \hline
5               & 99,85        & 22,38        & 98,67       & 30,59         & 69,41         & 1,19          & 98,81         \\ \hline
6               & 97,38        & 30,48        & 96,02       & 80,57         & 19,43         & 1,46          & 98,54         \\ \hline
7               & 99,57        & 12,51        & 98,1        & 66,46         & 33,54         & 1,49          & 98,51         \\ \hline
8               & 92,1         & 38,43        & 88,14       & 72,08         & 27,92         & 5,06          & 94,94         \\ \hline
9               & 3,66         & 30,14        & 4,19        & 99,36         & 0,64          & 28,15         & 71,85         \\ \hline
10              & 99,91        & 9,13         & 96,43       & 20,39         & 79,61         & 3,5           & 96,5          \\ \hline
11              & 98,54        & 49,75        & 96,34       & 38,34         & 61,66         & 2,35          & 97,65         \\ \hline
12              & 98,85        & 26,21        & 93,62       & 36,02         & 63,98         & 5,48          & 94,52         \\ \hline
13              & 98,66        & 34,15        & 95,64       & 44,35         & 55,65         & 3,18          & 96,82         \\ \hline
14              & 98,81        & 28,85        & 96,5        & 54,67         & 45,33         & 2,4           & 97,6          \\ \hline
15              & 95,03        & 36,16        & 93,19       & 81            & 19            & 2,12          & 97,88         \\ \hline
16              & 99,21        & 40,23        & 97,09       & 34,44         & 65,56         & 2,2           & 97,8          \\ \hline
17              & 95,94        & 33,3         & 94,81       & 86,92         & 13,08         & 1,26          & 98,74         \\ \hline
18              & 99,99        & 23,38        & 98,96       & 4,41          & 95,59         & 1,02          & 98,98         \\ \hline
19              & 98,15        & 23,06        & 93,83       & 56,78         & 43,22         & 4,57          & 95,43         \\ \hline
20              & 54,19        & 28,39        & 52,52       & 95,89         & 4,11          & 8,37          & 91,63         \\ \hline
21              & 99,53        & 30,54        & 98,21       & 44,1          & 55,9          & 1,34          & 98,66         \\ \hline
22              & 89,36        & 34,74        & 88,27       & 93,82         & 6,18          & 1,45          & 98,55         \\ \hline
23              & 99,91        & 22,16        & 99,27       & 31,83         & 68,17         & 0,65          & 99,35         \\ \hline
24              & 98,26        & 35,66        & 96,03       & 56,85         & 43,15         & 2,37          & 97,63         \\ \hline
25              & 99,21        & 8,79         & 95,77       & 69,39         & 30,61         & 3,51          & 96,49         \\ \hline
26              & 98,7         & 9,64         & 93,38       & 67,93         & 32,07         & 5,5           & 94,5          \\ \hline
27              & 99,71        & 8,68         & 99,03       & 81,61         & 18,39         & 0,69          & 99,31         \\ \hline
28              & 69,46        & 24,3         & 67,3        & 96,15         & 3,85          & 5,2           & 94,8          \\ \hline
29              & 99,76        & 21,57        & 96,75       & 22,14         & 77,86         & 3,04          & 96,96         \\ \hline
30              & 98,46        & 34,53        & 95,48       & 47,81         & 52,19         & 3,14          & 96,86         \\ \hline
31              & 91,91        & 25,61        & 88,87       & 86,82         & 13,18         & 3,74          & 96,26         \\ \hline
32              & 99,74        & 20,7         & 97,38       & 28,82         & 71,18         & 2,39          & 97,61         \\ \hline
33              & 98,49        & 24,97        & 95,67       & 60,18         & 39,82         & 2,95          & 97,05         \\ \hline
34              & 99,71        & 30,68        & 98,12       & 28,94         & 71,06         & 1,61          & 98,39         \\ \hline
35              & 96,77        & 25,84        & 93,53       & 72,3          & 27,7          & 3,54          & 96,46         \\ \hline
36              & 99,62        & 29,47        & 97,19       & 26,62         & 73,38         & 2,48          & 97,52         \\ \hline
37              & 99,42        & 34,04        & 97          & 30,74         & 69,26         & 2,49          & 97,51         \\ \hline
38              & 97,91        & 34,28        & 94,84       & 54,52         & 45,48         & 3,3           & 96,7          \\ \hline
39              & 99,82        & 38,44        & 98,59       & 19,1          & 80,9          & 1,24          & 98,76         \\ \hline
40              & 99,79        & 29,97        & 97,96       & 20,84         & 79,16         & 1,85          & 98,15         \\ \hline
41              & 98,52        & 25,31        & 94,3        & 48,84         & 51,16         & 4,43          & 95,57         \\ \hline
42              & 99,88        & 26,9         & 98,68       & 21,8          & 78,2          & 1,2           & 98,8          \\ \hline
43              & 98,41        & 33,57        & 97,08       & 69,38         & 30,62         & 1,4           & 98,6          \\ \hline
44              & 98,06        & 29,51        & 95,41       & 62,05         & 37,95         & 2,81          & 97,19         \\ \hline
45              & 57,93        & 35,03        & 57,15       & 97,17         & 2,83          & 3,78          & 96,22         \\ \hline
46              & 99,64        & 39,72        & 97,98       & 24,04         & 75,96         & 1,7           & 98,3          \\ \hline
47              & 75,39        & 14,42        & 72,49       & 97,16         & 2,84          & 5,36          & 94,64         \\ \hline
48              & 95,15        & 28,13        & 94,06       & 91,22         & 8,78          & 1,24          & 98,76         \\ \hline
49              & 74,04        & 11,41        & 72,12       & 98,63         & 1,37          & 3,64          & 96,36         \\ \hline
50              & 95,19        & 23,21        & 91,92       & 81,33         & 18,67         & 3,7           & 96,3          \\ \hline
51              & 99,87        & 27,47        & 98,43       & 18,64         & 81,36         & 1,46          & 98,54         \\ \hline
52              & 99,61        & 29,05        & 97,57       & 31,23         & 68,77         & 2,08          & 97,92         \\ \hline
53              & 90,62        & 18,58        & 89,33       & 96,51         & 3,49          & 1,61          & 98,39         \\ \hline
54              & 98,74        & 28,63        & 96,06       & 52,58         & 47,42         & 2,79          & 97,21         \\ \hline
55              & 92,9         & 32,14        & 91,8        & 92,24         & 7,76          & 1,34          & 98,66         \\ \hline
56              & 97,65        & 31,77        & 95,14       & 65,04         & 34,96         & 2,7           & 97,3          \\ \hline
57              & 1,5          & 17,56        & 2,48        & 98,85         & 1,15          & 78,21         & 21,79         \\ \hline
58              & 97,82        & 31,26        & 93,51       & 50,15         & 49,85         & 4,64          & 95,36         \\ \hline
59              & 91,56        & 27,78        & 90,41       & 94,33         & 5,67          & 1,42          & 98,58         \\ \hline
60              & 79,18        & 30,49        & 76,45       & 91,97         & 8,03          & 4,97          & 95,03         \\ \hline
61              & 96,74        & 27,53        & 95,55       & 87,13         & 12,87         & 1,3           & 98,7          \\ \hline
62              & 80,44        & 27,38        & 78,66       & 95,36         & 4,64          & 3,04          & 96,96         \\ \hline
63              & 99,7         & 33,27        & 98,82       & 40,07         & 59,93         & 0,89          & 99,11         \\ \hline
64              & 99,74        & 26,8         & 98,67       & 39,24         & 60,76         & 1,08          & 98,92         \\ \hline
65              & 1,87         & 23,35        & 2,92        & 98,79         & 1,21          & 67,95         & 32,05         \\ \hline
66              & 94,49        & 2,62         & 92,47       & 98,94         & 1,06          & 2,26          & 97,74         \\ \hline
67              & 99,61        & 24,79        & 98,05       & 42,52         & 57,48         & 1,58          & 98,42         \\ \hline
68              & 96,55        & 19,86        & 93,64       & 81,47         & 18,53         & 3,17          & 96,83         \\ \hline
69              & 99,71        & 29,77        & 98,99       & 48,73         & 51,27         & 0,72          & 99,28         \\ \hline
70              & 3,57         & 25,3         & 4,24        & 99,17         & 0,83          & 40,04         & 59,96         \\ \hline
71              & 99,45        & 29,77        & 97,99       & 46,12         & 53,88         & 1,49          & 98,51         \\ \hline
72              & 99,96        & 23,43        & 99,3        & 14,91         & 85,09         & 0,66          & 99,34         \\ \hline
73              & 72,74        & 25,58        & 70,01       & 94,55         & 5,45          & 5,92          & 94,08         \\ \hline
74              & 99,53        & 21,65        & 97,49       & 44,58         & 55,42         & 2,07          & 97,93         \\ \hline
75              & 98,64        & 21,92        & 97,06       & 74,66         & 25,34         & 1,64          & 98,36         \\ \hline
76              & 80,05        & 23,76        & 79,11       & 98,03         & 1,97          & 1,58          & 98,42         \\ \hline
77              & 98,05        & 27,03        & 95,91       & 69,89         & 30,11         & 2,26          & 97,74         \\ \hline
78              & 98,85        & 24,58        & 97,6        & 73,23         & 26,77         & 1,29          & 98,71         \\ \hline
79              & 72,42        & 31,93        & 71,56       & 97,56         & 2,44          & 1,99          & 98,01         \\ \hline
80              & 91,14        & 26,79        & 88,48       & 88,47         & 11,53         & 3,35          & 96,65         \\ \hline
81              & 22,88        & 17,47        & 22,73       & 99,36         & 0,64          & 9,34          & 90,66         \\ \hline
82              & 9,57         & 32,59        & 10,09       & 99,17         & 0,83          & 14,01         & 85,99         \\ \hline
83              & 90,55        & 7,82         & 88,61       & 98,05         & 1,95          & 2,39          & 97,61         \\ \hline
84              & 97,75        & 24,93        & 96,49       & 83,65         & 16,35         & 1,34          & 98,66         \\ \hline
85              & 89,18        & 35,72        & 86,97       & 87,52         & 12,48         & 3,02          & 96,98         \\ \hline
86              & 81,63        & 26,93        & 80,16       & 96,12         & 3,88          & 2,4           & 97,6          \\ \hline
87              & 97           & 26,1         & 94,32       & 74,51         & 25,49         & 2,91          & 97,09         \\ \hline
88              & 99,14        & 21,36        & 96,73       & 55,83         & 44,17         & 2,47          & 97,53         \\ \hline
89              & 99,3         & 33,29        & 98,41       & 60,35         & 39,65         & 0,91          & 99,09         \\ \hline
90              & 77,29        & 17,54        & 75,8        & 98,05         & 1,95          & 2,67          & 97,33         \\ \hline
91              & 92,11        & 13,85        & 88,99       & 93,2          & 6,8           & 3,74          & 96,26         \\ \hline
92              & 98,02        & 22,89        & 95,76       & 73,55         & 26,45         & 2,38          & 97,62         \\ \hline
93              & 98,6         & 22,06        & 96,65       & 70,9          & 29,1          & 2,02          & 97,98         \\ \hline
94              & 98,91        & 27,83        & 98,14       & 77,97         & 22,03         & 0,8           & 99,2          \\ \hline
95              & 96,71        & 29,22        & 95,25       & 83,51         & 16,49         & 1,6           & 98,4          \\ \hline
96              & 99,33        & 23,86        & 97,6        & 54,37         & 45,63         & 1,77          & 98,23         \\ \hline
97              & 99,18        & 35,16        & 96,9        & 38,54         & 61,46         & 2,36          & 97,64         \\ \hline
98              & 84,82        & 24,49        & 82,94       & 95,07         & 4,93          & 2,78          & 97,22         \\ \hline
99              & 84,79        & 28,84        & 84,19       & 97,97         & 2,03          & 0,91          & 99,09         \\ \hline
100             & 97,52        & 23,88        & 94,81       & 73,16         & 26,84         & 2,89          & 97,11         \\ \hline
\label{anx:moda}
\end{longtable}
\section{MODA-MAXIMO-RGB}
\begin{longtable}[c]{|l|l|l|l|l|l|l|l|}
\hline
\textbf{Imagen} & \textbf{ESP} & \textbf{SEN} & \textbf{EX} & \textbf{FP\%} & \textbf{VP\%} & \textbf{FN\%} & \textbf{VN\%} \\ \hline
\endfirsthead
%
\endhead
%
1               & 99,64        & 84,23        & 99,42       & 22,92         & 77,08         & 0,23          & 99,77         \\ \hline
2               & 99,89        & 74,29        & 99,77       & 22,96         & 77,04         & 0,13          & 99,87         \\ \hline
3               & 99,43        & 24,47        & 95,02       & 27,02         & 72,98         & 4,54          & 95,46         \\ \hline
4               & 99,55        & 39,6         & 98,62       & 42,16         & 57,84         & 0,94          & 99,06         \\ \hline
5               & 99,85        & 22,38        & 98,67       & 30,59         & 69,41         & 1,19          & 98,81         \\ \hline
6               & 97,38        & 30,48        & 96,02       & 80,57         & 19,43         & 1,46          & 98,54         \\ \hline
7               & 99,57        & 12,51        & 98,1        & 66,46         & 33,54         & 1,49          & 98,51         \\ \hline
8               & 92,1         & 38,43        & 88,14       & 72,08         & 27,92         & 5,06          & 94,94         \\ \hline
9               & 3,66         & 30,14        & 4,19        & 99,36         & 0,64          & 28,15         & 71,85         \\ \hline
10              & 99,91        & 9,13         & 96,43       & 20,39         & 79,61         & 3,5           & 96,5          \\ \hline
11              & 98,54        & 49,75        & 96,34       & 38,34         & 61,66         & 2,35          & 97,65         \\ \hline
12              & 98,85        & 26,21        & 93,62       & 36,02         & 63,98         & 5,48          & 94,52         \\ \hline
13              & 98,66        & 34,15        & 95,64       & 44,35         & 55,65         & 3,18          & 96,82         \\ \hline
14              & 98,81        & 28,85        & 96,5        & 54,67         & 45,33         & 2,4           & 97,6          \\ \hline
15              & 95,03        & 36,16        & 93,19       & 81            & 19            & 2,12          & 97,88         \\ \hline
16              & 99,21        & 40,23        & 97,09       & 34,44         & 65,56         & 2,2           & 97,8          \\ \hline
17              & 95,94        & 33,3         & 94,81       & 86,92         & 13,08         & 1,26          & 98,74         \\ \hline
18              & 99,99        & 23,38        & 98,96       & 4,41          & 95,59         & 1,02          & 98,98         \\ \hline
19              & 98,15        & 23,06        & 93,83       & 56,78         & 43,22         & 4,57          & 95,43         \\ \hline
20              & 54,19        & 28,39        & 52,52       & 95,89         & 4,11          & 8,37          & 91,63         \\ \hline
21              & 99,53        & 30,54        & 98,21       & 44,1          & 55,9          & 1,34          & 98,66         \\ \hline
22              & 89,36        & 34,74        & 88,27       & 93,82         & 6,18          & 1,45          & 98,55         \\ \hline
23              & 99,91        & 22,16        & 99,27       & 31,83         & 68,17         & 0,65          & 99,35         \\ \hline
24              & 98,26        & 35,66        & 96,03       & 56,85         & 43,15         & 2,37          & 97,63         \\ \hline
25              & 99,21        & 8,79         & 95,77       & 69,39         & 30,61         & 3,51          & 96,49         \\ \hline
26              & 98,7         & 9,64         & 93,38       & 67,93         & 32,07         & 5,5           & 94,5          \\ \hline
27              & 99,71        & 8,68         & 99,03       & 81,61         & 18,39         & 0,69          & 99,31         \\ \hline
28              & 69,46        & 24,3         & 67,3        & 96,15         & 3,85          & 5,2           & 94,8          \\ \hline
29              & 99,76        & 21,57        & 96,75       & 22,14         & 77,86         & 3,04          & 96,96         \\ \hline
30              & 98,46        & 34,53        & 95,48       & 47,81         & 52,19         & 3,14          & 96,86         \\ \hline
31              & 91,91        & 25,61        & 88,87       & 86,82         & 13,18         & 3,74          & 96,26         \\ \hline
32              & 99,74        & 20,7         & 97,38       & 28,82         & 71,18         & 2,39          & 97,61         \\ \hline
33              & 98,49        & 24,97        & 95,67       & 60,18         & 39,82         & 2,95          & 97,05         \\ \hline
34              & 99,71        & 30,68        & 98,12       & 28,94         & 71,06         & 1,61          & 98,39         \\ \hline
35              & 96,77        & 25,84        & 93,53       & 72,3          & 27,7          & 3,54          & 96,46         \\ \hline
36              & 99,62        & 29,47        & 97,19       & 26,62         & 73,38         & 2,48          & 97,52         \\ \hline
37              & 99,42        & 34,04        & 97          & 30,74         & 69,26         & 2,49          & 97,51         \\ \hline
38              & 97,91        & 34,28        & 94,84       & 54,52         & 45,48         & 3,3           & 96,7          \\ \hline
39              & 99,82        & 38,44        & 98,59       & 19,1          & 80,9          & 1,24          & 98,76         \\ \hline
40              & 99,79        & 29,97        & 97,96       & 20,84         & 79,16         & 1,85          & 98,15         \\ \hline
41              & 98,52        & 25,31        & 94,3        & 48,84         & 51,16         & 4,43          & 95,57         \\ \hline
42              & 99,88        & 26,9         & 98,68       & 21,8          & 78,2          & 1,2           & 98,8          \\ \hline
43              & 98,41        & 33,57        & 97,08       & 69,38         & 30,62         & 1,4           & 98,6          \\ \hline
44              & 98,06        & 29,51        & 95,41       & 62,05         & 37,95         & 2,81          & 97,19         \\ \hline
45              & 57,93        & 35,03        & 57,15       & 97,17         & 2,83          & 3,78          & 96,22         \\ \hline
46              & 99,64        & 39,72        & 97,98       & 24,04         & 75,96         & 1,7           & 98,3          \\ \hline
47              & 75,39        & 14,42        & 72,49       & 97,16         & 2,84          & 5,36          & 94,64         \\ \hline
48              & 95,15        & 28,13        & 94,06       & 91,22         & 8,78          & 1,24          & 98,76         \\ \hline
49              & 74,04        & 11,41        & 72,12       & 98,63         & 1,37          & 3,64          & 96,36         \\ \hline
50              & 95,19        & 23,21        & 91,92       & 81,33         & 18,67         & 3,7           & 96,3          \\ \hline
51              & 99,87        & 27,47        & 98,43       & 18,64         & 81,36         & 1,46          & 98,54         \\ \hline
52              & 99,61        & 29,05        & 97,57       & 31,23         & 68,77         & 2,08          & 97,92         \\ \hline
53              & 90,62        & 18,58        & 89,33       & 96,51         & 3,49          & 1,61          & 98,39         \\ \hline
54              & 98,74        & 28,63        & 96,06       & 52,58         & 47,42         & 2,79          & 97,21         \\ \hline
55              & 92,9         & 32,14        & 91,8        & 92,24         & 7,76          & 1,34          & 98,66         \\ \hline
56              & 97,65        & 31,77        & 95,14       & 65,04         & 34,96         & 2,7           & 97,3          \\ \hline
57              & 1,5          & 17,56        & 2,48        & 98,85         & 1,15          & 78,21         & 21,79         \\ \hline
58              & 97,82        & 31,26        & 93,51       & 50,15         & 49,85         & 4,64          & 95,36         \\ \hline
59              & 91,56        & 27,78        & 90,41       & 94,33         & 5,67          & 1,42          & 98,58         \\ \hline
60              & 79,18        & 30,49        & 76,45       & 91,97         & 8,03          & 4,97          & 95,03         \\ \hline
61              & 96,74        & 27,53        & 95,55       & 87,13         & 12,87         & 1,3           & 98,7          \\ \hline
62              & 80,44        & 27,38        & 78,66       & 95,36         & 4,64          & 3,04          & 96,96         \\ \hline
63              & 99,7         & 33,27        & 98,82       & 40,07         & 59,93         & 0,89          & 99,11         \\ \hline
64              & 99,74        & 26,8         & 98,67       & 39,24         & 60,76         & 1,08          & 98,92         \\ \hline
65              & 1,87         & 23,35        & 2,92        & 98,79         & 1,21          & 67,95         & 32,05         \\ \hline
66              & 94,49        & 2,62         & 92,47       & 98,94         & 1,06          & 2,26          & 97,74         \\ \hline
67              & 99,61        & 24,79        & 98,05       & 42,52         & 57,48         & 1,58          & 98,42         \\ \hline
68              & 96,55        & 19,86        & 93,64       & 81,47         & 18,53         & 3,17          & 96,83         \\ \hline
69              & 99,71        & 29,77        & 98,99       & 48,73         & 51,27         & 0,72          & 99,28         \\ \hline
70              & 3,57         & 25,3         & 4,24        & 99,17         & 0,83          & 40,04         & 59,96         \\ \hline
71              & 99,45        & 29,77        & 97,99       & 46,12         & 53,88         & 1,49          & 98,51         \\ \hline
72              & 99,96        & 23,43        & 99,3        & 14,91         & 85,09         & 0,66          & 99,34         \\ \hline
73              & 72,74        & 25,58        & 70,01       & 94,55         & 5,45          & 5,92          & 94,08         \\ \hline
74              & 99,53        & 21,65        & 97,49       & 44,58         & 55,42         & 2,07          & 97,93         \\ \hline
75              & 98,64        & 21,92        & 97,06       & 74,66         & 25,34         & 1,64          & 98,36         \\ \hline
76              & 80,05        & 23,76        & 79,11       & 98,03         & 1,97          & 1,58          & 98,42         \\ \hline
77              & 98,05        & 27,03        & 95,91       & 69,89         & 30,11         & 2,26          & 97,74         \\ \hline
78              & 98,85        & 24,58        & 97,6        & 73,23         & 26,77         & 1,29          & 98,71         \\ \hline
79              & 72,42        & 31,93        & 71,56       & 97,56         & 2,44          & 1,99          & 98,01         \\ \hline
80              & 91,14        & 26,79        & 88,48       & 88,47         & 11,53         & 3,35          & 96,65         \\ \hline
81              & 22,88        & 17,47        & 22,73       & 99,36         & 0,64          & 9,34          & 90,66         \\ \hline
82              & 9,57         & 32,59        & 10,09       & 99,17         & 0,83          & 14,01         & 85,99         \\ \hline
83              & 90,55        & 7,82         & 88,61       & 98,05         & 1,95          & 2,39          & 97,61         \\ \hline
84              & 97,75        & 24,93        & 96,49       & 83,65         & 16,35         & 1,34          & 98,66         \\ \hline
85              & 89,18        & 35,72        & 86,97       & 87,52         & 12,48         & 3,02          & 96,98         \\ \hline
86              & 81,63        & 26,93        & 80,16       & 96,12         & 3,88          & 2,4           & 97,6          \\ \hline
87              & 97           & 26,1         & 94,32       & 74,51         & 25,49         & 2,91          & 97,09         \\ \hline
88              & 99,14        & 21,36        & 96,73       & 55,83         & 44,17         & 2,47          & 97,53         \\ \hline
89              & 99,3         & 33,29        & 98,41       & 60,35         & 39,65         & 0,91          & 99,09         \\ \hline
90              & 77,29        & 17,54        & 75,8        & 98,05         & 1,95          & 2,67          & 97,33         \\ \hline
91              & 92,11        & 13,85        & 88,99       & 93,2          & 6,8           & 3,74          & 96,26         \\ \hline
92              & 98,02        & 22,89        & 95,76       & 73,55         & 26,45         & 2,38          & 97,62         \\ \hline
93              & 98,6         & 22,06        & 96,65       & 70,9          & 29,1          & 2,02          & 97,98         \\ \hline
94              & 98,91        & 27,83        & 98,14       & 77,97         & 22,03         & 0,8           & 99,2          \\ \hline
95              & 96,71        & 29,22        & 95,25       & 83,51         & 16,49         & 1,6           & 98,4          \\ \hline
96              & 99,33        & 23,86        & 97,6        & 54,37         & 45,63         & 1,77          & 98,23         \\ \hline
97              & 99,18        & 35,16        & 96,9        & 38,54         & 61,46         & 2,36          & 97,64         \\ \hline
98              & 84,82        & 24,49        & 82,94       & 95,07         & 4,93          & 2,78          & 97,22         \\ \hline
99              & 84,79        & 28,84        & 84,19       & 97,97         & 2,03          & 0,91          & 99,09         \\ \hline
100             & 97,52        & 23,88        & 94,81       & 73,16         & 26,84         & 2,89          & 97,11         \\ \hline
\label{anx:modmax}
\end{longtable}
\section{MODA-MINIMO-RGB}
\begin{longtable}[c]{|l|l|l|l|l|l|l|l|}
\hline
\textbf{Imagen} & \textbf{ESP} & \textbf{SEN} & \textbf{EX} & \textbf{FP\%} & \textbf{VP\%} & \textbf{FN\%} & \textbf{VN\%} \\ \hline
\endfirsthead
%
\endhead
%
1               & 99,64        & 84,23        & 99,42       & 22,92         & 77,08         & 0,23          & 99,77         \\ \hline
2               & 99,89        & 74,29        & 99,77       & 22,96         & 77,04         & 0,13          & 99,87         \\ \hline
3               & 99,43        & 24,47        & 95,02       & 27,02         & 72,98         & 4,54          & 95,46         \\ \hline
4               & 99,55        & 39,6         & 98,62       & 42,16         & 57,84         & 0,94          & 99,06         \\ \hline
5               & 99,85        & 22,38        & 98,67       & 30,59         & 69,41         & 1,19          & 98,81         \\ \hline
6               & 97,38        & 30,48        & 96,02       & 80,57         & 19,43         & 1,46          & 98,54         \\ \hline
7               & 99,57        & 12,51        & 98,1        & 66,46         & 33,54         & 1,49          & 98,51         \\ \hline
8               & 92,1         & 38,43        & 88,14       & 72,08         & 27,92         & 5,06          & 94,94         \\ \hline
9               & 3,66         & 30,14        & 4,19        & 99,36         & 0,64          & 28,15         & 71,85         \\ \hline
10              & 99,91        & 9,13         & 96,43       & 20,39         & 79,61         & 3,5           & 96,5          \\ \hline
11              & 98,54        & 49,75        & 96,34       & 38,34         & 61,66         & 2,35          & 97,65         \\ \hline
12              & 98,85        & 26,21        & 93,62       & 36,02         & 63,98         & 5,48          & 94,52         \\ \hline
13              & 98,66        & 34,15        & 95,64       & 44,35         & 55,65         & 3,18          & 96,82         \\ \hline
14              & 98,81        & 28,85        & 96,5        & 54,67         & 45,33         & 2,4           & 97,6          \\ \hline
15              & 95,03        & 36,16        & 93,19       & 81            & 19            & 2,12          & 97,88         \\ \hline
16              & 99,21        & 40,23        & 97,09       & 34,44         & 65,56         & 2,2           & 97,8          \\ \hline
17              & 95,94        & 33,3         & 94,81       & 86,92         & 13,08         & 1,26          & 98,74         \\ \hline
18              & 99,99        & 23,38        & 98,96       & 4,41          & 95,59         & 1,02          & 98,98         \\ \hline
19              & 98,15        & 23,06        & 93,83       & 56,78         & 43,22         & 4,57          & 95,43         \\ \hline
20              & 54,19        & 28,39        & 52,52       & 95,89         & 4,11          & 8,37          & 91,63         \\ \hline
21              & 99,53        & 30,54        & 98,21       & 44,1          & 55,9          & 1,34          & 98,66         \\ \hline
22              & 89,36        & 34,74        & 88,27       & 93,82         & 6,18          & 1,45          & 98,55         \\ \hline
23              & 99,91        & 22,16        & 99,27       & 31,83         & 68,17         & 0,65          & 99,35         \\ \hline
24              & 98,26        & 35,66        & 96,03       & 56,85         & 43,15         & 2,37          & 97,63         \\ \hline
25              & 99,21        & 8,79         & 95,77       & 69,39         & 30,61         & 3,51          & 96,49         \\ \hline
26              & 98,7         & 9,64         & 93,38       & 67,93         & 32,07         & 5,5           & 94,5          \\ \hline
27              & 99,71        & 8,68         & 99,03       & 81,61         & 18,39         & 0,69          & 99,31         \\ \hline
28              & 69,46        & 24,3         & 67,3        & 96,15         & 3,85          & 5,2           & 94,8          \\ \hline
29              & 99,76        & 21,57        & 96,75       & 22,14         & 77,86         & 3,04          & 96,96         \\ \hline
30              & 98,46        & 34,53        & 95,48       & 47,81         & 52,19         & 3,14          & 96,86         \\ \hline
31              & 91,91        & 25,61        & 88,87       & 86,82         & 13,18         & 3,74          & 96,26         \\ \hline
32              & 99,74        & 20,7         & 97,38       & 28,82         & 71,18         & 2,39          & 97,61         \\ \hline
33              & 98,49        & 24,97        & 95,67       & 60,18         & 39,82         & 2,95          & 97,05         \\ \hline
34              & 99,71        & 30,68        & 98,12       & 28,94         & 71,06         & 1,61          & 98,39         \\ \hline
35              & 96,77        & 25,84        & 93,53       & 72,3          & 27,7          & 3,54          & 96,46         \\ \hline
36              & 99,62        & 29,47        & 97,19       & 26,62         & 73,38         & 2,48          & 97,52         \\ \hline
37              & 99,42        & 34,04        & 97          & 30,74         & 69,26         & 2,49          & 97,51         \\ \hline
38              & 97,91        & 34,28        & 94,84       & 54,52         & 45,48         & 3,3           & 96,7          \\ \hline
39              & 99,82        & 38,44        & 98,59       & 19,1          & 80,9          & 1,24          & 98,76         \\ \hline
40              & 99,79        & 29,97        & 97,96       & 20,84         & 79,16         & 1,85          & 98,15         \\ \hline
41              & 98,52        & 25,31        & 94,3        & 48,84         & 51,16         & 4,43          & 95,57         \\ \hline
42              & 99,88        & 26,9         & 98,68       & 21,8          & 78,2          & 1,2           & 98,8          \\ \hline
43              & 98,41        & 33,57        & 97,08       & 69,38         & 30,62         & 1,4           & 98,6          \\ \hline
44              & 98,06        & 29,51        & 95,41       & 62,05         & 37,95         & 2,81          & 97,19         \\ \hline
45              & 57,93        & 35,03        & 57,15       & 97,17         & 2,83          & 3,78          & 96,22         \\ \hline
46              & 99,64        & 39,72        & 97,98       & 24,04         & 75,96         & 1,7           & 98,3          \\ \hline
47              & 75,39        & 14,42        & 72,49       & 97,16         & 2,84          & 5,36          & 94,64         \\ \hline
48              & 95,15        & 28,13        & 94,06       & 91,22         & 8,78          & 1,24          & 98,76         \\ \hline
49              & 74,04        & 11,41        & 72,12       & 98,63         & 1,37          & 3,64          & 96,36         \\ \hline
50              & 95,19        & 23,21        & 91,92       & 81,33         & 18,67         & 3,7           & 96,3          \\ \hline
51              & 99,87        & 27,47        & 98,43       & 18,64         & 81,36         & 1,46          & 98,54         \\ \hline
52              & 99,61        & 29,05        & 97,57       & 31,23         & 68,77         & 2,08          & 97,92         \\ \hline
53              & 90,62        & 18,58        & 89,33       & 96,51         & 3,49          & 1,61          & 98,39         \\ \hline
54              & 98,74        & 28,63        & 96,06       & 52,58         & 47,42         & 2,79          & 97,21         \\ \hline
55              & 92,9         & 32,14        & 91,8        & 92,24         & 7,76          & 1,34          & 98,66         \\ \hline
56              & 97,65        & 31,77        & 95,14       & 65,04         & 34,96         & 2,7           & 97,3          \\ \hline
57              & 1,5          & 17,56        & 2,48        & 98,85         & 1,15          & 78,21         & 21,79         \\ \hline
58              & 97,82        & 31,26        & 93,51       & 50,15         & 49,85         & 4,64          & 95,36         \\ \hline
59              & 91,56        & 27,78        & 90,41       & 94,33         & 5,67          & 1,42          & 98,58         \\ \hline
60              & 79,18        & 30,49        & 76,45       & 91,97         & 8,03          & 4,97          & 95,03         \\ \hline
61              & 96,74        & 27,53        & 95,55       & 87,13         & 12,87         & 1,3           & 98,7          \\ \hline
62              & 80,44        & 27,38        & 78,66       & 95,36         & 4,64          & 3,04          & 96,96         \\ \hline
63              & 99,7         & 33,27        & 98,82       & 40,07         & 59,93         & 0,89          & 99,11         \\ \hline
64              & 99,74        & 26,8         & 98,67       & 39,24         & 60,76         & 1,08          & 98,92         \\ \hline
65              & 1,87         & 23,35        & 2,92        & 98,79         & 1,21          & 67,95         & 32,05         \\ \hline
66              & 94,49        & 2,62         & 92,47       & 98,94         & 1,06          & 2,26          & 97,74         \\ \hline
67              & 99,61        & 24,79        & 98,05       & 42,52         & 57,48         & 1,58          & 98,42         \\ \hline
68              & 96,55        & 19,86        & 93,64       & 81,47         & 18,53         & 3,17          & 96,83         \\ \hline
69              & 99,71        & 29,77        & 98,99       & 48,73         & 51,27         & 0,72          & 99,28         \\ \hline
70              & 3,57         & 25,3         & 4,24        & 99,17         & 0,83          & 40,04         & 59,96         \\ \hline
71              & 99,45        & 29,77        & 97,99       & 46,12         & 53,88         & 1,49          & 98,51         \\ \hline
72              & 99,96        & 23,43        & 99,3        & 14,91         & 85,09         & 0,66          & 99,34         \\ \hline
73              & 72,74        & 25,58        & 70,01       & 94,55         & 5,45          & 5,92          & 94,08         \\ \hline
74              & 99,53        & 21,65        & 97,49       & 44,58         & 55,42         & 2,07          & 97,93         \\ \hline
75              & 98,64        & 21,92        & 97,06       & 74,66         & 25,34         & 1,64          & 98,36         \\ \hline
76              & 80,05        & 23,76        & 79,11       & 98,03         & 1,97          & 1,58          & 98,42         \\ \hline
77              & 98,05        & 27,03        & 95,91       & 69,89         & 30,11         & 2,26          & 97,74         \\ \hline
78              & 98,85        & 24,58        & 97,6        & 73,23         & 26,77         & 1,29          & 98,71         \\ \hline
79              & 72,42        & 31,93        & 71,56       & 97,56         & 2,44          & 1,99          & 98,01         \\ \hline
80              & 91,14        & 26,79        & 88,48       & 88,47         & 11,53         & 3,35          & 96,65         \\ \hline
81              & 22,88        & 17,47        & 22,73       & 99,36         & 0,64          & 9,34          & 90,66         \\ \hline
82              & 9,57         & 32,59        & 10,09       & 99,17         & 0,83          & 14,01         & 85,99         \\ \hline
83              & 90,55        & 7,82         & 88,61       & 98,05         & 1,95          & 2,39          & 97,61         \\ \hline
84              & 97,75        & 24,93        & 96,49       & 83,65         & 16,35         & 1,34          & 98,66         \\ \hline
85              & 89,18        & 35,72        & 86,97       & 87,52         & 12,48         & 3,02          & 96,98         \\ \hline
86              & 81,63        & 26,93        & 80,16       & 96,12         & 3,88          & 2,4           & 97,6          \\ \hline
87              & 97           & 26,1         & 94,32       & 74,51         & 25,49         & 2,91          & 97,09         \\ \hline
88              & 99,14        & 21,36        & 96,73       & 55,83         & 44,17         & 2,47          & 97,53         \\ \hline
89              & 99,3         & 33,29        & 98,41       & 60,35         & 39,65         & 0,91          & 99,09         \\ \hline
90              & 77,29        & 17,54        & 75,8        & 98,05         & 1,95          & 2,67          & 97,33         \\ \hline
91              & 92,11        & 13,85        & 88,99       & 93,2          & 6,8           & 3,74          & 96,26         \\ \hline
92              & 98,02        & 22,89        & 95,76       & 73,55         & 26,45         & 2,38          & 97,62         \\ \hline
93              & 98,6         & 22,06        & 96,65       & 70,9          & 29,1          & 2,02          & 97,98         \\ \hline
94              & 98,91        & 27,83        & 98,14       & 77,97         & 22,03         & 0,8           & 99,2          \\ \hline
95              & 96,71        & 29,22        & 95,25       & 83,51         & 16,49         & 1,6           & 98,4          \\ \hline
96              & 99,33        & 23,86        & 97,6        & 54,37         & 45,63         & 1,77          & 98,23         \\ \hline
97              & 99,18        & 35,16        & 96,9        & 38,54         & 61,46         & 2,36          & 97,64         \\ \hline
98              & 84,82        & 24,49        & 82,94       & 95,07         & 4,93          & 2,78          & 97,22         \\ \hline
99              & 84,79        & 28,84        & 84,19       & 97,97         & 2,03          & 0,91          & 99,09         \\ \hline
100             & 97,52        & 23,88        & 94,81       & 73,16         & 26,84         & 2,89          & 97,11         \\ \hline
\label{anx:modmin}
\end{longtable}
\section{SUAVIDAD-RGB}
\begin{longtable}[c]{|l|l|l|l|l|l|l|l|}
\hline
\textbf{Imagen} & \textbf{ESP} & \textbf{SEN} & \textbf{EX} & \textbf{FP\%} & \textbf{VP\%} & \textbf{FN\%} & \textbf{VN\%} \\ \hline
\endfirsthead
%
\endhead
%
1               & 99,65        & 87,57        & 99,48       & 21,59         & 78,41         & 0,18          & 99,82         \\ \hline
2               & 99,9         & 76,43        & 99,78       & 21,9          & 78,1          & 0,11          & 99,89         \\ \hline
3               & 99,12        & 25,59        & 94,79       & 35,37         & 64,63         & 4,49          & 95,51         \\ \hline
4               & 99,64        & 39,6         & 98,72       & 36,36         & 63,64         & 0,94          & 99,06         \\ \hline
5               & 99,85        & 21,31        & 98,66       & 31,16         & 68,84         & 1,2           & 98,8          \\ \hline
6               & 97,19        & 29,31        & 95,81       & 82,21         & 17,79         & 1,49          & 98,51         \\ \hline
7               & 99,93        & 10,45        & 98,42       & 27,2          & 72,8          & 1,52          & 98,48         \\ \hline
8               & 93,96        & 38,99        & 89,91       & 66,04         & 33,96         & 4,92          & 95,08         \\ \hline
9               & 5,24         & 30,14        & 5,74        & 99,35         & 0,65          & 21,47         & 78,53         \\ \hline
10              & 98,16        & 12,31        & 94,87       & 78,93         & 21,07         & 3,44          & 96,56         \\ \hline
11              & 97,18        & 50,05        & 95,06       & 54,34         & 45,66         & 2,37          & 97,63         \\ \hline
12              & 97,56        & 28,87        & 92,61       & 52,07         & 47,93         & 5,36          & 94,64         \\ \hline
13              & 98,66        & 34,15        & 95,64       & 44,35         & 55,65         & 3,18          & 96,82         \\ \hline
14              & 71,39        & 30,32        & 70,03       & 96,51         & 3,49          & 3,23          & 96,77         \\ \hline
15              & 95,04        & 36,37        & 93,21       & 80,88         & 19,12         & 2,11          & 97,89         \\ \hline
16              & 99,44        & 39,31        & 97,28       & 27,55         & 72,45         & 2,23          & 97,77         \\ \hline
17              & 97,94        & 32,71        & 96,76       & 77,47         & 22,53         & 1,24          & 98,76         \\ \hline
18              & 100          & 23,38        & 98,98       & 0             & 100           & 1,02          & 98,98         \\ \hline
19              & 97,97        & 22,98        & 93,66       & 59,15         & 40,85         & 4,58          & 95,42         \\ \hline
20              & 54,19        & 28,39        & 52,52       & 95,89         & 4,11          & 8,37          & 91,63         \\ \hline
21              & 99,52        & 28,97        & 98,17       & 45,9          & 54,1          & 1,37          & 98,63         \\ \hline
22              & 88,8         & 36,08        & 87,75       & 93,89         & 6,11          & 1,43          & 98,57         \\ \hline
23              & 99,93        & 20,47        & 99,27       & 30,27         & 69,73         & 0,66          & 99,34         \\ \hline
24              & 98,55        & 33,72        & 96,23       & 53,78         & 46,22         & 2,43          & 97,57         \\ \hline
25              & 99,19        & 8,79         & 95,75       & 69,83         & 30,17         & 3,51          & 96,49         \\ \hline
26              & 97,69        & 9,48         & 92,42       & 79,31         & 20,69         & 5,56          & 94,44         \\ \hline
27              & 99,69        & 8,66         & 99,01       & 82,65         & 17,35         & 0,69          & 99,31         \\ \hline
28              & 73,49        & 24,17        & 71,13       & 95,62         & 4,38          & 4,93          & 95,07         \\ \hline
29              & 99,7         & 21,63        & 96,7        & 25,81         & 74,19         & 3,04          & 96,96         \\ \hline
30              & 98,62        & 32,49        & 95,55       & 46,5          & 53,5          & 3,23          & 96,77         \\ \hline
31              & 81,21        & 25,82        & 78,68       & 93,81         & 6,19          & 4,2           & 95,8          \\ \hline
32              & 99,76        & 20,18        & 97,38       & 27,88         & 72,12         & 2,41          & 97,59         \\ \hline
33              & 98,28        & 24,66        & 95,45       & 63,62         & 36,38         & 2,97          & 97,03         \\ \hline
34              & 99,7         & 30,14        & 98,11       & 29,55         & 70,45         & 1,62          & 98,38         \\ \hline
35              & 97,16        & 24,49        & 93,84       & 70,8          & 29,2          & 3,59          & 96,41         \\ \hline
36              & 99,59        & 29,88        & 97,18       & 27,57         & 72,43         & 2,46          & 97,54         \\ \hline
37              & 99,27        & 34,95        & 96,89       & 35,1          & 64,9          & 2,46          & 97,54         \\ \hline
38              & 98,77        & 32,84        & 95,59       & 42,37         & 57,63         & 3,34          & 96,66         \\ \hline
39              & 99,81        & 38,39        & 98,58       & 19,85         & 80,15         & 1,24          & 98,76         \\ \hline
40              & 99,78        & 29,72        & 97,95       & 21,51         & 78,49         & 1,85          & 98,15         \\ \hline
41              & 98,98        & 24,69        & 94,7        & 40,26         & 59,74         & 4,45          & 95,55         \\ \hline
42              & 99,81        & 26,88        & 98,61       & 30,16         & 69,84         & 1,21          & 98,79         \\ \hline
43              & 98,48        & 33,45        & 97,14       & 68,46         & 31,54         & 1,4           & 98,6          \\ \hline
44              & 98,72        & 26,52        & 95,93       & 54,57         & 45,43         & 2,9           & 97,1          \\ \hline
45              & 54,52        & 36,03        & 53,9        & 97,3          & 2,7           & 3,95          & 96,05         \\ \hline
46              & 99,5         & 40,31        & 97,85       & 30,47         & 69,53         & 1,68          & 98,32         \\ \hline
47              & 74,63        & 14,39        & 71,77       & 97,25         & 2,75          & 5,41          & 94,59         \\ \hline
48              & 88,37        & 28,18        & 87,39       & 96,14         & 3,86          & 1,33          & 98,67         \\ \hline
49              & 74,98        & 11,2         & 73,03       & 98,61         & 1,39          & 3,6           & 96,4          \\ \hline
50              & 92,86        & 23,17        & 89,7        & 86,63         & 13,37         & 3,79          & 96,21         \\ \hline
51              & 99,89        & 27,72        & 98,45       & 16,5          & 83,5          & 1,45          & 98,55         \\ \hline
52              & 99,64        & 28,87        & 97,59       & 29,68         & 70,32         & 2,08          & 97,92         \\ \hline
53              & 90,42        & 18           & 89,12       & 96,69         & 3,31          & 1,63          & 98,37         \\ \hline
54              & 99,21        & 27,49        & 96,47       & 41,95         & 58,05         & 2,82          & 97,18         \\ \hline
55              & 90,39        & 34,24        & 89,36       & 93,79         & 6,21          & 1,33          & 98,67         \\ \hline
56              & 97,31        & 30,9         & 94,78       & 68,65         & 31,35         & 2,74          & 97,26         \\ \hline
57              & 2,46         & 17,56        & 3,38        & 98,84         & 1,16          & 68,56         & 31,44         \\ \hline
58              & 97,65        & 32,06        & 93,41       & 51,39         & 48,61         & 4,59          & 95,41         \\ \hline
59              & 91,58        & 27,08        & 90,42       & 94,44         & 5,56          & 1,44          & 98,56         \\ \hline
60              & 79,3         & 31,97        & 76,64       & 91,57         & 8,43          & 4,86          & 95,14         \\ \hline
61              & 95,49        & 31,52        & 94,39       & 89,09         & 10,91         & 1,24          & 98,76         \\ \hline
62              & 74,14        & 26,7         & 72,55       & 96,54         & 3,46          & 3,32          & 96,68         \\ \hline
63              & 99,5         & 33,72        & 98,63       & 52,29         & 47,71         & 0,88          & 99,12         \\ \hline
64              & 99,73        & 26,65        & 98,66       & 40,21         & 59,79         & 1,08          & 98,92         \\ \hline
65              & 1,88         & 23,35        & 2,93        & 98,79         & 1,21          & 67,84         & 32,16         \\ \hline
66              & 94,74        & 2,54         & 92,72       & 98,92         & 1,08          & 2,26          & 97,74         \\ \hline
67              & 99,62        & 24,79        & 98,06       & 42,11         & 57,89         & 1,58          & 98,42         \\ \hline
68              & 97,64        & 19,87        & 94,69       & 75,03         & 24,97         & 3,14          & 96,86         \\ \hline
69              & 99,7         & 29,5         & 98,98       & 49,93         & 50,07         & 0,73          & 99,27         \\ \hline
70              & 4,12         & 25,3         & 4,77        & 99,17         & 0,83          & 36,66         & 63,34         \\ \hline
71              & 99,15        & 33,42        & 97,77       & 54,19         & 45,81         & 1,42          & 98,58         \\ \hline
72              & 85,71        & 21,5         & 85,15       & 98,7          & 1,3           & 0,79          & 99,21         \\ \hline
73              & 73,66        & 24,83        & 70,83       & 94,52         & 5,48          & 5,9           & 94,1          \\ \hline
74              & 99,29        & 21,85        & 97,27       & 54,57         & 45,43         & 2,07          & 97,93         \\ \hline
75              & 96,82        & 24,15        & 95,32       & 86,23         & 13,77         & 1,62          & 98,38         \\ \hline
76              & 95,95        & 23,84        & 94,75       & 90,95         & 9,05          & 1,32          & 98,68         \\ \hline
77              & 94,34        & 27,01        & 92,31       & 87,08         & 12,92         & 2,35          & 97,65         \\ \hline
78              & 99,08        & 24,95        & 97,84       & 68,32         & 31,68         & 1,28          & 98,72         \\ \hline
79              & 72,29        & 31,99        & 71,43       & 97,56         & 2,44          & 1,99          & 98,01         \\ \hline
80              & 58,34        & 26,61        & 57,03       & 97,32         & 2,68          & 5,14          & 94,86         \\ \hline
81              & 74,08        & 16,71        & 72,49       & 98,19         & 1,81          & 3,11          & 96,89         \\ \hline
82              & 24,93        & 31,35        & 25,08       & 99,04         & 0,96          & 5,99          & 94,01         \\ \hline
83              & 90,27        & 7,75         & 88,33       & 98,12         & 1,88          & 2,4           & 97,6          \\ \hline
84              & 97,73        & 25,02        & 96,47       & 83,75         & 16,25         & 1,33          & 98,67         \\ \hline
85              & 88,87        & 36,31        & 86,69       & 87,66         & 12,34         & 3             & 97            \\ \hline
86              & 81,66        & 26,72        & 80,18       & 96,14         & 3,86          & 2,41          & 97,59         \\ \hline
87              & 83,39        & 26,4         & 81,23       & 94,12         & 5,88          & 3,35          & 96,65         \\ \hline
88              & 99,16        & 20,99        & 96,74       & 55,59         & 44,41         & 2,48          & 97,52         \\ \hline
89              & 99,1         & 33,82        & 98,22       & 65,83         & 34,17         & 0,91          & 99,09         \\ \hline
90              & 78,35        & 17,69        & 76,83       & 97,94         & 2,06          & 2,63          & 97,37         \\ \hline
91              & 90,75        & 13,81        & 87,68       & 94,15         & 5,85          & 3,8           & 96,2          \\ \hline
92              & 98,26        & 22,66        & 95,98       & 71,26         & 28,74         & 2,38          & 97,62         \\ \hline
93              & 98,6         & 22,06        & 96,65       & 70,9          & 29,1          & 2,02          & 97,98         \\ \hline
94              & 98,87        & 27,38        & 98,09       & 78,88         & 21,12         & 0,8           & 99,2          \\ \hline
95              & 93,81        & 29,07        & 92,41       & 90,55         & 9,45          & 1,65          & 98,35         \\ \hline
96              & 99,35        & 23,86        & 97,62       & 53,57         & 46,43         & 1,77          & 98,23         \\ \hline
97              & 99,1         & 35,85        & 96,84       & 40,43         & 59,57         & 2,34          & 97,66         \\ \hline
98              & 82,34        & 25,68        & 80,58       & 95,53         & 4,47          & 2,82          & 97,18         \\ \hline
99              & 84,55        & 28,41        & 83,94       & 98,03         & 1,97          & 0,92          & 99,08         \\ \hline
100             & 54,32        & 25,83        & 53,27       & 97,89         & 2,11          & 4,95          & 95,05         \\ \hline
\label{anx:suavidad}
\end{longtable}
\section{VARIANZA-RGB}
\begin{longtable}[c]{|l|l|l|l|l|l|l|l|}
\hline
\textbf{Imagen} & \textbf{ESP} & \textbf{SEN} & \textbf{EX} & \textbf{FP\%} & \textbf{VP\%} & \textbf{FN\%} & \textbf{VN\%} \\ \hline
\endfirsthead
%
\endhead
%
1               & 99,64        & 88,87        & 99,48       & 22,24         & 77,76         & 0,16          & 99,84         \\ \hline
2               & 99,9         & 75           & 99,78       & 21,05         & 78,95         & 0,12          & 99,88         \\ \hline
3               & 99,12        & 25,56        & 94,78       & 35,53         & 64,47         & 4,49          & 95,51         \\ \hline
4               & 99,53        & 38,94        & 98,59       & 43,68         & 56,32         & 0,95          & 99,05         \\ \hline
5               & 99,85        & 21,31        & 98,66       & 31,16         & 68,84         & 1,2           & 98,8          \\ \hline
6               & 97,18        & 30,96        & 95,83       & 81,44         & 18,56         & 1,45          & 98,55         \\ \hline
7               & 99,94        & 10,45        & 98,42       & 26,02         & 73,98         & 1,52          & 98,48         \\ \hline
8               & 97,56        & 38,69        & 93,22       & 44,19         & 55,81         & 4,77          & 95,23         \\ \hline
9               & 3,68         & 30,14        & 4,21        & 99,36         & 0,64          & 28,04         & 71,96         \\ \hline
10              & 97,5         & 12,31        & 94,23       & 83,61         & 16,39         & 3,46          & 96,54         \\ \hline
11              & 98,54        & 49,65        & 96,34       & 38,29         & 61,71         & 2,36          & 97,64         \\ \hline
12              & 97,57        & 28,91        & 92,62       & 52,01         & 47,99         & 5,36          & 94,64         \\ \hline
13              & 98,71        & 33,63        & 95,66       & 43,76         & 56,24         & 3,2           & 96,8          \\ \hline
14              & 98,77        & 29,11        & 96,47       & 55,35         & 44,65         & 2,39          & 97,61         \\ \hline
15              & 94,75        & 36,61        & 92,93       & 81,65         & 18,35         & 2,11          & 97,89         \\ \hline
16              & 98,92        & 39,93        & 96,79       & 42,13         & 57,87         & 2,22          & 97,78         \\ \hline
17              & 97,92        & 32,13        & 96,74       & 77,88         & 22,12         & 1,26          & 98,74         \\ \hline
18              & 100          & 23,38        & 98,98       & 0             & 100           & 1,02          & 98,98         \\ \hline
19              & 97,98        & 22,79        & 93,66       & 59,18         & 40,82         & 4,59          & 95,41         \\ \hline
20              & 55,39        & 28,31        & 53,64       & 95,8          & 4,2           & 8,21          & 91,79         \\ \hline
21              & 99,52        & 28,87        & 98,17       & 46,17         & 53,83         & 1,37          & 98,63         \\ \hline
22              & 89,02        & 35,7         & 87,96       & 93,84         & 6,16          & 1,44          & 98,56         \\ \hline
23              & 99,93        & 20,47        & 99,27       & 30,27         & 69,73         & 0,66          & 99,34         \\ \hline
24              & 98,55        & 33,72        & 96,23       & 53,81         & 46,19         & 2,43          & 97,57         \\ \hline
25              & 99,2         & 8,79         & 95,76       & 69,64         & 30,36         & 3,51          & 96,49         \\ \hline
26              & 97,67        & 9,48         & 92,4        & 79,48         & 20,52         & 5,56          & 94,44         \\ \hline
27              & 99,69        & 8,63         & 99,01       & 82,4          & 17,6          & 0,69          & 99,31         \\ \hline
28              & 33,69        & 24,95        & 33,27       & 98,14         & 1,86          & 10,07         & 89,93         \\ \hline
29              & 99,7         & 21,66        & 96,7        & 25,79         & 74,21         & 3,04          & 96,96         \\ \hline
30              & 98,59        & 32,49        & 95,51       & 47,16         & 52,84         & 3,23          & 96,77         \\ \hline
31              & 81,27        & 25,81        & 78,73       & 93,8          & 6,2           & 4,2           & 95,8          \\ \hline
32              & 99,78        & 20,18        & 97,39       & 26,35         & 73,65         & 2,41          & 97,59         \\ \hline
33              & 98,29        & 24,66        & 95,46       & 63,45         & 36,55         & 2,97          & 97,03         \\ \hline
34              & 99,7         & 30,04        & 98,1        & 30,01         & 69,99         & 1,62          & 98,38         \\ \hline
35              & 96,75        & 25,76        & 93,5        & 72,52         & 27,48         & 3,54          & 96,46         \\ \hline
36              & 99,64        & 29,15        & 97,2        & 25,38         & 74,62         & 2,49          & 97,51         \\ \hline
37              & 99,3         & 34,73        & 96,91       & 34,27         & 65,73         & 2,47          & 97,53         \\ \hline
38              & 99           & 32,67        & 95,79       & 37,64         & 62,36         & 3,34          & 96,66         \\ \hline
39              & 99,86        & 37,71        & 98,62       & 15,88         & 84,12         & 1,25          & 98,75         \\ \hline
40              & 99,78        & 29,26        & 97,94       & 22,03         & 77,97         & 1,87          & 98,13         \\ \hline
41              & 92,15        & 24,45        & 88,25       & 84            & 16            & 4,78          & 95,22         \\ \hline
42              & 99,78        & 27,53        & 98,6        & 32,43         & 67,57         & 1,2           & 98,8          \\ \hline
43              & 99,3         & 30,83        & 97,9        & 51,91         & 48,09         & 1,44          & 98,56         \\ \hline
44              & 89,11        & 26,56        & 86,7        & 91,08         & 8,92          & 3,2           & 96,8          \\ \hline
45              & 98,25        & 32,83        & 96,04       & 60,31         & 39,69         & 2,34          & 97,66         \\ \hline
46              & 99,5         & 40,5         & 97,87       & 30,09         & 69,91         & 1,68          & 98,32         \\ \hline
47              & 74,47        & 14,39        & 71,61       & 97,27         & 2,73          & 5,42          & 94,58         \\ \hline
48              & 89,19        & 28,53        & 88,2        & 95,81         & 4,19          & 1,31          & 98,69         \\ \hline
49              & 38,86        & 11,3         & 38,02       & 99,42         & 0,58          & 6,72          & 93,28         \\ \hline
50              & 72,49        & 23,32        & 70,26       & 96,12         & 3,88          & 4,79          & 95,21         \\ \hline
51              & 99,9         & 27,8         & 98,46       & 15,63         & 84,37         & 1,45          & 98,55         \\ \hline
52              & 99,65        & 28,95        & 97,6        & 29,05         & 70,95         & 2,08          & 97,92         \\ \hline
53              & 90,61        & 18,05        & 89,31       & 96,62         & 3,38          & 1,62          & 98,38         \\ \hline
54              & 99,2         & 27,39        & 96,46       & 42,34         & 57,66         & 2,83          & 97,17         \\ \hline
55              & 91,53        & 33,62        & 90,47       & 93,13         & 6,87          & 1,33          & 98,67         \\ \hline
56              & 97,41        & 30,72        & 94,86       & 67,99         & 32,01         & 2,75          & 97,25         \\ \hline
57              & 1,48         & 17,56        & 2,47        & 98,85         & 1,15          & 78,33         & 21,67         \\ \hline
58              & 97,83        & 31,41        & 93,53       & 49,94         & 50,06         & 4,63          & 95,37         \\ \hline
59              & 91,65        & 26,89        & 90,48       & 94,44         & 5,56          & 1,44          & 98,56         \\ \hline
60              & 79,26        & 30,67        & 76,53       & 91,9          & 8,1           & 4,95          & 95,05         \\ \hline
61              & 95,48        & 31,24        & 94,38       & 89,2          & 10,8          & 1,25          & 98,75         \\ \hline
62              & 77,33        & 27,85        & 75,67       & 95,9          & 4,1           & 3,14          & 96,86         \\ \hline
63              & 99,64        & 36,9         & 98,81       & 41,94         & 58,06         & 0,84          & 99,16         \\ \hline
64              & 99,71        & 26,95        & 98,64       & 42,3          & 57,7          & 1,08          & 98,92         \\ \hline
65              & 1,85         & 23,35        & 2,9         & 98,79         & 1,21          & 68,21         & 31,79         \\ \hline
66              & 95           & 2,74         & 92,97       & 98,78         & 1,22          & 2,25          & 97,75         \\ \hline
67              & 99,61        & 24,83        & 98,05       & 42,49         & 57,51         & 1,58          & 98,42         \\ \hline
68              & 97,65        & 19,78        & 94,7        & 75,03         & 24,97         & 3,14          & 96,86         \\ \hline
69              & 99,69        & 29,5         & 98,98       & 50,07         & 49,93         & 0,73          & 99,27         \\ \hline
70              & 2,01         & 25,3         & 2,73        & 99,18         & 0,82          & 54,24         & 45,76         \\ \hline
71              & 99,35        & 31           & 97,92       & 49,29         & 50,71         & 1,47          & 98,53         \\ \hline
72              & 85,79        & 23,91        & 85,26       & 98,55         & 1,45          & 0,77          & 99,23         \\ \hline
73              & 73,63        & 24,82        & 70,8        & 94,53         & 5,47          & 5,91          & 94,09         \\ \hline
74              & 99,53        & 21,65        & 97,49       & 44,74         & 55,26         & 2,07          & 97,93         \\ \hline
75              & 98,12        & 22,69        & 96,57       & 79,71         & 20,29         & 1,63          & 98,37         \\ \hline
76              & 97           & 23,27        & 95,77       & 88,41         & 11,59         & 1,32          & 98,68         \\ \hline
77              & 94,04        & 27           & 92,02       & 87,66         & 12,34         & 2,36          & 97,64         \\ \hline
78              & 99,06        & 24,95        & 97,81       & 68,91         & 31,09         & 1,28          & 98,72         \\ \hline
79              & 80,74        & 31,44        & 79,7        & 96,59         & 3,41          & 1,8           & 98,2          \\ \hline
80              & 94           & 26,25        & 91,2        & 84,15         & 15,85         & 3,27          & 96,73         \\ \hline
81              & 72,05        & 16,62        & 70,51       & 98,33         & 1,67          & 3,2           & 96,8          \\ \hline
82              & 9,42         & 32,66        & 9,94        & 99,17         & 0,83          & 14,2          & 85,8          \\ \hline
83              & 90,26        & 7,79         & 88,33       & 98,11         & 1,89          & 2,4           & 97,6          \\ \hline
84              & 97,73        & 24,78        & 96,47       & 83,86         & 16,14         & 1,34          & 98,66         \\ \hline
85              & 89,03        & 34,98        & 86,79       & 87,9          & 12,1          & 3,06          & 96,94         \\ \hline
86              & 82,18        & 26,39        & 80,69       & 96,08         & 3,92          & 2,41          & 97,59         \\ \hline
87              & 96,94        & 26,21        & 94,26       & 74,82         & 25,18         & 2,91          & 97,09         \\ \hline
88              & 99,09        & 20,9         & 96,67       & 57,57         & 42,43         & 2,49          & 97,51         \\ \hline
89              & 99,23        & 34,08        & 98,35       & 62,1          & 37,9          & 0,9           & 99,1          \\ \hline
90              & 80,54        & 17,59        & 78,96       & 97,73         & 2,27          & 2,56          & 97,44         \\ \hline
91              & 85,12        & 13,84        & 82,27       & 96,28         & 3,72          & 4,04          & 95,96         \\ \hline
92              & 98,43        & 22,51        & 96,15       & 69,18         & 30,82         & 2,38          & 97,62         \\ \hline
93              & 98,16        & 22,08        & 96,23       & 76,15         & 23,85         & 2,03          & 97,97         \\ \hline
94              & 98,87        & 27,33        & 98,09       & 78,99         & 21,01         & 0,8           & 99,2          \\ \hline
95              & 93,82        & 28,97        & 92,41       & 90,58         & 9,42          & 1,65          & 98,35         \\ \hline
96              & 99,37        & 23,62        & 97,63       & 53,36         & 46,64         & 1,77          & 98,23         \\ \hline
97              & 99,09        & 35,89        & 96,83       & 40,77         & 59,23         & 2,34          & 97,66         \\ \hline
98              & 80,72        & 27,15        & 79,06       & 95,67         & 4,33          & 2,82          & 97,18         \\ \hline
99              & 84,62        & 28,48        & 84,02       & 98,02         & 1,98          & 0,91          & 99,09         \\ \hline
100             & 50,61        & 25,62        & 49,69       & 98,06         & 1,94          & 5,31          & 94,69         \\ \hline
\label{anx:varianza}
\end{longtable}

% estos comandos generan la bibliografica
\printbibliography
\end{document}
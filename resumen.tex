\begin{resumen}

Este trabajo tiene como objetivo mejorar la accesibilidad a la información de contratación pública haciendo que esta información sea procesable para las máquinas con el fin de proporcionar transparencia al proceso de contratación pública. Se alcanzó una interoperabilidad sintáctica y semántica del dominio de contrataciones públicas para integrar datos y enriquecerlos con datos de otros dominios de conocimiento.

Se modeló una ontología y un proceso de transformación de los datos de contrataciones públicas a RDF para proporcionar una capa semántica formal a un estándar de datos abiertos para la publicación de datos estructurados denominado Estándar de Datos de Contratación Abierta (de sus siglas en ingles OCDS) aplicado a la Dirección Nacional de Contrataciones Públicas (DNCP) utilizando la metodología NeOn para el proceso de desarrollo de la ontología.
    
\textbf{Palabras clave:} Web Semántica,  Ontologías, Contrataciones Públicas, Open Contracting, OWL, SPARQL.
\end{resumen}



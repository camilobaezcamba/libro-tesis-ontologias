\begin{resumen}



Este trabajo tiene como objetivo mejorar la accesibilidad a la información de contratación pública haciendo que esta sea procesable para las máquinas con el fin de proporcionar mayor transparencia al proceso. Donde se entiende por mayor transparencia una mayor disponibilidad de información para ser utilizada por los ciudadanos. Se alcanzó una interoperabilidad sintáctica y semántica del dominio de contrataciones públicas para integrar datos y enriquecerlos con datos de otros dominios de conocimiento. 

Se modeló una ontología y un proceso de transformación de los datos de la Dirección Nacional de Contrataciones Publicas (DNCP) al formato RDF con el fin de añadir una capa semántica formal sobre el Estándar de Datos de Contratación Abierta (de sus siglas en ingles OCDS) aplicado a la Dirección Nacional de Contrataciones Públicas (DNCP) utilizando la metodología NeOn para el proceso de desarrollo de la ontología.
    
\textbf{Palabras clave:} Web Semántica,  Ontologías, Contrataciones Públicas, Open Contracting, OWL, SPARQL.
\end{resumen}


